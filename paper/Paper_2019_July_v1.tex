\documentclass[12pt]{article}

\usepackage{amssymb,amsmath,amsthm}
\usepackage{mathrsfs}
\usepackage{bbm}
\usepackage{fancyhdr}
\usepackage[mmddyyyy,hhmmss]{datetime}
\usepackage{pstricks, sgamevar, egameps}
\usepackage{chapterbib}
\usepackage{graphicx}
\usepackage{setspace}
\usepackage{cases}
\usepackage{url}
\usepackage{placeins}
\usepackage{longtable}
\usepackage{tabularx} 
\usepackage{caption}
\usepackage{threeparttable}
\usepackage{booktabs}
\usepackage{outlines}
\usepackage{float}
\usepackage{mathabx}
\usepackage{setspace}
\usepackage[breaklinks,hidelinks]{hyperref}
\usepackage{authblk}

\usepackage{apacite}
\doublespacing

\newtheorem{assumption}{Assumption}
\newtheorem{definition}{Definition}

\DeclareMathOperator*{\argmin}{arg\,min}
\DeclareMathOperator*{\argmax}{arg\,max}

\topmargin=-1in      
\evensidemargin=0in     
\oddsidemargin=0in      
\textwidth=6.5in        
\textheight=9.0in       
\headsep=0.25in

%\renewcommand{\baselinestretch}{2.0}

\usepackage[margin = 1in]{geometry}

\newcommand*{\graphicPath}{"/Users/Miriam/OneDrive/Box Sync/TYPlocal/output/graphs/matlabwages/"}% 
\newcommand*{\TablePath}{"/Users/Miriam/OneDrive/Box Sync/TYPlocal/output/tables/"}% 
\graphicspath{{\graphicPath}}
\makeatletter
\def\input@path{{\TablePath}}
\makeatother

\title{Gender Segregation in Occupations: Preferences or Homophily?}
%\title{Equal Pay for Equal Work: Unintended Consequences in the Presence of Homophily}
%\title{Gender Segregation in Occupations: The Consequences of Equal Pay Laws and Homophily}
\author{Miriam Larson-Koester\footnote{Federal Trade Commission, mlarsonkoester@ftc.gov. The views expressed are those of the author and do not necessarily reflect those of the Federal Trade Commission. I am grateful to Francesca Molinari, Richard Mansfield and Victoria Prowse for their advice and support. For helpful discussions and insights I thank Francine Blau, Panle Barwick, Mallika Thomas, Larry Blume, JF Houde, Yang Zhang, Angela Cools, Jorgen Harris, Caroline Walker, and numerous seminar participants. All errors are mine.}}
%\affil[1]{Department of Economics, Cornell University}



%\date{Updated \today \\
%\textit{for latest version click} \href{https://drive.google.com/file/d/0B6RyLRYAyJn_U0I0RkU5NkZhME0/view?usp=sharing}{\underline{here}}}

%This research was made possible through the use of Cornell University's Economics Compute Cluster organization, which was partially funded through NSF grant \#0922005.
 
\begin{document}

\pagenumbering{roman}
 
\maketitle

%\begin{center}
%\large{For latest version click \href{https://drive.google.com/file/d/0B6RyLRYAyJn_U0I0RkU5NkZhME0/view?usp=sharing}{\underline{here}}.}
%\end{center}

% with angela edits
% The sorting of men and women into different occupations is a persistent feature of the labor market and is increasingly important in explaining the gender wage gap in the U.S. 

\abstract{Women remain concentrated in certain occupations despite the drastic increase in women's labor force participation in the U.S. since 1960. I examine whether occupations remain segregated because workers prefer to enter occupations that already employ more of their own gender. I build a model of labor supply and demand in which firms care about the gender and wages of their employees, and workers get utility from their occupation, wage, and the number of women in their occupation. Using a Bartik instrumental variables strategy, I find that women prefer to enter into female-dominated occupations, but men show no evidence of gender preference. In a world without women's gender preference, occupations would be 50 percent less segregated and the gender wage gap 18 percent smaller. Homophily can mean that a given occupation could be either male or female dominated depending on how many women were in it initially. However, women's preference for working with women would have to be almost twice as strong for the number of women in the past to affect whether an occupation is male or female dominated today.}

% I built a model ... before us data labor market where workers have gender preferences and labor supply and demand... build a model of labor supply and demand where  gender prefrences affect occupational sorting

% in a world without gender pref, 50% less segregation and 18% less wage gap. wages adjust to compensating women for their preference. wage adjusment counteracts the preference

% gitkraken... sublime

%The sorting of men and women into different occupations is a persistent feature of the labor market and is increasingly important in explaining the gender wage gap in the U.S. \cite{Blau2017}. With U.S. data, I structurally estimate preferences over worker preferences over the gender in their occupation, and the reservation wage gap for men and women. Using a Bartik instrumental variables strategy, I find that women prefer to enter into female-dominated occupations, but men show no evidence of gender preference. Womens' preference for gender accounts for almost 50\% of  gender segregation. Wages adjust to compensate women for their preference, which reduces segregation, and explains 18\% of the gender wage gap. Simulations indicate that wage adjustment eliminates historical path dependence in segregation patterns.}

%The sorting of men and women into different occupations is a persistent feature of the labor market and is increasingly important in explaining the gender wage gap in the U.S. \cite{Blau2017}. Using U.S. data, I structurally estimate the willingness-to-pay for male and female workers by occupation, and the preferences of male and female workers for wages, exogenous occupation attributes, and occupation gender composition. Using a shift-share instrumental variables strategy, I find that women prefer to enter into more female occupations, and that this preference increases segregation in the cross-section and accentuates the impact of labor supply and demand shocks on gender composition over time. This gender preference is so strong that based on labor supply alone, most occupations have multiple equilibria, meaning that historical segregation could determine whether an occupation is male or female. However in the labor market, unlike in a classic Schelling model, compensating differentials moderate the impact of the gender preference, and ensure that all occupations have a unique equilibrium in fraction female. Therefore even at the high level of gender preference I estimate, manipulating the gender composition of an occupation alone has no long run impact on segregation.



%This paper examines the extent to which workers care about existing segregation when they make their occupation choice, and what segregation dynamics emerge from this preference. I adapt a transferable utility matching model to estimate both how much male and female workers value occupations, and how those occupations value male and female workers. Using a shift-share instrumental variables strategy, I find that more men in an occupation decreases the willingness of women to enter that occupation. This preference increases segregation and causes rapid changes in gender composition, or ``tipping". However, market clearing wages mitigate the impact of this gender preference through compensating differentials, thereby reducing segregation and ``tipping". In fact, market clearing wages ensure that there is a unique equilibrium in the fraction female by occupation unless gender preferences are twice their estimated values. This means that observed ``tipping" is not the result of multiple equilibria, but rather of the gender preference reinforcing the impact of changes to other parameters. Therefore policies to reduce gender segregation must change fundamentals such as workplace amenities and compensation. Manipulating the gender composition of an occupation alone will have no long-run impact because reducing gender segregation is not a matter of overcoming historical inertia.

%The policy implication is that manipulating the gender composition of an occupation alone will have no long-run impact, unless the policy does so through changing fundamentals such as workplace amenities and compensation. Reducing gender segregation is not a matter of overcoming historical inertia.}

%I then simulate the endogenous evolution of the fraction female across cohorts and find that gender preferences may lead some occupations to become rapidly more male or female in the future, a phenomenon called tipping. However, market clearing wages eliminate tipping because compensating differentials moderate the impact of gender preference. Market-clearing wages cause there to be only one unique equilibrium in the fraction female for each occupation. As a result, we should expect segregation levels to be stable and resilient to short-run shocks. Policies to reduce segregation need to do more than overcome historical inertia in order to have long run impact.}

% in the style of \citeA{Choo2006} and \citeA{Salanie2013a}

%motivating sentence, no methodology, narrative
%
%but for wages we would see more dramatic
%
%why are they persistently sorted: reason is wages stabilizing occupation .. I investigate role of wages and gender pref and elasticities two sided, with blah blah..... and then hook despite strong gender prefs wages moderate.
%
%without wages we would see crazy volatility. sorting, therefore wages act as this.
%
%then intuition for hook results (sentence 3 or 4)

%\abstract{I estimate a strong preference on the part of women against entering male-dominated occupations using a shift-share instrumental variables strategy and structural estimation of reservation wages. I then examine the consequences of this preference for segregation dynamics. In particular, I look for the existence of multiple equilibria in the fraction female by occupation. Using a transferable utility matching model, I find that firm preferences and market-clearing wages ensure than each occupation has a unique equilibrium in the fraction female, and therefore that the current sorting of men and women into occupations is generally stable and not historically dependent. A stronger gender preference, or stickiness of market clearing wages, would be necessary to produce multiple equilibria or the ``tipping" phenomena documented by \citeA{Pan2010}.}



% sorting is persistent. I separate out all the reasons why. it turns out that wages stop gender prefs from causing tipping, the intuition is compensating differentials.

% I empirically estimate the preferences of male and female workers for wages, exogenous occupation attributes, and gender composition (an endogenous occupation attribute) in the U.S. To do so I model lifetime earnings as transfers in a one-to-one matching model of workers to jobs. The model allows me to estimate a distribution of reservation earnings, which combined with a shift-share instrumental variables strategy, identifies worker preferences. I use these estimates within the matching model to predict which occupations have stable and unstable gender composition across generations of workers. I find that the current sorting of men and women into occupations is generally stable and not historically dependent, despite a strong preference on the part of women to enter female occupations. Although previous literature suggests that in-group preferences can produce multiple equilibria, I find that each occupation has a unique fraction female in the long-run equilibrium. Allowing wages to adjust as compensating differentials is key to this result.

% I separately estimate firm and worker preferences by modeling lifetime wages as transfers in a one-to-one matching model of the labor market. Using the estimated distribution of reservation wages I am able to disentangle male and female preferences for wages, gender composition, and exogenous occupation attributes using a shift-share instrumental variables strategy, which exploits variation in occupation exposure to industries, male and female labor force participation, and estimated firm willingness to pay. I then use these estimates and equilibrium wages from the matching model to predict which occupations have stable and unstable gender composition across generations. I find that with some notable exceptions that the current segregation patterns are stable, and not historically dependent, despite a strong preference on the part of women to enter female occupations.

%Patterns of occupation segregation may be due in part to preferences for working with one's own gender. In this paper, I use structural estimation and instrumental variables to estimate firms' willingness-to-pay for male and female workers separately from worker preferences for occupation attributes and occupation gender composition. I separately estimate firm and worker preferences by modeling lifetime wages as transfers in a one-to-one matching model of the labor market. Using the estimated distribution of reservation wages I am able to disentangle male and female preferences for wages, gender composition, and exogenous occupation attributes using a shift-share instrumental variables strategy, which exploits variation in occupation exposure to industries, male and female labor force participation, and estimated firm willingness to pay. I then use these estimates to predict which occupations have stable and unstable gender composition. I find that with some notable exceptions in STEM fields, which will continue to feminize, and machine operators, which will masculinize, the current segregation patterns are stable, and not historically dependent. If all occupations were at parity in 1960, current observed segregation would be similar to today.

%\abstract{Persistent segregation, and dramatic changes in segregation, may be due in part to preferences for working with one's own gender \cite{Pan2010}. I use structural estimation to predict which occupations have stable and unstable gender composition. With some notable exceptions in STEM fields, which will continue to feminize, and machine operators, which will masculinize, the current segregation patterns are stable, and not historically dependent. Understanding segregation patterns over time requires estimates of worker preference for occupation attributes (exogenous) and gender composition (endogenous), and firms' willingness to pay for male and female workers. I estimate firm and worker preferences by modeling lifetime wages as transfers in a 1:1 matching model of the labor market. Using the estimated distribution of reservation wages I am able to disentangle worker preferences for wages, gender composition, and exogenous occupation attributes using a shift-share instrumental variables strategy, which exploits variation in occupation exposure to industries, male and female labor force participation, and estimated firm willingness to pay. }





%\abstract{The sorting of occupations by gender remains a key source of the gender wage gap in the U.S. \cite{Blau2017}. In this paper I take seriously the idea that the fraction female in an occupation is an endogenous attribute valued by workers. Using a panel instrumental variables strategy, I find that women, more so than men, care about the number of women in their chosen occupation. I adapt a transferable utility matching model in the style of \citeA{Choo2006} and \citeA{Salanie2013a} to simulate the long-run evolution of the fraction female as an endogenous attribute and find that gender preferences on the part of women may lead some occupations to tip male and others female in the future. Allowing the market clearing wage to emerge endogenously from the matching model causes occupations to become more female faster, especially in the case of managerial and professional occupations.}

%gender preferences may cause health diagnosing occupations and engineers architects and surveyors to tip female while food preparation and service, construction excluding precision production, and farming all become more male.
%equilibrium outcomes of gendered policy interventions on the

%I find a preference over the fraction female in an occupation on the part of women and use a transferable utility matching model to simulate the dynamics of segregation, where segregation is an endogenous attribute.

 %Simultaneously, the gender ratio of some occupations has evolved non-linearly over time suggesting a preference over the gender of one's occupation peers \cite{Pan2010}. Possible causes include gender identity, stigma, or correlated occupation attributes. The contribution of this paper is to estimate the impact of such preferences on occupation choice. In a world with such preferences, the gender ratio of occupations evolves endogenously over time, a structural model is needed to simulate the dynamic consequences of policies meant to integrate occupations, whether it be pushing men into nursing or women into STEM fields. This paper adapts a transferable utility matching model in the style of \citeA{Choo2006} and \citeA{Salanie2013a} to flexibly estimate worker and firm preferences and simulate the long-run equilibrium outcomes of gendered policy interventions.}

% angela's notes: too much background get to methodology in second sentence and results in third sentence

%The sorting of occupations by gender remains a key source of the gender wage gap in the U.S. \cite{Kahn2016a}. Simultaneously, the gender ratio of some occupations has evolved non-linearly over time suggesting a preference over the gender of one's occupation peers \cite{Pan2010}. Possible causes include gender identity, stigma, or correlated occupation attributes. The contribution of this paper is to estimate the impact of such preferences on occupation choice. In a world with such preferences, the gender ratio of occupations evolves endogenously over time, a structural model is needed to simulate the dynamic consequences of policies meant to integrate occupations, whether it be pushing men into nursing or women into STEM fields. This paper adapts a transferable utility matching model in the style of \citeA{Choo2006} and \citeA{Salanie2013a} to flexibly estimate worker and firm preferences and simulate the long-run equilibrium outcomes of gendered policy interventions.


%The increase in the female labor force participation rate over the past 60 years has had a disparate impact on the femininity of occupations. While women entered many occupations at rates that could have been predicted given past shares, some occupations remained entirely male dominated, and still others ``tipped" and became female dominated. These gendered patterns are not well understood, with some widely explored mechanisms including preferences, productivity differences, and discrimination. Understanding the dynamics of occupation gender segregation could be important if preference against being the minority gender is acting as an entry barrier distorting occupation choice. Preferences over the fraction female in the occupation itself rather than occupation tasks and amenities may lead to segregation that is not socially optimal due to mismatch. To understand the causes and consequences of occupation gender segregation, I introduce a transferable utility matching model adapted from \citeA{Choo2006} and \citeA{Salanie2013a}, and identify it using wages as transfers. The estimated model will allow me to separately identify the worker and job match surpluses and then test my hypothesis of gender ratio as an entry barrier to the worker. If my hypothesis is true, incentivizing entry into opposite gender occupations could cause long run movement towards an unsegregated equilibrium by eliminating distortion in the occupation choice of future generations.

\newpage
\tableofcontents

\newpage
\section{Introduction}
\pagenumbering{arabic}

%stylized facts, where the research is at, wheres the gap, how I fill the gap, methods, and policy relevance

%Evidence from job satisfaction data is suggestive of worker preferences over the gender composition of occupations \cite{Usui2008, Lordan2015}. In addition, 

% LFP has skyrocketed but occupations still super segregated. policymakers care about this and why it is. we don't know. my question. why it's hard. incorporate lit review

Women's labor force participation has skyrocketed in the U.S. since 1960, but men and women still go into very different occupations. As of 2009, approximately 50\% of women would need change jobs in order to to achieve an equal number of men and women in every occupation, and this gap is unlikely to close soon \cite{Blau2013}. Recent literature has identified some contributing causes of segregation: preferences over amenities (\citeA{Olivieri2014a}, \citeA{Wiswall2013})\footnote{For overview see \citeA{ Bertrand2011} and \citeA{Cortes2017a}.}, productivity and skill differences (\citeA{Baker2015}), and occupational and educational barriers (\citeA{Hsieh2016}). I identify another possible cause of the persistence of segregation: workers valuing the gender composition of occupations.

In particular, this paper examines whether current segregation depends on past segregation because workers care about the gender of their occupation in addition to intrinsic characteristics. This is important because the right policy to address segregation, if any, will depend on what mechanisms are at play. Policymakers have expressed concern about occupational segregation, for example the lack of women in STEM and men in nursing. If these occupations are male or female for historical reasons, then segregation itself may be a barrier to individual careers and the efficiency of the economy as a whole. If, on the other hand, occupations are male or female based on intrinsic characteristics of workers or jobs, then interventions that seek to change the gender composition of occupations alone are likely inappropriate, and policymakers concerned about gender segregation should examine policies to address more specific causes as they deem necessary.

%Recent research using job satisfaction data suggests that women prefer not to work in male occupations \cite{Usui2008, Lordan2015}. Such a preference could operate at the occupation level, through gender identity effects \cite{Akerlof2000}, or through the work environment itself. Since the fraction female in an occupation evolves endogenously according to worker choices, preferences over the fraction female can produce rapid changes in gender composition at the occupation level. Rapid changes in composition can be referred to as ``tipping", and this was documented in a number of U.S. occupations from 1940-1980 by \citeA{Pan2010}. 

%Tipping can result from two distinct mechanisms. First, group preferences can create a feedback loop that accelerates changes in composition caused by changes in the underlying attractiveness of a occupation, or labor demand. Second, if group preferences are strong enough, it may become profitable for an occupation to hire only men or women. This can mean that whether an occupation was historically male or female can affect its gender composition today.

%%%%%%%%%%%%%%%%%%%%%%%%%%%%%%%%%%%%%%%%%%%%%%%%%%%%%%%%%
%(ii) why is it hard to answer, 
%(iii) how does the paper answer the question, 
%(iv) what it novel about that method/data used to answer the question, and 
%%%%%%%%%%%%%%%%%%%%%%%%%%%%%%%%%%%%%%%%%%%%%%%%%%%%%%%%%

%(ii) why is it hard to answer, 
Identifying a preference over occupation fraction female is a difficult empirical problem that involves disentangling many other causes of gender segregation. Workers likely care about intrinsic characteristics of occupations, such as skills and job amenities (see for example \citeA{DeLeire2004} and \citeA{Reed1994}), as much as about the gender of their coworkers (see for example \citeA{Lordan2015} and \citeA{Usui2008}). At the same time, firms may care about which gender is perceived as more productive or valuable, as well as which gender of worker is cheaper to hire. Simply looking at which occupations contain more women will not allow me to distinguish firm and worker preferences, or worker preferences over the fraction female.


%(iii) how does the paper answer the question, 
The key innovation of this paper, building on the insights of \citeA{Choo2006}, \citeA{Salanie2014a}, and \citeA{Dupuy2017}, is that the distribution of wages can be used to separately identify worker and firm preferences. The intuition is that the right tail of wages tells us more about what firms are willing to pay for workers of a given type, while the left tail tells us more about what workers are willing to accept for a given type of job. Assuming that wage offers are distributed lognormally, and that firms do not care about the productivity of individuals within gender,\footnote{Unobserved productivity of individual workers could be included in theory, but would result in very weak identification and likely computational intractability because it would involve joint estimation of the model and integration over multiple sources of unobserved heterogeneity to interpret the wage distribution.} one can use Maximum Likelihood to estimate firm preferences and the wage offer distribution for male and female workers by occupation. Although the estimation strategy in this paper was developed specifically for segregation in the labor market, it could be used in any context in which a price mechanism clears a two-sided market.

The second identification challenge is to disentangle preferences for the number of women in an occupation from other occupation attributes. If an occupation is becoming more attractive to women, we will observe the fraction female in the occupation going up, and more women entering the occupation, and may incorrectly infer that women are entering the occupation because there are more women, a classic problem of omitted variable bias. To solve this problem I find variation in the fraction female by occupation that is plausibly exogenous to changes in other occupation amenities using Bartik-style instrumental variables. The main identifying assumption is that changes to occupations that workers care about are not correlated across occupations.

For tractability I treat occupation choice as a static choice. Workers choose occupation once at the beginning of their career, based on the contemporary characteristics of the occupations, including fraction female. Therefore, the fraction female in each occupation only updates across cohorts of workers.\footnote{Allowing workers to choose occupation myopically more than once during their lifetime, would lead to faster changes in occupation fraction females, but not otherwise change the dynamics of the model.} Lifetime wages are set to clear the market in static equilibrium for each cohort of men and women, which requires a large market assumption (continuum of workers and jobs of each type) to guarantee a unique wage equilibrium.

The data moments to identify the model are the shares of male and female workers by gender from the 1960-2000 U.S. Censuses and 2012 3-year ACS, and estimates of lifetime income by occupation, gender, and year constructed using income quantiles from the Census data combined with transition rates from the SIPP 2004 and 2008 panels. I take a two-step estimation approach, first estimating the firm side using maximum likelihood, then the worker side using instrumental variables regression, taking the wage offer distribution estimated on the firm side as data. The structure of the model allows me to simulate transition paths in the fraction female by cohort and determine there is path dependency, all while solving for equilibrium wages and fixing other firm and worker preferences.
 
%In the first stage of estimation, I find that firms are often willing to pay much more to hire male than female workers, but the degree to which this is true varies greatly across occupation. This gap may reflect discrimination, differences in labor force attachment over the lifetime, skill differences, or job sorting not captured in my occupation categories. The gap in the willingness to pay for female vs. male workers is lower in female dominated occupations. This might be because women are selecting into occupations in which they are more valued or have comparative advantage, or because the way work is valued might vary with the fraction female in the occupation \cite{Levanon2009, Harris2018}, or some combination of both.

%In the instrumental variables decomposition of worker utility into exogenous amenities and wages and fraction female components, 

I find that women care strongly about the number of women in an occupation, but no evidence that men care about the number of women. The point estimate in my preferred specification is very high, with an occupation change from 25\% to 75\% female being equivalent to an extra \$3 million in lifetime income for a woman. The gender preference leads to more segregated outcomes. With no gender preference, the model predicts no occupations with fraction female greater than 70\% or less than 10\%, and a Duncan segregation index of 24\% in 2012, meaning 24\% of male or female workers would have to change occupations to make all occupations 50\% female. With the gender preference, the Duncan index is predicted to be almost twice as high at 47\%.\footnote{The Duncan index in my observed 2012 data is 41\%.}

% *cite tipping lit more thoroughly? CONSISTENT WITH THESE STYLIZED FACTS

Labor supply or demand shocks that affect the fraction female are reinforced by a feedback loop from the gender preference. An illustrative case study is insurance adjustors, which is one of many occupations observed to have moved from male to female by \citeA{Pan2010}. I find that more women began to become insurance adjusters because firms' demand for women in this occupation rose. Then as more women entered, the occupation became more attractive to women. This in turn made women cheaper to hire, which increased labor demand for female insurance adjusters, providing further reinforcement of the feminization of the occupation. My model predicts that if it were not for women preferring to work with women, insurance adjusters would have become only 50\% female, rather than the observed over 70\% female (from a starting point of 20\% female in 1960). Thus my estimated model provides one possible mechanism that is consistent with the stylized fact that occupations tend to move rapidly from male to female. This phenomenon is referred to as ``tipping" by \citeA{Pan2010}.\footnote{``Tipping" is also a documented phenomenon in racial segregation by neighborhoods \cite{Card2008} and racial composition of schools \cite{Caetano2017}.}

%If either men or women have preferences over the fraction female of occupations then we could observe multiple equilibria and ``tipping" patterns that result from moving between these equilibria. However, despite the extremely strong in-group preference that I estimate on the part of women, I find that equilibrium wages are able to adjust to prevent the emergence of multiple equilibria, and therefore tipping between equilibria. An even stronger preference would be required to produce these phenomena. I cannot rule out such a strong preference given the imprecision of my preferred specification, but such a strong preference seems economically implausible given the already large magnitude of my preferred specification.

% going from 25% to 75% female is equivalent to about $3 million in additional lifetime income
% going from 0% to 100% female is about equivalent to an extra 6 million dollars in lifetime income... yes...
% and yet I still don't see multiple equilibria????

%This distinction is key because it means that there are no ``better" sorting equilibria, just changes to the location of the unique equilibrium resulting from changes in labor demand and supply.

% wage movement is key. contrast to previous work on housing segregation. long run predictions with and without wages, multiple equilibria with and without wages. preview of case study results, how much of wage gap is from gender preference??? illustrate with wages suppressed in insurance adjustors??

Although the gender preference I estimate is very strong, it is largely mitigated by compensating differentials. In a model without endogenous wages, all occupations eventually converge to either 0\% or 100\% female. Wage adjustment leads to more moderate outcomes. As women enter an occupation, firms can hire the same number of women at lower cost, and lower wages dampen the increase in female labor supply. Likewise as women exit an occupation, firms must pay the remaining women more, which slows the exit of women. I find that these compensating differentials explain 18\% of the difference in lifetime income between men and women, with women in highly female dominated occupations suffering the largest loss in earnings. The result that as women enter an occupation, wages go down is consistent with previous literature \cite{Levanon2009, Harris2018}.

%For example, if it were not for womens' preference to work with women, women in the female-dominated fields of ``Health Service Occupations" and ``Health Assessment and Treating" would earn just as much as men.

%insurance adjusters the wage went down? mention levanon england harris?

%how much the wage gap is from gender preference? average W= .5418195 without pref vs. W=.119695 with pref
% for insurance adjustors What2_2012= .70884 no pref and What2_2012= -.19423 with pref 
% what does this mean for actual wages??? export usigma for both specs then run datagen

%My model matches the tipping of insurance adjustors documented by \citeA{Pan2010} through an increase in the willingness-to-pay of firms for female insurance adjustors, accompanied by relatively stagnant female reservation wages, caused by the compensating differentials.

%One occupation documented by \citeA{Pan2010} to have moved rapidly from male to female during my sample period is insurance adjustors. By estimating my model I can see that this data pattern is actually explained by an increase in firms' willingness-to-pay for female insurance adjustors. The gender preference contributes to the rapid movement by suppressing women's wages. As more women become insurance adjusters, the occupation becomes more attractive to women, and firms are able to hire more women without raising wages, causing them to demand even more women. I predict that if it were not for women preferring to work with women, insurance adjusters would have become only 50\% female, rather than the observed over 70\% female (from a starting point of 20\% female in 1960). So although the observed ``tipping" pattern is not caused by multiple equilibria, the preference over fraction female plays an important role in reinforcing movements in the fraction female.



The presence of a gender preference could make it most profitable for occupations to hire only men or majority women. In this case, the segregation today may have been selected due to initial conditions such as historical barriers or norms \cite{Schelling1971, Pan2010}. However the gender preference I estimate would have to be about twice as strong to cause current occupation sorting to depend on past occupation sorting. 

There are two important takeaways from this paper for policymakers. First, short run shocks to the fraction female in occupations, such as temporarily pushing more men into nursing or women into STEM, will not have long run consequences since there is no historical inertia to overcome. Policymakers seeking to reduce segregation should find specific supply and demand factors that concern them and address them directly, whether it be occupation amenities or discrimination. Second, policymakers should keep in mind that changes to labor supply and demand might have an outsized impact on segregation due to the feedback loop of women wanting to work with women. Making an occupation more attractive to women might have the unintended consequence of making the occupation highly female dominated, and thereby lowering the wages for those women over time through a compensating differential. 

Lastly, future research should address the source of the gender preference to determine it is due to gender identity, or workplace environment and amenities. If workplace environment and amenities are the cause, then policymakers who care about reducing segregation and the gender wage gap might consider how to address these to make male-dominated occupations more welcoming to women.

%There are two policy implications from this result: First, policies that only temporarily change the number of men and women in an occupation have no impact in the long run. Because you can't shock policy. and because you can't care about segregation in and of itself, need to care about direct causes. Second, sorting patterns are not historically dependent. For example, if all occupations were 50\% female in 1960, gender segregation would still be very similar to what it is today. 

%The distinction is key because it means that, all else fixed, there are no ``better" stable sorting patterns that can be achieved by manipulating the gender composition of occupations alone. 
 
Section \ref{model} introduces the structure of the transferable utility matching model and equilibrium wages. Section \ref{empirical} describes the empirical specification including recovering the firm side parameters and wage offer distribution, and the decomposition of worker utility into fixed occupation attributes and gender preference. Section \ref{data} describes the data sources. Section \ref{results} presents results, and Section \ref{counterfactuals} presents simulations of segregation dynamics, including counterfactuals.



\section{Model} \label{model}

The identification strategy in this paper builds off empirical applications of matching models to marriage markets (e.g. \citeA{Choo2006} and \citeA{Chiappori2015}), but differs in that I use data on transfers (wages) to separately identify worker and firm payoffs. In the marriage market, a lack of data on transfers means that only the sum of wage and non-wage utility is identified.\footnote{\citeA{Fox2008c} is able to estimate payoffs from the interaction of characteristics of both sides of the market but not full payoff functions.}. Another key difference from previous literature is that I use the distribution of transfers (wages) for identification.

%Because my research question is how non-wage utility varies with occupation fraction female, it is critical that I separately identify non-wage utility from utility from wages. 

% REPLACE
%The difficulty is that observetd transfers are the sum of an idiosyncratic and an aggregate component, as discussed above, and for counterfactuals these must be separately identified. The observed distribution of wages is also the result of optimization over these components and other model parameters to be estimated, specifically the willingness-to-pay for workers by gender and the variance of the distribution of unobserved heterogeneity in job amenities. 

The first step towards separate identification of firm and worker non-wage parameters is to assume that observed wages are the transfers between job and worker that clear the market. Although some transfers in the labor market may be non-wage, such as enhanced benefits, wages are a natural first-order approximation. Previous literature applying transferable utility to the marriage market has assumed a logit assumption for unobserved transfers because logit is most tractable. I model wages as lognormal in order to match the observed wage distribution, which is greater than zero and has a long right tail.\footnote{Unlike the logit case, the variance of the distribution of the maximum of lognormals depends on other model parameters to be estimated, so the variance needs to be jointly estimated unlike in \citeA{Salanie2014a} for example.} Second, I define an unmatched worker to be a worker who is unemployed or out of the labor force, and an unmatched job to be a posted vacancy that has remained unfilled for a period of time. Since vacancy data is potentially a poor proxy for what it means for a job to be unmatched, I try to limit the importance of the vacancy data in my estimation procedure by choosing a specification that in principle does not require it, by assuming non-negative profit on the job side. Thus identification relies more heavily on the wage distribution.


%The empirical strategy consists of two stages: first I use maximum likelihood to disentangle the selection effects of worker and firm optimization on observed matches and wages. I am able to separately identify the willingness-to-pay of the firm and the reservation wage distributions of the workers. This effectively separates the two sides of the market, unlike previous applications of transferable utility matching models, such as \citeA{Choo2006} and \citeA{Chiappori2015}, who were only able identify total match surplus because they lacked data on transfers.\footnote{\citeA{Fox2008c} is able to estimate payoffs from the interaction of characteristics of both sides of the market but not full payoff functions.}

%Second, the estimated reservation wage distributions from the maximum likelihood estimation are used to estimate the worker's utility. Workers care about non-wage amenities, wages, and the fraction female. In the second stage of estimation, a panel regression for each gender is run of shares of workers in each occupation on reservation wages, occupation intercepts, and fraction female. Instruments are used to control for omitted variables correlated with both fraction female and wages over time, and provide clean variation in the fraction female and reservation wage to trace out labor supply.

%Theoretically it would be feasible to combined both stages into joint GMM estimation. This would have the advantage of using all data variation to identify all parameters. However joint estimation would require pooling all years of data and searching for 224 parameters instead of 138, so it would be computationally challenging.

%This would have the advantage of using all data variation, including the number of unemployed workers by gender, to identify willingness-to-pay, reservation wages, and overall utility levels. However joint estimation would require pooling all years of data and searching for 224 parameters instead of 138, so it would be computationally challenging.

For clarity of identification, I estimate the two sides of the market, workers and jobs, separately. The job side of the model is a maximum likelihood estimation exploiting the full variation of the observed wage distribution at the individual match level by year. The worker side, by contrast, is an instrumental variables regression at the occupation-year level, where instruments provide clean variation in the fraction female and reservation wage to trace out labor supply.\footnote{Combining the two estimation steps using joint GMM is theoretically feasible and would have the advantage that all data moments are used to identify all parameters. However the number of parameters (914 parameters) and the need to pool all years of data (7 million observations) makes joint estimation computationally unattractive.}






%Importantly, all workers and all jobs participate in the auction, so there is no role for search frictions in the model. This assumption is made more reasonable by the fact that I am considering lifetime occupation choices, not individual jobs. 

%The gender ratio of the occupation is determined at the occupation level. There are no firms in the model so I treat all of the utility of the occupation as exogenously given at the occupation level by the occupation production technology, rather than provided by firms at cost as in a hedonic model. When a worker is considering their choice of lifetime occupation, they will consider the overall characteristics of the occupations rather than a particular job or firm. 




%% put firm STUFF HERE

%unlike the marriage market, it is not clear what it means for a job to be ``unmatched". Theoretically, the number of unmatched jobs would be the number of jobs that would exist if labor were free. Rather than estimate production functions and capacity constraints by occupation, I choose to use data on posted vacancies that remain unfilled for some time. 

%The decision to post a vacancy is endogenous to market conditions, so vacancy data is an imperfect proxy for unfilled jobs. 

%In order to limit the importance of the vacancy data in my estimation, I choose a specification that in principle does not require it, by assuming non-negative profit on the job side. However in practice this relies very heavily on the functional form assumption on the tail of the distribution of job dis-amenities. As a result in most specifications I use data on vacancies by occupation imputed from the Job Openings and Labor Turnover Survey \cite{JOLTS}.






%can you estimate the sigmas at all without using the mle?? probably can't since the sigma distributions depends on the pi but the pi depends on the sigma? explore this explanation. but without truncation on the firm side then I should be able to get everything immediately. lnsj-lns0 gives you the surplus on each side. the variance of wages is the variance of the underlying heterogeneity given that max results. then the mean observed wage is W plus some shading term from the optimization that depends on the variance? is that shading term still tractable now I am using normal dist. instead of logit? what is the distribution of the max of normals? apparently under certain convergence conditions the max of normals converges to extreme value but only for large n where n is the number of choices!! for two choices the dist. seems to depend on both means, which are unknown??? In this case the variance also depends on the means so it becomes intractable even at the stage of getting the variance.





%\footnote{$\xi^g_j$ could also be reinterpreted as the job-specific productivity of gender $g$, but in this case, it will not appear in the wage in equilibrium, discussed below.}

%This assumption also implies that all wage heterogeneity is the result of compensating differentials. In reality wage heterogeneity is the result of both compensating differentials and individual worker productivity (see for example \citeA{Sorkin2015a, Taber2011a} for comparison of these sources).



 %I plan to run simulations to assess the level of this bias.

%\footnote{My model will have slightly different dynamics in terms of the evolution of the mean wage by occupation. Overcrowding in my model will lack an additional decrease on wages from drawing in lower skilled workers into the overcrowded occupation.} 



%Unfortunately allowing both forms of heterogeneity to enter the wage makes it impossible to learn from individual wage observations, since wages would reflect the sum of two unobserved terms. We may still be able to learn from the moments of the wage distribution, integrating out both sources of heterogeneity, as suggested in \citeA{Salanie2013a}, and in future work I may explore this possibility.\footnote{In a model with both unobserved productivities and unobserved amenities reflected in wages, workers would choose based on the sum of two heterogeneity terms, one of which is due to tastes and not reflected in the wage and one of which is due to productivity and is reflected in the wage. Wages would reflect the sum of worker productivity heterogeneity, and job amenity heterogeneity.} 



%since the observed wages would then depend on the discrete choice problems on both sides of the market, and more importantly, on a selection problem of dimension equal to the number of occupations rather than the number of genders.



% Since underlying heterogeneity in both productivity and amenities is unobserved, assuming one distribution away and making a distributional assumption on the other is equivalent to making a distributional assumption on their sum. Thus while strictly speaking the the model wage heterogeneity reflects unobserved heterogeneity in job amenities, it can be interpreted as a combination of that and unobserved heterogeneity in individual productivity.Unobserved to the econometrician are preferences over individual workers and jobs, which vary by type. Workers of type $X$ all share the same preferences over jobs $j$ within type $Y$, and worker $i$ has unobserved preferences over types of jobs $Y$. Jobs care only about the type $X$ of the worker they hire and the wage that they must pay to attract individual $i$. Intuitively, this structure implies that conditional on job and worker type, workers are all equally productive so the job cares only about the wage. The workers share the same ranking of jobs within occupations, conditional on type, but have idiosyncratic preferences over occupations.

%$$\pi^g_j = \pi^g_o - w^g_j $$ and the utility of worker $i$ of accepting job $j$: $$u_{ij} = u^g_o + w^g_j + \xi^g_j + \eta^i_o$$ Crucially preferences satisfy additive separability, meaning that preferences can only depend on the unobserved attributes of both sides of the market in an additive way. This means that the matching problem can be divided into two separate optimization problems connected only through transfers.

\subsection{Payoff Functions}
\subsubsection{Firm}

%Jobs choose the gender of worker that will maximize productivity per dollar spent, which may be differentially valued by men and women. This is consistent with a model of firms where jobs within a firm are filled independently, perhaps due to costly coordination or low complementarity. Individual jobs optimizing over rate of return does not imply a specific firm-level production function, but it produces similar behavior as firms minimizing cost with production quotas. Without data on firm size, I take this as an approximation of firm behavior.

% the relative productivity of men and women and the wage that they will have to pay to an individual worker.

Firms are willing to pay different amounts for male and female workers. I do not take a stance as to the cause of the difference. Let $WTP^g_o$ be the most that a job in occupation $o$ would be willing to pay to hire a worker of gender $g$. The total payoff to job $j$, $\pi^g_j$, also depends on the cost of hiring a worker, ${Wage^g_j}$, which varies at the job level according to how attractive the job is to workers.

\begin{align} \label{firm}
 \pi^g_j = \frac{WTP^g_o}{Wage^g_j} 
\end{align}
%$$$$ % = \frac{WTP^g_o}{W^g_o * \xi^g_j}

I specify the payoff function as a ratio for both economic and computational reasons. First, maximizing rate of return is consistent with a firm only filling one vacancy at a time, and allows me to avoid making assumptions about the number of vacancies at a firm or complementarities between those vacancies. Second, the multiplicative specification allows wages to be lognormally distributed, which is common in labor economics to match the observed shape of the wage distribution, while still additive and normal when logged, which is computationally attractive. 

The log payoff to the firm $j$ is:

%First, since firms in the model only fill one vacancy at a time, rate of return is a reasonable object for firms to care about that does not require making assumptions about the number of total vacancies at a firm. By allowing the firm to maximize return on a single vacancy at a time I make a reasonable job-level approximation of a firm level profit function. 


%Second, the rate of return specification is more tractable for estimation under the assumption that wages are lognormal. I later allow wages to be distributed lognormal so that log wages is normal and additive in the firm payoff.

%This is because in a model with firms making hiring decisions, the firm would likely be hiring multiple workers to meet a certain production level. Therefore an individual job is less likely to care about the level of output of an individual worker (as in an additive profit function) than the return per unit of wage. 
\begin{align} \label{logfirm}
  log(\pi^g_j)=&  log(WTP^g_o) -  log(Wage^g_j) \\ \nonumber
  \equiv & \widebar{WTP}^g_o - \widebar{Wage}^g_j
\end{align}







\subsubsection{Worker}
%In a competitive equilibrium firms will produce at the production level that minimizes average cost, or equivalently maximizes willingness-to-pay over cost.

Workers choose an occupation at the beginning of their lives, based on their individual tastes for occupations and the wage offer for their gender in that occupation. The job choice is binding for life.\footnote{Since there is no switching jobs, I assume that education choice is tied to job choice and do not treat education as a separate decision. Lowering switching cost would lead to faster convergence to a fixed point in the fraction female by occupation.} I assume that workers are myopic and do not anticipate future changes in the market that would lead them to want to switch occupations.\footnote{Relaxing myopia would mean that the endogenous attribute of interest, the fraction female, would depend on expectations over other workers' occupation choices, which in turn are affected by the fraction female. This would pose a challenge to tractability.}

The worker's taste for the occupation consists of two components: $u^g_o$ which is common to all workers of gender $g$ matching to jobs in occupation $o$, and $\eta^i_o$ which is worker $i$'s specific utility from occupation $o$. The dis-amenities of job $j$, which are denoted $\xi^g_j$, are the same for everyone conditional on gender.

%Non-employment is defined as either not in the labor force or unemployed, and the payoff to non-employment is $ \bar{\eta}^i_N $, the idiosyncratic taste for non-employment.

The payoff to worker $i$ from job $j$ is specified as:
\begin{align} \label{worker}
u^i_j &= \frac{ u^g_o*Wage^g_j *  \eta^i_o } {\xi^g_j}    
\end{align}

Defining log utility to be $\bar{u}^{i}_j$ and re-parameterizing in terms of logs we have:

\begin{align} \label{logworker}
\bar{u}^{i}_j \equiv log(u^i_j) &= log(u^g_o) + log(Wage^g_j)  + log(\eta^i_o) - log(\xi^g_j)   \\ \nonumber
&\equiv  \bar{u}^{g}_o + \widebar{Wage}^g_j   + \bar{\eta}^i_o - \bar{\xi}^g_j 
\end{align}

%Workers choose a job $j$ to maximize log utility $\bar{u}^{i}_j$. The common taste parameter for non-employment $\bar{u}^g_N$, where non-employment means either not in the labor force or unemployed, is normalized to zero. It is also assumed that in non-employment workers receive no wages or value from job amenities. The payoff to non-employment ($\bar{u}^i_N$) is therefore  $ \bar{\eta}^i_N $, the idiosyncratic taste for non-employment.



%\footnote{The log utility parameters for working, $\bar{u}^g_o$, include the disutility of working at all and are expected to be negative. However the underlying utility parameters $u^g_o = exp(\bar{u}^g_o)$ will always be positive leading to complementarity in wage and non-wage utility. Similarly, ${\xi^g_j}$ will be assumed to be log-normally distributed and therefore always positive, so the higher the dis-amenity value of the job the lower the utility. }

Furthermore worker utility will be decomposed into a common component and a dependence on the fraction female in the occupation, which is the endogenous amenity of interest. Let $F_o$ be the fraction female in occupation $o$. To avoid simultaneity, the fraction female is defined as the fraction female of the occupation among older workers (ages 35-65), that is, the fraction female that would be observed by the worker as they make their occupation decisions. To make the timing assumption clear I use subscript $t-1$ for the fraction female.

\begin{align*}
\bar{u}^{i}_j &= \bar{u}^{g}_o + \widebar{Wage}^g_j   + \bar{\eta}^i_o - \bar{\xi}^g_j \\
    &= \alpha^g_o +   \gamma^g F_{o,t-1} + \widebar{Wage}^g_j   + \bar{\eta}^i_o - \bar{\xi}^g_j
\end{align*}
% \frac{\beta^g}{\sigma^g_{\eta}}  X_o+



\subsection{Market Clearing Wages}

% REPLACE
%The lifetime income distributions by gender and occupation needed to clear the market are determined in equilibrium to equate demand and supply, and are made up of two components: a job-specific component, and an aggregate component that varies by occupation and gender. The job-specific component reflects the idiosyncratic disutility of taking a specific job. In equilibrium firms must make workers indifferent across individual jobs within an occupation, and therefore firms perfectly compensate workers for idiosyncratic disutility. Wage heterogeneity within occupation and gender emerges from these job-specific compensating differentials. 

%In addition to an individual job level compensating differential, there is a component of wages which is common across individuals within gender and occupation, and serves to equate supply and demand. It is a function of all parameters in the model, specifically, both of the utility that workers receive from occupations as well as the value that jobs have for male and female workers. 



%\footnote{For reference, simulations of an ascending price auction with sample size of 20 already produce a wage vector similar to the equilibrium vector assumed by the large sample. I perform the simulations using a modified version of the auction mechanism outlined in \citeA{RothSotomayor} page 209.}

The equilibrium wages I propose, $\widebar{Wage}^g_j$, equate supply and demand for workers by gender and jobs by occupation. They do so by compensating the worker for the idiosyncratic dis-amenities of a particular job within occupation, and through a common component $W^g_o$, which varies by gender and occupation. These equilibrium wages support the pairwise stable matching of workers and jobs through the optimization on both sides of the market. See appendix \ref{sec.Stability} for proof of stability.

\begin{align}
Wage^g_j = W^g_o * \xi^g_j 
\end{align}


A firm receives a draw of $\xi^M_j$ and $\xi^F_j$, which is the amenity value of the job to men and women respectively. The firm then chooses to hire a male or female worker based on the overall productivity of men and women in that occupation ($WTP^g_o$) and the cost of hiring men and women which varies in order to compensate exactly for the utility or disutility experienced by the worker due to $\xi^M_j$ and $\xi^F_j$. This is a compensating differential at the job level. 

Compensating differentials also emerge at the occupation level through the common component of wages, $W^g_o$, which will vary according to the utility that workers receive from occupations, as well as the value that jobs have for male and female workers, in order to equate supply and demand. This component of wages can be thought of as the result of a market-wide open-ended ascending price auction where jobs make wage bids for workers, but each job can only ``win" one worker.\footnote{In order for equilibrium wages to not depend on sample size I must assume a large number of workers of each gender and jobs of each occupation \cite{Galichon2013b}.}

% , and there are no search frictions. The assumption of no search frictions is made more reasonable by the fact that I am considering lifetime occupation choices, not individual jobs.



Plugging in equilibrium wages we have the following equilibrium utility for workers:

\begin{align} \label{utility}
\bar{u}^i_o &= \bar{u}^{g}_o + \widebar{Wage}^g_j   + \bar{\eta}^i_o - \bar{\xi}^g_j  \\ \nonumber
&= \bar{u}^{g}_o +  \bar{W}^g_o + \bar{\xi}^g_j  + \bar{\eta}^i_o - \bar{\xi}^g_j  \\ \nonumber
&=  \bar{u}^{g}_o + \bar{W}^g_o  + \bar{\eta}^i_o  \\ \nonumber
&=  \alpha^g_o + \gamma^g F_{o,t-1} + \bar{W}^g_o  + \bar{\eta}^i_o 
\end{align}

Workers are exactly compensated for the job dis-amenities, $ \xi^g_j$, in equilibrium, leaving the taste for occupation, $\bar{\eta}^i_o$, as the heterogeneity at the individual worker level. 

%The worker utility function therefore takes into account the job dis-amenities $\xi^M_j$ and $\xi^F_j$ as well as the workers taste for that occupation, and because $log(Wage^g_j) = log(W^g_o * \xi^g_j) = \bar{W^g_o} + \bar{\xi^g_j}$, 

The equilibrium payoff to a firm is

\begin{align}
  \bar{\pi}^g_j =& \widebar{WTP}^g_o - \widebar{Wage}^g_j \\ \nonumber
  =& \widebar{WTP}^g_o -  \bar{W}^g_o - \bar{\xi}^g_j
\end{align}

Firms care only about the value proposition of hiring a man or woman ($\widebar{WTP}^g_o -  \bar{W}^g_o$), and the idiosyncratic appeal of their job to men and women, $\bar{\xi}^g_j$. This means that, as on the worker side, there is only one source of heterogeneity in the equilibrium payoff. This is a critical assumption because it allows me to separate the matching problem into two separate discrete choice problems, one for each side of the market \cite{Galichon2013}. The cost of this assumption is that there is no individual worker productivity heterogeneity. Selection on productivity is orthogonal to my research question, but it may lead me to underestimate the value of amenities of small occupations and overestimate the appeal of large occupations.

%The counterfactual dynamics I am interested in are at the level of occupation and gender, and productivity and utility are allowed to vary freely by occupation and gender. Therefore including individual worker productivity heterogeneity should not have a major impact on my results. In terms of biasing parameter estimates, I may overestimate how undesirable small occupations are. High wages in small occupations may be due to selecting the most skilled workers rather than dis-amenities. Similarly I may overestimate the value of amenities in large occupations if wages are low due to a lower than average level of worker skill, rather than attractive amenities.
% JUSTIFY MORE WHY ORTHOGONAL



%%%% CUT AND REWRITE
 %% PUT IN FOOTNOTE why sigmas matter, but CUT THIS
%Lastly, I estimate four variance parameters in total: variance of taste for occupation for men and women, and variance of job dis-amenities for men and women. These parameters are important because they govern the elasticity of labor supply to changes in wages or other non-wage amenities such as the gender ratio in the counterfactual, as well as the demand response to changes in equilibrium wage. Unlike \citeA{Chiappori2015}, who uses many markets to identify these parameters, I use the observed wage distribution as suggested in \citeA{Salanie2014a}. However I estimate using maximum likelihood in order to use the information from the wage distribution efficiently, and because the lognormal assumption means that the scale parameters cannot be separately estimated from all other parameters, as noted above. For all these reasons maximum likelihood is the most natural strategy for my application.

\section{Empirical Strategy} \label{empirical}


Estimation follows from distributional assumptions on the unobserved heterogeneity on each side of the market.

\subsection{Firm Estimation}

\begin{assumption}
Let the heterogeneity in job amenities for each gender, $\xi^g_j$, be distributed lognormal and independent across $j$, such that $\bar{\xi}^g_j = ln(\xi^g_j)$ is distributed $\mathcal{N}(0, \sigma^g_{\xi}$).
\end{assumption}


I also assume that if a job is unattractive enough, it will not be filled. This allows me to identify the model without use of vacancy data on the firm side. In practice I use vacancy data in estimation for stability of the estimator and to avoid relying too much on extreme data points for identification.

%I make the assumption of a fixed outside option because then the model is identified without use of data on vacancies, allowing me to test robustness to the use of the vacancy data. I can estimate the willingness-to-way of the firms with the maximum observed wage by gender and occupation. This strategy is less stable than using vacancy data. 



%The payoff to the firm of not hiring any worker, $\pi^N_j$, is normalized to one, so that the log payoff to not filling a vacancy is zero, ($ \bar{\pi}^N_j = 0$). There is no equivalent on the firm side to the idiosyncratic preference for non-employment that exists on the worker side. Heterogeneity in the outside option for a firm could reflect heterogeneity in the cost of hiring for different jobs, or the substitutability of a particular job in the firm production function. 


%
%Let $\Phi_{0,\sigma^g_{\xi}}$ and $\phi_{0,\sigma^g_{\xi}}$ are the cdf and pdf of the normal distribution with location zero and scale $\sigma^g_{\xi}$. The normality assumption implies the following choice probability.
%
%\begin{align*}
%Pr( \text{firm hires } F ) &=  Pr(\widebar{WTP}^F_o - \bar{W}^F_o + \xi^F_j \geq \widebar{WTP}^M_o - \bar{W}^M_o + \xi^M_j)\\
%&=  Pr(\widebar{WTP}^F_o - \bar{W}^F_o + \xi^F_j - (\widebar{WTP}^M_o - \bar{W}^M_o) \geq  \xi^M_j)\\
%&= \int_{-\infty}^{\infty} {\Phi_{0,\sigma^M_{\xi}}(\widebar{WTP}^F_o - \bar{W}^F_o - \widebar{WTP}^M_o - \bar{W}^M_o + \xi^F_j)}  \phi_{0,\sigma^F_{\xi}}(\xi^F_j) d\xi^F_j\\
%\end{align*}

%A firm will only hire a worker if the willingness-to-pay for that worker is higher than the wage the worker will accept in equilibrium, which means the payoff $ \pi^g_j $ must be greater than one. 

%$$ \pi^g_j = \frac{WTP^g_o}{Wage^g_j} \geq 1 $$

\begin{assumption}
A firm will not hire a worker if the wage for men is higher than the willingness-to-pay for men ($WTP^M_o$) and the wage for women is higher than the firm's willingness-to-pay for women ($WTP^F_o$).
\end{assumption}

$$ \bar{\pi}^g_j = \widebar{WTP}^g_o - \widebar{Wage}^g_j <  0 \hspace{3mm} \text{for} \hspace{3mm} g \in \{M,F\}  \hspace{3mm} \implies \hspace{3mm} \text{$j$ unfilled}$$

%This differs from the classic multinomial choice model because there is no unobserved heterogeneity in the outside option. That is to say, t

%The above assumption implies that jobs are unfilled when both $\bar{\xi}^M_j$ and $\bar{\xi}^F_j$ are very high. High $\bar{\xi}^M_j$ and $\bar{\xi}^F_j$ means very high dis-amenity value to both men and women, requiring high wages to compensate. 

We observe log wages ($\widebar{Wage}^g_j$), but the only wages that are observed are the wages that maximize the job's choice over male, female, or not hiring any worker, so the observed data are the result of both selection and truncation. The selection and truncation depends on the $\bar{W}^g_o$, which is an unknown equilibrium object to be estimated, as well as the scale of the unobserved job heterogeneity, $\sigma^g_\xi $, which is unknown.\footnote{Estimating scale is important because is governs the elasticity of labor supply to changes in wages or other non-wage amenities such as the gender ratio in the counterfactual, as well as the demand response to changes in equilibrium wage. Unlike \citeA{Chiappori2015}, who uses many markets to identify these parameters, I use the observed wage distribution as suggested in \citeA{Salanie2014a}.} To back out the unobserved wage offer distribution from the observed wages I use Tobit Type 5 maximum likelihood estimation. This allows me to estimate $WTP^g_o$, $W^g_o$ and $\sigma^g_{\xi}$ while accounting for selection and truncation. See Appendix \ref{sec.likelihood} for the likelihood function in the notation of \citeA{Amemiya1985} and my notation.

%Since we observe log wages $\widebar{Wage}^g_j$, the expectation of observed wages suggests itself as an estimator for $\bar{W}^g_o$. Unfortunately there is a wrinkle to this strategy. There is a selection problem in that . 



\subsection{Worker Estimation}

\begin{assumption}
Let the worker taste heterogeneity for occupations, $\eta^i_o$, be independently distributed extreme value type 1.\footnote{Also known as gumbel for maxima or logit.} 
\end{assumption}

The location parameter of $\eta^i_o$ is normalized to zero, and the scale parameter is estimated separately for each gender, resulting in two scale parameters $\sigma^M_{\eta}$ and $\sigma^F_{\eta}$ for men and women respectively. I leverage well known properties of extreme value distributions to obtain occupation choice probabilities in terms of utility parameters. Recall that $log(u^i_g) = log(u^g_o) + log(W^g_o)  + \eta^i_o \equiv  \bar{u}^{g}_o + \bar{W}^g_o   + \eta^i_o  $.

%In \citeA{Chiappori2015} the scale parameters require many markets to estimate but with observed transfers we can rely on variation in lifetime income across occupations to trace out the scale.



%$$ Pr(j \in Y \text{ chooses } i \in X) = \frac{exp(\frac{\Pi^g_o}{\sigma^g_{\xi}})}{\sum_X exp(\frac{\Pi^g_o}{\sigma^g_{\xi}})}$$

$$ Pr(i \in g \text{ chooses } \forall  j \in o) = \frac{exp(\frac{\bar{u}^{g}_o + \bar{W}^g_o}{\sigma^g_{\eta}})}{\sum_{k \in O} exp(\frac{\bar{u}^{g}_k + \bar{W}^g_k}{\sigma^g_{\eta}})}$$ 


\begin{assumption}
I normalize the log utility from non-employment to be zero ($\bar{u}^g_N=0$). 
\end{assumption}

%$$ \frac{Pr(j \in Y \text{ chooses } i \in X)}{Pr(j \in Y \text{ chooses } 0)} = \frac{ \frac{exp(\frac{\Pi^g_o}{\sigma^g_{\xi}})}{\sum_X exp(\frac{\Pi^g_o}{\sigma^g_{\xi}})} } {\frac{1}{\sum_X exp(\frac{\Pi^g_o}{\sigma^g_{\xi}})}} =  exp(\frac{\Pi^g_o}{\sigma^g_{\xi}})$$

Since no wages are received in non-employment, this leaves only the idiosyncratic taste for non-employment $\eta^i_N$. We can then define the probability of choosing occupation $o$ relative to the probability of choosing non-employment in terms of the share of workers of gender $g$ who match to occupation $o$ ($s^g_o$), and  the share that choose non-employment ($s^g_N$), which are observed. Recall that the non-wage utility ($\bar{u}^{g}_o$) can then be decomposed into utility from the occupation ($\alpha^{g}_o$) and utility from the fraction female ($\gamma^g F_{o,t-1}$).

%$$ \frac{Pr(i \in g \text{ chooses } \forall j \in o)}{ Pr(i \in g \text{ chooses }N)}= \frac{ \frac{exp(\frac{\bar{u}^{g}_o + \bar{W}^g_o}{\sigma^g_{\eta}})}{\sum_Y exp(\frac{\bar{u}^{g}_o + \bar{W}^g_o}{\sigma^g_{\eta}})}  } {\frac{1}{\sum_Y exp(\frac{\bar{u}^{g}_o + \bar{W}^g_o}{\sigma^g_{\eta}})} } = exp(\frac{\bar{u}^{g}_o + \bar{W}^g_o}{\sigma^g_{\eta}})$$


\begin{equation} \label{eq:1}
\begin{split}
ln \left( \frac{Pr(i \in g \text{ chooses } \forall j \in o)}{ Pr(i \in g \text{ chooses }N)} \right) &= ln(s^g_o) - ln(s^g_N) \\
 &= \frac{\bar{u}^{g}_o + \bar{W}^g_o}{ \sigma^g_{\eta} } = \frac{\alpha^{g}_o + \gamma^g F_{o,t-1}  + \bar{W}^g_o}{ \sigma^g_{\eta} } 
 \end{split}
\end{equation}


This equation \ref{eq:1} cannot be estimated in the cross section because there are only as many moments as occupations, and I must estimate $\gamma^g$ and $\sigma^g_{\eta}$ in addition to an $\alpha^g_o$ for each occupation. Therefore I pool all six cross sections of data (1960-2012), allowing me to estimate the occupation-specific intercepts, $\alpha^g_o$ and use time variation in fraction female and reservation wages to identify coefficients $\frac{\gamma^g}{\sigma^g_{\eta}}$ and $\frac{1}{\sigma^g_{\eta}}$. In order to better match time variation in shares, a time effect $\beta^g_t$ is added. In addition, in the spirit of \citeA{Berry1994}, $\epsilon^g_{o,t}$ represents changes over time in the utility of workers due to changes in unobserved occupation attributes, so changes not due to movement in the fraction female or the reservation wages.

\begin{align*}
ln(s^g_{o,t}) - ln(s^g_{N,t}) =   \frac{ \beta^g_t + \alpha^g_o +  \gamma^g F_{o} +  \bar{W}^g_{o,t} + \epsilon^g_{o,t}}{\sigma^g_{\eta}}  \\
\end{align*}

%In the sections below I introduce several strategies to identify the parameters $\frac{\gamma^g}{\sigma^g_{\eta}}$ and $\frac{1}{\sigma^g_{\eta}}$. 

Unfortunately using time variation means that estimates of $\frac{\gamma^g}{\sigma^g_{\eta}}$ and $\frac{1}{\sigma^g_{\eta}}$ likely suffer from omitted variable bias. Changes over time in both the gender ratio and the reservation wage are likely correlated with changes in unobserved occupation attributes $\epsilon^g_{o,t}$.

For example if an occupation is becoming more family friendly over time, and this causes more women to enter the occupation, the coefficient on fraction female will be biased upward for women. Similarly if occupation amenities deteriorate over time, this may be correlated with increases in wages, causing a downward bias on the coefficient on the reservation wage $W^g_o$.



% \subsection{Fixed Effects Specification}
 
% Recall from equation \ref{utility} that worker utility is given by 
%
%$$\bar{u}^i_o =  \alpha^g_o + \gamma^g_o F^g_o + \bar{W}^g_o  + \bar{\eta}^i_o $$

 
% Let worker utility in occupation $o$ in time $t$ for either men or women be given by
%\begin{align*}
%u^{i}_{o,t} &= \alpha_o +   \gamma_t +  \gamma F_{o,t} + \delta log(W^g_{o,t}) + \eta^i_{o,t} + \epsilon_{o,t}   \\
%\end{align*}



%One concern with estimating this equation is that, leaving aside the structural interpretation of the discrete choice model, it is unclear if the coefficients reflect labor demand or labor supply. Skill requirements of occupations for example is one factor, unmodeled on the firm side, that could lead to a spurious negative correlation between the wage and the share of workers in an occupation. Simultaneously, there is the problem of omitted variable bias due to correlation between time varying occupation amenities $\epsilon_{o,t}$ and $F_{o,t}$ and $W^g_{o,t}$ discussed above. Ideally instruments driven by labor demand will also be uncorrelated with $\epsilon_{o,t}$ and therefore solve the endogeneity problem.

%In order to trace out the labor supply curve, variation in the fraction female and the wage due to labor demand shocks is needed.

% Instruments that use shocks to labor demand will ensure that $\gamma$ and $\delta$ reflect labor supply. 


%ADD MORE INTRO TO IVs HERE???
In the worker identification section below I introduce Bartik-style instruments that exploit variation in the industry composition of occupations to isolate changes over time in the fraction female and wage by occupation that are caused by labor demand shifts, allowing me to get clean identification of the labor supply parameters of the worker.



% \footnote{There are 138 firm side parameters estimated in the first stage in each of 6 waves of data, for a total of 828 parameters. There are 86 worker parameters estimated in the second stage over all waves of data.}

%Combining the two estimation steps would be feasible but would be less natural because the worker side is well suited to gMM and the job side to MLE. It would be possible to estimate the worker mean utility levels simultaneously in the job side MLE, however the incorporation of instruments would still require a separate second stage of estimation, unless joint normality is assumed for the reduced form errors to allow for LIML. Estimation of both sides simultaneously using gMM would also be possible, but in this case information from the shape of wage distribution would need to be collapsed into an arbitrary set of higher order moments, which is not as desirable as using the full information in MLE.

%%%%%In the following sections I lay out the assumptions that allow the ML estimation of the willingness-to-pay and reservation wages. The reservation wages are then taken as data in the estimation of worker utility parameters, which is described in the following section.

%The job MLE produces the locations of the unconditional reservation wage distributions faced by workers in equilibrium as they make their occupation choices. These parameters are then taken as data in the worker estimation. 



%\footnote{Making this alternative assumption would not substantially change the identification strategy, but would reduce reliance on the tails of the wage distribution and increase reliance on the shape more generally.} 

%In a fully specified firm model this could make sense with a complementary production function and other inputs, but without firms such complementarities should already be approximated in the job level willingness-to-pay term. 



%The joint probability of hiring male or female and hiring any worker at all can be broken out into conditional components as follows, in the case of hiring female:

%e use the distribution of $\bar{\xi}^F_j$ conditional on $\widebar{WTP}^F_o - \bar{W}^F_o + \bar{\xi}^F_j>0$:

%\begin{align*}
%Pr(j \in o \text{ choose } F , j \text{ filled}) &= \\
%  Pr(\widebar{WTP}^F_o - \bar{W}^F_o + \bar{\xi}^F_j \geq \widebar{WTP}^M_o - \bar{W}^M_o + \bar{\xi}^M_j | \widebar{WTP}^F_o - \bar{W}^F_o + \bar{\xi}^F_j \geq 0) ( 1-Pr ( \bar{\xi}^F_j < {-\widebar{WTP}^F_o + \bar{W}^F_o} ))   \\
%%&= \int_{0}^{\infty} \frac{ exp(-e^{-(\widebar{WTP}^F_o - \bar{W}^F_o - \widebar{WTP}^M_o - \bar{W}^M_o + \xi^F_j)} ) e^{-\xi^F_j} exp(-e^{-\xi^F_j})}{ 1-exp(-e^{\widebar{WTP}^F_o - \bar{W}^F_o} ) } d\xi^F_j\\
%%&= \frac{exp(\widebar{WTP}^F_o - \bar{W}^F_o)}{exp(\widebar{WTP}^F_o - \bar{W}^F_o) + exp(\widebar{WTP}^M_o - \bar{W}^M_o)} (1 - exp(- e^{\widebar{WTP}^F_o - \bar{W}^F_o} - e^{\widebar{WTP}^M_o - \bar{W}^M_o})) \frac{1}{1-exp(-e^{\widebar{WTP}^F_o - \bar{W}^F_o} ) } \\
%\end{align*}

%\begin{align*}
%Pr(j \in Y \text{ choose } F ,  \widebar{WTP}^F_o - \bar{W}^F_o + \xi^F_j \geq 0) &=Pr(j \in Y \text{ choose } F |  \widebar{WTP}^F_o - \bar{W}^F_o + \xi^F_j \geq 0) * Pr(  \widebar{WTP}^F_o - \bar{W}^F_o + \xi^F_j \geq 0)\\
%&= \frac{exp(\widebar{WTP}^F_o - \bar{W}^F_o)}{ exp(\widebar{WTP}^F_o - \bar{W}^F_o) + exp(\widebar{WTP}^M_o - \bar{W}^M_o)}  (1 - exp(- e^{\widebar{WTP}^F_o - \bar{W}^F_o}) - exp(- e^{\widebar{WTP}^M_o - \bar{W}^M_o})) \\
%\end{align*}




%\subsubsection{Likelihood Function}
%%The parameters that we need to identify for the job side of the market are the type level market clearing common component of wages $W^g_o$, the willingness-to-pays $WTP^g_o$,  and the scale parameter of the job amenities shock $\xi^g_j$, which is $\sigma^g_{\xi}$.
%
%Parameters of the model are
%
%$$ \theta = \{ WTP^g_o,  \alpha^g_o, \gamma^g, \sigma^g_\eta, \sigma^g_\xi  \} $$
%
%Recall that $WTP^g_o$ are the firms' willingness-to-pay parameters, $\alpha^g_o$ the workers' non-wage utility unrelated to the fraction female, $\gamma^g$ the value of the fraction female to the worker, $\sigma^g_\eta$ the scale of the worker taste heterogeneity, and $\sigma^g_\xi$ the scale of the job dis-amenities heterogeneity. The unconditional centers of the reservation wage distributions, $W^g_o$, are reduced form outcomes determined in equilibrium as a function of the primitives $\theta$. Nevertheless recovering the $W^g_o$ from the observed conditional wage distributions is critical to recovering the fundamental parameters in the worker's utility function.
%



%%%Recall the log wage specification, $$ \widebar{Wage}^g_j = \bar{W}^g_o -  \bar{\xi}^g_j $$

%\footnote{The distributions of the $\epsilon$ given they are observed at the maximized utility only is the same as the unconditional distributions \cite{DePalma2007}, therefore we can estimate the scale parameter  $\sigma^g_{\xi}$ using the individual wage data as suggested in \citeA{Salanie2014}. The variance of observed wages $w^g_j = W^g_o -  \xi^g_j$ will be $var(W^g_o) + var(\xi^g_j)$. However since $var(W^g_o)$ is a constant within $X,Y$ and the $\xi^g_j$ are independent of $X,Y$, we can estimate $\sigma_{X}$ with the sample variance of $w^g_j$ within each type $X,Y$. }

%Since we observe $w^g_j$, the worker-job match wage, for the matches that are made, it appears we could estimate the $\sigma^g_{\xi}$ from the scale of the observed wage distribution (especially given that the scale of a maximum of extreme value distributions is the same as scale of the original distributions\footnote{See Appendix}). However it is not possible to take this approach because of the positive willingness-to-pay product assumption, which implies a truncation of the job payoffs, and therefore a truncation of the observed wages at a truncation point for each type ($ \pi^g_o + W^g_o + \xi^g_j \geq 0 \implies \xi^g_j \geq - \pi^g_o - W^g_o$). This truncation point is unknown because $W^g_o$ is unknown.


\subsection{Identification}
I first discuss the identification in the first stage maximum likelihood estimation of the firm parameters ($\widebar{WTP}^g_{o}$ and $\sigma^g_{\xi}$) and centers of the equilibrium wage offer distributions ($W^g_{o}$). Then I discuss identification in the second stage instrumental variables regression of the worker parameters ($\alpha^g_o$, $\gamma^g$, and $\sigma^g_{\eta}$).

\subsubsection{Firm Identification}

I observe the share of jobs that are filled by men and women in each occupation, and the $Wage^F_o$ and $Wage^M_o$ for those matches that do occur. From these moments I need to identify how much the firm is willing to pay for men and women in each occupation ($WTP^F_o$ and $WTP^M_o$) and the center of the wage offer distributions for men and women by occupation ($W^F_{o}$ and $W^M_{o}$). The willingness-to-pay will be primarily pinned down by the shares and the right tail of the wage distribution (the most firms are willing to offer for workers of a gender in an occupation), while the wage offers will be primarily pinned down by the left tail of the wage distribution (the least workers are willing to accept for a job in an occupation).

%Separate identification of the willingness-to-pay and reservation wage parameters ($WTP^g_o$ and $W^g_o$) relies on the shape of the wage distribution, and in practice, the addition of vacancy data. 
%Intuitively, the shape of the observed wage distribution will vary according to where the reservation wage distribution is centered, while what portion of the reservation wage distribution we see depends on the $WTP^g_o$, that is jobs relative valuation of male or female workers. 
%To give intuition in terms of the likelihood function, $W^F_{o}$ and $W^M_{o}$ are pinned down by the $Pr(Wage^g_{j})$ component, while the relative difference between $WTP^F_o$ and $WTP^M_o$ is pinned down by the $Pr(\text{$j$ hire g} | Wage^g_{j} ) $ component, and the level of both $WTP^F_o$ and $WTP^M_o$ is pinned down by the denominator $Pr(\text{$j$ unfilled})$.

Figures \ref{sales} and \ref{health} illustrate identification by comparing the observed wage distribution in two very different occupations. In both occupations, women are paid less than men, but in Sales Representatives, Finance, and Business Services, this is due to firms being willing to pay women less, while in Health Service Occupations, this is the result of women being willing to work for less.

The willingness of firms to pay more for men in Sales Representatives, Finance, and Business Services manifests itself in the wage distribution through a long right tail male wages, well beyond the support of the female wage distribution. By contrast, in Health Service Occupations, the estimated willingness-to-pay gap is very small because the support of the right tails of the male and female wage distributions broadly overlaps.

Without the use of vacancy data the willingness-to-pay of the firm would in fact be identified from the maximum wage observed for each gender. Because this relies heavily on a single data point that could be an outlier or sampling error, I prefer to use a specification with vacancy data to help pin down how many jobs remain unfilled in the far right tail.

For women in Health Services Occupations, the left tail of the observed wage distribution has a lot of density close to zero, implying that female workers are willing to work for very little in Health Service Occupations, so we would expect wage offers for women to be low in this occupation. The observed wage distribution for women in Sales Representatives, Finance, and Business Services is centered much further to the right, implying women need to be compensated more highly to work in this occupation, so we would expect the center of the wage offer distribution to be higher. Therefore the shape of the observed wage distributions and the shares together identify the locations of the wage offer distributions ($W^F_{o}$ and $W^M_{o}$) for each gender and occupation. Wage offers are be needed to back out worker utility parameters.

%In summary using the model I am able to distinguish between the scenario in Figure \ref{sales}, where female workers are less valued than male workers, and the scenario in Figure \ref{health}, where female workers are cheaper to hire than male workers. In both cases the observed wages for women are lower, but the mechanisms are very different. Separately identifying reservation wages from willingness-to-pay is important because taking reservation wages as given is what allows me to estimate the worker side of the market. $W^F_{o}$ and $W^M_{o}$

%To illustrate identification consider the plots in Figures \ref{sales} and \ref{health}. These figures compare the reservation wage distributions implied by the model, with the wage distribution of matches, also implied by the model. The vertical lines denote the willingness-to-pay for male and female workers respectively, $\widebar{WTP}^M_{o}$ and $\widebar{WTP}^F_{o}$. 

%In both example occupations, the wage distribution predicted for matches (top panel) is lower for women than men. In the first example in Figure \ref{sales}, Sales Representatives, Finance, and Business Services, this is the result of firm preferences. In the second example in Figure \ref{health}, Health Service occupations, this is the result of worker preferences.

%In the bottom panel of Figure \ref{sales}, we see that in Sales Representatives, Finance, and Business Services, the reservation wages of women and men are centered at approximately the same location. This implies that men and women value the amenities of the occupation similarly and have similar outside options. Differences in the wage distribution of predicted matches are driven instead by firms being willing to pay higher wages for male workers. This manifests itself in the wage distribution by a long right tail male wages, beyond the support of the female wage distribution. 

%By contrast in Figure \ref{health}, in Health Service occupations, we see that the reservation wage distribution for men is much higher than for women. This means that men do not value this occupation, or have higher outside options. The high reservation wages for men drives the wage gap in this occupation. The estimated willingness-to-pay gap is very small because the support of the right tails of the male and female wage distributions broadly overlaps.




%Recall that worker utility depends on non-wage utility from occupation $o$, denoted $\alpha^g_o$, the fraction female through coefficient $\gamma^g$, and the reservation wages through the scale parameter of the worker heterogeneity in tastes for occupation $\sigma^g_{\eta}$, where $\frac{1}{\sigma^g_{\eta}}$ is the coefficient on the reservation wage.

%In the first scenario we will see a low center of the female wage distribution and with women being drawn primarily from the lower end of that distribution. In the second scenario we will see a large spread in the female wage distribution with women being matched relatively uniformly from all parts of it. In summary, the center of the observed distribution identifies the $W^g_o$ and the relative spread and shape identifies how many workers from that part of the distribution are being chosen relative to the other gender. 



%Depending on the relative values of $WTP^F_o$ and $WTP^M_o$ we will observe a higher share of jobs hiring women or men in a given occupation. Theoretically the scale of the willingness-to-pay parameters is identified also from the shape of the wage distribution, specifically from the degree of truncation on the right tail. In practice this source of variation does not perform well, in part because the right tail of finite sample wage data is censored and thin. The direct implication of the scale of both $WTP^F_o$ and $WTP^M_o$ is that we will see more or fewer jobs left unfilled, which is not directly observable without additional data. 



%To further illustrate identification consider the following plots of the reservation wage distributions for men and women, and the observed wage distributions for men and women. The vertical black lines denote $W^F_{o}$ and $W^M_{o}$, the centers of the offer distributions.

%\begin{center}
%%\includegraphics[width=.9\textwidth]{offerwages1reg1trunc}
%\includegraphics[width=.9\textwidth]{offerwages_occfinal8reg1trunc}
%\end{center}
%
%In this example\footnote{For more examples see Appendix.} we see that the willingness-to-pay from hiring a man must be higher than a woman since we observe male wages well to the right of female wages. Simultaneously the cost of hiring a man is relatively high because we observe that the male wage distribution is flattened


%\begin{align*}
%Pr(\text{$j$ hire M} | Wage_{Mj} ) &=exp(-e^{\frac{-(  \pi_{MY} - Wage_{Mj} -( \pi_{FY} - W_{FY}) )}{\sigma_{XY}}} ) \\
%Pr(Wage_{Mj} ) &=e^{-\frac{ W_{MY} -Wage_{Mj}   }{\sigma_{XY}}} exp(-e^{-\frac{W_{MY} -Wage_{Mj}  }{\sigma_{XY}}}) \\
%Pr(\text{$j$ hire F} | Wage_{Fj}) &=exp(-e^{\frac{-(\pi_{FY} - Wage_{Fj} - (\pi_{MY} - W_{MY}))}{\sigma_{XY}}} ) \\
%Pr(Wage_{Fj}) &=e^{-\frac{W_{FY} -Wage_{Fj} }{\sigma_{XY}}} exp(-e^{-\frac{W_{FY} - Wage_{Fj} }{\sigma_{XY}}}) \\
%Pr(\text{$j$ unmatched}) &=  exp (- \sum_{X \in M,F} e^{ \frac{\pi_{XY} - W_{XY}}{\sigma_{X}}  } ) \\
%\end{align*}

%\begin{align*}
%F_{1}(0|y_{3j}) &=exp(-e^{\frac{-(  \widebar{WTP}^M_o - \bar{W}^M_o - y_{3j}  -( \widebar{WTP}^F_o - \bar{W}^F_o - W_{FY}) )}{\sigma^g_{\xi}}} ) \\
%f_3(y_{3j}) &= \frac{1}{\sigma^g_{\xi}} e^{-\frac{ W_{MY} -y_{3j}  }{\sigma^g_{\xi}}} exp(-e^{-\frac{W_{MY} -y_{3j} }{\sigma^g_{\xi}}}) \\
%F_{-1}( 0| y_{2j}) &=exp(-e^{\frac{-(\widebar{WTP}^F_o - \bar{W}^F_o - y_{2j} - (\widebar{WTP}^M_o - \bar{W}^M_o - W_{MY}))}{\sigma^g_{\xi}}} ) \\
%f_2(y_{2j}) &= \frac{1}{\sigma^g_{\xi}} e^{-\frac{W_{FY} - y_{2j} }{\sigma^g_{\xi}}} exp(-e^{-\frac{W_{FY} - y_{2j} }{\sigma^g_{\xi}}}) \\
%\end{align*}


%Note also that once we know $ \widebar{WTP}^g_o$ and  $W^g_o$, we can solve for the number of jobs that remain unfilled. This will be important for the counterfactual because unfilled jobs becoming filled will affect the wage dynamics because less desirable jobs must compensate workers more for their dis-amenities $\xi^g_j$.

%Lastly, $W^g_o$ are taken as data to estimate the components of the worker utility function. 



%$$ \sigma^g_{\xi} ln(  \mu^g_o) - \sigma^g_{\xi}  ln(\mu_{0Y}) = \Pi^g_o = \pi^g_o - W^g_o $$
%$$ \implies  ln(\mu_{0Y}) = ln(  \mu^g_o) +\frac{W^g_o- \pi^g_o  }{\sigma^g_{\xi}} $$



\subsubsection{Worker Identification}

Recall that the worker's occupation choice depends on a fixed value for the occupation, as well as wages and the fraction female, which may change across cohorts of workers. Since the occupation fixed effects ($\alpha^g_o$) capture the fixed value of the occupations to men and women, the identification concern is that we would expect changes in the fraction female and the wage offer to be correlated with changes in the value of the occupation not captured in the $\alpha^g_o$. I therefore need instruments to get clean variation in $W^g_o$ and $F_{o,t-1}$ to identify the worker utility parameters.

Occupations exist in a variety of different industries, and these industries have different wage offers and fraction females, and also experience different changes in wage offers and fraction female. The idea behind my first two instruments is to use industry level changes to predict wage offers and fraction female by occupation, under the assumption that industry level changes are driven by labor demand and not confounded by labor supply changes. For example, if wages in manufacturing are going up over time because the production technology in manufacturing has become more efficient, then we would expect this to have an impact on the wages of workers in occupations employed in manufacturing that is due to labor demand and independent of occupation amenities. Identification will be threatened if worker preferences for industry vary over time, or if worker preferences for occupations are changing over time in a way that is correlated across occupations. This type of instrument is commonly called a Bartik instrument.
 
 %\footnote{The instrument is similar in style to \citeA{Autor2013e} where import demand is assumed uncorrelated across countries but import supply correlated across countries. It is also similar to Hausman instruments where product cost shocks are assumed correlated across markets but demand shocks are independent \cite{Hausman1996a, Nevo2001}.} 



%Intuition:
%Occupations exist in different industries
%Industries experience growth and changes to fraction female and wages We can predict changes in occupation fraction female and wages using changes in industries
%(Using labor demand shifts to trace out labor supply)
%Threats to identification:
%Worker preferences for industry vary over time
%Changes in worker preferences for occupations are correlated across occupations

%The identifying assumptions for the first set of instruments are that changes in industry wage levels and gender ratios over time are uncorrelated with changes in how workers value industries, and that changes in occupation attributes are independent across occupations. The identifying assumptions for the second pair of instruments is that changes in industry size over time that are correlated with initial wage and gender ratios, are uncorrelated with changes in the worker valuation of the industry. Lastly I introduce an instrument that interacts the initial gender ratio by occupation and the relative growth rates of men's and women's labor force participation over time. The identifying assumption is that changes in unobserved occupation attributes are independent across occupations.

 %two main variations on a Bartik instrument. The first, which can be constructed in the cross section, are the occupation wage levels and gender ratios that could be predicted solely from the industry composition of the occupation. The key assumption for these is the independence across occupations of worker preferences over industries. The second, which relies on panel data, are the changes in wage and gender ratio due solely to changes in industry size or the relative labor force size of men and women.
 

%\subsubsection{Instrumental Variables}
%\subsubsection{Changes Over Time in Industry Wage and Gender Ratio}
%The first set of instruments uses industry variation in wages and gender ratios, and variation in the presence of industries within occupations, to predict occupation level wages and gender ratios. For example, the reservation wage for administrative assistants will be the weighted sum of the wages by industry for all industries that exist in that occupation.

%The basic idea is that various labor demand factors may cause variation in the reservation wage or gender ratio by industry. For example production technology or consumer demand may lead reservation wages for administrative assistants in manufacturing to be higher than in retail trade, and the gender ratio in that occupation to be highest in professional service and lowest in manufacturing. 

Let $p_{Io}$ be the fraction of occupation $o$ in industry $I$, and $\hat{F}_{Io}$ the fraction female in industry $I$ excluding workers in occupation $o$. I exclude the occupation that is being instrumented for so that changes in occupation amenities will not be included in estimates of changes in industry wages and fraction female. The industry composition $p_{Io}$ is be fixed in 1950, prior to the sample data. Then the predicted fraction female in occupation $o$ in time $t$ is as follows:

$$\hat{F}_{o,t} = \sum_I  p_{Io}*\hat{F}_{Io,t} $$

The predicted wage is similarly the sum over industry of the industry wage by the industry composition. For the instrument to work it is necessary that each industry contain multiple occupations. At the level of aggregation I use, 14 major industries,\footnote{Industries used are aggregates of the harmonized IPUMS codes of ind1990. Industries are as follows: Agriculture, Forestry and Fisheries; Mining; Construction; Manufacturing; Transportation, Communications, and other public utilities; Wholesale Trade; Retail Trade; Finance, Insurance, and Real Estate; Business and Repair Service; Personal Services; Entertainment and Recreation Services; Professional and Related Services; Public Administration.} every industry has employment in almost all occupations.

%\footnote{Note that in order for industry level fraction female or wage level to be estimated excluding occupation $o$, it is necessary that each industry contain multiple occupations. In fact at the level of aggregation I use, 14 major industries, every industry has employment in almost all occupations.} 

 %Fraction female is measured as the fraction female in the older generations (ages 36-65) in the Census cross section. Industry wages are measured by gender as lifetime incomes of those beginning their careers in a given industry as simulated using the Census and SIPP data described above.

%Pooling all six waves of data (1960-2010) is critical to the quality of the instruments in two ways. First, the industry composition can be fixed in 1950, prior to the sample data. This alleviates any concern that changes in industry composition are correlated with changes in unobserved occupation amenities, which is critical for the validity of the instrument \cite{SorkinBartik}. 

%Second, occupation fixed effects absorb the mean utility coming from the industry composition. Therefore we must only assume that \textit{changes} in how workers value industries are independent across occupation, but changes in industry demand are correlated across occupations. 



%One might worry that the reason reservation wages are increasing in manufacturing is due to a compensating differential from exactly the unobserved amenities that confound the OLS regression on reservation wages. To avoid this concern, the reservation wage level in the industry is calculated excluding the reservation wages in the occupation being instrumented. So for example in the case of administrative assistants, manufacturing wages would be calculated taking the mean over only other occupations in manufacturing. 

%The instrument will be invalidated if changes in industry amenities are correlated across occupation. For example, if the working conditions for all workers in manufacturing are getting worse over time, and this is causing the working conditions for administrative assistants in the manufacturing sector to get worse over time, then the instrument will predict an increase in wages among administrative assistants that is correlated with the change in working conditions.



%\subsubsection{Growth in Industry Employment}
I also use a more standard Bartik instrument, which predicts changes in occupation wage offers and fraction female due to changes in the relative size of industries. For example, if manufacturing is declining, the number of administrative assistants who work in manufacturing will decrease, and we can use this to predict changes in the wage offer and fraction female of administrative assistants. The key identifying assumption is that changes in the prominence of certain industries due to production technology or demand side factors, not labor supply changes.


%The growth rate of industries since 1950 is the treatment and the initial wages and industry fraction female\footnote{Industry composition by gender is set in 1950. Wages are set in 1960 because the sample size in the 1950 Census is too small to calculate lifetime wages} are the exposure to the treatment. Changes in employment by industry is a fundamentally different source of variation. 

%The instrument predicts changes in occupation wage and gender ratio over time due to t, not including the impact of these factors on changes in wage or gender ratio since these are fixed in the initial period, and therefore controlled by the occupation fixed effects.

Let the occupation*industry composition, $p_{Io,initial}$, and the fraction female, $ F_{Io,initial}$ be fixed at the initial period (1950). Let $size_{Io,t}$ be the total employment level in industry $I$ excluding occupation $o$ in year $t$. I exclude the instrumented occupation in the estimation of industry size. The predicted occupation fraction female is then:

$$\hat{F}_{o,t} = \sum_I p_{Io,initial}*F_{Io,initial}*\frac{size_{Io,t}}{size_{Io,initial}}$$

%The changes over time in the predicted $\hat{F}_{o,t}$ are driven by changes in industry size.

%One requirement for validity of the instrument is that the growth in the size of industries, and therefore the exposure of occupations to the wages and gender ratios predicted by those industries, is due to product demand or productive efficiency changes over time. So for example say that demand for manufacturing is growing over time and therefore more jobs are available in the occupations in manufacturing. We would expect the wages and gender ratios in occupations found in manufacturing to be more and more reflective of the growing prevalence of manufacturing.

%The main threat to validity of the instrument is if the wages and gender ratio in manufacturing are reflective of underlying amenity values that are common across all occupations in manufacturing. In that case, even if the growth of an industry is exogenous, its increased prevalence in an occupation will be correlated with a change in the amenities in that occupation. This is less important to the extent that my paper is concerned with occupation amenities not industry amenities.

%This instrument would be invalid if industry growth rates are driven by changes in occupation amenities, and these changes in amenities are correlated with wage levels or fraction female in the initial period. We might worry, for example, that industries with initially low wages grow at a faster rate and this growth is due to the expansion of unobserved amenities correlated with the low wages in the initial period. If this is the case then we would be predicting low wages in occupations that 

%Workers could be valuing being a manager in manufacturing more and more and this could be growing manufacturing and therefore impacting the wage and gender ratio in management and also how much workers value management as an occupation. As long as the things workers like about being a manager in manufacturing are unique to being a manager and not correlated across all occupations in manufacturing, then we should be capturing only labor demand factors using this instrument since the initial wages and gender composition are captured by the fixed effects. 

% I could also argue the independence of industry amenities across occupation here but I do not think it is necessary given the paragraph below.

The last Bartik-style instrument predicts changes in the fraction female of occupations over time by interacting changes in the overall labor force participation rate of relative to women with the initial fraction female by occupation. The idea is that as the relative value of home and work changes, women may be more likely to enter occupations that historically have had more women due to higher fixed values of those occupations ($\alpha^o_g$).\footnote{This is similar to instrumenting for immigration patterns based on overall flows of immigrants and initial shares \cite{Altonji1991a}.} 

The instrument is constructed as follows. Let $F_{o,initial}$ be the fraction female in the occupation in the initial period, 1950. Let the number of men and women employed in all occupations except $o$ in time period $t$ be $\#M_t$ and $\#W_t$. I define the relative growth in female vs. male employment $r_t$ as

$$ r_t = \frac{ \frac{\#F_t}{\#M_t} }{ \frac{\#F_{initial}}{\#M_{initial}} }$$

Then the fraction female in occupation $o$ predicted by the instrument in time t, $\hat{F}_{o,t}$ is as follows:

$$\hat{F}_{o,t} = F_{o,initial} * r_t $$

The instrument will be invalid if the changes in labor force attachment are driven by changes in occupation amenities, and these changes are correlated with initial amenities. For example, if high fraction female occupations in the initial period are becoming relatively more attractive to women over time, and this change is driving the increase in female labor force participation, the instrument will be invalid.
 
 %I assume that the growth in the ratio of employed to non-employed in all other occupations will not be correlated with growth in amenities in the instrumented occupation. 
% So for example we see a high fraction female for teaching non-postsecondary in 1960 and also growth in the number of female teachers over time. Under the assumption of the instrument the growth in the number of female teachers is due to an overall increase in female labor force participation and the fixed amenities component of teaching being attractive for women. However, if the increase in female labor force participation is driven by growth in the amenity value of teaching, and the amenity value growth is correlated with the initial amenity value, then the instrument is endogenous.


%\subsubsection{Changes Over Time in Employment by Gender}
%Another source of variation made possible by the panel is variation in the relative value of employment and non-employment for men vs. women over time. Fixing occupation gender ratios in the initial period, the instrument predicts changes in occupation gender ratios due only to the gender ratio of those employed in all other occupations. Assuming changes in occupation attributes are independent across occupation, growth in the ratio of employed to non-employed in all other occupations will not be correlated with growth in amenities in the given occupation. 

%The identifying variation is changes in the relative value of home vs. work by gender, projected onto current occupation gender ratios based on previous gender ratios. This is similar to instrumenting for immigration patterns based on overall flows of immigrants and initial shares \cite{Altonji1991a}. One might expect initial fraction female to impact flows into an occupation through occupation attributes that are fixed over time. Since fixed attributes are controlled by fixed effects, variation over time is assumed to be due only to changes in labor force attachment of men relative to women relative to the initial period. For recent evidence of changes in labor force attachment by gender see \citeA{Albanesi2017}. 





%\subsubsection{Firm Willingness-to-Pay Parameters}
Lastly, I include the willingness-to-pay estimates $WTP^g_o$ from the firm side of the model as instruments. The $WTP^g_o$ are a measure of how much firms value workers should be uncorrelated with unobserved amenities, assuming amenities are fixed, and therefore are a good proxy for labor demand side factors that will shift the reservation wage and the relative number of men or women through firm preference for hiring men vs. women, whether this be consumer demand driven, productivity differences, or discrimination.

%The ratio of $\frac{WTP^M_o}{WTP^F_o}$ is also used as a proxy for labor demand.

\section{Data} \label{data}
The primary data elements needed to estimate the model described above are: expectations of lifetime labor income by occupation and gender, shares of workers by gender and age cohort choosing each occupation and non-employment, and a measure of unfilled jobs by occupation.

While the Census and ACS provide cross sectional wage and occupation, the SIPP is needed to provide a panel for the construction of lifetime labor income estimates.\footnote{Public use Census 1960, 1970, 1980, 1990, 2000, and 2012 three-year ACS data obtained from IPUMS \cite{IPUMSUSA}. SIPP data from the 2004 and 2008 panels are constructed using the NBER files \cite{SIPP}. Occupation codes are constructed by aggregation of the IPUMS harmonized codes (occ1990) to achieve sufficient sample size. See appendix \ref{occupations} for the full list of occupation codes.} Using pooled data from the 2004 and 2008 SIPP panels, which are four and five years long respectively, I construct transition rates through five quantiles of earnings and occupations by worker age and gender. Because of limited sample size, I assume that transition rates depend only on the current state not the past history. This transition matrix is then used to simulate worker career paths from the starting point of workers aged 25-35 in the Census and ACS. Their assigned choice of occupation is taken as the occupation they start out in at ages 25-35 as observed in the Census, which means that the occupation choice can be interpreted as including the expectation of all future transitions.

%\footnote{CPS or CPS MORG data could be added to get an earlier estimate of these transition rates.}
%The lifetime income assigned to each observation in the Census or ACS is the sum of their simulated career path through the SIPP transition matrix. 

%For a comparison of estimated lifetime income paths to observed lifetime income paths in the PSID, see appendix \ref{PSID}. 

%The five quantile cut offs are estimated in the Census data where the larger overall sample size allows for more accuracy. 

%Lifetime wages and shares of workers by gender and occupation are estimated using a combination of Census data and SIPP data.

% While using cross sectional data may be sufficient to create estimates of lifetime income when only a few moments are needed, for my identification strategy the shape of the wage distribution is critical, and I therefore cannot assume that every worker obtains, for example, the median income for their chosen occupation at each age. The resulting distribution is mechanically quite lumpy. One reason that the lifetime income distribution in reality might be much smoother than a cross-sectional approximation is that workers transition stochastically over time between wage levels and occupations.

%To model worker transitions over time as probabilistic, panel data from the SIPP is essential. 



%Income quantiles from Census, by age, gender, and occupation, are denoted $quan$ below. Occupation is denoted $occ$, and five year age bracket $age$. Transition rates from year to year, where $x'$ indicates the value of $x$ in the following year, are then estimated non-parametrically as follows:

%\begin{align*}
%& Pr(quan',occ'|quan,occ,age)=  \frac{\sum_{i} \mathcal{I}(i \in quan',occ',quan,occ,age)}{ \sum_{i} \mathcal{I}(i \in quan,occ,age)}\\
%\end{align*}



The Job openings and Labor Turnover Survey \cite{JOLTS} is used to construct a measure of unfilled jobs by occupation. Since JOLTS does not directly contain occupation, only NAICS industry codes, industries are projected into occupations using contemporaneous occupation industry shares estimated in CPS \cite{IPUMSCPS}. The estimated openings by occupation is then divided by the total number of people employed in the occupation to get the ratio of openings to employed.

%I also make an estimate of the number of openings that remain open at the end of the year by assuming that the probability that a job is filled is uniform across time and jobs. Then the daily rate of hiring is equal to the total number of hires that month divided by the number of days in the month. Then the probability a job is not filled on a given day is one minus this daily hire rate, and the probability a job is unfilled for the year is this probability to the power of 365. This measure of unfilled jobs will be used for robustness.

%Additional data on occupation attributes from the Dictionary of Occupation Titles (1977), the ONET database (1998-2016), the CPS work supplement, and recently collected job opening data from \citeA{Atalay2017}, are included in some specifications. The DoT contains a number of measures of skill requirements of the occupation and occupation tasks in including the level of complexity with which a worker relates to people, data, and things on the job. The ONET contains a myriad of job context and task measures, most importantly competition, contact with others and time pressure, previously identified by \citeA{Goldin2014} to be of potential differential interest to women and men.\footnote{The ONET is updated continuously such that all occupations have been updated at least once during the period 1998-2016. Thus the oNET itself can be used as a measure of changes in occupation attributes but only during this limited time period. Similarly some occupations in the DoT were updated in 1991, allowing for some longitudinal variation between 1977-1991, and potential for rough mapping of the 1991 DoT into the 1998 oNET.} IPUMS CPS work supplement data from 1991, 1997, 2001, 2004 is pooled to obtain a measure of flexible working hours and nonstandard end time of job. I also use occupation attribute data from \citeA{Atalay2017}, who use changes over time in job opening ads in several major newspapers to get measures of occupation characteristics that vary over time.



To assess the impact of my assumption that workers make lifetime occupation choices, I examine the extent to which workers switch occupations during their working lifetime. In the PSID, the average worker who spends most years working spends 80\% of working years in the same occupation\footnote{I include observations that appear in the PSID for at least 25 years between 1968 and 2011, beginning before age 30 and ending after age 55, and are not missing occupation data. This results in a sample of 764 workers.} Furthermore, for 85\% of workers in the PSID the modal occupation for ages 25-35 is also the modal occupation for ages 25-55. My simulated lifetime income sample overestimates occupation transitions relative to the PSID, likely because occupation choice is more history dependent in reality. Either way, occupation choice appears to be fairly stable for a lot of people, and to the extent that people do transition, this is included in the expected value of the initial occupation choice.

%and 57\% of workers in the simulated SIPP, 

%In my 1960 data, simulated from the Survey of Income and Program Participation (SIPP), 14\% of workers change occupation each year on average.\footnote{\citeA{Kambourov2008} find that on average 13\% of workers change occupation each year, by comparing retrospective and concurrent occupation data in the Panel Study of Income Dynamics (PSID). In my unadjusted PSID sample this number is 21\%.} 

%However the statistic that matters is how many workers spend many or most of their working years in the same occupation, mirroring the lifetime occupation choice dictated by the model.  



%In my simulated 1960 SIPP sample this number is only 64\%. In reality dependence on past occupation likely extends beyond the immediate previous period, so this number is likely a lower bound due to my first-order Markov assumption in simulating the data. 



%\footnote{The higher rate of transitions but lower rate of years in the same occupation in my simulated data is consistent with the first-order Markov assumption being overly simplistic. In reality dependence on past occupation likely extends beyond the immediately previous period.} 




\section{Model Estimates} \label{results}
\subsection{Model Parameters and Fit}


Figures \ref{modelwages1} and \ref{modelwages2} compare the observed wage distributions and the model predicted wage distributions for a few example occupations. The model predicted wages look very similar to observed wages. The full set of job side model parameter values and histograms of the model fit of the lifetime income distributions and shares can be found in the online appendix.

Table \ref{modelestimates} shows the model estimates from the first stage of estimation by occupation, averaged across years. The first column is the fraction female, varying from .02 to .89, and all 34 occupations are ordered from highest fraction female (administrative support) to the lowest fraction female (construction and extraction). In the second column we see the ratio of the female wage offer to male wage offer ($  \frac{\bar{W}^F_o}{\bar{W}^M_o}$, which varies from .36 to 2.66, and in the third column we see the ratio of the firms' wilingness-to-pay for women vs. men ($ \frac{\widebar{WTP}^F_o}{\widebar{WTP}^M_o}$), which varies from .37 to 1.14. 



In general women have lower reservation wages than men ($  \frac{\bar{W}^F_o}{\bar{W}^M_o} < 1$) and are less valued by jobs ($ \frac{\widebar{WTP}^F_o}{\widebar{WTP}^M_o} < 1$). In general the higher the fraction female, the lower the wage offers for women are in the occupation, relative to men. On the other hand firms tend to be willing to pay more for women relative to men in high fraction female occupations. Intuitively, women like, and are valued most at, highly female dominated occupations such as Administrative Support, Financial Records Processing Occupations, and Health Service Occupations, and women dislike and are valued least at highly male occupations such as Engineers, Architects, and Surveyors, Mechanics and Repairers, and Construction and Extraction.

%This is consistent with a Roy model with women sorting into occupations in which they are more valued by jobs.

% (thought not through a wage mechanism necessarily since both observed wages and reservation wages are lower for women in female-dominated occupations).

It has been noted that both men and women have lower wages the higher the female share in an occupation (see eg. \citeA{Macpherson1995a, Levanon2009, Addison2017,Harris2018}). Unlike the mean wage by occupation, the centers of the wage offer distributions $\bar{W}^g_o $ control for the fact that in observed wages, we see only the most attractive jobs filled in each occupation, and by the workers that satisfy the firm's payoff maximization. Thus using the model allows me to look at the correlation between wage offers and fraction female unconditional on selection effects, which would downward bias any estimated correlation. I do not find a statistically significant correlation between my estimates of lifetime income and average fraction female in an occupation, but I do find that female wage offers are negatively correlated with the fraction female (correlation coefficient $-0.77$), while male wage offers are positively correlated with the fraction female (correlation coefficient $0.6$). 

%and reflects the common component of the reservation wage to every worker of gender $g$ at any job in occupation $o$. 

%Estimation results indicate that in contrast to observed wages, the model estimated reservation wages for men, $\bar{W}^M_o $, are positively correlated with the fraction female (estimated correlation coefficient of $0.6$), where the unit of observation is an occupation-year. The correlation coefficient between fraction female and observed labor income is $-0.16$. Reservation wages for women ($\bar{W}^F_o $) are negatively correlated with the fraction female (correlation coefficient $-0.77$), just like observed wages (correlation coefficient $-0.12$). 



% \cite{Baker2001} for canada evidence and non-linearities

%Reservation wages are more closely related to the value that the worker has for an occupation than observed wages, since observed wages reflect additional selection by the firm. This explains why reservation wages could be so strongly correlated with the fraction female, but lifetime income not so much. 

%Although reservation wages for men in female-dominated occupations are quite high, in the observed wage data we see only those jobs within the occupation that are so attractive to men that the men can be hired an comparable wages to women. In contrast, the higher the fraction female in an occupation, the lower the reservation wage to women $\bar{W}^F_o$, but since women are cheap to hire relative to men, we see even the least attractive jobs being filled by women. 

Lower wage offers for women in female-dominated occupations could be consistent with a female preference for working with women producing a compensating differential, but could also be the result of a strong preference for certain occupations. Similarly, the high wage offers for men in female occupations could be consistent with a male preference against working with women or strong tastes for occupations. In the next section I present the results of the instrumental variables regression of worker utility, and discuss whether the fraction female has a causal impact on worker utility.

% this follows mechanically from the model maybe??

%*** include table comparing $\bar{W}^g_o$ to the mean observed lifetime income by occupation and gender?




%\begin{tablenotes}
%      \footnotesize
%  things
%    \end{tablenotes}
%blah
%\captionof{figure}{my table}
%\end{minipage}





%*** include the parameter table here with confidence intervals comparing to the counterfactual wage vector

%My first step is to see if my simulation of the model over cohorts can match the actual evolution of the fraction female over cohorts. The potential problem is that in real life workers do change careers after the first period, meaning that my model predicted older generation gender ratio will not match the actual observed older generation gender ratio exactly. I need to see how good of an approximation my model is to reality before I can interpret any counterfactual results.



\subsection{Gender Preference Results}
I find a strong and robust preference on the part of women for entering into more female-dominated occupations, which is around four times the preference for log wages, meaning that if the log reservation wage in an occupation went up by 10\% this would have an equivalent effect on log utility ($\bar{u}^F_o$) of an increase to the percent female in the occupation of around 5\%. I do not find evidence for a preference on the part of men against entering into more female-dominated occupations.

% 10 \% increase in wages implies a 10/100 increase in log utility
% then 4* change in F implies a 10/100 increase in log utility
% then change in F must be 10/400= 2.5% or .025... 10/200 would be .05

Graphical representation of the preference structure under the assumption of linear, quadratic or cubic form is given in Figure \ref{prefs}. Women have increasing utility in the fraction female, and the increase is steeper the fewer women there are in the occupation. By contrast for men, utility is relatively flat in the fraction female and not statistically different from zero.\footnote{Results from a beta distribution specification follow a similar pattern.}

Results for the industry and employment growth panel instruments are in Tables \ref{IV1} and \ref{IV2}. All tables report fixed effects regressions on the pooled data. Estimation is done using limited information maximum likelihood for robustness to weak instruments, but two stage GMM results are similar. Standard errors clustered at the occupation level.\footnote{I expect errors correlated within occupation due to occupation fixed effects and possible differences in model fit across occupations.} I report only the linear specification for men and the cubic specification for women because these seem to fully capture the functional form and obtain the highest first stage F statistics in the preferred IV specification.

The first column is an un-instrumented fixed effects regression. The second column (IV1) includes the instruments using variation in industry wage and fraction female over time. The third column (IV2) includes these instruments and subsequently includes instruments from the variation in occupation size over time. The fourth column (IV3) also includes all previous instruments and also adds in the instrument exploiting variation in the size of the male and female labor force relative to the initial period. The last column (IV4) additionally includes the firm willingness-to-pay parameters $WTP^g_o$ for men and women respectively and the ratio of firm willingness-to-pay parameters  $\frac{WTP^M_o}{WTP^F_o}$. This final column is the preferred specification because it has the strongest first stage.

%Pooling all six waves of data (1960-2010) allows for the addition of occupation and time fixed effects. These fixed effects should improve the quality of the instruments in two ways. First, the industry composition can be fixed in 1950, prior to the sample data. This alleviates any concern that changes in industry composition are correlated with changes in unobserved occupation amenities. Second, occupation fixed effects makes the assumption that industry amenities are independent across occupation easier to swallow since the assumption becomes industry amenities are independent net of mean occupation utility. 

%Finally, a classic Bartik IV exploiting change over time in the the size of industries can be used to instrument for $W^g_o$ and the gender ratio respectively. In this case the growth rate of industries since 1950 is the treatment and the initial wages and industry composition by gender\footnote{Industry composition by gender is set in 1950. Wages are set in 1960 because the sample size in the 1950 Census is too small to calculate lifetime wages} are the exposure to the treatment. Results for these instruments are below.


%\input{Stata_pvfirstW_sex1_}
%\input{Stata_pvfirstg_sex1_}
%\input{Stata_pvIV_sex1_}
%\input{Stata10_pvIV_sex1_}
%\input{Stata10_pvIV_PDVsex1_}
%\input{Stata10_pvIV_PDVstartsex1_}
%\input{Stata_pvfirstW_sex2_}
%\input{Stata_pvfirstg_sex2_}
%\input{Stata_pvIV_sex2_}
%\input{Stata10_pvIV_sex2_}
%\input{Stata10_pvIV_PDVsex2_}
%\input{Stata10_pvIV_PDVstartsex2_}



In the un-instrumented regression in the first column, both men and women have a negative wage coefficient. This implies that even though the fixed effects have controlled for any time invariant omitted variables, the changes over time may still be picking up unobserved labor demand factors, changes in occupation attributes, or other omitted variables. The instrumented specifications should avoid this endogeneity by identifying off of labor demand shocks. The wage coefficient is positive for men and women in all instrumented specifications. The last column, using all instruments discussed above, achieves an Kleibergen-Papp F statistic of around 10 for women and 8.6 for men.\footnote{Denoted ``KP rk F" in the tables.}

In no specification do men have a significant preference over the fraction female. The instrumented point estimates are close to zero in the preferred specification, but the standard errors are high, so I cannot rule out moderate effects in either direction. The lack of precision could be due to a lack of variation over time in men's labor market outcomes, which is corroborated by the much higher total sum of squares in the female regression.

%In fact, a regression of the log difference in shares on only time and occupation dummies produces and F statistic of 823 for women and 3167 for men.



%For women the results imply a preference for entering female occupations on the same order of magnitude as the preference for wage. Under the fixed effects specification, moving the fraction female from 0\% to 100\% would have an equivalent impact on log utility as increasing lifetime income by \$700,000 for all workers in this occupation. Under the $IV4$ specification this figure is \$1.56 million.



% PRoBLEM: this is log wages and log utility.... so if gender ratio moved from 0 to 1 then Log utility goes up by 1, similarly if wages go up by 1 million, Log utility goes up by 1

% so ubar =  beta * F  + delta * W, what is the meaning of beta?? but also we have log(u) = ubar + Wbar + eta 

% and workers maximize log utility which is equivalent to maximizing utilty for an individual worker but in terms of whether ToTAL surplus gets maximized could be different since u+p relatively equal > u +p unequal (when logged). I could make it so that multiplicative surplus is maximized so pairwise stability defined in terms of u*p then the wage cancels on the side where the match actually occurs (which is the counterfactual side) so u*p <= u'*p' where u' and p' are with whoever they are with which  might not be each other. if workers are maximizing logged utiltiy and firms logged profit than they should be maximizing also total logged surplus... so this is a different allocation than the standard but it is the allocation that is consistent with my context.

% but if I do this for a single occupation this will have an impact on other occupations through the wage equilibrium... so this is a partial equilibrium outcome... should I calculate expected utility if all occupations are 0% female vs. 100% female including the equilibrium wage effects?? or expected utility in one occupation if it is 0% or 100% including equilibrium wage effect

%So if lifetime income went up by one million dollars, utility would increase similarly as moving the number of women in the chosen occupation from 0\% to 100\%. 

%%%%%%%%%%%%%%%%%%%%%%%%%%%%%%%%%%%%%%%%%
% why am I reporting partial equilibrium results? report it with the wage moving??
% Problem: need to solve for new equilibrium 34 times for p25 for each occupation and the 34 more times for p75 for each occupation... need to run this overnight. at least I can do men and women at the same time?

% also do 10-20, 80-90 instead?

% also forget moving the wage since wage is in equilibrium? so make two columns, row 1 moving 10-20\% female, row 2, 80-90\% female, first column labor supply effect, second column equilibrium effect.

For women the estimated preference is economically significant in terms of the impact on occupation choice. Moving the fraction female in a single occupation from 20\% ($F=.2$) to 80\% ($F=.8$) would have an average marginal effect of moving 1352\% more women into that occupation under the $IV4$ specification (see Table \ref{sumstat}). However this effect is exaggerated because it is partial equilibrium. Holding equilibrium wages fixed means that this result only reflects the decisions of workers without firm response. When firms are allowed to adjust wages in equilibrium, the effect of moving from $F=.2$ to $F=.8$ is only 124\% on average. So if an occupation moved from relatively male dominated to relatively female dominated that would just over double the number of women who would enter that occupation in equilibrium.

%The impact of moving the log reservation wage from the 25th to 75th percentile is offered for comparison purposes and is much smaller, only an average marginal effect of 90\% more women or 53\% more men. A change in fraction female from 0 to 100 would be equivalent to about a 300\% increase in log wages in terms of the effect on women's utility.

%Fraction female from 0 to 100 increases female utility about 6 in cubic spec. so 6 = beta/100%*change in wage. Beta is 2. So 6 = 2/100%* change in wage. So the equivalent change in log wage is 100%*6/2, or 3*100% so 300%

To check the sensibility of the model it is useful to look at the estimated fixed effects from the regressions in Tables \ref{IV1} and \ref{IV2}. These correspond to the estimated non-wage utility of men and women for each occupation. The occupation fixed effects by gender can be found in Table \ref{FEs}. The excluded category is ``Teachers, Postsecondary". In combination with the estimated willingness-to-pay from Table \ref{modelestimates}, they track relatively closely with the observed fraction female by occupation. Occupations with higher female utility and higher willingness-to-pay for women have more women in them. 

% this would be a nice spot to run the counterfactuals setting um=uf and pim=pif and contrasting which contributes more to segregation



%\newpage
%\subsection{Adding Quadratic in Fraction Female}
%\input{Stata10_pvIV_PDVstartsex1__v2_s}
%\input{Stata10_pvIV_PDVstartsex2__v2_s}
%\clearpage
%
%%\input{Stata10_pvIV_sex1_s}
%%\input{Stata10_pvIV_PDVstartsex1_s}
%%\input{Stata10_pvIV_sex2_s}
%%\input{Stata10_pvIV_PDVstartsex2_s}
%%\clearpage
%
%\newpage
%\subsection{Adding Cubic in Fraction Female}
%\input{Stata10_pvIV_PDVstartsex1__v2_sc}
%\input{Stata10_pvIV_PDVstartsex2__v2_sc}
%\clearpage


%\subsection{Adding distance from parity in the fraction female}
%Let genderM be defined as distance from parity in a male dominated occupation and genderF be the same in a female dominated occupation.
%
%\input{Stata10_pvIV_sex1_MF}
%\input{Stata10_pvIV_PDVstartsex1_MF}
%\input{Stata10_pvIV_sex2_MF}
%\input{Stata10_pvIV_PDVstartsex2_MF}



%\citeA{Stock2008} recommend against the use of robust standard errors in the case of panel data with fixed time periods. 

%\subsection{Scaling in terms of wage coefficient}
%\input{testing}

%\newpage
%\subsection{Adding in interaction of Wage and gender.}
%
%\input{Stata_pvfirstW_sex1_Wgender}
%\input{Stata_pvfirstg_sex1_Wgender}
%\input{Stata_pvIV_sex1_Wgender}
%\clearpage
%\input{Stata_pvfirstW_sex2_Wgender}
%\input{Stata_pvfirstg_sex2_Wgender}
%\input{Stata_pvIV_sex2_Wgender}


%  also IV with size of the labor force???

%BLP instruments have no variation across time so they should be almost collinear with the occupation fixed effects as is the problem with Nevo's cereal application

\section{Counterfactuals} \label{counterfactuals}
The dynamics of the model result from updating of the fraction female in each occupation with each successive cohort of workers. In each ten year period, only young workers ages 25-34 choose a new occupation, while older cohorts in age brackets 35-44, 45-54, 55-65 are fixed in their occupation. When the current cohort of workers, ages 25-34, make occupational decisions, they face the gender ratio produced by older cohorts of workers. The occupation choice of each worker is fixed for the rest of their working lifetime (assumed to be 4 periods, or 40 years), and workers do not take into account predicted future evolution of the fraction female. 

%Workers and jobs are assumed to be numerous enough that no single worker need take into account their own impact on the fraction female in their occupation of choice.

%Let the fraction female in occupation $o$ observed by the young cohort before making their decisions be $F_o$. Let the fraction female among cohort $C$ in occupation $o$ be $F_{o,C}$. Note that $F_{o,C}$ results only from the decisions made by cohort $C$. Then the fraction female in occupation $o$ observed by the current generation when they make their decisions is $F_o$ and can be decomposed into the weighted average of decisions of previous generations as follows, where the weights are the cohort sizes $n_{o,C}$.

%$$ F_o = \frac{n_{o,C-1}F_{o,C-1} + n_{o,C-2} F_{o,C-2} + n_{o,C-3} F_{o,C-3}}{n_{o,C-1}+n_{o,C-2}+n_{o,C-3}}$$ 
Let the fraction female in an occupation observed by the young cohort in time $t$ before making their decisions be $F_t$, where the occupation subscript $_o$ is omitted for convenience. Then we can write $F$ as a function of the number of men and women ($n^F$ and $n^M$) choosing that occupation in the previous three periods.


$$F_t = \frac{n^F_{t-1} + n^F_{t-2} + n^F_{t-3}}{(n^F_{t-1} + n^M_{t-1}) + (n^F_{t-2} + n^M_{t-2}) + (n^F_{t-3} + n^M_{t-3}) } $$

Given this observation of $F_t$ in each occupation, the young cohort makes their own occupation decisions. Their decisions determine the number of men and women in the occupation in time $t$, or $n^F_t$ and $n^M_t$, which will influence the fraction female observed by the next three cohorts of workers.



\subsection{Dynamics}

%\subsubsection{Finding Stable Equlibria}

In Figures \ref{fig:sq1}-\ref{fig:sq4} below, I simulate future generations of workers to determine if the current level of sorting is stable, and if not, to what fraction females the occupations will converge in the future. For counterfactual simulations I assume there are no changes to parameters, which isolates the impact of the endogenous movement of the fraction female across generations. In particular, I assume that the willingness-to-pay parameters $WTP^g_o$ on the firm side are fixed, and that worker utility only changes due to changes in the fraction female $F_{o,t-1}$, and changes to the equilibrium wage. All other inputs are fixed at the 2012 values, the last year of data. 

% ($\bar{u}^i_j =  \alpha^g_o + \gamma^g F_o + \bar{W}^g_o  + \bar{\eta}^i_o $)
% $\bar{W}^g_o$

%We can get a sense of model fit by comparing the model simulated segregation patterns over the time period for which we observe real segregation patterns (1960-2012).

%For the purposes of these stylized simulations I assume that workers are fixed in their starting occupation for the rest of their lives, rather than impute later life changes in occupation which may be biased or noisy. 

In the top right panels of Figures \ref{fig:sq1}-\ref{fig:sq4}, I assume that the wage levels by occupation are fixed over time, simulating the outcome in a one-sided model of occupation choice. In this case, all occupations eventually converge to fully male or female due to the preference for fraction female.



In the bottom panel I allow the wages to update through the market clearing condition of the matching model described above. I solve for the new $ \bar{W}^{*g}_o$ the equates the supply and demand for workers and jobs, given the changes to worker utility due to the evolution of fraction female. 


%%%%
%\subsection{Counterfactual Wage Equilibrium}

%The main counterfactual will be a manipulation of the fraction female in each occupation, affecting worker share of match surplus, $u^g_o$. To perform any counterfactual simulations, including changes in the size of the female labor force, changes in amenities, and changes in $F$, we must leverage the fact that we identified the job payoffs $WTP^g_o$. For counterfactuals we must assume $u^g_o$ and $WTP^g_o$ remain fixed, other than any specific changes to $u^g_o$ that are part of the counterfactual.

%and the number of jobs left unfilled in each occupation $ln(\mu_{0Y})$

%In the dynamic simulations I assume that the willingness-to-pay parameters $WTP^g_o$ are fixed. The choices of the previous cohort produces new values of worker utility ($u^g_o \rightarrow u^{*g}_o$) through the fraction female. 

%I then solve for the new type level transfers $\bar{W}^{*g}_o$ that equate the market. $ \bar{W}^{*g}_o$ is then the center of the counterfactual reservation wage distribution which equates the new supply and demand for workers and jobs, assuming the distribution of individual job and worker taste heterogeneity is held fixed. 

%$\bar{W}^{*g}_o$ must equate the number of jobs in $o$ choosing gender $g$ to the number of workers of gender $g$ choosing occupation $o$. This must hold for any gender $g \in G$ and occupation $o \in O$. Intuitively, with equilibrium wages there can be no workers or jobs that want to match but cannot. 

%$$ Pr(\text{worker of gender $g$ chooses occupation } o) = Pr(\text{job in occupation $o$ chooses gender } g)$$
%$$ \text{Workers of gender $g$ choosing occupation } o = \text{Number of jobs in occupation $o$ choosing gender } g)$$

In equilibrium, there can be no workers or jobs that want to match but cannot. Recall that $n_o$ is the number of jobs in occupation $o$, and $n^g$ the number of workers of gender $g$. Then for all occupations $o$ and gender $g$ the $ \bar{W}^{*g}_o$ must solve

$$n^g*Pr(\text{$i \in g$ chooses } o)  =n_o*Pr(\text{$j \in o$ chooses } g) $$

which in terms of the choice probabilities defined above is


\begin{align*}
n^g\frac{exp(\frac{\bar{u}^{g}_o + \bar{W}^g_o}{\sigma^g_{\eta}})}{\sum_o exp(\frac{\bar{u}^{g}_o + \bar{W}^g_o}{\sigma^g_{\eta}})} =& n_oPr(\widebar{WTP}^g_o - \bar{W}^g_o + \bar{\xi}^g_j \geq \widebar{WTP}^{-g}_o - \bar{W}^{-g}_o + \bar{\xi}^{-g}_j) \\ 
& *Pr ( \bar{\xi}^F_j \geq {-\widebar{WTP}^F_o + \bar{W}^F_o} ) *Pr ( \bar{\xi}^M_j \geq {-\widebar{WTP}^M_o + \bar{W}^M_o} )     \\
\end{align*}


%%%


In with equilibrium wages clearing the market, there is much less movement in segregation patterns\footnote{Although the male preference for working with men is weakly identified (and not statistically significant) these predicted future patterns hold broadly even if men are given a coefficient on fraction female at the lower bound of the confidence interval. This is the case in which we would assume men would be most likely to flee female fields and reinforce tipping.} because most occupations are very close to their unique equilibrium. However the model does predict changes in the gender composition of some occupations. Occupations predicted to become more than 10\% more female in the long run include Health diagnosing occupations (38\% to 60\%), engineers, architects and surveyors (17\% to 30\%), math computer and natural science (32\% to 43\%), and precision production occupations (25\% to 38\%). Male occupations predicted to become less female in the future include various machine operators fabricators assemblers testers (20\% to 5\%) and metal wood plastic print textile (30\% to 10\%).

%We also see executive, administrative and managerial, post-secondary teachers, and management related occupations become more female. 



%Some occupations, e.g. nursing and administrative support, remain very female dominated. 

%Other occupations such as manufacturing and farming converge to fully male. 





%However, future adjustment in the fraction female, with equilibrium wages, has the impact of furthering the feminization of some occupations. 



%Occupations that appear to tip male include food preparation and service and retail. 


%Whether or not management related occupations becomes male or female depends on whether the market clearing wage is allowed to update. If the wage is fixed this occupation becomes highly male dominated. However if the market clearing wage is adjusted, it becomes female dominated. 

%This is because the reservation wage decreases for women relative to men.

%\subsubsection{Status Quo Simulation Results}
%% Linear for men cubic for women, no wage updating
%% in these the wage updates only after 2022
%\begin{center}
%\includegraphics[width=.8\textwidth]{"/Users/Miriam/OneDrive/Box Sync/TYPlocal/output/graphs/Final_v2_clin_occ61"}
%\includegraphics[width=.8\textwidth]{"/Users/Miriam/OneDrive/Box Sync/TYPlocal/output/graphs/Final_v2_clin_occ62"}
%\includegraphics[width=.8\textwidth]{"/Users/Miriam/OneDrive/Box Sync/TYPlocal/output/graphs/Final_v2_clin_occ63"}
%\includegraphics[width=.8\textwidth]{"/Users/Miriam/OneDrive/Box Sync/TYPlocal/output/graphs/Final_v2_clin_occ64"}
%\end{center}



%These results are surprising given the observed slowing down of integration of occupations since the 1980s. With the estimated gender preferences we continue to see drastic movement in the fraction female in some occupation over the next 100 years based solely on the evolution of the fraction female as an endogenous attribute. In fact, all occupations do not appear to to be in stable fixed points in the fraction female until after 2200. This occurs in the absence of any shocks to overall labor supply of men or women to the market. 

The fit of the model can be assessed by comparing the top left panel, which is observed fraction female by occupation in the Census data,\footnote{Both observed and simulated fraction females are for age ranges 35-64} to the right and bottom panels, which are model simulations. Note that the model simulations do not exactly match the Census data patterns in the top left because the dynamic updating of the fraction female across cohorts was not a moment targeted by the model. In the data workers also do not necessarily stay in their starting occupation for their lifetime.

%Another reason for discrepancy between the model predictions prior to 2012 and the observed data prior to 2012 is that the model prediction fixes workers in their starting occupation for their entire lifetime. Therefore we would expect a discrepancy to the extent that some occupations have a different fraction female among younger workers than older workers, for example if there is differential attrition due to fertility among women by occupation.







%especially in the case where I do not allow the equilibrium wage to update. However overall, 



%In general, with a starting point at parity, occupations converge to a gender ratio closer to parity than reality.

% Linear for men cubic for women
%\begin{center}
%\includegraphics[width=.8\textwidth]{"/Users/Miriam/OneDrive/Box Sync/TYPlocal/output/graphs/Final_v2_clin_f_nowW_occ61"}
%\includegraphics[width=.8\textwidth]{"/Users/Miriam/OneDrive/Box Sync/TYPlocal/output/graphs/Final_v2_clin_f_nowW_occ62"}
%\includegraphics[width=.8\textwidth]{"/Users/Miriam/OneDrive/Box Sync/TYPlocal/output/graphs/Final_v2_clin_f_nowW_occ63"}
%\includegraphics[width=.8\textwidth]{"/Users/Miriam/OneDrive/Box Sync/TYPlocal/output/graphs/Final_v2_clin_f_nowW_occ64"}
%\end{center}





%If the fraction female in each occupation does converge to a fixed point, a follow up question is how many generations does this take to occur. I expect to find that we are approaching a stable equilibrium. This would be consistent with the slowing down in the integration of occupations since the 1980s. I do not expect to see drastic movement in the fraction female without major changes to other parameters, some of which will be experimented with below.
	


%\clearpage
%\subsubsection{Cubic specification: recentered, impose top of CI for male wage coef}
%\begin{center}
%\includegraphics[width=.8\textwidth]{"/Users/Miriam/OneDrive/Box Sync/TYPlocal/output/graphs/Final_occ61"}
%\includegraphics[width=.8\textwidth]{"/Users/Miriam/OneDrive/Box Sync/TYPlocal/output/graphs/Final_occ62"}
%\includegraphics[width=.8\textwidth]{"/Users/Miriam/OneDrive/Box Sync/TYPlocal/output/graphs/Final_occ63"}
%\includegraphics[width=.8\textwidth]{"/Users/Miriam/OneDrive/Box Sync/TYPlocal/output/graphs/Final_occ64"}
%\includegraphics[width=.8\textwidth]{"/Users/Miriam/OneDrive/Box Sync/TYPlocal/output/graphs/Final_occ65"}
%\end{center}


%\clearpage
%\subsubsection{Linear Specification}
%\begin{center}
%\includegraphics[width=.8\textwidth]{"/Users/Miriam/OneDrive/Box Sync/TYPlocal/output/graphs/Final_lin_occ61"}
%\includegraphics[width=.8\textwidth]{"/Users/Miriam/OneDrive/Box Sync/TYPlocal/output/graphs/Final_lin_occ62"}
%\includegraphics[width=.8\textwidth]{"/Users/Miriam/OneDrive/Box Sync/TYPlocal/output/graphs/Final_lin_occ63"}
%\includegraphics[width=.8\textwidth]{"/Users/Miriam/OneDrive/Box Sync/TYPlocal/output/graphs/Final_lin_occ64"}
%\includegraphics[width=.8\textwidth]{"/Users/Miriam/OneDrive/Box Sync/TYPlocal/output/graphs/Final_lin_occ65"}
%\end{center}

%\clearpage
%\subsubsection{Linear for men cubic for women, wage updating, need to re-run with liml}
%\begin{center}
%\includegraphics[width=.8\textwidth]{"/Users/Miriam/OneDrive/Box Sync/TYPlocal/output/graphs/Final_clin_occ61"}
%\includegraphics[width=.8\textwidth]{"/Users/Miriam/OneDrive/Box Sync/TYPlocal/output/graphs/Final_clin_occ62"}
%\includegraphics[width=.8\textwidth]{"/Users/Miriam/OneDrive/Box Sync/TYPlocal/output/graphs/Final_clin_occ63"}
%\includegraphics[width=.8\textwidth]{"/Users/Miriam/OneDrive/Box Sync/TYPlocal/output/graphs/Final_clin_occ64"}
%\end{center}

%\clearpage
%\subsubsection{Simulation Results}
%% Linear for men cubic for women, no wage updating
%\begin{center}
%\includegraphics[width=.8\textwidth]{"/Users/Miriam/OneDrive/Box Sync/TYPlocal/output/graphs/Final_clin_nowW_occ61"}
%\includegraphics[width=.8\textwidth]{"/Users/Miriam/OneDrive/Box Sync/TYPlocal/output/graphs/Final_clin_nowW_occ62"}
%\includegraphics[width=.8\textwidth]{"/Users/Miriam/OneDrive/Box Sync/TYPlocal/output/graphs/Final_clin_nowW_occ63"}
%\includegraphics[width=.8\textwidth]{"/Users/Miriam/OneDrive/Box Sync/TYPlocal/output/graphs/Final_clin_nowW_occ64"}
%\end{center}


%
%\clearpage
%\subsubsection{Quadratic for both, no wage updating}
%\begin{center}
%\includegraphics[width=.8\textwidth]{"/Users/Miriam/OneDrive/Box Sync/TYPlocal/output/graphs/Final_qq_nowW_occ61"}
%\includegraphics[width=.8\textwidth]{"/Users/Miriam/OneDrive/Box Sync/TYPlocal/output/graphs/Final_qq_nowW_occ62"}
%\includegraphics[width=.8\textwidth]{"/Users/Miriam/OneDrive/Box Sync/TYPlocal/output/graphs/Final_qq_nowW_occ63"}
%\includegraphics[width=.8\textwidth]{"/Users/Miriam/OneDrive/Box Sync/TYPlocal/output/graphs/Final_qq_nowW_occ64"}
%\end{center}

%
%	
%	Second I plan to simulate the sensitivity of equilibria to initial sorting patterns. For example I may change the initial starting point of the fraction female faced by the 1960 cohort, and observe the ramifications through subsequent cohorts. This is a compelling exercise if one views initial sorting patterns as somewhat arbitrary, perhaps due to idiosyncratic cultural factors (cite the occupation ladder book here). In this exercise again we fix all parameters at their estimated values. Much of the variation over time will be explained by the time*gender dummies and the changes in the firm side willingness-to-pay parameters estimated in each cross section. 
%	
%	I will observe how the counterfactual evolution of the gender ratio and wages compare to the evolution from the starting point in the data, and how long the segregation takes to converge to the same or a different equilibrium from this different starting point. I expect that the ultimate stable equilibrium will be highly sensitive to the initial condition due to the preferences over the gender ratio.
	
%\subsubsection{Welfare Impact of Policies}
%
%	If gender segregation is viewed as a problem, the welfare impact of policies to combat it is important. I simulate several policies that could be used to pull workers into occupations in which they are the minority, for example men into nursing and women into STEM. I can imagine this hypothetical policy in two ways. First one could increase the utility value of that occupation by a certain percentage for the minority gender. In reality this could reflect adding amenities valued by men or women, perhaps through government mandate or subsidy. Second one could incentivize affirmative action on the part of employers by increasing the value to the employer of hiring a worker of a given gender. One could imagine this through a tax rebate or other subsidy mechanism. I expect only fairly dramatic fixed alterations to the firm or worker side payoffs will result in convergence to a different equilibrium.
%	
%	The last simulation exercise will be to erode the gender preference itself. This makes most sense if the preference arises from a malleable source such as a stigma and not actual amenities correlated with the fraction female. One could imagine a public relations campaign encouraging welcoming environments or inspiring workers to envision themselves in a particular occupation that previously seemed off limits. I will start the simulation at the initial 1960 period but this time with no gender preferences and compare the evolution of segregation and wages.




%such that the number of jobs demanded by each worker type equals the number of jobs provided. The supply of workers is equal to the share of workers picking each type of job times the total number of potential workers in the economy. The demand for workers is likewise the share of jobs choosing each type of worker times the total number of jobs in the economy.

%$$ Pr(i \in g \text{ chooses } j \in o) = \frac{exp(\frac{\bar{u}^{g}_o + \bar{W}^g_o}{\sigma^g_{\eta}})}{\sum_o exp(\frac{\bar{u}^{g}_o + \bar{W}^g_o}{\sigma^g_{\eta}})}$$

%\begin{align*}
%Pr(j \in o \text{ choose } F , \widebar{WTP}^F_o - \bar{W}^F_o + \bar{\xi}^F_j \geq 0) &=  Pr(\widebar{WTP}^F_o - \bar{W}^F_o + \bar{\xi}^F_j \geq \widebar{WTP}^M_o - \bar{W}^M_o + \bar{\xi}^M_j) ( 1-Pr ( \bar{\xi}^F_j < {-\widebar{WTP}^F_o + \bar{W}^F_o} ))   \\
%\end{align*}

%\begin{align*}
% \frac{exp(\frac{u'^g_o+W'^g_o}{\sigma_Y})}{\sum_{y} exp(\frac{u'^g_o+W'^g_o}{\sigma_Y})} = \frac{exp(\frac{U'^g_o}{\sigma_Y})}{\sum_{y} exp(\frac{U'^g_o}{\sigma_Y})} = \frac{exp(\frac{\Pi'^g_o}{\sigma^g_{\xi}})}{\sum_{x} exp(\frac{\Pi'^g_o}{\sigma^g_{\xi}})}=  \frac{exp(\frac{\pi^g_o-W'^g_o}{\sigma^g_{\xi}})}{\sum_{x} exp(\frac{\pi^g_o-W'^g_o}{\sigma^g_{\xi}})}\\
%\end{align*}

%\input{Estimation_occfinal_sex_1_year_2012}

\subsection{Steady States by Occupation}
I search for all equilibria in the fraction female for each occupation individually. To do so I graph the mapping between fraction female in the current period and the next period. Recall that fraction female chosen today affects the choices in the next period through women's preference for higher fraction female. By examining these graphs it is easy to find fixed points, where the fraction female this period is the same as the fraction female in next period, by observing intersections with the 45 degree line. I fix all attributes of other occupations, including the fraction female, to focus only on the occupation at hand. I graph ten equidistant starting points between 0\% female and 100\% female using 2012 parameter values, and show a fitted line through these points on the graphs.

I find that every occupation has one unique equilibrium in the fraction female. The equilibria are close to the observed 2012 values of fraction female, and where there is deviation from the observed fraction female they are close to the long run equilibria shown in Figures \ref{fig:sq1} through \ref{fig:sq4} where the fraction female is allowed to evolve in all occupations at once. The transition graphs for some selected occupations are shown in Figures \ref{transitions17} through \ref{transitions83}.

%\footnote{As discussed above, in Figures \ref{fig:sq1} through \ref{fig:sq4} all occupations are allowed to evolve to a fixed point at the same time. In these graphs, changes in fraction female in any one occupation have repercussions for the fraction female in all other occupations.}

%The transition graphs also show that the further away the starting point is from the stable equilibrium, the faster the convergence towards that equilibrium. 

If the preference for working with women is strong enough, as women leave a male occupation female wages will rise enough to price women out of the occupation and produce a 0\% female equilibrium. Because I do not estimate a preference on the part of men to work with men, they will not be similarly priced out of female occupations, so a 100\% female equilibrium is unlikely. As a result with a female preference, I would expect occupations to have equilibria at either 0\% female, somewhere in between 0\% and 100\% female, or both. Figure \ref{fig:tipping} illustrates the stylized model for the case in which the gender preference produces both an all male and a mixed equilibrium (mixed meaning between 0\% and 100\% female). 

% IS THIS INTERESTING?? CUT THIS?
%In simulating future sorting patterns with endogenous wages, I see that some occupations still converge to almost 0\% female, whereas no occupation lands above 90\% female. This is because firms are willing to pay more for men, and men do not have a preference against working with women. So if enough women leave a male occupation, female wages can become so high that it is no longer profitable to hire any women. On the other hand, as men leave a female occupation, female wages cannot go negative, so firms are still willing to pay for some men.


I estimate that all occupations have only the mixed equilibrium. This is because the gender preference is not strong enough to cause female labor supply to actually cross male labor supply, meaning that employers are always able to hire a few women who really love the job for cheap, making a 0\% female occupation unsustainable.

%So rather than having stable equilibria at 0\% or 100\% female as might be predicted by the \citeA{Pan2010} model, starting at such extreme points actually leads to large movements to the more moderate stable equilibrium in most cases. 

Furthermore, only a few occupations are characterized by very male or female dominated equilibria. At greater than 80\% female we have only Health Services Occupations with a fixed point at around 87\% female, and Health Assessment and Treating and Therapists at around 85\% female. At under 20\% female we have Agriculture Forestry and Fishing at around 12\% female, Machine Operators Fabricators Assemblers and Testers at around 15\% female, Road Rail and Water Transportation at around 10\%, and both Mechanics and Repairers, and Construction and Extraction, at around 5\% female.

%The location of the stable equilibria depend on the location and shape of the male and female labor supply curves. More occupations have an equilibrium at close to 0\% female than 100\% female. This is partly due to the lack of male preferences over fraction female, meaning that it is possible for women to be priced out of a very male occupation, but difficult for men to be priced out of female occupations. This is exacerbated by the fact that in most occupations firms have a higher willingness to pay for male workers.

\subsubsection{Conditions for Multiple Steady States}
Although the estimated parameter values produce only one equilibrium in the fraction female for each occupation, the model does allow for multiple equilibria to emerge. Below I explore two scenarios that could lead to multiple equilibria in the fraction female. First, doubling the magnitude of the preference for women to work with women, and second, fixing equilibrium wages so they are not allowed to adjust and form compensating differentials.

%predict multiple equilibria to emerge with a stronger gender preference, or by fixing equilibrium wages. Below I run simulations to find the multiple equilibria by occupation with a higher gender preference, and with a fixed wage. In the following section I further discuss the mechanisms by which wages prevent multiple equilibria in the fraction female.

I find that doubling the preference over the fraction female produces multiple equilibria in some occupations, such as Postsecondary Teachers, and Engineers Architects and Surveyors. These occupations have one equilibrium at close to 0\% female and another at majority female. Intuitively, a very strong gender preference makes it cheapest for an occupation to hire either all men or majority women. The initial fraction female would determine to which equilibrium the occupation converges. 

%It turns out that allowing wages to vary freely to equate supply and demand plays a key role in eliminating multiple equilibria.

Next I fix the wage so that the gender composition reflects only labor supply, and the fixed wage will no longer be able to clear the market as the fraction female changes. In the case of fixed wages, 27 of the 34 occupations have two stable equilibria as opposed to one. Figures \ref{ftransitions17} through \ref{ftransitions83} show the same set of occupations as Figures \ref{transitions17} through \ref{transitions83} but with fixed wages. In all of these examples we have an equilibrium that is majority female, and an equilibrium that is close to zero percent female, implying that allowing wages to vary freely to equate supply and demand plays a key role in eliminating multiple equilibria..

The wage moderates the impact of the preference of women to enter more female occupations, as it would over any endogenous amenity. As more women enter an occupation it becomes more attractive, but at the same time wages go down as employers are able to attract more women at lower cost, thus ultimately dampening the supply of women to the occupation. Likewise as women leave an occupation, the wage offered to women in that occupation goes up, which increase female labor supply to that occupation and dampens the movement towards 0\% female.


%% INCLUDE THESE WAGE RATIO RESULTS???
%To explore the role of wages I run two sets of simulations. First I fix wages at their current value, allowing for no market clearing adjustment, and use transitions graphs as described above to observe the equilibria in fraction female. Next I impose only that the ratio of male to female wages is fixed, but that otherwise wages can adjust freely up or down. In both sets of simulations multiple equilibria emerge in some occupations.

%In the next set of simulations, I allow wages to adjust up and down, but I fix the ratio of male to female wages. I fix wage ratio at the ratio of male to female labor force participation, as measured in hours over the lifetime by gender and occupation. This roughly approximates an equal pay for equal work law that would prohibit wage variation based on productivity or labor supply factors (differential taste for job amenities for example). Similarly to the case of fixing wages entirely, I find that with a fixed wage ratio many occupations have multiple equilibria in the fraction female. In fact fixing the male-female wage ratio appears to produce more multiple equilibria by increasing the sensitivity to the fraction female without allowing employers to impose their desired gender composition by changing relative wages.


% no wage adjustment results:
% Engineers only has female equilibrium
% Only male equilibrium: Machine operators, fabricators, assemblers, testers; metal wood plastic print textile; food preparation and service occupations; sales workers, retail, and personal services; writers artists entertainers athletes; social scientists, lawyers, judges, urban planners, librarians; teachers except postsecondary, 

% how does teachers only have a male equilibrium??? women like it more and are good at it......

% two stable equilibria, Sf is upward sloping for first few women, then downward sloping as more women enter, then upward sloping again at high fraction female. WHY? the equilibrium at 0% makes more sense than the once I'm seeing at like 20% female. but this is in the pan model where wages adjust. here we are talking no wage adjustment.



%In the case of a fixed male-to-female wage ratio, many more occupations have multiple equilibria in the fraction female. NEED TO RUN ALL OF THESE TO COUNT THEM!!  Why does fixed ratio mean more multiple equilibria than the completely fixed simulation? accelerates convergence but employer still can't impose its preference over male vs. female so we still don't get the single equilibrium??

	%include fixedratio graphs ?
	
%To explore the role of wages, I fix wages at their current value, allowing for no market clearing adjustment, and use transitions graphs as described above to observe the equilibria in fraction female. In the case of fully fixed wages, 27 of the 34 occupations have two stable equilibria as opposed to one. Figures \ref{ftransitions17} through \ref{ftransitions83} show the same set of occupations as Figures \ref{transitions17} through \ref{transitions83} but with fixed wages. 

%Since wages are fixed, these simulations effectively hold the employers' responses fixed. In the absence of firms adjusting wages to meet demand, the impact of fraction female on worker labor supply is less dramatic. The result is that whether an occupation converges to more male or female can depend on starting point, see for example \ref{ftransitions17}. Freely moving wages reinforce convergence, and convergence to a single equilibrium.

% start here
	



\subsection{Impulse Response Examples}
To take a closer look at how wages adjust to compensate for the preference over fraction female, I plot the response of wage to a change in the fraction female in two case studies. Specifically I set a female dominated occupation to be 0\% female, and separately, a male dominated occupation to be 100\% female, and observe how wage adjustment facilitates convergence back to the unique stable equilibria, which occurs after about eight cohorts of workers.

% How do I explain that the transition graphs show no state dependence even without wage adjustment, but the graphs with all of them moving do so state dependence??? mechanics and repairers stays female without wage adjustment...... and nurses stays male.... what the hell... moving all occupatoins at the same time causes more polarization?


%Firm preferences generally mitigate the impact of the worker side gender preferences in determining sorting patterns. 

In the first simulation I set nursing (``Health Technologists and Technicians") to be 0\% female in 1960. In reality nursing was close to 100\% female in 1960. If wages were fixed, 0\% female would be a stable equilibrium for nursing. However, the gender preference is not strong enough to completely price women out of nursing once wages adjust to reflect labor demand. It is still cost-effective for firms to hire some women who really love nursing jobs, and so the female wage offer rate must be set high to equate supply and demand by compensating women for the disutility of the low fraction female.

%In the simulation, 0\% male is a stable equilibrium if we fix wages and thereby ignore firm preferences. Without equilibrium wages, no women want to enter nursing when it is 0\% female because of the preference for working with women. 

The wage adjustment process for the simulation is shown in Figure \ref{nurses}. Solid lines are simulated counterfactual wage offer levels, while dotted lines are actual estimated wage offers. We can see that although simulated female wages in nursing start out way higher than reality due to the compensating differential, they very quickly drop as nursing feminizes. % and women are cheaper to hire.

Women with particularly high utility from nursing are enticed to enter the occupation by the high wages, which in turn makes nursing more attractive for the next cohort. This in turn lowers the simulated reservation wages of the next cohort of workers, which also makes women cheaper for firms to hire. This process continues until nursing is female-dominated and simulated wage offers have dropped to the levels estimated in the data.

In the second example, I set ``Mechanics and Repairers", a male dominated occupation, to be a 100\% female in 1960. If wages were fixed, the occupation would then converge to its majority female stable equilibrium at around 80\% female. However, with wage adjustment, which can be seen in Figure \ref{mechanics}, female wages quickly skyrocket as more and more men start to become Mechanics and Repairers. After about eight cohorts, women are no longer affordable to hire and the occupation has converged to its unique stable equilibrium at around 0\% female.




%Movement to 100\% female dominated is mitigated by the fact that wages are bounded below by zero, so without male gender preferences moving the male reservation wage distribution up, it is unlikely that every woman will be cheaper to hire than every man. This is exacerbated by the fact that in most occupations firms have a higher willingness to pay for male workers.

%On the other hand, in occupations with low fraction female, wage adjustment does allow movement to close to 100\% male. In these cases, the female reservation wage distribution moves up to the point that it is entirely above the firm's willingness-to-pay. This is again a phenomenon that hinges on the willingness-to-pay gap.

%The extent to which occupations exhibit more or less tipping depends on the extent to which they are not yet in a stable equilibrium, which depends on the relative non-wage utility of male and female workers, the willingness-to-pay gap, and also these values in all other occupations since the workers' and firms' outside options matter for labor supply and demand. 

\subsection{Simulation from Initial Parity}
If preferences over an endogenous amenity like the fraction female are strong enough, there could be multiple equilibria. In this case long run sorting patterns could depend on the initial conditions, or historical segregation, which would determine which equilibrium is selected. As a first pass at testing this hypothesis, I begin my simulation with all occupations at 50\% female in the initial year of 1960. The figures \ref{fig:ip1} through \ref{fig:ip4} show that this does not affect the long run outcome. The patterns are shockingly similar today as if occupations had begun from the observed segregated position. 

This result suggests that we are currently in a stable equilibrium, and that parity is close enough to this stable equilibrium that it leads to convergence. It could also suggest that each occupation has in fact only one stable equilibrium in the fraction female, and would converge to this equilibrium regardless of any starting point. In the next section I further explore this question by searching for all equilibria in the fraction for each occupation separately.


\section{Conclusion}

%Most occupations appear to be close to their stable equilibrium based on the limited amount of movement that occurs both in the status quo future simulation, and in counterfactuals.

%I examine the consequences of gender segregation as an endogenous occupation attribute and find that current preferences could result in long run tipping patterns in the future akin to those identified in \citeA{Pan2010}. However 



%\textit{``Supporting women STEM students and researchers is not only an essential part of America's strategy to out-innovate, out-educate, and out-build the rest of the world; it is also important to women themselves"} (Office of Science and Technology Policy under the Obama administration) %\cite{ObamaSTEM}.

%\textit{``One of the things that I really strongly believe in is that we need to have more girls interested in math, science, and engineering. We've got half the population that is way underrepresented in those fields and that means that we've got a whole bunch of talent...not being encouraged the way they need to."} (President Obama 2013)\footnote{\url{https://obamawhitehouse.archives.gov/administration/eop/ostp/women}}


\textit{``We've got half the population that is way underrepresented in those fields [math, science, and engineering] and that means that we've got a whole bunch of talent...not being encouraged the way they need to."} (President Obama 2013)\footnote{\url{https://obamawhitehouse.archives.gov/administration/eop/ostp/women}}


%Policymakers view reducing occupation gender segregation as a way to narrow the gender wage gap and even increase economic productivity. 

%From the Office of Science and Technology Policy under the Obama administration: 

% GAO or BLS reports on gender... inspector general
% former policy people, paul krugman, 
% we need everyone in STEM and compsci
% obama quote on pushing women into stem, then my results indicate maybe this is a bad idea

%One way to justify the economic benefit of reducing segregation is if workers have preferences over the gender of their colleagues, and this results in historical dependence. 

It is an open policy question as to what sort of encouragement would lead more women to enter male occupations and vice versa. Ideally we might see that a few men or women entering a field might lead to a flood of followers. I find women do prefer to go into occupations that already have more women. However, womens' preference is not strong enough that simply putting more women in a field will lead more women to enter in the long run. I also find no evidence that men prefer to enter occupations that already have more men. 

%On the other hand my simulations indicate that changes to occupation amenities or the perceived productivity of men or women would lead to long run changes in the gender of occupations.

%Could encouragement of say women to enter STEM, or men to enter nursing, lead to long run changes in the gender composition of these fields? The answer is that it depends on the type of encouragement. I find a strong preference on the part of women to work with other women, but that this preference is not strong enough that simply putting more women in a field will lead more women to enter in the long run. Long run gender composition is only responsive to changes in labor supply and demand, such as occupation amenities or the valuation of men and women by firms.

The reason that putting more men or women in an occupation has no long-run effect on segregation is that wages are free to adjust. As the fraction female goes up, wages for women go down, which slows the entry of women into the occupation. The level of gender preference would have to be two times as large as I estimate in order to overcome the tendency to converge back to the original sorting pattern through wage adjustment.

% If wages were not allowed to adjust in my model, the level of gender preference that I estimate would imply that occupations could ``tip" between male or female based on small changes in composition. 

%A model without endogenous wages would in fact imply multiple equilibria in most occupations. If the gender preference were twice as strong this would also be too strong for wages to be able to compensate, causing multiple equilibria. This suggests that thinking about how changes to occupations interact dynamically with changes to wages, and gender preference, could be important for future work on gender segregation.

The preference on the part of women to work with women increases gender segregation by creating a feedback loop that amplifies the impact of gender differences in labor supply and demand. According to my simulations, the``tipping" patterns documented by \citeA{Pan2010} might be the result of changes in the perceived productivity of men and women, compounded by a feedback loop from the preference of women to work with women. This feedback loop mechanism also exacerbates the gender wage gap by lowering wages in female dominated occupations through compensating differentials.

% I estimate that occupations do not have the potential to be either male or female depending on historical patterns. Therefore short-run shocks to the fraction female in occupations do not cause tipping, rather convergence back to the original fraction female.
%Policymakers hoping to reduce gender segregation should focus on changing fundamental economic parameters such as non-wage amenities of jobs and the perceived productivity of men and women, as these things can move the stable equlibrium and cause long run changes in segregation. 


%\citeA{Schelling1971} showed that even a small degree of homophily can result dramatic changes in group composition. \citeA{Pan2010} documents such dramatic changes in occupation gender composition in the United States. This motivates the question of whether slight encouragement of say, women to enter STEM, could result in tipping patterns as observed in the past. 

%I find a strong preference on the part of women to work with other women. However, I estimate that this preference is not strong enough to produce multiple stable equilibria in the fraction female. Therefore, according to my estimates, putting more women in an occupation does not mean that more women will follow. In fact in my simulations the occupation will always converge back to its unique stable equilibrium in the long-run.

%According to my estimates, convergence to a more productive stable equilibrium is not possible based on shocks to gender composition alone.



% consistent with tipping from stable male to stable female equilibria. This motivates the question of whether current segregation patterns are optimal, or could some occupations produce more surplus at a more male or female equilibrium.

%However, this preference is not strong enough to produce tipping points in the fraction female by occupation as proposed by \citeA{Pan2010}. I estimate that all occupations have only one stable fraction female. 



%Sorting patterns depend on the extent to which men and women value occupations differently, and occupations value men and women differently. Changes to occupational attributes or the perceived productivity of men and women could therefore move the stable equlibrium and cause long run changes in segregation.

%  talk about the firm vs. worker preferences simulation here???

The estimates of this model are likely imperfectly predictive of the future of occupation gender segregation, but the model does prove useful for learning about the role of wages in two-sided matching with an endogenous amenity. This paper focused on the fraction female in occupations, but future work could use this model to look at race, age, or any other group preference or endogenous amenity, and the moderating effect of price adjustment.

In addition, although I do not find evidence of tipping between multiple equilibria given my estimated parameter values, it is clear that this could occur under different circumstances, such as a stronger gender preference or stickier wages (for example equal pay for equal work laws). Future work is needed to fully understand occupation gender segregation, and could benefit from considering how tipping is mitigated by compensating differentials.

%how changes to labor supply and demand interact dynamically with wages, and gender preference to cause changes in segregation.

%or in a more general context, how allowing prices to vary freely to equate supply and demand can moderate the outcome of a group preference.

%We are stable! Don't push people around it's not optimal! but keep an eye out for male prefs in the future!? also for equal pay laws! Cuz tipping is real and could happen, just not now given my estimates!



%The evidence that I uncover points to the importance of womens' preferences for entering female-dominated occupations and against entering male-dominated occupations, with little to no role for mens' preferences against entering female-dominated occupations. 

%Furthermore the matching model allows market clearing wages to adjust, which can exacerbate or attenuate the tipping dynamics depending on the preference structure, and where an occupation is relative to its stable equilibrium. 

%Most occupations appear to be close to their stable equilibrium based on the limited amount of movement that occurs both in the status quo future simulation, and in the counterfactual simulations of initial parity and shocks to nurses and mechanics.

%In future work I will simulate the impact of integrative policies such as wage subsidies or amenity requirements on segregation dynamics. 

%\section{Preliminary First Stage: 386 occupations}
%Below are preliminary results of the decomposition of approximately $u^g_o$. \textbf{Results need to be adjusted for scale estimation!}
%
%\begin{align*}
% u^g_o = Z_y\beta_X + Z^g_o\beta_X + g_X(F)+ \xi_{y} \\ 
%\end{align*}
%
%Where the parameters of interest are the $\delta$ in $g(F)$:
%$$g_X(F)= \delta_{x}^1 F + \delta_{x}^2 \mathcal{I}(F<.5)(F-.5)^2 + \delta_{x}^3\mathcal{I}(F>.5)(F-.5)^2 $$
%
%The following plots are of $g_X(F)$ for each type of worker.

%Each column represents a type of worker, with columns 1-4 education categories (less than high school, high school degree, some college, college degree). The tables are for childless males, males with children, childless females, and females with children, in that order.


%\clearpage
%\subsection{Linear and preference against minority in fraction female}
%\begin{center}
%\includegraphics[width=.8\textwidth]{utility_sexratio_3}
%\includegraphics[width=.8\textwidth]{utility_sexratio_4}
%\includegraphics[width=.8\textwidth]{utility_sexratio_1}
%\includegraphics[width=.8\textwidth]{utility_sexratio_2}
%\end{center}


%
%\section{Instrument for Fraction Female}
%With a strong instrument it would no longer be necessary to make $g_X(F)$ a flexible non-linear function, although these results highlighting a quadratic preference against being a minority for women are interesting. With an instrument we may believe that any impact of the fraction female is due to preferences over the fraction female rather than unobserved attributes in $\xi_Y$ that are correlated with fraction female. 
%
%I propose to use a Bartik instrument to 

%\section{Questions}
%\begin{enumerate}
%\item Do limitations on the sign of transfers (wages) preclude socially optimal outcome?
%\item What are the implications of assuming the coefficient on wages is one for both sides of the market?
%\item Can I make substitution patterns more reasonable, eg. nested logit?
%\item Can I allow the data to pick types of workers and tipping point for $g_X(F)$?
%\item Extension to continuous types?
%\item Convert wages to PDV lifetime income using method outlined by Hoxby or other.
%\end{enumerate}

%
%\section{Further work}
%\begin{enumerate}
%\item To what extent is sex ratio relationship mechanical if not instrumented?
%\item What is the intuition for the separation of worker and firm utility? 
%\item What is the intuition behind the additive separability assumption?
%\item What is the best choice for firm outside option?
%\item Find an instrument for fraction female in the oLS stage, use state variation? maternity leave policies?
%\item Do limitations on the sign of transfers (wages) preclude socially optimal outcome?
%\item Continuous types (avoid two stage estimation process), guarantees unique equilibrium?
%\item Identify types and occupation specific tipping points and use them instead of .5 in the parametric form of $g(F)$, or use semi-parametric estimator in logit.
%\end{enumerate}


\newpage

\clearpage

\section{Appendix}


\subsection{Figures}



\begin{figure}[H]
\centering
\caption{Sales Representatives, Finance, and Business Services: Observed vs. Model Reservation Wages}
\label{sales}
%\footnotesize{\textbf{Sales Representatives, Finance, and Business Services}}
%\includegraphics[width=.7\textwidth]{"/Users/Miriam/OneDrive/Box Sync/TYPlocal/output/graphs/matlabwages/offerwages_occfinal8reg1trunc_1960_v3"}
\includegraphics[width=.5\textwidth]{"/Users/Miriam/OneDrive/Box Sync/TYPlocal/output/graphs/matlabwages/pofferwages_occfinal8reg1trunc_1960"}
\end{figure}

\begin{figure}[H]
\centering
\caption{Health Service occupations: Observed vs. Model Reservation Wages}
\label{health}
%\footnotesize{\textbf{Health Service occupations}}
%\includegraphics[width=.7\textwidth]{"/Users/Miriam/OneDrive/Box Sync/TYPlocal/output/graphs/matlabwages/offerwages_occfinal18reg1trunc_1960_v3"}
\includegraphics[width=.5\textwidth]{"/Users/Miriam/OneDrive/Box Sync/TYPlocal/output/graphs/matlabwages/pofferwages_occfinal18reg1trunc_1960"}
\end{figure}

\begin{figure}[H]
\centering
\label{modelwages1}
\includegraphics[width=.8\textwidth]{logwages13reg1trunc_occfinal}
\end{figure}

\begin{figure}[H]
\centering
\label{modelwages2}
\includegraphics[width=.8\textwidth]{logwages14reg1trunc_occfinal}
\end{figure}

\begin{figure}[H]
\caption{Male and Female Log Utility by Fraction Female in Occupation}
\label{prefs}
\begin{center}
\includegraphics[width=.8\textwidth]{"/Users/Miriam/OneDrive/Box Sync/TYPlocal/output/graphs/Prefs_linear_quadratic_cubic_dashed"}
%\includegraphics[width=.8\textwidth]{"/Users/Miriam/OneDrive/Box Sync/TYPlocal/output/graphs/Prefs_cubic"}
\begin{minipage}{.8\textwidth}
\begin{tablenotes}
\footnotesize
\item Results of instrumental variables regressions with occupation fixed effects on 6 waves of Census and ACS data (1960-2012), 34 occupations, using reservation wages estimated earlier using MLE.
\end{tablenotes}
\end{minipage}
\end{center}
\end{figure}


\begin{center}
% in these the wage updates after 1960
\begin{figure}[H]
\centering
\caption{Status Quo: Simulated Occupation Segregation Patterns}
\label{fig:sq1}
\includegraphics[width=.8\textwidth]{"/Users/Miriam/OneDrive/Box Sync/TYPlocal/output/graphs/Final_v2_clinv2_nowW_occ61"}
\end{figure}
\begin{figure}[H]
\centering
\caption{Status Quo: Simulated Occupation Segregation Patterns}
\label{fig:sq2}
\includegraphics[width=.8\textwidth]{"/Users/Miriam/OneDrive/Box Sync/TYPlocal/output/graphs/Final_v2_clinv2_nowW_occ62"}
\end{figure}
\begin{figure}[H]
\centering
\caption{Status Quo: Simulated Occupation Segregation Patterns}
\label{fig:sq3}
\includegraphics[width=.8\textwidth]{"/Users/Miriam/OneDrive/Box Sync/TYPlocal/output/graphs/Final_v2_clinv2_nowW_occ63"}
\end{figure}
\begin{figure}[H]
\centering
\caption{Status Quo: Simulated Occupation Segregation Patterns}
\label{fig:sq4}
\includegraphics[width=.8\textwidth]{"/Users/Miriam/OneDrive/Box Sync/TYPlocal/output/graphs/Final_v2_clinv2_nowW_occ64"}
\end{figure}
\end{center}

\begin{center}
\begin{figure}[H]
\centering
\caption{Initial Parity: Simulated Occupationx Segregation Patterns}
\label{fig:ip1}
\includegraphics[width=.8\textwidth]{"/Users/Miriam/OneDrive/Box Sync/TYPlocal/output/graphs/Final_v2_clinv2_f_occ61"}
\end{figure}
\begin{figure}[H]
\centering
\caption{Initial Parity: Simulated Occupation Segregation Patterns}
\label{fig:ip2}
\includegraphics[width=.8\textwidth]{"/Users/Miriam/OneDrive/Box Sync/TYPlocal/output/graphs/Final_v2_clinv2_f_occ62"}
\end{figure}
\begin{figure}[H]
\centering
\caption{Initial Parity: Simulated Occupation Segregation Patterns}
\label{fig:ip3}
\includegraphics[width=.8\textwidth]{"/Users/Miriam/OneDrive/Box Sync/TYPlocal/output/graphs/Final_v2_clinv2_f_occ63"}
\end{figure}
\begin{figure}[H]
\centering
\caption{Initial Parity: Simulated Occupation Segregation Patterns}
\label{fig:ip4}
\includegraphics[width=.8\textwidth]{"/Users/Miriam/OneDrive/Box Sync/TYPlocal/output/graphs/Final_v2_clinv2_f_occ64"}
\end{figure}
\end{center}


% transitions with moving wage
\begin{figure}[H]
\centering
\caption{Transitions in Fraction Female Across Periods}
\label{transitions10}
\includegraphics[width=.6\textwidth]{"/Users/Miriam/OneDrive/Box Sync/TYPlocal/output/graphs/Splot_occ_10v3"}
\end{figure}

\begin{figure}[H]
\centering
\caption{Transitions in Fraction Female Across Periods}
\label{transitions17}
\includegraphics[width=.6\textwidth]{"/Users/Miriam/OneDrive/Box Sync/TYPlocal/output/graphs/Splot_occ_17v3"}
\end{figure}

\begin{figure}[H]
\centering
\caption{Transitions in Fraction Female Across Periods}
\label{transitions35}
\includegraphics[width=.6\textwidth]{"/Users/Miriam/OneDrive/Box Sync/TYPlocal/output/graphs/Splot_occ_35v3"}
\end{figure}

\begin{figure}[H]
\centering
\caption{Transitions in Fraction Female Across Periods}
\label{transitions51}
\includegraphics[width=.6\textwidth]{"/Users/Miriam/OneDrive/Box Sync/TYPlocal/output/graphs/Splot_occ_51v3"}
\end{figure}

\begin{figure}[H]
\centering
\caption{Transitions in Fraction Female Across Periods}
\label{transitions83}
\includegraphics[width=.6\textwidth]{"/Users/Miriam/OneDrive/Box Sync/TYPlocal/output/graphs/Splot_occ_83v3"}
\end{figure}

% transitions with fixed wage
\begin{figure}[H]
\centering
\caption{Transitions in Fraction Female Across Periods: Fixed Wages}
\label{ftransitions10}
\includegraphics[width=.6\textwidth]{"/Users/Miriam/OneDrive/Box Sync/TYPlocal/output/graphs/Splot_nowW_occ_10"}
\end{figure}

\begin{figure}[H]
\centering
\caption{Transitions in Fraction Female Across Periods: Fixed Wages}
\label{ftransitions17}
\includegraphics[width=.6\textwidth]{"/Users/Miriam/OneDrive/Box Sync/TYPlocal/output/graphs/Splot_nowW_occ_17"}
\end{figure}

\begin{figure}[H]
\centering
\caption{Transitions in Fraction Female Across Periods: Fixed Wages}
\label{ftransitions35}
\includegraphics[width=.6\textwidth]{"/Users/Miriam/OneDrive/Box Sync/TYPlocal/output/graphs/Splot_nowW_occ_35"}
\end{figure}

\begin{figure}[H]
\centering
\caption{Transitions in Fraction Female Across Periods: Fixed Wages}
\label{ftransitions51}
\includegraphics[width=.6\textwidth]{"/Users/Miriam/OneDrive/Box Sync/TYPlocal/output/graphs/Splot_nowW_occ_51"}
\end{figure}

\begin{figure}[H]
\centering
\caption{Transitions in Fraction Female Across Periods: Fixed Wages}
\label{ftransitions83}
\includegraphics[width=.6\textwidth]{"/Users/Miriam/OneDrive/Box Sync/TYPlocal/output/graphs/Splot_nowW_occ_83"}
\end{figure}


%  transitions with double the preference parameter for women
\begin{figure}[H]
\centering
\caption{Transitions in Fraction Female Across Periods: Gender Preference Doubled}
\label{ftransitions10}
\includegraphics[width=.6\textwidth]{"/Users/Miriam/OneDrive/Box Sync/TYPlocal/output/graphs/Splot_occ_10double"}
\end{figure}

\begin{figure}[H]
\centering
\caption{Transitions in Fraction Female Across Periods: Gender Preference Doubled}
\label{ftransitions17}
\includegraphics[width=.6\textwidth]{"/Users/Miriam/OneDrive/Box Sync/TYPlocal/output/graphs/Splot_occ_17double"}
\end{figure}

\begin{figure}[H]
\centering
\caption{Transitions in Fraction Female Across Periods: Gender Preference Doubled}
\label{ftransitions35}
\includegraphics[width=.6\textwidth]{"/Users/Miriam/OneDrive/Box Sync/TYPlocal/output/graphs/Splot_occ_35double"}
\end{figure}

\begin{figure}[H]
\centering
\caption{Transitions in Fraction Female Across Periods: Gender Preference Doubled}
\label{ftransitions51}
\includegraphics[width=.6\textwidth]{"/Users/Miriam/OneDrive/Box Sync/TYPlocal/output/graphs/Splot_occ_51double"}
\end{figure}

\begin{figure}[H]
\centering
\caption{Transitions in Fraction Female Across Periods: Gender Preference Doubled}
\label{ftransitions83}
\includegraphics[width=.6\textwidth]{"/Users/Miriam/OneDrive/Box Sync/TYPlocal/output/graphs/Splot_occ_83double"}
\end{figure}

% 

\begin{figure}[H]
\centering
\caption{Status Quo: Simulated Occupation Segregation Patterns}
\label{nurses}
\includegraphics[width=.6\textwidth]{"/Users/Miriam/OneDrive/Box Sync/TYPlocal/output/graphs/WgPDVstart_occ17_clinv2_0_17sc"}
\end{figure}
\begin{figure}[H]
\centering
\caption{Status Quo: Simulated Occupation Segregation Patterns}
\label{mechanics}
\includegraphics[width=.6\textwidth]{"/Users/Miriam/OneDrive/Box Sync/TYPlocal/output/graphs/WgPDVstart_occ51_clinv2_1_51sc"}
\end{figure}


\newpage
\subsection{Tables}

\input{Model_estimates_v4}

%\subsection{Comparison of Average Shares: Modal occupation vs. starting occupation 1960-2012}
%
%\input{share_census_lifeinc}

\input{Stata10_pvIV_PDVstartsex1__v2_}

%\input{Stata10_pvIV_PDVstartsex2__v2_}

\input{Stata10_pvIV_PDVstartsex2__v2_sc}

%\input{peffect}

%\input{peffect2018}  corrected for coding error not updating deltas


\input{peffect20182080}  % corrected for other coding error no updating wage

\input{FE_nowW_PDVstart_clinv2_sc}\label{FEs}

\input{"/Users/Miriam/OneDrive/Box Sync/TYPlocal/output/tables/Wage_gap_occs_clinv4_mw0.tex"}

\clearpage
\newpage

\subsection{General Model Structure}
When a worker and a job match, total surplus is created from the match. In the worker's case the value of a match reflects the amenities of the job. A job might have a particularly collegial environment, or free child care for example. Amenities may be valued differently by gender. On the job side the payoff is the willingness-to-pay for a worker, which could reflect productivity, and differ by gender due to gender differences in turnover, differences in search cost by gender, differences in productivity, or devaluation, for example. The wage determines the split of the total surplus between the worker and the firm.

The most general payoff structure in a matching model would allow each possible match between a worker $i$ and a job $j$ to have its own unobserved match quality. To make the problem empirically tractable, I assume that no portion of the payoff depends on unobservable characteristics of both firm and worker, which is a standard assumption in empirical matching. So although the surplus may depend on $i$ or $j$, it may not depend on $i$ and $j$.

\begin{assumption}
Additive Separability: No component of surplus depends on unobserved characteristics of both workers and firms.
\end{assumption}

Formally, let $g$ denote gender, which is observed as either male ($M$) or female ($F$) in this model. Let $o$ denote occupation. We therefore have workers $i \in g \in G= \{M,F\}$ and jobs $j \in o \in O = \{1,2,...,34\}$.\footnote{Thirty-four occupations are chosen according to data constraints discussed below.} Under additive separability we have that the total surplus from a match between worker $i$ and job $j$, $S^i_j$, can be decomposed:

\begin{align}
S^i_j = S^g_o + \eta^i_o + \xi^g_j
\end{align}


Note that there are components that vary at the occupation*gender level ($S^g_o$), the occupation*worker level ($\eta^i_o$), and the gender*job level ($\xi^g_j$), but never the worker*job level. In other words, additive separability implies that there is no $\xi^i_j$ or $\eta^i_j$. This assumption is important because it allows me to separate the matching problem into two separate discrete choice problems, one for each side of the market \cite{Galichon2013}.

The components of total surplus that depend on unobservables of either the worker ($\eta^i_o$) or the job ($\xi^g_j$) can theoretically come from the worker's utility function, the job payoff function, or both. In order to gain identifying power from the observed wage distribution, and because my research question is focused the the role of worker utility in occupation choice, I assume all unobserved components of surplus originate from the worker's utility. This means that only workers have preferences over unobservables, and jobs care only about whether they chose to hire a male or female worker.

\begin{assumption}
$\eta^i_o$ and $\xi^g_j$ are primitives in the worker's utility function.
\end{assumption}

In other words, each worker has an individual taste for each occupation ($\eta^i_o$) and each job differs in how attractive it is to men and women ($\xi^g_j$). The job amenity heterogeneity can be thought of as any component of the attractiveness of a job that is orthogonal to the overall attractiveness of the occupation, which is included in $S^g_o$. For example child care offerings at a particular employer might differ relative to the average child care offerings in that occupation.


\section{Equilibrium Wages and Stability} \label{equilibrium}
In the following section I outline conditions for a matching to be feasible and stable. I then introduce the equilibrium wage vector and show that it supports feasibility and stability.

\subsection{Feasibility}
A matching is feasible if every worker is matched to at most one job and every job matched to at most one worker. Formally, following \citeA{Salanie2013a}, let $\mu^i_j$ be equal to either $0$ or $1$ where $1$ indicates a match between worker $i$ and job $j$. Then for every $i$ and $j$ a feasible matching has

$$ \sum_{k \in \mathcal{J}} \mu^i_k \leq 1 \text{ and }  \sum_{k \in \mathcal{I}} \mu^k_j \leq 1$$

Similarly following \citeA{Salanie2013a}, the matching must be feasible given the number of men and women and jobs in each occupation available in the market, or

$$ \sum_{j \in \mathcal{J}} \mu^g_j \leq n_g, \text{ } \forall g \text{ and }  \sum_{i \in \mathcal{I}} \mu^i_o \leq n_o, \text{ } \forall o$$

\subsection{Stability}
%Pairwise stability implies that for any worker and any job in the candidate equilibrium, the sum of their individual surplus must be greater than the total surplus if they were to form a match with each other. 

Intuitively, pairwise stability implies that no worker and job that are not currently matched with each other, would prefer to match with each other. Let $i$ and $j$ be a so called ``blocking pair", and let $i$ be currently matched to $j(i)$ and $j$ to $i(j)$. Then pairwise stability states that the sum of the individual surpluses from the existing matches ($i$ with $j(i)$ and $j$ with $i(j)$) must be greater than the surplus of the blocking pair ($i$ and $j$). Therefore even with any possible transfer, $i$ and $j$ will not both prefer to match with each other, because the total possible surplus is lower.

\begin{definition}
Pairwise Stability: In a matching where $i$ is paired with $j(i)$ and $j$ is paired with $i(j)$, it must be the case that $\bar{u}^{i}_{j(i)}+ \bar{\pi}^{i(j)}_j \geq \bar{u}^{i}_{j} + \bar{\pi}^{i}_j, \hspace{3mm}  \forall i,j$. In addition, each worker and job must attain higher surplus than their outside option, or $\bar{u}^{i}_{j(i)} \geq \bar{u}^{i}_{N}$ and $\bar{\pi}^{i(j)}_j \geq \bar{\pi}^{N}_j $, where $N$ represents not working for the worker, and not hiring for the firm.
 \end{definition}
 
Note that on the left hand side $\bar{u}^{i}_{j(i)}+ \bar{\pi}^{i(j)}_j$ includes the wage paid out to the worker and by the job in their respective matches. On the right hand side $\bar{u}^{i}_{j} + \bar{\pi}^{i}_j$ the wage will cancel within the match leaving the underlying total surplus. 

%Therefore, even though it could be the case that in the absence of equilibrium wages both the worker and the job would prefer to match with each other, it must be the case that the total surplus they get from their respective matchings including wages must be higher.

Following \citeA{Shapley1972a}, the pairwise stable matching will be unique and the competitive equilibrium will coincide with the pairwise stable matching, but the competitive equilibrium wage vector may not be unique. I assume the observed wages are the equilibrium wages described in \citeA{Salanie2013a} and \citeA{Salanie2014}. These are the wages that make workers indifferent over jobs within each occupation, and jobs indifferent over workers within each gender. As the sample size of men and women goes to infinity, the equilibrium wages will be unique \cite{Salanie2013a}.

\subsection{Proof of Pairwise Stability}\label{sec.Stability}

Workers choose an occupation to maximize utility, and firms choose a worker to maximize rate of return, so the chosen job $j^*$ and worker $i^*$ respectively must satisfy

\begin{align*}
j^* \in o^* &= \argmax_o ( \bar{u}^{g}_o + \bar{W^g_o}   + \eta^i_o ) \\
i^* \in g^* &= \argmax_g (\widebar{WTP}^g_o -  \bar{W^g_o} - \bar{\xi^g_j})
\end{align*}

From this it is clear than within an occupation, workers are indifferent to which job they are matched to, and likewise within gender, jobs are indifferent to which worker they are matched to.

This implies that if worker $i$ were to match with a different job within the same occupation, we would have $\bar{u}^{i}_j =\bar{u}^{i}_{-j}$, and likewise for job $j$, $\bar{\pi}^{i}_j = \bar{\pi}^{-i}_j $, therefore the pairwise stability inequality holds trivially for observationally equivalent (same $g$ and $o$) candidate matches:

$$\bar{u}^{i}_{j(i)}+ \bar{\pi}^{i(j)}_j = \bar{u}^{i}_{j} + \bar{\pi}^{i}_j \hspace{5mm} \forall i,i(j) \in g \hspace{3mm} \forall j,j(i) \in o$$

Now consider matching worker $i$ to a job in a different occupation. Both workers and jobs choose the occupation or gender that produces the highest payoff for them, given the wage vector. Let the optimal occupation be $o^*$ and optimal gender $g^*$. Therefore we know that for worker $i$

$$\bar{u}^{i}_{j(i)} > \bar{u}^{i}_{j} \hspace{5mm} \forall j(i) \in o^* \text{  and  } \forall j \in o \neq o^*$$

and for job $j$

$$\bar{\pi}^{i(j)}_{j} > \bar{\pi}^{i}_j \hspace{5mm} \forall i(j) \in g^* \text{  and  } \forall i \in g \neq g^*$$

Therefore pairwise stability holds with strict inequality for all candidate matches that are not observationally equivalent (different $g$ or $o$) to the competitive equilibrium.

The second part of pairwise stability is the requirement that the choice payoffs be greater than the outside option payoffs. Recall that the outside option for the worker is remaining unemployed is equal to the idiosyncratic taste for non-employment, $ \bar{u}^i_N = \bar{\eta}^i_N$. The value to the firm of not hiring a worker is simply zero, $\bar{\pi}^j_N = 0$.

%An interesting result from \citeA{Salanie2013a} (p.15) is that although the preference structure may be such that the jobs have an ordering $\eta^i_o$ over workers and workers have an ordering $\xi^g_j$ over jobs, the post-transfer utility will always accumulate such that job $j$ receives $\xi^g_j$ and worker $i$ receives $\eta^i_o$. Intuitively this results from the transfers serving to make workers indifferent over jobs within each type, and jobs indifferent over workers within each type.


Another key aspect of the equilibrium wage vector is that it must be feasible, which in the case of this labor market is equivalent to equating supply and demand at the level of male and female workers and occupations. \citeA{Crawford1981} and \citeA{RothSotomayor} prove the existence of such an equilibrium in a model with transfers. Intuitively, as long as the common component of wage, or $W^g_o$, is free to adjust, supply and demand can adjust until the market clears. The empirical implications of market clearing will be discussed in the empirical section below.

% Roth and Sotomayor chapter 9: we need continuous utilty functions and must assume that some amount of money could induce you to switch jobs. They also assume that the utility functions map from R to R...  this is ok for me since log(W) is normal and therefore infinites support

% Crawford and Knoer also lay out the mechanism for feasible matching.


%\begin{align*}
%u^i_g &= \frac{ u^g_o*W^g_o * \xi^g_j * e^{ \eta^i_o} } {\xi^g_j}    \\
%&= u^g_o*W^g_o * e^{ \eta^i_o} 
%\end{align*}



\clearpage 
\newpage
\subsection{Pan 2010 Tipping Model}\label{sec.Pan}
To illustrate the concept of ``tipping" I refer to the model of \citeA{Pan2010}, illustrated below in Figure \ref{fig:tipping}. In this stylized model, an occupation can hire either male or female workers to fill a fixed number of positions. As more women are hired, fewer men are hired, and the fraction female goes up. Assuming upward sloping supply curves for men and women, wages must go up as more workers of either gender are hired. Therefore, assuming equal productivity, firms continue to hire men or women until wages are equalized, which depends on the shape of the supply curves. Tipping points emerge if one or more of the supply curves are not always upward sloping. In the righthand side of Figure \ref{fig:tipping}, we see the case where the female labor supply curve becomes downward sloping at very low fractions female, illustrating a scenario where women are so strongly averse to entering male dominated occupations that hiring more women actually allows the occupation to pay lower wages to women.

%In the righthand graph, the labor supply of women depends also on the fraction female in the occupation. Women are so strongly averse to entering male dominated occupations that in fact the female labor supply curve becomes downward sloping at very low fractions female. More women are enticed into the occupation even as wages go down. 

In this stylized model, the location of stable and unstable sorting equilibria depends only on the male and female labor supply curves in the occupation, which could depend on how non-wage amenities, wages, and fraction female are valued by men and women.\footnote{In \citeA{Pan2010} the locations of the unstable equilibria, and therefore the tipping points, are fixed to be the same across occupations within blue collar or white collar categories.} Another key factor affecting the gender mix in the occupation is that firms may not have the same willingness to pay for male and female workers. Firms may values workers of one gender less due to productivity or skill differences, or taste-based or statistical discrimination, or devaluation. Such a gap willingness-to-pay on the firm side will drive a wedge between male and female wages and push the mixed-gender equilibrium in an occupation up or down. Thus both supply and demand factors impact the fraction female in an occupation, and also the tipping point or stability of that fraction female, if there are preferences over occupation gender in the labor supply curves.

\begin{figure}[H]
\centering
\caption{Stylized Model of Tipping from Pan 2010}
\label{fig:tipping}
\includegraphics[width=.8\textwidth]{"/Users/Miriam/OneDrive/Box Sync/TYPlocal/output/graphs/Pan_equilibria_both"}
\end{figure}


\subsection{List of occupation Codes} \label{occupations}

Note that not all component occupations exist in all years of data. 

%Teachers, Postsecondary 
%
%\indent{\textit{earth, environmental, and marine science instructors;
%biological science instructors;
%chemistry instructors;
%physics instructors;
%psychology instructors;
%economics instructors;
%history instructors;
%sociology instructors;
%engineering instructors;
%math instructors;
%education instructors;
%law instructors;
%theology instructors;
%home economics instructors;
%humanities profs/instructors, college, nec;
%	subject instructors (hs/college);}}

   
   Adjusters and Investigators
   
   Administrative Support
   
   Agriculture, Forestry and Fishing
   
   Cleaning and Building Service 
   
   Construction and Extraction
   
   Engineers, Architects, and Surveyors
   
   Executive, Administrative, and Managerial
   
   Financial Records Processing 
   
   Food Preparation and Service 
   
   Health Assessment and Treating and Therapists
   
   Health Diagnosing
   
   Health Service
   
   Health Technologists and Technicians
   
   Machine operators, Fabricators, Assemblers, Testers
   
   Mail and Material Distribution
   
   Management Related
   
   Material Moving, Laborers
   
   Math, Computer, and Natural Science
   
   Mechanics and Repairers
   
   Metal, Wood, Plastic, Print, Textile
   
   Miscellaneous Administrative Support
   
   Precision Production
   
   Private Household and Personal Services
   
   Protective Service
   
   Records Processing
   
   Road, Rail and Water Transportation
   
   Sales Representatives, Finance and Business Services
   
   Sales Workers, Retail and Personal Services
   
   Social Scientists, Lawyers, Judges
   
   Social, Recreation, and Religious Workers
   
   Teachers, Except Postsecondary
   
   Teachers, Postsecondary
   
   Technicians except health
   
   Writers, Artists, Entertainers, and Athletes
   
   
 \subsection{Joint Likelihood Tobit Type 5}\label{sec.likelihood}


\begin{align*}
y_{1j}^* &= \widebar{WTP}^F_o - \bar{W}^F_o  - \xi^F_j - (\widebar{WTP}^M_o - \bar{W}^M_o  - \xi^M_j) \\
y_{2j}^* &= \bar{W}^F_o + \bar{\xi}^F_j \\
y_{3j}^* &= \bar{W}^M_o + \bar{\xi}^M_j \\
y_{2j} &= y_{2j}^* \hspace{10mm} &\text{if} \hspace{2mm} y_{1j}^* >0 \\
y_{2j} &= 0 \hspace{10mm} &\text{if} \hspace{2mm} y_{1j}^* \leq 0 \\
y_{3j} &= y_{3j}^* \hspace{10mm} &\text{if} \hspace{2mm} y_{1j}^* \leq 0 \\
y_{3j} &= 0 \hspace{10mm} &\text{if} \hspace{2mm} y_{1j}^* > 0 \\
\end{align*}


Let $f_{1,3}$ be the joint density of $y_{1j}^*$ and $y_{3j}^*$, and likewise $f_{1,2}$.
\begin{align*}
L &= \prod_F \int_{-\infty}^0 f_{1,3}(y_{1j}^*,y_{3j}) dy_{1j}^* \prod_M \int_{0}^{\infty} f_{1,2}(y_{1j}^*,y_{2j}) dy_{1j}^* \\
&= \prod_j Pr(y_{1j}^* \leq 0, y_{3j})^{\mathcal{I}(y_{3j})}*Pr(y_{1j}^* > 0, y_{2j})^{\mathcal{I}(y_{2j})} \\
&= \prod_j (Pr(y_{1j}^* \leq 0 | y_{3j})*Pr(y_{3j}))^{\mathcal{I}(y_{3j})}*(Pr(y_{1j}^* > 0| y_{2j})*Pr(y_{2j}))^{\mathcal{I}(y_{2j})}\\
&= \prod_j (F_{1}(0|y_{3j})*f_3(y_{3j}))^{\mathcal{I}(y_{3j})}*(F_{-1}( 0| y_{2j})*f_2(y_{2j}))^{\mathcal{I}(y_{2j})}\\
\end{align*}

Where $F_1$ is the cdf of $y_{1j}^*$, $F_{-1}$ the cdf of $-y_{1j}^*$, $f_3$ is the pdf of $y_{3j}^*$, and $f_2$ the pdf of $y_{2j}^*$.
\begin{align*}
 y_{2j}=y_{3j}=0 \hspace{10mm} &\text{if} \hspace{5mm} \widebar{WTP}^M_o - \bar{W}^M_o - \bar{\xi}^M_j < 0\\
 &\text{and} \hspace{5mm} \widebar{WTP}^F_o - \bar{W}^F_o -  \bar{\xi}^F_j < 0 \\
\end{align*}

Translating into my notation, if the job is filled, its contribution to the likelihood function takes the form 

% this decomposition is not quite right when I am using the outside option data because I am using law of total probabiltiy
% this method would be right in the case that I maximize conditinal likelihood without outside option data
%\begin{align*}
% LL_{j_{\text{filled}}} &= \prod_{j_{\text{filled}}}  Pr(\text{$j$ hire g},Wage^g_{j},\text{$j$ filled}) \\
% &= \prod_{j_{\text{filled}}}  Pr(\text{$j$ hire g},Wage^g_{j} | \text{$j$ filled})*Pr( \text{$j$ filled}) \\
%  &=  \prod_{j_{\text{filled}}}  Pr(\text{$j$ hire g} | Wage^g_{j}, \text{$j$ filled})*Pr(Wage^g_{j} | \text{$j$ filled})*Pr( \text{$j$ filled}) \\
% % &= \prod_j \frac{Pr(\text{$j$ hire g} | Wage^g_{j}) Pr(Wage^g_{j})}{Pr( \text{$j$ filled})}  \\
%\end{align*}

\begin{align*}
 LL_{j_{\text{filled}}} &= \prod_{j_{\text{filled}}}  Pr(\text{$j$ hire g},Wage^g_{j})*Pr(\text{$j$ filled}) \\
 %&= \prod_{j_{\text{filled}}}  Pr(\text{$j$ hire g},Wage^g_{j} | \text{$j$ filled})*Pr( \text{$j$ filled}) \\
  &=  \prod_{j_{\text{filled}}}  Pr(\text{$j$ hire g} | Wage^g_{j})*Pr(Wage^g_{j})*Pr( \text{$j$ filled}) \\
 % &= \prod_j \frac{Pr(\text{$j$ hire g} | Wage^g_{j}) Pr(Wage^g_{j})}{Pr( \text{$j$ filled})}  \\
\end{align*}

If the job is unfilled, its contribution is
\begin{align*}
 LL_{j_{\text{unfilled}}} &= \prod_{j_{\text{unfilled}}}  Pr(\text{$j$ unfilled}) \\
\end{align*}

$\mathcal{I}( \text{ j unfilled})$ be the indicator function equal to one if the job $j$ is unfilled, and similarly for $\mathcal{I}(\text{$j$ filled})$. Then the total likelihood is given by

\begin{align*}
LL_j  &=  \prod_j ( Pr(\text{$j$ hire g} | Wage^g_{j})*Pr(Wage^g_{j})*Pr( \text{$j$ filled}))^{\mathcal{I}(\text{$j$ filled})}*(Pr(\text{$j$ unfilled}))^{\mathcal{I}(\text{$j$ unfilled})}  \\
 % &= \prod_j \frac{Pr(\text{$j$ hire g} | Wage^g_{j}) Pr(Wage^g_{j})}{Pr( \text{$j$ filled})}  \\
\end{align*}


%$$ LL = \prod_j  Pr(\text{$j$ hire g},Wage^g_{j},\text{$j$ filled}) = Pr(\text{$j$ hire g},Wage^g_{j} | \text{$j$ filled})*Pr( \text{$j$ filled})$$

% With vacancies:
%Pr(choice, wage, notrunc) = Pr(choice,wage | notrunc)*Pr( notrunc) = Pr(choice | notrunc ,wage) * Pr(wage | notrunc) * P(notrunc)

% without vacancies
%=Pr(choice, notrunc | wage) * pr(wage) = Pr(choice | notrunc, wage) * Pr(notrunc | wage) * pr(wage)
%where P(notrunc | wage) is either zero or one if wage > or< pi. discrete jump in likelihood => impose constraints


%$$ LL = \prod_j \frac{Pr(\text{$j$ hire g} | Wage^g_{j}) Pr(Wage^g_{j})}{1-Pr(\text{$j$ unfilled})} $$

$Pr(\text{$j$ unfilled}) $ is the probability that we do not observe a match, which I impute from the JOLTS vacancy data.\footnote{Results do not appear sensitive to imputation method.} This occurs when hiring neither women nor men produces willingness-to-pay higher than cost.

\begin{align*}
 \widebar{WTP}^M_o - \bar{W}^M_o - \bar{\xi}^M_j < 0 \hspace{5mm} \text{and} \hspace{5mm} \widebar{WTP}^F_o - \bar{W}^F_o -  \bar{\xi}^F_j < 0 \\
\end{align*}

The $Pr( \text{$j$ filled}) = 1 - Pr( \text{$j$ unfilled}) $ is then
\begin{align*}
1- Pr(\xi^F_j< -(\widebar{WTP}^F_o - \bar{W}^F_o), \hspace{3mm} \xi^M_j< -(\widebar{WTP}^M_o - \bar{W}^M_o )) & \\
= 1 - \Phi_{0,\sigma^F_{\xi}}(-(\widebar{WTP}^F_o - \bar{W}^F_o)* \Phi_{0,\sigma^M_{\xi}}(-(\widebar{WTP}^M_o - \bar{W}^M_o))
\end{align*}

Let $\Phi_{0,\sigma^g_{\xi}}$ and $\phi_{0,\sigma^g_{\xi}}$ are the cdf and pdf of the normal distribution with location zero and scale $\sigma^g_{\xi}$. Recall that

$$ \widebar{Wage}^g_j = \bar{W}^g_o + \bar{\xi}^g_j $$

Then other terms in the likelihood are as follows:

\begin{align*}
Pr(\text{$j$ hire M} | \widebar{Wage}^M_{j} ) &= \Phi_{0,\sigma^F_{\xi}}( \widebar{WTP}^M_o -  \widebar{Wage}^M_j) - (\widebar{WTP}^F_o - \bar{W}^F_o))  \\
Pr(\widebar{Wage}^M_{j} ) &= \phi_{0,\sigma^M_{\xi}}( \widebar{Wage}^M_o - \bar{W}^M_o)  \\
Pr(\text{$j$ hire F} | \widebar{Wage}^F_{j}) &= \Phi_{0,\sigma^M_{\xi}}( \widebar{WTP}^F_o -  \widebar{Wage}^F_j) - (\widebar{WTP}^M_o - \bar{W}^M_o))  \\
Pr(\widebar{Wage}^F_{j}) &=  \phi_{0,\sigma^F_{\xi}}( \widebar{Wage}^F_j - \bar{W}^F_o)  \\
\end{align*}




%%% THESE gRAPHS USE STARTINg oCCUPATIoN AND 2000-2012 DATA FoR THE SIPP NoT 1960
%\begin{center}
%\clearpage
%\includegraphics[width=.8\textwidth]{"/Users/Miriam/OneDrive/Box Sync/TYPlocal/output/graphs/SIPPlifeinc_age_occ_0"}
%\newline
%\includegraphics[width=.8\textwidth]{"/Users/Miriam/OneDrive/Box Sync/TYPlocal/output/graphs/PSIDlifeinc_age_occ_0"}
%\clearpage
%\includegraphics[width=.8\textwidth]{"/Users/Miriam/OneDrive/Box Sync/TYPlocal/output/graphs/SIPPlifeinc_age_occ_1"}
%\newline
%\includegraphics[width=.8\textwidth]{"/Users/Miriam/OneDrive/Box Sync/TYPlocal/output/graphs/PSIDlifeinc_age_occ_1"}
%\clearpage
%\includegraphics[width=.8\textwidth]{"/Users/Miriam/OneDrive/Box Sync/TYPlocal/output/graphs/SIPPlifeinc_age_occ_2"}
%\newline
%\includegraphics[width=.8\textwidth]{"/Users/Miriam/OneDrive/Box Sync/TYPlocal/output/graphs/PSIDlifeinc_age_occ_2"}
%\clearpage
%\includegraphics[width=.8\textwidth]{"/Users/Miriam/OneDrive/Box Sync/TYPlocal/output/graphs/SIPPlifeinc_age_occ_3"}
%\newline
%\includegraphics[width=.8\textwidth]{"/Users/Miriam/OneDrive/Box Sync/TYPlocal/output/graphs/PSIDlifeinc_age_occ_3"}
%\clearpage
%\includegraphics[width=.8\textwidth]{"/Users/Miriam/OneDrive/Box Sync/TYPlocal/output/graphs/SIPPlifeinc_age_occ_4"}
%\newline
%\includegraphics[width=.8\textwidth]{"/Users/Miriam/OneDrive/Box Sync/TYPlocal/output/graphs/PSIDlifeinc_age_occ_4"}
%\clearpage
%\includegraphics[width=.8\textwidth]{"/Users/Miriam/OneDrive/Box Sync/TYPlocal/output/graphs/SIPPlifeinc_age_occ_5"}
%\newline
%\includegraphics[width=.8\textwidth]{"/Users/Miriam/OneDrive/Box Sync/TYPlocal/output/graphs/PSIDlifeinc_age_occ_5"}
%\clearpage
%\includegraphics[width=.8\textwidth]{"/Users/Miriam/OneDrive/Box Sync/TYPlocal/output/graphs/SIPPlifeinc_age_occ_6"}
%\newline
%\includegraphics[width=.8\textwidth]{"/Users/Miriam/OneDrive/Box Sync/TYPlocal/output/graphs/PSIDlifeinc_age_occ_6"}
%\clearpage
%\end{center}







%\subsection{Plots of Model Estimates over Time}
%\begin{center}
%\includegraphics[width=.4\textwidth]{"/Users/Miriam/OneDrive/Box Sync/TYPlocal/output/graphs/Wpi_results_2"}
%\includegraphics[width=.4\textwidth]{"/Users/Miriam/OneDrive/Box Sync/TYPlocal/output/graphs/Wpi_results_3"}
%\newline
%\includegraphics[width=.4\textwidth]{"/Users/Miriam/OneDrive/Box Sync/TYPlocal/output/graphs/Wpi_results_4"}
%\includegraphics[width=.4\textwidth]{"/Users/Miriam/OneDrive/Box Sync/TYPlocal/output/graphs/Wpi_results_5"}
%\newline
%\includegraphics[width=.4\textwidth]{"/Users/Miriam/OneDrive/Box Sync/TYPlocal/output/graphs/Wpi_results_6"}
%\includegraphics[width=.4\textwidth]{"/Users/Miriam/OneDrive/Box Sync/TYPlocal/output/graphs/Wpi_results_7"}
%\newline
%\includegraphics[width=.4\textwidth]{"/Users/Miriam/OneDrive/Box Sync/TYPlocal/output/graphs/Wpi_results_8"}
%\includegraphics[width=.4\textwidth]{"/Users/Miriam/OneDrive/Box Sync/TYPlocal/output/graphs/Wpi_results_9"}
%\end{center}
%\clearpage
%
%\begin{center}
%\includegraphics[width=.4\textwidth]{"/Users/Miriam/OneDrive/Box Sync/TYPlocal/output/graphs/Wpi_results_10"}
%\includegraphics[width=.4\textwidth]{"/Users/Miriam/OneDrive/Box Sync/TYPlocal/output/graphs/Wpi_results_11"}
%\newline
%\includegraphics[width=.4\textwidth]{"/Users/Miriam/OneDrive/Box Sync/TYPlocal/output/graphs/Wpi_results_12"}
%\includegraphics[width=.4\textwidth]{"/Users/Miriam/OneDrive/Box Sync/TYPlocal/output/graphs/Wpi_results_13"}
%\newline
%\includegraphics[width=.4\textwidth]{"/Users/Miriam/OneDrive/Box Sync/TYPlocal/output/graphs/Wpi_results_14"}
%\includegraphics[width=.4\textwidth]{"/Users/Miriam/OneDrive/Box Sync/TYPlocal/output/graphs/Wpi_results_15"}
%\newline
%\includegraphics[width=.4\textwidth]{"/Users/Miriam/OneDrive/Box Sync/TYPlocal/output/graphs/Wpi_results_16"}
%\includegraphics[width=.4\textwidth]{"/Users/Miriam/OneDrive/Box Sync/TYPlocal/output/graphs/Wpi_results_17"}
%\end{center}
%\clearpage
%
%\begin{center}
%\includegraphics[width=.4\textwidth]{"/Users/Miriam/OneDrive/Box Sync/TYPlocal/output/graphs/Wpi_results_18"}
%\includegraphics[width=.4\textwidth]{"/Users/Miriam/OneDrive/Box Sync/TYPlocal/output/graphs/Wpi_results_19"}
%\newline
%\includegraphics[width=.4\textwidth]{"/Users/Miriam/OneDrive/Box Sync/TYPlocal/output/graphs/Wpi_results_20"}
%\includegraphics[width=.4\textwidth]{"/Users/Miriam/OneDrive/Box Sync/TYPlocal/output/graphs/Wpi_results_21"}
%\newline
%\includegraphics[width=.4\textwidth]{"/Users/Miriam/OneDrive/Box Sync/TYPlocal/output/graphs/Wpi_results_22"}
%\includegraphics[width=.4\textwidth]{"/Users/Miriam/OneDrive/Box Sync/TYPlocal/output/graphs/Wpi_results_23"}
%\newline
%\includegraphics[width=.4\textwidth]{"/Users/Miriam/OneDrive/Box Sync/TYPlocal/output/graphs/Wpi_results_24"}
%\includegraphics[width=.4\textwidth]{"/Users/Miriam/OneDrive/Box Sync/TYPlocal/output/graphs/Wpi_results_25"}
%\end{center}
%\clearpage
%
%\begin{center}
%\includegraphics[width=.4\textwidth]{"/Users/Miriam/OneDrive/Box Sync/TYPlocal/output/graphs/Wpi_results_26"}
%\includegraphics[width=.4\textwidth]{"/Users/Miriam/OneDrive/Box Sync/TYPlocal/output/graphs/Wpi_results_27"}
%\newline
%\includegraphics[width=.4\textwidth]{"/Users/Miriam/OneDrive/Box Sync/TYPlocal/output/graphs/Wpi_results_28"}
%\includegraphics[width=.4\textwidth]{"/Users/Miriam/OneDrive/Box Sync/TYPlocal/output/graphs/Wpi_results_29"}
%\newline
%\includegraphics[width=.4\textwidth]{"/Users/Miriam/OneDrive/Box Sync/TYPlocal/output/graphs/Wpi_results_30"}
%\includegraphics[width=.4\textwidth]{"/Users/Miriam/OneDrive/Box Sync/TYPlocal/output/graphs/Wpi_results_31"}
%\newline
%\includegraphics[width=.4\textwidth]{"/Users/Miriam/OneDrive/Box Sync/TYPlocal/output/graphs/Wpi_results_31"}
%\includegraphics[width=.4\textwidth]{"/Users/Miriam/OneDrive/Box Sync/TYPlocal/output/graphs/Wpi_results_32"}
%\end{center}
%\clearpage
%
%\begin{center}
%\includegraphics[width=.4\textwidth]{"/Users/Miriam/OneDrive/Box Sync/TYPlocal/output/graphs/Wpi_results_33"}
%\includegraphics[width=.4\textwidth]{"/Users/Miriam/OneDrive/Box Sync/TYPlocal/output/graphs/Wpi_results_34"}
%\newline
%\includegraphics[width=.4\textwidth]{"/Users/Miriam/OneDrive/Box Sync/TYPlocal/output/graphs/Wpi_results_35"}
%\end{center}
%\clearpage


%\subsection{Static Worker Utility Decomposition}
%
%\input{Estimation_utility_year_1960}
%\input{Estimation_utility_year_1960IV}
%\input{Estimation_utility_year_1960IV_nocont}
%\clearpage
%\input{Estimation_utility_year_1970}
%\input{Estimation_utility_year_1970IV}
%\input{Estimation_utility_year_1970IV_nocont}
%\clearpage
%\input{Estimation_utility_year_1980}
%\input{Estimation_utility_year_1980IV}
%\input{Estimation_utility_year_1980IV_nocont}
%\clearpage
%\input{Estimation_utility_year_1990}
%\input{Estimation_utility_year_1990IV}
%\input{Estimation_utility_year_1990IV_nocont}
%\clearpage
%\input{Estimation_utility_year_2000}
%\input{Estimation_utility_year_2000IV}
%\input{Estimation_utility_year_2000IV_nocont}
%\clearpage
%\input{Estimation_utility_year_2012}
%\input{Estimation_utility_year_2012IV}
%\input{Estimation_utility_year_2012IV_nocont}



%\subsection{Reservation Wage Distributions}
%
%\begin{center}
%%\includegraphics[width=.9\textwidth]{offerwages1reg1trunc}
%\includegraphics[width=.9\textwidth]{offerwages_occfinal30reg1trunc}
%\end{center}
%\begin{center}
%%\includegraphics[width=.9\textwidth]{offerwages1reg1trunc}
%\includegraphics[width=.9\textwidth]{offerwages_occfinal1reg1trunc}
%\end{center}
%\begin{center}
%%\includegraphics[width=.9\textwidth]{offerwages1reg1trunc}
%\includegraphics[width=.9\textwidth]{offerwages_occfinal2reg1trunc}
%\end{center}
%\begin{center}
%%\includegraphics[width=.9\textwidth]{offerwages1reg1trunc}
%\includegraphics[width=.9\textwidth]{offerwages_occfinal3reg1trunc}
%\end{center}
%\begin{center}
%%\includegraphics[width=.9\textwidth]{offerwages1reg1trunc}
%\includegraphics[width=.9\textwidth]{offerwages_occfinal4reg1trunc}
%\end{center}
%\begin{center}
%%\includegraphics[width=.9\textwidth]{offerwages1reg1trunc}
%\includegraphics[width=.9\textwidth]{offerwages_occfinal5reg1trunc}
%\end{center}
%\begin{center}
%%\includegraphics[width=.9\textwidth]{offerwages1reg1trunc}
%\includegraphics[width=.9\textwidth]{offerwages_occfinal6reg1trunc}
%\end{center}
%\begin{center}
%%\includegraphics[width=.9\textwidth]{offerwages1reg1trunc}
%\includegraphics[width=.9\textwidth]{offerwages_occfinal7reg1trunc}
%\end{center}
%\begin{center}
%%\includegraphics[width=.9\textwidth]{offerwages1reg1trunc}
%\includegraphics[width=.9\textwidth]{offerwages_occfinal8reg1trunc}
%\end{center}
%\begin{center}
%%\includegraphics[width=.9\textwidth]{offerwages1reg1trunc}
%\includegraphics[width=.9\textwidth]{offerwages_occfinal9reg1trunc}
%\end{center}
%\begin{center}
%%\includegraphics[width=.9\textwidth]{offerwages1reg1trunc}
%\includegraphics[width=.9\textwidth]{offerwages_occfinal10reg1trunc}
%\end{center}




%
%
%\subsection{Distribution of Maximum of Extreme Value Distributions}
%
%The distributions of the $\epsilon$ given they are observed at the maximized utility only is the same as the unconditional distributions \cite{DePalma2007}, therefore we can estimate the scale parameter  $\sigma^g_{\xi}$ using the individual wage data as suggested in \citeA{Salanie2014}. To see this, follow the original proof of from \citeA{Domenchich1975} page 64, outlined here.
%
%First, note that 
%
%$$Pr(\Pi^g_o + \xi^g_j \leq Z) = Pr( \xi^g_j \leq Z - \Pi^g_o )=exp(-e^{-\frac{1}{\sigma^g_{\xi}} (Z - \Pi_{X_jY} + \sigma^g_{\xi} )})$$
%
%So we can define the CDF associated with alternative $X_k$ as the $F_{X_k}$, the CDF of $\xi^g_j$ with mean parameter shifted by $\Pi_{X_kY}$.
%
%$$F_{X_k} = exp(-e^{-\frac{1}{\sigma^g_{\xi}} (Z - \Pi_{X_kY} + \sigma^g_{\xi} )})$$
%
%Then we can express the probability of choosing a worker of type $X_1$ over a worker of type $X_2$ as 
%\begin{align*}
% Pr(\Pi_{X_2Y} + \epsilon_{X_2j} \leq \Pi_{X_1Y} + \epsilon_{X_1j}) &= F_{X_2}(  \Pi_{X_1Y} + \epsilon_{X_1j}) \\
%\end{align*}
%
%
%Then the probability that a number $Z$ is the maximum over all choices in $X$, indexed $k=1,...,n_k$:
%\begin{align*}
%Pr( \max_k \Pi_{X_kY} + \epsilon_{X_kj} \leq Z) &=  \prod_{k} F_{X_k}(  Z) \\
%&= \prod_{k} exp(-e^{-\frac{1}{\sigma^g_{\xi}} (Z - \Pi_{X_kY} + \sigma^g_{\xi} )}) \\
%&= exp(- \sum_{k} e^{-\frac{1}{\sigma^g_{\xi}} (Z - \Pi_{X_kY} + \sigma^g_{\xi} )}) \\
%&= exp(- e^{-\frac{1}{\sigma^g_{\xi}} (Z + \sigma^g_{\xi} )}  \sum_{k}  e^{\frac{\Pi_{X_jY}}{\sigma^g_{\xi}} }) \\
%\end{align*}
%
%Now we have the CDF of the maximum value of the random variable, it remains to recognize the scale and location parameters. Let the location parameter of the distribution of the maximum value be $\mu$:
%
%\begin{align*}
%e^{\frac{\mu}{\sigma^g_{\xi}}} = \sum_{k}  e^{\frac{\Pi_{X_jY}}{\sigma^g_{\xi}} }
%\end{align*}
%
%Then we have $\mu =  \sigma^g_{\xi} ln \sum_{k} e^{\frac{\Pi_{X_jY}}{\sigma^g_{\xi}}}$, or $\sigma^g_{\xi}$ times the log inclusive value. Thus the CDF of the maximized random variable is
%
%$$exp(- e^{ - \frac{1}{\sigma^g_{\xi}} (Z - \mu + \sigma^g_{\xi} )}  ) $$
%
%Therefore the scale parameter of the maximum value remains $\sigma^g_{\xi}$, the scale parameter of $\xi^g_j$.
%

%\subsection{Industry Wages by Sex and Year}
%\includegraphics[width=.8\textwidth]{"/Users/Miriam/OneDrive/Box Sync/TYPlocal/output/graphs/Industry_sex1"}
%\newline
%\includegraphics[width=.8\textwidth]{"/Users/Miriam/OneDrive/Box Sync/TYPlocal/output/graphs/Industry_sex2"}
%
%\newpage
%\includegraphics[width=.8\textwidth]{"/Users/Miriam/OneDrive/Box Sync/TYPlocal/output/graphs/Industry_lifeinc_sex1"}
%\newline
%\includegraphics[width=.8\textwidth]{"/Users/Miriam/OneDrive/Box Sync/TYPlocal/output/graphs/Industry_lifeinc_sex2"}


%
%
%\begin{center}
%\begin{scriptsize}
%\input{logit_female_nokid_v2_occ1990}
%\end{scriptsize}
%\end{center}
%
%\begin{center}
%\begin{scriptsize}
%\input{logit_female_kid_v2_occ1990}
%\end{scriptsize}
%\end{center}
%
%\begin{center}
%\begin{scriptsize}
%\input{logit_male_nokid_v2_occ1990}
%\end{scriptsize}
%\end{center}
%
%\begin{center}
%\begin{scriptsize}
%\input{logit_male_kid_v2_occ1990}
%\end{scriptsize}
%\end{center}
%
%
%\clearpage
%\subsection{Linear in fraction female}
%\begin{center}
%\includegraphics[width=.8\textwidth]{utility_sexratio_3_line}
%\includegraphics[width=.8\textwidth]{utility_sexratio_4_line}
%\includegraphics[width=.8\textwidth]{utility_sexratio_1_line}
%\includegraphics[width=.8\textwidth]{utility_sexratio_2_line}
%\end{center}
%
%
%
%\clearpage
%\subsection{$F+F^2+F^3$}
%\begin{center}
%\includegraphics[width=.8\textwidth]{utility_sexratio_3_cube}
%\includegraphics[width=.8\textwidth]{utility_sexratio_4_cube}
%\includegraphics[width=.8\textwidth]{utility_sexratio_1_cube}
%\includegraphics[width=.8\textwidth]{utility_sexratio_2_cube}
%\end{center}
%
%\clearpage
%\subsection{$F+F^2+F^3+F^4$}
%\begin{center}
%\includegraphics[width=.8\textwidth]{utility_sexratio_34}
%\includegraphics[width=.8\textwidth]{utility_sexratio_44}
%\includegraphics[width=.8\textwidth]{utility_sexratio_14}
%\includegraphics[width=.8\textwidth]{utility_sexratio_24}
%\end{center}
%
%\clearpage
%\section{Preliminary First Stage: 80 occupations}
%
%\begin{center}
%\begin{scriptsize}
%\input{logit_male_nokid_occ80}
%\end{scriptsize}
%\end{center}
%
%\begin{center}
%\begin{scriptsize}
%\input{logit_male_kid_occ80}
%\end{scriptsize}
%\end{center}
%
%\begin{center}
%\begin{scriptsize}
%\input{logit_female_nokid_occ80}
%\end{scriptsize}
%\end{center}
%
%\begin{center}
%\begin{scriptsize}
%\input{logit_female_kid_occ80}
%\end{scriptsize}
%\end{center}


\newpage

\bibliographystyle{apacite}
\bibliography{library} 





\end{document}