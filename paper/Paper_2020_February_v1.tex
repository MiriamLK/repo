\documentclass[12pt]{article}

\usepackage{amssymb,amsmath,amsthm}
\usepackage{mathrsfs}
\usepackage{bbm}
\usepackage{fancyhdr}
\usepackage[mmddyyyy,hhmmss]{datetime}
\usepackage{pstricks, sgamevar, egameps}
\usepackage{chapterbib}
\usepackage{graphicx}
\usepackage{setspace}
\usepackage{cases}
\usepackage{url}
\usepackage{placeins}
\usepackage{longtable}
\usepackage{tabularx} 
\usepackage{caption}
\usepackage{threeparttable}
\usepackage{booktabs}
\usepackage{outlines}
\usepackage{float}
\usepackage{mathabx}
\usepackage{setspace}
\usepackage[breaklinks,hidelinks]{hyperref}
\usepackage{authblk}

\usepackage{apacite}
\doublespacing

\newtheorem{assumption}{Assumption}
\newtheorem{definition}{Definition}

\DeclareMathOperator*{\argmin}{arg\,min}
\DeclareMathOperator*{\argmax}{arg\,max}

\topmargin=-1in      
\evensidemargin=0in     
\oddsidemargin=0in      
\textwidth=6.5in        
\textheight=9.0in       
\headsep=0.25in

%\renewcommand{\baselinestretch}{2.0}

\usepackage[margin = 1in]{geometry}

\newcommand*{\graphicPath}{"/Users/Miriam/OneDrive/Box Sync/TYPlocal/output/graphs/matlabwages/"}% 
\newcommand*{\TablePath}{"/Users/Miriam/OneDrive/Box Sync/TYPlocal/output/tables/"}% 
\graphicspath{{\graphicPath}}
\makeatletter
\def\input@path{{\TablePath}}
\makeatother

%\title{Gender Segregation in Occupations: Preferences or Homophily?}
%\title{Equal Pay for Equal Work: Unintended Consequences in the Presence of Homophily}
%\title{Gender Segregation in Occupations: The Consequences of Equal Pay Laws and Homophily}
%\title{A Unique Equilibrium in Occupation Gender Composition}
\title{Occupation Gender Segregation: Empirical Evidence from a Matching Model with Transfers }
%\title{Gender Composition of Occupations in the Presence of Homophily}
\author{Miriam Larson-Koester\footnote{Federal Trade Commission, 600 Pennsylvania Ave. NW, Washington, DC 20580, mlarsonkoester@ftc.gov. The views expressed are those of the author and do not necessarily reflect those of the Federal Trade Commission. I am grateful to Francesca Molinari, Richard Mansfield and Victoria Prowse for their advice and support. For helpful discussions and insights I thank Francine Blau, Panle Barwick, Mallika Thomas, Larry Blume, JF Houde, Yang Zhang, Angela Cools, Jorgen Harris, Caroline Walker, Devesh Raval, James Thomas, Nathan Wilson and numerous seminar participants. All errors are mine.}}
%\affil[1]{Department of Economics, Cornell University}



%\date{Updated \today \\
%\textit{for latest version click} \href{https://drive.google.com/file/d/0B6RyLRYAyJn_U0I0RkU5NkZhME0/view?usp=sharing}{\underline{here}}}

%This research was made possible through the use of Cornell University's Economics Compute Cluster organization, which was partially funded through NSF grant \#0922005.
 
\begin{document}

\pagenumbering{roman}
 
\maketitle

%\begin{center}
%\large{For latest version click \href{https://drive.google.com/file/d/0B6RyLRYAyJn_U0I0RkU5NkZhME0/view?usp=sharing}{\underline{here}}.}
%\end{center}


\abstract{Women have increased their labor force participation dramatically since 1960, but remain concentrated in certain occupations. I examine whether this concentration reflects workers' preference to work with their own gender as opposed to other worker, or firm, preferences. I build a model of labor supply and demand in which firms maximize profit over the gender and wages of their employees, and workers maximize utility over occupation, wage, and the fraction female in their occupation. Using a Bartik instrumental variables strategy, I find that women strongly prefer to enter into female-dominated occupations, but men show no evidence of gender preference. Given these estimates, equilibrium simulations indicate that equal pay for equal work laws could in fact increase segregation by preventing employers from compensating for a gender preference. }

%Counterintuitively, my model implies that several policies to promote gender equity might not be effective, and might even backfire, given the preference over gender composition that I estimate. Temporarily encouraging workers to enter certain occupations has not long run impact on segregation, and equal pay for equal work laws could in fact increase both segregation and the wage gap. 

%My model predicts that the gender wage gap is 18\% higher relative to a counterfactual with no preferences over occupation gender.

%Despite this preference, I find that each occupation tends towards a unique fraction female regardless of initial conditions. However, fixing wages, such as under equal pay for equal work laws, means that many occupations could tend towards either male or female depending on initial conditions.

%\newpage
%\tableofcontents

\newpage
\section{Introduction}
\pagenumbering{arabic}

%%%%%%%%%%%%%%%%%%%%%%%%%%%%%%%%%%%%%%%%%%%%%%%%%%%%%%%%%
%(ii) why is it hard to answer, 
%(iii) how does the paper answer the question, 
%(iv) what it novel about that method/data used to answer the question, and 
%%%%%%%%%%%%%%%%%%%%%%%%%%%%%%%%%%%%%%%%%%%%%%%%%%%%%%%%%

%Women's labor force participation has skyrocketed in the U.S. since 1960, but men and women still go into very different occupations. 

As of 2009, approximately 50\% of women would need to change occupation in order to have an equal number of men and women in every occupation, and this gap is unlikely to close soon \cite{Blau2013}. This segregation is also a major driver of the gender pay gap, with women more concentrated in lower paying occupations \cite{Blau2017}. In this paper I explore one possible driver of segregation and the gender wage gap: workers preferring to work with their own gender (homophily).

%Policies such as encouraging women to enter STEM, and enforcing equal pay for equal work, are meant to address these gaps. In this paper I explore the equilibrium consequences of such policies when workers prefer to work with coworkers of the same gender (homophily), and wage gaps emerge partially as compensating differentials for this preference.

%Recent literature has identified some contributing causes of segregation: preferences over amenities (\citeA{Olivieri2014a}, \citeA{Wiswall2013})\footnote{For overview see \citeA{ Bertrand2011} and \citeA{Cortes2017a}.}, productivity and skill differences (\citeA{Baker2015}), and occupational and educational barriers (\citeA{Hsieh2016}). I study another possible cause of the persistence of segregation: workers valuing the gender composition of occupations. Previous literature such as \citeA{Usui2008, Lordan2015} has looked at whether women prefer to work with women using job satisfaction surveys and employment transitions, but I am the first to quantify the impact of homophily on overall segregation and wage patterns.

I build an empirical model to separately identify homophily from workers' preferences for occupations and firms' preferences for workers. I use my model to quantify the impact of homophily on observed segregation and wage gaps, and to simulate the equilibrium consequences of equal pay for equal work and encouraging entry into occupations in the presence of homophily. Understanding homophily is particularly important for gender equity policy because it could prolong the impact of historical preferences or discrimination, for example if nursing is only female because it was historically female.


%Taking homophily into account is important when examining gender equity policies, such as equal pay for equal work and encouraging entry into occupations.could prolong the impact of historical preferences or discrimination. 
%In the absence of intervention, workers may be forced to choose between a job that is a better match, and one that contains sufficient members of their own gender. 
%This could invite a role for gender equity policies to reduce inefficient mismatch in the labor market. 

%But many gender equity policies, such as equal pay for equal work and encouraging entry into occupations, are targeting equilibrium outcomes rather than fundamentals. To predict the consequences of such policies we need to know the fundamentals: workers' preferences for occupations and who they work with, as well as firms' preferences for workers. 

%I distinguish preferences over the gender composition of occupations from two other possible causes of persistent segregation: worker preferences for occupations, and firm preferences for male or female workers.

%If workers value the gender composition of occupations, then occupational sorting by gender may be largely explained by historical barriers, such that without intervention workers may be forced to choose between an occupation that is a better match and one that contains sufficient members of the same gender. However if segregation is the result of other preferences, then intervention to change the gender composition of occupations on its own is likely inappropriate. 

%(ii) why is it hard to answer, 
% There are two main identification challenges. I can't just look at shares. move these details to previous paragraph or cut?
%There are two main identification challenges to estimating a preference over occupation fraction female: separating firm and worker preferences, and separating worker preferences for occupations from preferences for the gender of coworkers.
%It is a difficult empirical problem to identify homophily. The number of women observed in an occupation reflects both firm preferences over gender and worker preferences over occupations, including homophily. 

To identify homophily I must first separately identify worker and firm preferences using data on shares and wages, which are equilibrium outcomes. To do so I assume that firms are not able to observe the productivity of individual workers beyond their gender. Then, the highest wages paid in an occupation help identify differences in firms' willingness-to-pay by gender, while differences in lowest wages workers are willing to accept help identify gender differences in worker preferences. The second challenge is that there may be unobserved changes to occupations that are correlated with changes to the gender composition. To distinguish these stories, I use the fact that occupations exist in different industries. For example, suppose that accounting in the manufacturing industry is more male-dominated. Then as manufacturing declines, workers may be less likely to view accounting as a male-dominated occupation. I assume that wage and gender composition are the only attributes of occupations that are affected by changes to industries.

%For example, suppose that some accountants work in the manufacturing industry, and this industry is more male dominated. Then if manufacturing declines as an industry, we would expect accounting to become more female dominated for reasons unrelated to worker preferences for accounting.

%I assume that changes in industries can affect occupation wage and gender composition without affecting unobserved occupation amenities. 



% I assume that changes in industry sizes, wages, and gender composition, are not correlated with unobserved changes in occupation amenities. 

%I use the fact that occupations exist in different industries, and industries experience growth and changes to the fraction female that may be unrelated to labor supply. For example, suppose that some accountants work in the manufacturing industry, and this industry is more male dominated. Then if manufacturing declines as an industry, we would expect accounting to become more female dominated for reasons unrelated to worker preferences for accounting.

% rise of health industry
% network/role model story, now I see my aunt in healthcare accounting, think it's female

%A key innovation of this paper is to use the wage distribution to separately identify worker and firm preferences. Intuitively, if workers really like an occupation, then they are more willing to accept jobs in that occupation for low wages. Simultaneously, if firms view workers of a given gender as very productive, then they are willing to hire workers of that gender at high wages. To leverage this intuition for identification, I assume that firms are not able to observe the productivity of individual workers beyond their gender, which allows me to focus on differences across gender as opposed to within gender.

%This last assumption may bias parameter estimates, but is orthogonal to the predictions of the model.

%The shape of the observed wage distribution can be used to back out firm preferences, and the wage offer distribution, an equilibrium object that reflects both firm and worker preferences. 

% Simultaneously, a long and thick right tail implies that firms have a high willingness-to-pay, since even the most unattractive jobs, who have to pay the highest wages, are willing to hire a worker of a given gender in that occupation.

%Although the focus of the paper is overall gaps between male and female workers, wages vary a lot at the individual level as a result of unobserved worker and job heterogeneity. This is both an opportunity to learn from more variation, and a source of measurement error.

%In order to focus on gaps between male and female workers, I assume that workers are equally productive within gender and occupation, or that if individual productivity differences do exist they are unobservable to firms. Thus any wage gaps across gender will reflect the expected productivity of workers of that gender and selection effects across jobs within occupation. Both unobserved worker heterogeneity and unobserved job heterogeneity may cause measurement error in my estimates of overall gender gaps.



%The question of how firms select individual workers, though important, is orthogonal to my question of male vs. female workers. 

%In order to interpret wages, I need to shut down some forms of unobserved heterogeneity. \citeA{Hsieh2016} shut down unobserved job amenity heterogeneity, but since compensating differentials are a key component of my paper and the question of how firms select individual workers is orthogonal to my question, I instead shut down unobserved worker productivity heterogeneity, meaning that firms cannot distinguish workers within gender and occupation. 

%Leveraging this intuition, the shape of the observed wage distribution can be used to back out firm preferences, and the wage offer distribution, an equilibrium object that reflects both firm and worker preferences. An observed wage distribution centered near zero implies that workers of that gender really like that occupation and are more willing to accept low wages. Simultaneously, a long and thick right tail implies that firms have a high willingness-to-pay, since even the most unattractive jobs, who have to pay the highest wages, are willing to hire a worker of a given gender in that occupation.

%To do so, I assume that workers are equally productive within gender and occupation, or that if such differences do exist they are unobservable to firms. The question of how firms select individual workers, though important, is orthogonal to my question of male vs. female workers. 

%I also assume that wage offers are lognormally distributed, and only those wage offers that are accepted by both firms and workers are observed. 



%\footnote{Unobserved productivity of individual workers could be included in theory, but would result in very weak identification and likely computational intractability because it would involve joint estimation of the model and integration over multiple sources of unobserved heterogeneity to interpret the wage distribution.} 

%Although the estimation strategy in this paper was developed specifically for segregation in the labor market, it could be used in any context in which a price mechanism clears a two-sided market.




%if accountants can work in manufacturing or service sectors, and accounting in manufacturing is more male dominated, as manufacturing shrinks over time, we would expect accounting to become more female dominated for reasons unrelated to worker preferences. 

%The key identifying assumptions are that workers' preferences for industries do not vary over time, and that changes to occupation preferences are not correlated across occupations.


%%%%%%% MOVE SOME OF THIS TO TEXT BELOW???
% eg. two stage, myopia vs. switching costs, 
%For tractability I treat occupation choice as a static choice. Workers choose occupation once at the beginning of their career, based on the contemporary characteristics of the occupations, including fraction female. Therefore, the fraction female in each occupation only updates across cohorts of workers.\footnote{Allowing workers to choose occupation myopically more than once during their lifetime, would lead to faster changes in occupation fraction females, but not otherwise change the dynamics of the model.} Lifetime wages are set to clear the market in static equilibrium for each cohort of men and women, which requires a large market assumption (continuum of workers and jobs of each type) to guarantee a unique wage equilibrium.

%The data moments to identify the model are the shares of male and female workers by gender from the 1960-2000 U.S. Censuses and 2012 3-year ACS, and estimates of lifetime income by occupation, gender, and year constructed using income quantiles from the Census data combined with transition rates from the SIPP 2004 and 2008 panels. I take a two-step estimation approach, first estimating the firm side using maximum likelihood, then the worker side using instrumental variables regression, taking the wage offer distribution estimated on the firm side as data. The structure of the model allows me to simulate transition paths in the fraction female by cohort and determine there is path dependency, all while solving for equilibrium wages and fixing other firm and worker preferences.
 
 


%%%%%%%%%%%%%%%%%%%%%%%%%%%%%%%%
% Talk about firm vs. worker preferences here???
%In the first stage of estimation, I find that firms are often willing to pay much more to hire male than female workers, but the degree to which this is true varies greatly across occupation. This gap may reflect discrimination, differences in labor force attachment over the lifetime, skill differences, or job sorting not captured in my occupation categories. The gap in the willingness to pay for female vs. male workers is lower in female dominated occupations. This might be because women are selecting into occupations in which they are more valued or have comparative advantage, or because the way work is valued might vary with the fraction female in the occupation \cite{Levanon2009, Harris2018}, or some combination of both.


% *cite tipping lit more thoroughly? CONSISTENT WITH THESE STYLIZED FACTS: segregation, wage gaps. ALSO DYNAMICS: tipping and wage changes when women enter.
% then talk about multiple steady states and equal pay laws, why my model is uniquely suited to that.

I find that women care strongly about the number of women in an occupation, but find no evidence that men care about the number of women, consistent with recent survey evidence \cite{Delfino2019}. The point estimate in my preferred specification is very high, with an occupation moving from 25\% to 75\% female being equivalent to an extra \$3 million in lifetime income for a woman. Without this preference for higher fraction female, segregation would be much lower, with only 24\% as opposed to 41\% of workers having to change occupation for all occupations to be 50\% female. I also find women are paid on average \$275,000 less over the course of their lifetimes because of compensating differentials from the gender preference, with women in highly female-dominated occupations such as nursing losing the most income.

%This preference captures not only direct preference over the gender of coworkers, but also any changes to occupations themselves with a higher fraction female.

%I find that compensating differentials explain 18\% of the difference in lifetime income between men and women, with women in highly female dominated occupations losing the most income.
%This number includes any changes that occur to the occupation itself as a result of the change in the fraction female.
%In the status quo almost twice as many workers, or 47\%, would have to change occupation to achieve parity.



%For example, if it were not for womens' preference to work with women, women in the female-dominated fields of ``Health Service Occupations" and ``Health Assessment and Treating" would earn just as much as men.

%The gender preference leads to more segregated outcomes. With no gender preference, the model predicts no occupations with fraction female greater than 70\% or less than 10\%, and a Duncan segregation index of 24\% in 2012, meaning 24\% of male or female workers would have to change occupations to make all occupations 50\% female. With the gender preference, the Duncan index is predicted to be almost twice as high at 47\%.\footnote{The Duncan index in my observed 2012 data is 41\%.}

%Although the gender preference I estimate is very strong, it is largely mitigated by compensating differentials. In a model without endogenous wages, all occupations eventually converge to either 0\% or 100\% female. 

%The gender preference also affects the dynamics of sorting and wages in ways that are consistent with stylized facts from the literature. 

My equilibrium model with estimated gender preference is able to explain multiple stylized facts in the literature. First, I find that women are willing to accept lower wages in an occupation as it becomes more female, and firms only hire men whose preference for that occupation causes them to also accept these lower wages. This is consistent with previous literature which has found that as women enter an occupation, wages go down for both men and women in that field \cite{Levanon2009, Harris2018}. Second, I find that the gender preference causes more dramatic movement in the fraction female through a feedback loop in response to changes in worker and firm preferences. This is one possible mechanism for ``tipping", or rapid movement from male to female, which was found in occupations by \citeA{Pan2010}.\footnote{``Tipping" has also been found in racial segregation by neighborhoods \cite{Card2008} and racial composition of schools \cite{Caetano2017}. Tipping can refer to any rapid change in composition, or only rapid changes caused by moving between multiple equilibria. Here I find unique equilibria in each occupation, so I use tipping to mean any rapid changes.} 



%My model also predicts that labor supply or demand shocks that affect the fraction female are reinforced by a feedback loop from the gender preference. 
%An illustrative case study is insurance adjustors. My model predicts that, without a gender preference, as barriers to women in this occupation fell, the fraction female would have risen to only 50\% female from a starting point of 20\% female in 1960, rather than to the observed over 70\% female. This phenomenon is referred to as ``tipping" and documented in U.S. occupations by
% See Appendix \ref{sec.Pan} for an explanation of tipping in occupations.

%My model matches the tipping of insurance adjustors documented by \citeA{Pan2010} through an increase in the willingness-to-pay of firms for female insurance adjustors, accompanied by relatively stagnant female reservation wages, caused by the compensating differentials.

%I find that more women began to become insurance adjusters because firms' demand for women in this occupation rose. Then as more women entered, the occupation became more attractive to women. This in turn made women cheaper to hire, which increased labor demand for female insurance adjusters, providing further reinforcement of the feminization of the occupation. 

% going from 25% to 75% female is equivalent to about $3 million in additional lifetime income
% going from 0% to 100% female is about equivalent to an extra 6 million dollars in lifetime income... yes...
% and yet I still don't see multiple equilibria???? % I cannot rule out such a strong preference given the imprecision of my preferred specification, but such a strong preference seems economically implausible given the already large magnitude of my preferred specification.

I use my estimated model to find out if the fraction female in occupations depends on initial conditions. To do so I simulate the decisions of successive cohorts of workers who overlap in the labor market starting from various initial sorting patterns. I find that each occupation tends towards a unique steady state in the fraction female regardless of initial gender composition. For example, my model predicts that nursing would be female-dominated regardless of whether there were norms or barriers that led it to be female-dominated in 1960. This implies that a policy of temporarily pushing more men or women into certain occupations will not lead to convergence to new, and possibly better, sorting patterns, because the steady state sorting pattern is unique.

%Another possible implication of the gender preference is historical path dependence, meaning that occupations could remain male or female dominated based on past preferences and norms even if current worker and firm preferences might imply a different gender composition. To find out if any occupations are stuck at the ``wrong" fraction female, I find that in each occupation there is only one fraction female at which the number of women who enter the occupation is the same as the number of women who retire. 

However, if wages are not allowed to adjust to compensate workers for changes in the fraction female, then in some cases segregation \textit{does} depend on initial conditions. I simulate the consequences of one such policy, equal pay for equal work, in the context of the large estimated preference for women to work with women. The equal pay policy means that women cannot be compensated for working in male-dominated occupations. As a result, firms tend to hire either only men or women, and in many occupations there are multiple steady states at either highly male or female-dominated. Even though men and women are offered more comparable wages within each occupation, the dramatic increase in segregation due to the policy has the unintended consequence of leading to a larger gender wage gap.

%\footnote{I define equal pay for equal work as equal per hour pay within occupation across gender}

% this policy simulation segregation increases dramatically to the point that women's wages fall relative to men. Intuitively, men can no longer be paid more than women in female dominated occupations, making these occupations even more female dominated, and therefore even more low paying for women.

%In my model it is possible to impose restrictions on equilibrium wage adjustment. I simulate fixing wages at their current values, as well as imposing equal pay for equal work, as defined as equal per hour pay within occupation across gender. In both cases segregation increases dramatically to the point that women's wages fall relative to men. Intuitively, men can no longer be paid more than women in female dominated occupations, making these occupations even more female dominated, and therefore even more low paying for women.

% if employers have to pay women more in male dominated occupations, why don't we see more women? because wages for both men and women go down. Also women were paid more in these occupations before equal pay..... to compensate for the disamenity. Wage OFFERS for women were drastically higher than for men in very male occupations before equal pay. At the same time the outcome in terms of who matches, women are paid less. This is because men like the male occupations and accept relatively low wages, so the only women who get hired can compete with the lower male wages, and there is no long right tail of women in male occupations because they are undervalued.

%Women's wages go up relative to men in male dominated occupations, but fewer women enter these occupations, and down relative to men in female dominated occupations, and women crowd into these occupations.

%Intuitively, with sticky wages, women cannot be compensated for going into male occupations and therefore crowd into female occupations which tend to be lower paying.

% move to end of intro and add closest papers to mine?
Recent literature has identified a number of contributing causes of segregation: preferences over amenities (\citeA{Olivieri2014a}, \citeA{Wiswall2013})\footnote{For overview see \citeA{ Bertrand2011} and \citeA{Cortes2017a}.}, productivity and skill differences (\citeA{Baker2015}), and occupational and educational barriers (\citeA{Hsieh2016}). I study another possible cause of the persistence of segregation: workers valuing the gender composition of occupations. Previous literature such as \citeA{Usui2008, Lordan2015} has looked at whether women prefer to work with women using job satisfaction surveys and employment transitions, but I am the first to quantify the impact of homophily on overall segregation and wage patterns. To do so I build on the methodological insights of \citeA{Choo2006}, \citeA{Salanie2014a}, and \citeA{Dupuy2017} in empirical transferable utility matching.
 
Section \ref{model} introduces the model including firm and worker payoffs and equilibrium wages. Section \ref{data} describes the data sources, and Section \ref{empirical} the estimation and identification. I discuss the parameter estimates in Section \ref{results} and present simulations of segregation dynamics in Section \ref{counterfactual}, including counterfactuals. 



\section{Model} \label{model}




%The empirical strategy consists of two stages: first I use maximum likelihood to disentangle the selection effects of worker and firm optimization on observed matches and wages. I am able to separately identify the willingness-to-pay of the firm and the reservation wage distributions of the workers. This effectively separates the two sides of the market, unlike previous applications of transferable utility matching models, such as \citeA{Choo2006} and \citeA{Chiappori2015}, who were only able identify total match surplus because they lacked data on transfers.\footnote{\citeA{Fox2008c} is able to estimate payoffs from the interaction of characteristics of both sides of the market but not full payoff functions.}

%Second, the estimated reservation wage distributions from the maximum likelihood estimation are used to estimate the worker's utility. Workers care about non-wage amenities, wages, and the fraction female. In the second stage of estimation, a panel regression for each gender is run of shares of workers in each occupation on reservation wages, occupation intercepts, and fraction female. Instruments are used to control for omitted variables correlated with both fraction female and wages over time, and provide clean variation in the fraction female and reservation wage to trace out labor supply.

%Theoretically it would be feasible to combined both stages into joint GMM estimation. This would have the advantage of using all data variation to identify all parameters. However joint estimation would require pooling all years of data and searching for 224 parameters instead of 138, so it would be computationally challenging.

%This would have the advantage of using all data variation, including the number of unemployed workers by gender, to identify willingness-to-pay, reservation wages, and overall utility levels. However joint estimation would require pooling all years of data and searching for 224 parameters instead of 138, so it would be computationally challenging.






%Importantly, all workers and all jobs participate in the auction, so there is no role for search frictions in the model. This assumption is made more reasonable by the fact that I am considering lifetime occupation choices, not individual jobs. 

%The gender ratio of the occupation is determined at the occupation level. There are no firms in the model so I treat all of the utility of the occupation as exogenously given at the occupation level by the occupation production technology, rather than provided by firms at cost as in a hedonic model. When a worker is considering their choice of lifetime occupation, they will consider the overall characteristics of the occupations rather than a particular job or firm. 




%% put firm STUFF HERE

%unlike the marriage market, it is not clear what it means for a job to be ``unmatched". Theoretically, the number of unmatched jobs would be the number of jobs that would exist if labor were free. Rather than estimate production functions and capacity constraints by occupation, I choose to use data on posted vacancies that remain unfilled for some time. 

%The decision to post a vacancy is endogenous to market conditions, so vacancy data is an imperfect proxy for unfilled jobs. 

%In order to limit the importance of the vacancy data in my estimation, I choose a specification that in principle does not require it, by assuming non-negative profit on the job side. However in practice this relies very heavily on the functional form assumption on the tail of the distribution of job dis-amenities. As a result in most specifications I use data on vacancies by occupation imputed from the Job Openings and Labor Turnover Survey \cite{JOLTS}.






%can you estimate the sigmas at all without using the mle?? probably can't since the sigma distributions depends on the pi but the pi depends on the sigma? explore this explanation. but without truncation on the firm side then I should be able to get everything immediately. lnsj-lns0 gives you the surplus on each side. the variance of wages is the variance of the underlying heterogeneity given that max results. then the mean observed wage is W plus some shading term from the optimization that depends on the variance? is that shading term still tractable now I am using normal dist. instead of logit? what is the distribution of the max of normals? apparently under certain convergence conditions the max of normals converges to extreme value but only for large n where n is the number of choices!! for two choices the dist. seems to depend on both means, which are unknown??? In this case the variance also depends on the means so it becomes intractable even at the stage of getting the variance.





%\footnote{$\xi^g_j$ could also be reinterpreted as the job-specific productivity of gender $g$, but in this case, it will not appear in the wage in equilibrium, discussed below.}

%This assumption also implies that all wage heterogeneity is the result of compensating differentials. In reality wage heterogeneity is the result of both compensating differentials and individual worker productivity (see for example \citeA{Sorkin2015a, Taber2011a} for comparison of these sources).



 %I plan to run simulations to assess the level of this bias.

%\footnote{My model will have slightly different dynamics in terms of the evolution of the mean wage by occupation. Overcrowding in my model will lack an additional decrease on wages from drawing in lower skilled workers into the overcrowded occupation.} 



%Unfortunately allowing both forms of heterogeneity to enter the wage makes it impossible to learn from individual wage observations, since wages would reflect the sum of two unobserved terms. We may still be able to learn from the moments of the wage distribution, integrating out both sources of heterogeneity, as suggested in \citeA{Salanie2013a}, and in future work I may explore this possibility.\footnote{In a model with both unobserved productivities and unobserved amenities reflected in wages, workers would choose based on the sum of two heterogeneity terms, one of which is due to tastes and not reflected in the wage and one of which is due to productivity and is reflected in the wage. Wages would reflect the sum of worker productivity heterogeneity, and job amenity heterogeneity.} 



%since the observed wages would then depend on the discrete choice problems on both sides of the market, and more importantly, on a selection problem of dimension equal to the number of occupations rather than the number of genders.



% Since underlying heterogeneity in both productivity and amenities is unobserved, assuming one distribution away and making a distributional assumption on the other is equivalent to making a distributional assumption on their sum. Thus while strictly speaking the the model wage heterogeneity reflects unobserved heterogeneity in job amenities, it can be interpreted as a combination of that and unobserved heterogeneity in individual productivity.Unobserved to the econometrician are preferences over individual workers and jobs, which vary by type. Workers of type $X$ all share the same preferences over jobs $j$ within type $Y$, and worker $i$ has unobserved preferences over types of jobs $Y$. Jobs care only about the type $X$ of the worker they hire and the wage that they must pay to attract individual $i$. Intuitively, this structure implies that conditional on job and worker type, workers are all equally productive so the job cares only about the wage. The workers share the same ranking of jobs within occupations, conditional on type, but have idiosyncratic preferences over occupations.

%$$\pi^g_j = \pi^g_o - w^g_j $$ and the utility of worker $i$ of accepting job $j$: $$u_{ij} = u^g_o + w^g_j + \xi^g_j + \eta^i_o$$ Crucially preferences satisfy additive separability, meaning that preferences can only depend on the unobserved attributes of both sides of the market in an additive way. This means that the matching problem can be divided into two separate optimization problems connected only through transfers.

\subsection{Firm Payoff}

%Jobs choose the gender of worker that will maximize productivity per dollar spent, which may be differentially valued by men and women. This is consistent with a model of firms where jobs within a firm are filled independently, perhaps due to costly coordination or low complementarity. Individual jobs optimizing over rate of return does not imply a specific firm-level production function, but it produces similar behavior as firms minimizing cost with production quotas. Without data on firm size, I take this as an approximation of firm behavior.

% the relative productivity of men and women and the wage that they will have to pay to an individual worker.
In the model, each firm ($j$) chooses to hire a male or female worker ($g \in \{M,F\}$) to fill a single vacancy. I allow firms to be willing to pay different amounts for male and female workers, but I do not take a stance as to why or how. The firm maximizes log profit $\pi^g_j$ to maximize its rate of return on its single vacancy. This allows me to abstract away from the number of vacancies at a firm or complementarities between those vacancies. Firm $j$'s payoff is as follows:

%\footnote{It also allows me to assume that wages are lognormally distributed, as is standard in the literature, while estimating normally distributed logged parameters, which is computationally attractive.} The firm solves the following problem:

\begin{align} \label{logfirm}
\pi^g_{j,t}=&  WTP^g_{o,t} -  Wage^g_{j,t}  \\ \nonumber
\end{align}

%\begin{align} \label{logfirm}
%  log(\pi^g_j)=&  log(WTP^g_o) -  log(Wage^g_j) \\ \nonumber
%  \equiv & \widebar{WTP}^g_o - \widebar{Wage}^g_j
%\end{align}


%\begin{align} \label{firm}
% \pi^g_j = \frac{WTP^g_o}{Wage^g_j} 
%\end{align}

The total payoff to firm $j$, $\pi^g_{j,t}$, depends on $WTP^g_{o,t}$, the log willingness-to-pay of a firm in occupation $o$ for a worker of gender $g  \in \{M,F\}$ in time period $t$. $WTP^g_{o,t}$ will be estimated for each gender and occupation and time period. The payoff also depends on the log cost of hiring a worker, ${Wage^g_{j,t}}$, which is observed but varies in equilibrium. 

In Appendix \ref{sec.Stability}, I show that log wages consist of two components in equilibrium:

\begin{align} \label{wages}
Wage^g_{j,t} = W^g_{o,t} + \xi^g_{j,t}
\end{align}

The first term, $W^g_{o,t}$, equates aggregate supply and demand in order to make the matching feasible. This term varies by gender and occupation. Below I refer to $W^g_{o,t} $ as the wage offer component because it is equal to the expected value of the log wage offer in occupation $o$ for gender $g$, ignoring selection over firms $j$.

The log dis-amenity value of a particular firm $j$ is denoted $\xi^g_{j,t}$. This dis-amenity heteroeneity value could be, for example, the presence of on-site childcare or other work environment factors, specific to firm $j$. I take these dis-amenities as fixed and exogenous for the purpose of the model. In equilibrium, wages must exactly compensate workers for the dis-amenity heterogeneity values, making workers indifferent across firms within an occupation. This assumption ensures pairwise stability, meaning that after optimization, no workers or firms could achieve a higher surplus by rematching. 

%A firm receives a draw of exogenous heterogeneity, $\xi^M_{j,t}$ and $\xi^F_{j,t}$, which is the amenity value of the firm to men and women respectively.

%Log equilibrium wages consist of both the compensation for job amenities component ($\xi^g_{j,t}$), and the aggregate feasibility component ($W^g_{o,t} $). 



Note that in order for equilibrium wages to be unique I assume a large number of workers of each gender and firms in each occupation \cite{Galichon2015}.  I also assume that log firm dis-amenity heterogeneity terms, $\xi^g_{j,t}$, are normally distributed, making wages lognormally distributed. This is a common assumption that matches observed wage distributions.

\begin{assumption} \label{lognormal}
Let the log heterogeneity in firm dis-amenities for each gender, $\xi^g_{j,t}$, be distributed $\mathcal{N}(0, \sigma_{\xi^g_t}$) and independent across $j$, such that $exp(\xi^g_{j,t})$ is distributed lognormal.
\end{assumption}


%% ADD WHAT WAGES ARE HERE AND LOGNORMALITY ASSUMPTION??



%\subsection{Market Clearing Wages}

% REPLACE
%The lifetime income distributions by gender and occupation needed to clear the market are determined in equilibrium to equate demand and supply, and are made up of two components: a job-specific component, and an aggregate component that varies by occupation and gender. The job-specific component reflects the idiosyncratic disutility of taking a specific job. In equilibrium firms must make workers indifferent across individual jobs within an occupation, and therefore firms perfectly compensate workers for idiosyncratic disutility. Wage heterogeneity within occupation and gender emerges from these job-specific compensating differentials. 

%In addition to an individual job level compensating differential, there is a component of wages which is common across individuals within gender and occupation, and serves to equate supply and demand. It is a function of all parameters in the model, specifically, both of the utility that workers receive from occupations as well as the value that jobs have for male and female workers. 



%\footnote{For reference, simulations of an ascending price auction with sample size of 20 already produce a wage vector similar to the equilibrium vector assumed by the large sample. I perform the simulations using a modified version of the auction mechanism outlined in \citeA{RothSotomayor} page 209.}






%Compensating differentials also emerge at the occupation level through the common component of wages, $W^g_o$, which will vary according to the utility that workers receive from occupations, as well as the value that jobs have for male and female workers, in order to equate supply and demand.\footnote{This component of wages can be thought of as the result of a market-wide open-ended ascending price auction where jobs make wage bids for workers, but each job can only ``win" one worker. In order for equilibrium wages to not depend on sample size I must assume a large number of workers of each gender and jobs of each occupation \cite{Galichon2013b}.}

% , and there are no search frictions. The assumption of no search frictions is made more reasonable by the fact that I am considering lifetime occupation choices, not individual jobs.





%The worker utility function therefore takes into account the job dis-amenities $\xi^M_j$ and $\xi^F_j$ as well as the workers taste for that occupation, and because $log(Wage^g_j) = log(W^g_o * \xi^g_j) = \bar{W^g_o} + \bar{\xi^g_j}$, 

I also assume that if the equilibrium wage the firm would have to pay to fill its vacancy is greater than its willingness-to-pay, the firm will not fill its job opening and receive a payoff of zero. This will occur for firms that receive large draws of both $\xi^M_{j,t}$ and $\xi^F_{j,t}$, meaning both men and women find the firm very unattractive.

\begin{assumption}\label{nonneg}
A firm will not hire if $\pi^g_{j,t} \leq 0$ for both male and female workers.
\end{assumption}

Combining Equations \ref{logfirm} and \ref{wages} and Assumption \ref{nonneg}, the firm solves the following problem in equilibrium:

\[
\max \left \{ \max_{g \in \{M,F\}}  (\pi^g_{j,t} =  WTP^g_{o,t} -  W^g_{o,t} - \xi^g_{j,t}), 0 \right  \} %\\
%\text{s.t.} \hspace{5mm} \pi^g_j > 0
\]



The firm chooses to hire a male or female worker based on how much they value hiring a man or woman ($WTP^g_{o,t}$), the overall equilibrium cost of hiring a man or women, ($W^g_{o,t}$), and the idiosyncratic cost of hiring men and women at firm $j$ ($\xi^M_{j,t}$ and $\xi^F_{j,t}$). The assumption that firms do not have preferences over individual workers within gender is critical because it allows me to separate the matching problem into two separate discrete choice problems, one for firms and one for workers \cite{Galichon2015}. 

%The cost of this assumption is that there is no individual worker productivity heterogeneity. Selection on productivity is orthogonal to my research question, but it may lead me to underestimate the value of amenities of small occupations and overestimate the appeal of large occupations. 




% for both economic and computational reasons. First, maximizing rate of return is consistent with a firm only filling one vacancy at a time, and allows me to avoid making assumptions about the number of vacancies at a firm or complementarities between those vacancies. Second, the multiplicative specification allows wages to be lognormally distributed, which is common in labor economics to match the observed shape of the wage distribution, while still additive and normal when logged, which is computationally attractive. 

%The log payoff to the firm $j$ is:

%First, since firms in the model only fill one vacancy at a time, rate of return is a reasonable object for firms to care about that does not require making assumptions about the number of total vacancies at a firm. By allowing the firm to maximize return on a single vacancy at a time I make a reasonable job-level approximation of a firm level profit function. 


%Second, the rate of return specification is more tractable for estimation under the assumption that wages are lognormal. I later allow wages to be distributed lognormal so that log wages is normal and additive in the firm payoff.

%This is because in a model with firms making hiring decisions, the firm would likely be hiring multiple workers to meet a certain production level. Therefore an individual job is less likely to care about the level of output of an individual worker (as in an additive profit function) than the return per unit of wage. 






%This allows me to identify the model without use of vacancy data on the firm side. In practice I use vacancy data in estimation for stability of the estimator and to avoid relying too much on extreme data points for identification.

%I make the assumption of a fixed outside option because then the model is identified without use of data on vacancies, allowing me to test robustness to the use of the vacancy data. I can estimate the willingness-to-way of the firms with the maximum observed wage by gender and occupation. This strategy is less stable than using vacancy data. 



%The payoff to the firm of not hiring any worker, $\pi^N_j$, is normalized to one, so that the log payoff to not filling a vacancy is zero, ($ \bar{\pi}^N_j = 0$). There is no equivalent on the firm side to the idiosyncratic preference for non-employment that exists on the worker side. Heterogeneity in the outside option for a firm could reflect heterogeneity in the cost of hiring for different jobs, or the substitutability of a particular job in the firm production function. 


%
%Let $\Phi_{0,\sigma^g_{\xi}}$ and $\phi_{0,\sigma^g_{\xi}}$ are the cdf and pdf of the normal distribution with location zero and scale $\sigma^g_{\xi}$. The normality assumption implies the following choice probability.
%
%\begin{align*}
%Pr( \text{firm hires } F ) &=  Pr(\widebar{WTP}^F_o - \bar{W}^F_o + \xi^F_j \geq \widebar{WTP}^M_o - \bar{W}^M_o + \xi^M_j)\\
%&=  Pr(\widebar{WTP}^F_o - \bar{W}^F_o + \xi^F_j - (\widebar{WTP}^M_o - \bar{W}^M_o) \geq  \xi^M_j)\\
%&= \int_{-\infty}^{\infty} {\Phi_{0,\sigma^M_{\xi}}(\widebar{WTP}^F_o - \bar{W}^F_o - \widebar{WTP}^M_o - \bar{W}^M_o + \xi^F_j)}  \phi_{0,\sigma^F_{\xi}}(\xi^F_j) d\xi^F_j\\
%\end{align*}

%A firm will only hire a worker if the willingness-to-pay for that worker is higher than the wage the worker will accept in equilibrium, which means the payoff $ \pi^g_j $ must be greater than one. 

%$$ \pi^g_j = \frac{WTP^g_o}{Wage^g_j} \geq 1 $$



%$$ \bar{\pi}^g_j = \widebar{WTP}^g_o - \widebar{Wage}^g_j <  0 \hspace{3mm} \text{for} \hspace{3mm} g \in \{M,F\}  \hspace{3mm} \implies \hspace{3mm} \text{$j$ unfilled}$$

%This differs from the classic multinomial choice model because there is no unobserved heterogeneity in the outside option. That is to say, t

%The above assumption implies that jobs are unfilled when both $\bar{\xi}^M_j$ and $\bar{\xi}^F_j$ are very high. High $\bar{\xi}^M_j$ and $\bar{\xi}^F_j$ means very high dis-amenity value to both men and women, requiring high wages to compensate. 




\subsubsection{Worker}
%In a competitive equilibrium firms will produce at the production level that minimizes average cost, or equivalently maximizes willingness-to-pay over cost.

%I assume that job choice is binding for life, which means that the decisions of previous cohorts affects the choice set of the current cohort through the fraction female. 

For the purposes of the model, I divide workers into four ten-year cohorts by age: ages 25-34, 35-44, 45-54, and 55-64. Every 10 years a young cohort of workers (ages 25-34) makes an occupation choice. The young cohort observes the fraction female in the occupation among the previous three cohorts of workers (ages 35-65), which I denote $F_{o,t-1}$. I assume that the job choice is binding for life,\footnote{To assess the impact of this assumption, I examine the extent to which workers switch occupations during their working lifetime in the PSID. I find that the average worker who spends most years working spends 80\% of working years in the same broad occupation category.} which implies that the choice of the young cohort of workers will then influence the fraction female observed by the next three cohorts of workers. Lowering switching costs would lead to faster convergence to a fixed point in the fraction female by occupation, but not change the substantive dynamics of the model. I also assume that workers are myopic and do not anticipate future changes in the market that would lead them to want to switch occupations.\footnote{Relaxing myopia would mean that the endogenous attribute of interest, the fraction female, would depend on expectations over other workers' occupation choices, which in turn are affected by the fraction female. This would pose a challenge to tractability.}

In the model, a worker $i$ of gender $g$ in time $t$ chooses a firm $j$ in occupation $o$ to maximize log utility as follows:

\begin{align} \label{logworker}
\max_{j } u^i_{j,t} &=  \alpha^g_{o}+ \beta^g_t  +  \gamma^g F_{o,t-1}   + Wage^g_{j,t} + \eta^i_{o,t} - \xi^g_{j,t}  \\ \nonumber
\end{align}

%Workers choose an occupation at the beginning of their lives, based on their individual tastes for occupations and the wage offer for their gender in that occupation. 


The main parameter of interest is $\gamma^g $, which reflects utility that gender $g$ receives from the observed fraction female in the occupation, $F_{o,t-1}$. The model assumes that workers care about the fraction female at the occupation level, which could reflect social norms regarding the gender of the occupation, or an expectation of working with colleagues within the same occupation. Recall that $F_{o,t-1}$ is subscripted $t-1$ to indicate that it reflects the decisions of the previous three cohorts of workers. The preference that all workers of gender $g$ have for occupation $o$ is given by $\alpha^g_o$. This could reflect any sort of gender specific taste for type of work that does not vary over time. Variation over time in the attractiveness of employment vs. non-employment by gender is captured by $\beta^g_t $. Workers also care about the wage they are offered by firm $j$, ($Wage^g_{j,t}$), and the dis-amenities at firm $j$, denoted $\xi^g_{j,t}$, which are the same for everyone conditional on gender. Worker $i$'s specific taste for occupation $o$ is $\eta^i_{o,t}$. 


Recall that in equilibrium the wage is given by  $Wage^g_{j,t} = W^g_{o,t} + \xi^g_{j,t}$. Therefore the log utility maximization problem of the worker in equilibrium can be simplified to:

\begin{align} \label{utility}
\max_{o \in O} u^i_o &=   \alpha^g_o + \beta^g_t  + \gamma^g F_{o,t-1} + W^g_{o,t}  + \eta^i_{o,t}
\end{align}

Workers are exactly compensated for the firm dis-amenities, $ \xi^g_{j,t}$, in equilibrium. This means that the taste for occupation, $\eta^i_{o,t}$, is the only heterogeneity at the individual worker level. The implication is that workers do not have preferences over individual firms within an occupation. As on the firm side, this allows me to separate the sides of the market into two discrete choice problems.

I make the following two assumptions, which are standard in the discrete choice literature \cite{Berry1994}: a logit assumption on taste heterogeneity and a normalization of the value of the outside option.

\begin{assumption}
Let the worker taste heterogeneity for occupations, $\eta^i_{o,t}$, be independently distributed extreme value type 1, with scale parameters scale parameters $\sigma^M_{\eta}$ and $\sigma^F_{\eta}$ to be estimated.
\end{assumption}

%The location parameter of $\eta^i_o$ is normalized to zero, and the scale parameter is estimated separately for each gender, resulting in two scale parameters $\sigma^M_{\eta}$ and $\sigma^F_{\eta}$ for men and women respectively. 

\begin{assumption}
Let the log utility from non-employment be normalized to zero ($u^g_N=0$). 
\end{assumption}

%$$ \frac{Pr(j \in Y \text{ chooses } i \in X)}{Pr(j \in Y \text{ chooses } 0)} = \frac{ \frac{exp(\frac{\Pi^g_o}{\sigma^g_{\xi}})}{\sum_X exp(\frac{\Pi^g_o}{\sigma^g_{\xi}})} } {\frac{1}{\sum_X exp(\frac{\Pi^g_o}{\sigma^g_{\xi}})}} =  exp(\frac{\Pi^g_o}{\sigma^g_{\xi}})$$

Since no wages are received in non-employment, this leaves only the idiosyncratic taste for non-employment $\eta^i_N$. I then leverage well-known properties of extreme value distributions to obtain relative occupation choice probabilities in terms of utility parameters. I denote the share of workers of gender $g$ who match to occupation $o$ in time $t$ as $s^g_{o,t}$, and  the share who choose non-employment as $s^g_{N,t}$.
\begin{equation} \label{eq:1}
\begin{split}
ln(s^g_{o,t}) - ln(s^g_{N,t}) &=  \frac{\alpha^{g}_o + \beta^g_t + \gamma^g F_{o,t-1}  + W^g_{o,t}}{ \sigma^g_{\eta} } 
 \end{split}
\end{equation}

We now have worker utility parameters written in terms of observed shares $s^g_{o,t}$ and $s^g_{N,t}$. Below I discuss the data used to identify the model, and then the estimation strategy for both the firm and worker sides of the model.

%Recall that $log(u^i_g) = log(u^g_o) + log(W^g_o)  + \eta^i_o \equiv  \bar{u}^{g}_o + \bar{W}^g_o   + \eta^i_o  $.

%In \citeA{Chiappori2015} the scale parameters require many markets to estimate but with observed transfers we can rely on variation in lifetime income across occupations to trace out the scale.



%$$ Pr(j \in Y \text{ chooses } i \in X) = \frac{exp(\frac{\Pi^g_o}{\sigma^g_{\xi}})}{\sum_X exp(\frac{\Pi^g_o}{\sigma^g_{\xi}})}$$

%$$ Pr(i \in g \text{ chooses } \forall  j \in o) = \frac{exp(\frac{\bar{u}^{g}_o + \bar{W}^g_o}{\sigma^g_{\eta}})}{\sum_{k \in O} exp(\frac{\bar{u}^{g}_k + \bar{W}^g_k}{\sigma^g_{\eta}})}$$ 




%$$ \frac{Pr(i \in g \text{ chooses } \forall j \in o)}{ Pr(i \in g \text{ chooses }N)}= \frac{ \frac{exp(\frac{\bar{u}^{g}_o + \bar{W}^g_o}{\sigma^g_{\eta}})}{\sum_Y exp(\frac{\bar{u}^{g}_o + \bar{W}^g_o}{\sigma^g_{\eta}})}  } {\frac{1}{\sum_Y exp(\frac{\bar{u}^{g}_o + \bar{W}^g_o}{\sigma^g_{\eta}})} } = exp(\frac{\bar{u}^{g}_o + \bar{W}^g_o}{\sigma^g_{\eta}})$$







%Non-employment is defined as either not in the labor force or unemployed, and the payoff to non-employment is $ \bar{\eta}^i_N $, the idiosyncratic taste for non-employment.
%
%The payoff to worker $i$ from job $j$ is specified as:
%\begin{align} \label{worker}
%u^i_j &= \frac{ u^g_o*Wage^g_j *  \eta^i_o } {\xi^g_j}    
%\end{align}
%
%Defining log utility to be $\bar{u}^{i}_j$ and re-parameterizing in terms of logs we have:



%Workers choose a job $j$ to maximize log utility $\bar{u}^{i}_j$. The common taste parameter for non-employment $\bar{u}^g_N$, where non-employment means either not in the labor force or unemployed, is normalized to zero. It is also assumed that in non-employment workers receive no wages or value from job amenities. The payoff to non-employment ($\bar{u}^i_N$) is therefore  $ \bar{\eta}^i_N $, the idiosyncratic taste for non-employment.



%\footnote{The log utility parameters for working, $\bar{u}^g_o$, include the disutility of working at all and are expected to be negative. However the underlying utility parameters $u^g_o = exp(\bar{u}^g_o)$ will always be positive leading to complementarity in wage and non-wage utility. Similarly, ${\xi^g_j}$ will be assumed to be log-normally distributed and therefore always positive, so the higher the dis-amenity value of the job the lower the utility. }

%Furthermore worker utility will be decomposed into a common component and a dependence on the fraction female in the occupation, which is the endogenous amenity of interest. And a time effect.
%
%\begin{align*}
%\bar{u}^{i}_j &= \bar{u}^{g}_o + \widebar{Wage}^g_j   + \bar{\eta}^i_o - \bar{\xi}^g_j \\
%    &= \alpha^g_o +   \gamma^g F_{o,t-1} + \widebar{Wage}^g_j   + \bar{\eta}^i_o - \bar{\xi}^g_j
%\end{align*}
%% \frac{\beta^g}{\sigma^g_{\eta}}  X_o+



%Let $F_{o,t-1}$ be the fraction female among older workers, ages 35-65, in occupation $o$ observed by the younger workers, ages 25-34, at time $t$ as they make their lifetime occupation decisions. 
%
%The youngest cohort, ages 25-34, observes $F_{o,t-1}$ and makes their occupation choices, producing the number of men and women in each occupation in the time $t$ cohort, $n^F_{o,t}$ and $n^M_{o,t}$. The dynamics of the model result from updating of the fraction female in each occupation with each successive cohort of workers.  Since the occupation choice of each worker is fixed for the rest of their working lifetime (assumed to be 40 years), their choice which will influence the fraction female observed by the next three cohorts of workers.
%
%%In each ten year period, only young workers ages 25-34 choose a new occupation, while older cohorts in age brackets 35-44, 45-54, 55-65 are fixed in their occupation.  To make the timing assumption clear I use subscript $t-1$ for the fraction female.
%
%
%
%$$F_{o,t-1} = \frac{n^F_{o,t-1} + n^F_{o,t-2} + n^F_{o,t-3}}{(n^F_{o,t-1} + n^M_{o,t-1}) + (n^F_{o,t-2} + n^M_{o,t-2}) + (n^F_{o,t-3} + n^M_{o,t-3}) } $$







%When the current cohort of workers, ages 25-34, make occupational decisions, they face the gender ratio produced by older cohorts of workers. The occupation choice of each worker is fixed for the rest of their working lifetime (assumed to be 4 periods, or 40 years), and workers do not take into account predicted future evolution of the fraction female. 

%Workers and jobs are assumed to be numerous enough that no single worker need take into account their own impact on the fraction female in their occupation of choice.

%Let the fraction female in occupation $o$ observed by the young cohort before making their decisions be $F_o$. Let the fraction female among cohort $C$ in occupation $o$ be $F_{o,C}$. Note that $F_{o,C}$ results only from the decisions made by cohort $C$. Then the fraction female in occupation $o$ observed by the current generation when they make their decisions is $F_o$ and can be decomposed into the weighted average of decisions of previous generations as follows, where the weights are the cohort sizes $n_{o,C}$.

%$$ F_o = \frac{n_{o,C-1}F_{o,C-1} + n_{o,C-2} F_{o,C-2} + n_{o,C-3} F_{o,C-3}}{n_{o,C-1}+n_{o,C-2}+n_{o,C-3}}$$ 
%Let the fraction female in an occupation observed by the young cohort in time $t$ before making their decisions be $F_t$, where the occupation subscript $_o$ is omitted for convenience. Then we can write $F$ as a function of the number of men and women ($n^F$ and $n^M$) choosing that occupation in the previous three periods.






%The counterfactual dynamics I am interested in are at the level of occupation and gender, and productivity and utility are allowed to vary freely by occupation and gender. Therefore including individual worker productivity heterogeneity should not have a major impact on my results. In terms of biasing parameter estimates, I may overestimate how undesirable small occupations are. High wages in small occupations may be due to selecting the most skilled workers rather than dis-amenities. Similarly I may overestimate the value of amenities in large occupations if wages are low due to a lower than average level of worker skill, rather than attractive amenities.
% JUSTIFY MORE WHY ORTHOGONAL



%%%% CUT AND REWRITE
 %% PUT IN FOOTNOTE why sigmas matter, but CUT THIS
%Lastly, I estimate four variance parameters in total: variance of taste for occupation for men and women, and variance of job dis-amenities for men and women. These parameters are important because they govern the elasticity of labor supply to changes in wages or other non-wage amenities such as the gender ratio in the counterfactual, as well as the demand response to changes in equilibrium wage. Unlike \citeA{Chiappori2015}, who uses many markets to identify these parameters, I use the observed wage distribution as suggested in \citeA{Salanie2014a}. However I estimate using maximum likelihood in order to use the information from the wage distribution efficiently, and because the lognormal assumption means that the scale parameters cannot be separately estimated from all other parameters, as noted above. For all these reasons maximum likelihood is the most natural strategy for my application.



\section{Data} \label{data}
The data elements needed to estimate the model described above are: expectations of lifetime labor income by occupation, gender, and cohort ($Wage^g_{j,t}$), shares of workers by gender and age cohort choosing each occupation and non-employment ($s^g_{o,t}$ and $s^g_{N,t}$), and a measure of unfilled jobs by occupation (jobs for which $\pi^g_{j,t} \leq 0$).

While the Census and ACS provide cross sectional wage and occupation, the Survey of Income and Program Participation (SIPP) is needed to provide a panel for the construction of lifetime labor income estimates.\footnote{Public use Census 1960, 1970, 1980, 1990, 2000, and 2012 three-year ACS data obtained from IPUMS \cite{IPUMSUSA}. SIPP data from the 2004 and 2008 panels are constructed using the NBER files \cite{SIPP}. Occupation codes are constructed by aggregation of the IPUMS harmonized codes (occ1990) to achieve sufficient sample size. See Table \ref{modelestimates} for the full list of occupation codes.} Using pooled data from the 2004 and 2008 SIPP panels, which are four and five years long respectively, I construct transition rates through five quantiles of earnings and occupations by worker age and gender.\footnote{ I assume that transition rates depend only on the current state not the past history, but this assumption could be relaxed with the addition of more data.} This transition matrix is then used to simulate worker career paths from the starting point of workers aged 25-35 in the Census and ACS. Their assigned choice of occupation is taken as the occupation they start out in at ages 25-35 as observed in the Census, which means that the occupation choice can be interpreted as including the expectation of all future transitions.

% INCLUDE A SUMMARY STATS TABLE??? would have lifetime income by occupation, fraction female by occupation, and unfilled jobs by occupation

%\footnote{CPS or CPS MORG data could be added to get an earlier estimate of these transition rates.}
%The lifetime income assigned to each observation in the Census or ACS is the sum of their simulated career path through the SIPP transition matrix. 

%For a comparison of estimated lifetime income paths to observed lifetime income paths in the PSID, see appendix \ref{PSID}. 

%The five quantile cut offs are estimated in the Census data where the larger overall sample size allows for more accuracy. 

%Lifetime wages and shares of workers by gender and occupation are estimated using a combination of Census data and SIPP data.

% While using cross sectional data may be sufficient to create estimates of lifetime income when only a few moments are needed, for my identification strategy the shape of the wage distribution is critical, and I therefore cannot assume that every worker obtains, for example, the median income for their chosen occupation at each age. The resulting distribution is mechanically quite lumpy. One reason that the lifetime income distribution in reality might be much smoother than a cross-sectional approximation is that workers transition stochastically over time between wage levels and occupations.

%To model worker transitions over time as probabilistic, panel data from the SIPP is essential. 



%Income quantiles from Census, by age, gender, and occupation, are denoted $quan$ below. Occupation is denoted $occ$, and five year age bracket $age$. Transition rates from year to year, where $x'$ indicates the value of $x$ in the following year, are then estimated non-parametrically as follows:

%\begin{align*}
%& Pr(quan',occ'|quan,occ,age)=  \frac{\sum_{i} \mathcal{I}(i \in quan',occ',quan,occ,age)}{ \sum_{i} \mathcal{I}(i \in quan,occ,age)}\\
%\end{align*}



The Job openings and Labor Turnover Survey \cite{JOLTS} is used to construct a measure of unfilled jobs by occupation. Since JOLTS does not directly contain occupation, only NAICS industry codes, industries are projected into occupations using contemporaneous occupation industry shares estimated in CPS \cite{IPUMSCPS}. The estimated openings by occupation is then divided by the total number of people employed in the occupation to get the ratio of openings to employed.

%I also make an estimate of the number of openings that remain open at the end of the year by assuming that the probability that a job is filled is uniform across time and jobs. Then the daily rate of hiring is equal to the total number of hires that month divided by the number of days in the month. Then the probability a job is not filled on a given day is one minus this daily hire rate, and the probability a job is unfilled for the year is this probability to the power of 365. This measure of unfilled jobs will be used for robustness.

%Additional data on occupation attributes from the Dictionary of Occupation Titles (1977), the ONET database (1998-2016), the CPS work supplement, and recently collected job opening data from \citeA{Atalay2017}, are included in some specifications. The DoT contains a number of measures of skill requirements of the occupation and occupation tasks in including the level of complexity with which a worker relates to people, data, and things on the job. The ONET contains a myriad of job context and task measures, most importantly competition, contact with others and time pressure, previously identified by \citeA{Goldin2014} to be of potential differential interest to women and men.\footnote{The ONET is updated continuously such that all occupations have been updated at least once during the period 1998-2016. Thus the oNET itself can be used as a measure of changes in occupation attributes but only during this limited time period. Similarly some occupations in the DoT were updated in 1991, allowing for some longitudinal variation between 1977-1991, and potential for rough mapping of the 1991 DoT into the 1998 oNET.} IPUMS CPS work supplement data from 1991, 1997, 2001, 2004 is pooled to obtain a measure of flexible working hours and nonstandard end time of job. I also use occupation attribute data from \citeA{Atalay2017}, who use changes over time in job opening ads in several major newspapers to get measures of occupation characteristics that vary over time.


%%%% PUT THIS BACK IN MAYBE???
%To assess the impact of my assumption that workers make lifetime occupation choices, I examine the extent to which workers switch occupations during their working lifetime. In the PSID, the average worker who spends most years working spends 80\% of working years in the same occupation\footnote{I include observations that appear in the PSID for at least 25 years between 1968 and 2011, beginning before age 30 and ending after age 55, and are not missing occupation data. This results in a sample of 764 workers.} Furthermore, for 85\% of workers in the PSID the modal occupation for ages 25-35 is also the modal occupation for ages 25-55. My simulated lifetime income sample overestimates occupation transitions relative to the PSID, likely because occupation choice is more history dependent in reality. Either way, occupation choice appears to be fairly stable for a lot of people, and to the extent that people do transition, this is included in the expected value of the initial occupation choice.

%and 57\% of workers in the simulated SIPP, 

%In my 1960 data, simulated from the Survey of Income and Program Participation (SIPP), 14\% of workers change occupation each year on average.\footnote{\citeA{Kambourov2008} find that on average 13\% of workers change occupation each year, by comparing retrospective and concurrent occupation data in the Panel Study of Income Dynamics (PSID). In my unadjusted PSID sample this number is 21\%.} 

%However the statistic that matters is how many workers spend many or most of their working years in the same occupation, mirroring the lifetime occupation choice dictated by the model.  



%In my simulated 1960 SIPP sample this number is only 64\%. In reality dependence on past occupation likely extends beyond the immediate previous period, so this number is likely a lower bound due to my first-order Markov assumption in simulating the data. 



%\footnote{The higher rate of transitions but lower rate of years in the same occupation in my simulated data is consistent with the first-order Markov assumption being overly simplistic. In reality dependence on past occupation likely extends beyond the immediately previous period.} 


\section{Empirical Strategy} \label{empirical}

I estimate the firm side and the worker side of the model separately in two stages. In the first stage, I estimate the firm parameters ($WTP^g_{o,t}$, $\sigma_{\xi^g_t} $) and equilibrium wage offers ($W^g_{o,t}$) using maximum likelihood estimation for each cross section of Census or ACS data.

%The data used in estimation is the imputed observed wage distribution at the individual level, shares of workers in each occupation, and imputed unfilled jobs,

In the second stage, I estimate the worker side using an instrumental variables regression at the occupation-year level. The common equilibrium wage offers ($W^g_{o,t}$), estimated in the first stage, are treated as data in the second stage regression. Instruments, discussed below, provide clean variation in the fraction female and reservation wage to trace out labor supply parameters ($\alpha^g_o$, $\gamma^g$, $\sigma^g_{\eta}$).


%\footnote{Combining the two estimation steps using joint GMM is theoretically feasible and would have the advantage that all data moments are used to identify all parameters. However the number of parameters (914 parameters) and the need to pool all years of data (7 million observations) makes joint estimation computationally unattractive.} 




%Estimation follows from distributional assumptions on the unobserved heterogeneity on each side of the market.

\subsection{Step 1: Firm Estimation and Identification}

\subsubsection{Firm Estimation}

We observe only those log wages ($Wage^g_{j,t}$) that maximize the firm's choice over male, female, or not hiring any worker. Therefore the observed data are the result of both selection and truncation. The selection and truncation depends on what firms are willing to pay for workers ($WTP^g_{o,t}$), the equilibrium wage offers ($W^g_{o,t}$) needed to clear the market, and the scale of the unobserved firm dis-amenity heterogeneity, $\sigma_{\xi^g_t} $.\footnote{Estimating scale is important because is governs the elasticity of labor supply to changes in wages or other non-wage amenities such as the gender ratio in the counterfactual, as well as the demand response to changes in equilibrium wage. Unlike \citeA{Chiappori2015}, who uses many markets to identify these parameters, I use the observed wage distribution as suggested in \citeA{Salanie2014a} and \citeA{Dupuy2017}.} All of these elements are unknown and to be estimated. To estimate $WTP^g_{o,t}$, $W^g_{o,t}$ and $\sigma_{\xi^g_t} $ jointly while accounting for selection and truncation, I use Tobit Type 5 maximum likelihood estimation. I estimate the likelihood separately on individual level data for each of six cohorts of workers between 1960 and 2012 ( $t \in \{1960,1970,1980,1990,2000,2012\} )$.

%See Appendix \ref{sec.likelihood} for the full likelihood function.

 %\subsection{Joint Likelihood Tobit Type 5}\label{sec.likelihood}


%\begin{align*}
%y_{1j}^* &= \widebar{WTP}^F_o - \bar{W}^F_o  - \xi^F_j - (\widebar{WTP}^M_o - \bar{W}^M_o  - \xi^M_j) \\
%y_{2j}^* &= \bar{W}^F_o + \bar{\xi}^F_j \\
%y_{3j}^* &= \bar{W}^M_o + \bar{\xi}^M_j \\
%y_{2j} &= y_{2j}^* \hspace{10mm} &\text{if} \hspace{2mm} y_{1j}^* >0 \\
%y_{2j} &= 0 \hspace{10mm} &\text{if} \hspace{2mm} y_{1j}^* \leq 0 \\
%y_{3j} &= y_{3j}^* \hspace{10mm} &\text{if} \hspace{2mm} y_{1j}^* \leq 0 \\
%y_{3j} &= 0 \hspace{10mm} &\text{if} \hspace{2mm} y_{1j}^* > 0 \\
%\end{align*}
%
%
%Let $f_{1,3}$ be the joint density of $y_{1j}^*$ and $y_{3j}^*$, and likewise $f_{1,2}$.
%\begin{align*}
%L &= \prod_F \int_{-\infty}^0 f_{1,3}(y_{1j}^*,y_{3j}) dy_{1j}^* \prod_M \int_{0}^{\infty} f_{1,2}(y_{1j}^*,y_{2j}) dy_{1j}^* \\
%&= \prod_j Pr(y_{1j}^* \leq 0, y_{3j})^{\mathcal{I}(y_{3j})}*Pr(y_{1j}^* > 0, y_{2j})^{\mathcal{I}(y_{2j})} \\
%&= \prod_j (Pr(y_{1j}^* \leq 0 | y_{3j})*Pr(y_{3j}))^{\mathcal{I}(y_{3j})}*(Pr(y_{1j}^* > 0| y_{2j})*Pr(y_{2j}))^{\mathcal{I}(y_{2j})}\\
%&= \prod_j (F_{1}(0|y_{3j})*f_3(y_{3j}))^{\mathcal{I}(y_{3j})}*(F_{-1}( 0| y_{2j})*f_2(y_{2j}))^{\mathcal{I}(y_{2j})}\\
%\end{align*}
%
%Where $F_1$ is the cdf of $y_{1j}^*$, $F_{-1}$ the cdf of $-y_{1j}^*$, $f_3$ is the pdf of $y_{3j}^*$, and $f_2$ the pdf of $y_{2j}^*$.
%\begin{align*}
% y_{2j}=y_{3j}=0 \hspace{10mm} &\text{if} \hspace{5mm} \widebar{WTP}^M_o - \bar{W}^M_o - \bar{\xi}^M_j < 0\\
% &\text{and} \hspace{5mm} \widebar{WTP}^F_o - \bar{W}^F_o -  \bar{\xi}^F_j < 0 \\
%\end{align*}

%Translating into my notation, 

%If a job is filled, its contribution to the likelihood function takes the form 

% this decomposition is not quite right when I am using the outside option data because I am using law of total probabiltiy
% this method would be right in the case that I maximize conditinal likelihood without outside option data
%\begin{align*}
% LL_{j_{\text{filled}}} &= \prod_{j_{\text{filled}}}  Pr(\text{$j$ hire g},Wage^g_{j},\text{$j$ filled}) \\
% &= \prod_{j_{\text{filled}}}  Pr(\text{$j$ hire g},Wage^g_{j} | \text{$j$ filled})*Pr( \text{$j$ filled}) \\
%  &=  \prod_{j_{\text{filled}}}  Pr(\text{$j$ hire g} | Wage^g_{j}, \text{$j$ filled})*Pr(Wage^g_{j} | \text{$j$ filled})*Pr( \text{$j$ filled}) \\
% % &= \prod_j \frac{Pr(\text{$j$ hire g} | Wage^g_{j}) Pr(Wage^g_{j})}{Pr( \text{$j$ filled})}  \\
%\end{align*}

%\begin{align*}
% LL_{j_{\text{filled}}} &= \prod_{j_{\text{filled}}}  Pr(\text{$j$ hire g},Wage^g_{j})*Pr(\text{$j$ filled}) \\
% %&= \prod_{j_{\text{filled}}}  Pr(\text{$j$ hire g},Wage^g_{j} | \text{$j$ filled})*Pr( \text{$j$ filled}) \\
%  &=  \prod_{j_{\text{filled}}}  Pr(\text{$j$ hire g} | Wage^g_{j})*Pr(Wage^g_{j})*Pr( \text{$j$ filled}) \\
% % &= \prod_j \frac{Pr(\text{$j$ hire g} | Wage^g_{j}) Pr(Wage^g_{j})}{Pr( \text{$j$ filled})}  \\
%\end{align*}
%
%If the job is unfilled, its contribution is
%\begin{align*}
% LL_{j_{\text{unfilled}}} &= \prod_{j_{\text{unfilled}}}  Pr(\text{$j$ unfilled}) \\
%\end{align*}

Let $\mathcal{I}( \text{ j unfilled})$ be equal to one if firm $j$'s job is unfilled. Then the likelihood contribution of firm $j$ is given by:

\begin{align*}
LL_j  =  \prod_j ( Pr&(\text{$j$ hire g} | Wage^g_{j,t})*Pr(Wage^g_{j,t})*(1-Pr( \text{$j$ unfilled})))^{(1-\mathcal{I}(\text{$j$ unfilled}))}\\
&*(Pr(\text{$j$ unfilled}))^{\mathcal{I}(\text{$j$ unfilled})} \\
\end{align*}

Recall that Assumption \ref{lognormal} states that log wages are normally distributed. Therefore we can rewrite the log likelihood in terms of normal distributions. Let $\Phi_{0,\sigma_{\xi^g_t}}$ and $\phi_{0,\sigma_{\xi^g_t}}$ be the normal cdf and pdf respectively, with location zero and scale $\sigma_{\xi^g_t}$. Then the components of the log likelihood can be written as follows:

\begin{align*}
 % &= \prod_j \frac{Pr(\text{$j$ hire g} | Wage^g_{j}) Pr(Wage^g_{j})}{Pr( \text{$j$ filled})}  \\
Pr(\text{$j$ hire g} | Wage^g_{j,t} ) &= \Phi_{0,\sigma_{\xi^{g\prime}_t}}( WTP^g_{o,t} -  Wage^g_{j,t}) - (WTP^{g\prime}_{o,t} - W^{g\prime}_{o,t})) , \\
Pr(Wage^g_{j,t} ) &= \phi_{0,\sigma_{\xi^g_t}}( Wage^g_{o,t} - W^g_{o,t}), \text{ and}  \\
Pr( \text{$j$ unfilled}) &= \Phi_{0,\sigma_{\xi^g_t}}(-(WTP^g_{o,t} - W^g_{o,t})* \Phi_{0,\sigma_{\xi^{g\prime}_t}}(-(WTP^{g\prime}_{o,t} - W^{g\prime}_{o,t})). \\
\end{align*}



Thus we have parameters $WTP^g_{o,t}$ and $\sigma_{\xi^g_t}$ and equilibrium wage offers $W^g_{o,t}$ as a function of data. $Pr(\text{$j$ unfilled}) $ is the probability that we do not observe a match, which I impute from the JOLTS vacancy data.\footnote{Results do not appear sensitive to imputation method.} Predicted shares, or $Pr(\text{$j$ hire g} | Wage^g_{j,t} )$, are observed in the Census/ACS, and wages, or $Wage^g_{j,t}$, are imputed using the Census and SIPP as described in Section \ref{data} above. This concludes the first stage of estimation.

\subsubsection{Firm Identification}

%Another key difference from previous literature is that I use the distribution of transfers (wages) for identification.

%Because my research question is how non-wage utility varies with occupation fraction female, it is critical that I separately identify non-wage utility from utility from wages. 

% REPLACE
%The difficulty is that observetd transfers are the sum of an idiosyncratic and an aggregate component, as discussed above, and for counterfactuals these must be separately identified. The observed distribution of wages is also the result of optimization over these components and other model parameters to be estimated, specifically the willingness-to-pay for workers by gender and the variance of the distribution of unobserved heterogeneity in job amenities. 

%  MODEL INTRODUCTION PARAGRAPHS, MOVE TO CONCLUSION??
%%%%%%%%%%%%%%%%%%%%%%
%The identification strategy in this paper builds off empirical applications of transferable utility matching models to marriage markets (e.g. \citeA{Choo2006} and \citeA{Chiappori2015}), but differs in that I use data on transfers (wages) to separately identify worker and firm payoffs. In the marriage market, a lack of data on transfers means that only the sum of wage and non-wage utility is identified.\footnote{\citeA{Fox2008c} is able to estimate payoffs from the interaction of characteristics of both sides of the market but not full payoff functions.} 

%The first step is to assume that observed wages are the transfers between job and worker that clear the market. Although some transfers in the labor market may be non-wage, such as enhanced benefits, wages are a natural first-order approximation. Previous literature applying transferable utility to the marriage market has assumed a logit assumption for unobserved transfers because logit is most tractable. I model wages as lognormal in order to match the observed wage distribution, which is greater than zero and has a long right tail.\footnote{Unlike the logit case, the variance of the distribution of the maximum of lognormals depends on other model parameters to be estimated, so the variance needs to be jointly estimated unlike in \citeA{Salanie2014a} for example.} Second, I define an unmatched worker to be a worker who is unemployed or out of the labor force, and an unmatched job to be a posted vacancy that has remained unfilled for a period of time. Since vacancy data is potentially a poor proxy for what it means for a job to be unmatched, I try to limit the importance of the vacancy data in my estimation procedure by choosing a specification that in principle does not require it, by assuming non-negative profit on the job side. Thus identification relies more heavily on the wage distribution.

%I observe the share of jobs that are filled by men and women in each occupation, and the wages for those matches that do occur. From these moments I need to identify how much the firm is willing to pay for men and women in each occupation ($WTP^F_o$ and $WTP^M_o$) and the center of the wage offer distributions for men and women by occupation ($W^F_{o}$ and $W^M_{o}$). 


%%%%%%%%%%%%%%%%%%%%%%%%

%Separate identification of the willingness-to-pay and reservation wage parameters ($WTP^g_o$ and $W^g_o$) relies on the shape of the wage distribution, and in practice, the addition of vacancy data. 
%Intuitively, the shape of the observed wage distribution will vary according to where the reservation wage distribution is centered, while what portion of the reservation wage distribution we see depends on the $WTP^g_o$, that is jobs relative valuation of male or female workers. 
%To give intuition in terms of the likelihood function, $W^F_{o}$ and $W^M_{o}$ are pinned down by the $Pr(Wage^g_{j})$ component, while the relative difference between $WTP^F_o$ and $WTP^M_o$ is pinned down by the $Pr(\text{$j$ hire g} | Wage^g_{j} ) $ component, and the level of both $WTP^F_o$ and $WTP^M_o$ is pinned down by the denominator $Pr(\text{$j$ unfilled})$.


Figures \ref{workerpref} and \ref{firmpref} illustrate the separate identification of firms' willingness-to-pay ($WTP^F_{o,t}$ and $WTP^M_{o,t}$) and center of the wage offer distributions for men and women by occupation ($W^F_{o,t}$ and $W^M_{o,t}$) by simulating observed wage distributions under different parameter values. 

The top panels of Figure \ref{workerpref} and \ref{firmpref} show the wage distributions of men and women when they have identical preferences over occupations, and firms have no preference over male and female workers (implying $WTP^F_{o,t}=WTP^M_{o,t}$ and $W^F_{o,t}=W^M_{o,t}$). In this baseline case it can be seen that the wage distributions of men and women are identical. 

The lower panel of Figure \ref{workerpref} illustrates the impact of worker preferences on the observed wage distribution. In this simulation I assume that men prefer to work in the occupation more than women. The male preference for the occupation implies that the wage offers required to clear the market for men are lower than those of women ($W^F_{o,t} > W^M_{o,t}$). The lower equilibrium wage offers for men mean than firms tend to hire more men into the occupation because they are cheaper to hire, and also that the male wage distribution has more mass closer to zero. Note also that in the bottom half of Figure \ref{workerpref} the right tails of the male and female distributions still overlap substantially. The right tail of the wage distribution represents firms who are compensating workers for bad firm amenity draws, that is high $\xi^g_{j,t}$. Because, in this simulation, firms' willingness-to-pay for both men and women is the same ($WTP^F_{o,t}=WTP^M_{o,t}$), both men and women can match with the firms with the worst amenities and therefore appear in the right tail of the wage distribution.

The lower panel of Figure \ref{firmpref} shows the impact of firm preferences on the observed wage distribution. In the bottom panel, as opposed to the top panel, firms prefer to hire men, holding all else equal. As in Figure \ref{workerpref}, this means that more men are hired into the occupation, but the implication for the observed wage distribution is very different. Examining the bottom panel of Figure \ref{firmpref} we see that in the case of firms preferring to hire men, the bulk of the male wage distribution is shifted to the right, in fact well past the right tail of the female wage distribution. This is because firms are willing to pay more for men and therefore even with a worse dis-amenity draw for men $\xi^M_{j,t} > \xi^F_{j,t}$, will choose to pay more to hire the man. The only women who are hired into this occupation are those that match to firms with very good amenity draws for women, and are therefore cheap to hire.\footnote{Identification could be confounded if workers select into occupations based on individual productivity (as in a Roy model). Such selection would lead my model to overestimate gaps in wage offers between men and women in segregated occupations. For example, relatively high wages for men in nursing may reflect wage premia for individual talent, not just dislike of the occupation. The more segregated the occupation, the greater the bias. Unfortunately incorporating Roy-style heterogeneity into the model would make the estimation routine described in the next section intractable.}

% Since the model does not include selection on individual productivity, it is illustrative to discuss how that might bias the identification of wage offers and willingness to pay. If workers differ by individual productivity we might expect them to select jobs at which they are most productive and for the matching stability, firms would pay workers their full productivity and thus still be indifferent across individual workers. In this case if there are an equal number of men and women in an occupation, we would not expect any systematic bias in the parameter estimates. If there are say more men than women in an occupation, we would expect those women to be generally more productive than the mean and therefore paid higher than the mean. Thus for example in figure 1 where we see a small number of women, we are seeing only the most productive women, and therefore we might underestimate how much women like the occupation, misconstruing women being paid more to compensate for amenities vs. more productive women taking the job. In figure 2 we also see a small number of women in the occupation and we again might underestimate how much women prefer the occupation since were we to hire more women (say if firms did not have a preference for men) we would see women accepting lower wages as less productive women are taking the job.

% basically seeing a small number of people at a high wage, we don't know if it is productivity selection or extreme aversion to the occupation. A model that included selection on productivity would infer larger gaps in the equilibrium wage offer. average wage offer is lower then when we assume that everyone is the same. So a small number of workers would mean an even bigger gap in wage offer. Does not affect WTP gap because that is based on the tail which is not affected.

%The willingness-to-pay will be primarily pinned down by the shares and the right tail of the wage distribution (the most firms are willing to offer for workers of a gender in an occupation), while the wage offers will be primarily pinned down by the left tail of the wage distribution (the least workers are willing to accept for a job in an occupation).

Figures \ref{health} and \ref{sales} further illustrate identification by comparing the actual (not simulated) observed wage distribution in two occupations with very different estimated parameter values: ``Sales Representatives, Finance, and Business Services," and ``Health Service Occupations."

%In both occupations, we observe women are paid less than men, but in Sales Representatives, Finance, and Business Services, I estimate that this is due to firms being willing to pay less for women, while in Health Service Occupations, I estimate that is the result of women being willing to work for less.

In ``Health Services Occupations" in Figure \ref{health}, we see that the observed wage distribution for women is centered to the left of the male wage distribution. At the same time, the right tails of the male and female wage distributions broadly overlap. This is a pattern similar to the simulated data in Figure \ref{workerpref}. The model rationalizes this pattern by estimating that the wage offer distribution for women is lower than for men in this occupation ($W^F_{o,t} < W^M_{o,t}$) while the willingness-to-pay on the part of firms is roughly similar ($WTP^F_{o,t} = WTP^M_{o,t}$).

In ``Sales Representatives, Finance, and Business Services" in Figure \ref{sales} we see that the observed wage distribution mirrors more closely the simulated patterns in Figure \ref{firmpref}. We see that the bulk of the density of male wages are centered to the right of the bulk of the density of female wages, and that the long right tail of male wages extends well beyond the support of the female wage distribution. The model rationalizes this pattern by estimating similar wage offer distributions for men and women ($W^F_{o,t} = W^M_{o,t}$), but a higher willingness-to-pay on the part of firms for men ($WTP^F_{o,t} < WTP^M_{o,t}$). The intuition from these examples generalizes to other occupations where both the wage offer distributions and the willingness-to-pay parameters may differ by gender at the same time. 

%is centered much further to the right, implying women need to be compensated more highly to work in this occupation, so we would expect the center of the wage offer distribution to be higher. Therefore the shape of the observed wage distributions and the shares together identify the locations of the wage offer distributions ($W^F_{o}$ and $W^M_{o}$) for each gender and occupation. Wage offers are be needed to back out worker utility parameters.

%The willingness of firms to pay more for men in Sales Representatives, Finance, and Business Services manifests itself in the wage distribution through a long right tail male wages, well beyond the support of the female wage distribution. By contrast, in Health Service Occupations, the estimated willingness-to-pay gap is very small because the support of the right tails of the male and female wage distributions broadly overlaps.

%Without the use of vacancy data the willingness-to-pay of the firm would in fact be identified from the maximum wage observed for each gender. Because this relies heavily on a single data point that could be an outlier or sampling error, I prefer to use a specification with vacancy data to help pin down how many jobs remain unfilled in the far right tail.



%In summary using the model I am able to distinguish between the scenario in Figure \ref{sales}, where female workers are less valued than male workers, and the scenario in Figure \ref{health}, where female workers are cheaper to hire than male workers. In both cases the observed wages for women are lower, but the mechanisms are very different. Separately identifying reservation wages from willingness-to-pay is important because taking reservation wages as given is what allows me to estimate the worker side of the market. $W^F_{o}$ and $W^M_{o}$

%To illustrate identification consider the plots in Figures \ref{sales} and \ref{health}. These figures compare the reservation wage distributions implied by the model, with the wage distribution of matches, also implied by the model. The vertical lines denote the willingness-to-pay for male and female workers respectively, $\widebar{WTP}^M_{o}$ and $\widebar{WTP}^F_{o}$. 

%In both example occupations, the wage distribution predicted for matches (top panel) is lower for women than men. In the first example in Figure \ref{sales}, Sales Representatives, Finance, and Business Services, this is the result of firm preferences. In the second example in Figure \ref{health}, Health Service occupations, this is the result of worker preferences.

%In the bottom panel of Figure \ref{sales}, we see that in Sales Representatives, Finance, and Business Services, the reservation wages of women and men are centered at approximately the same location. This implies that men and women value the amenities of the occupation similarly and have similar outside options. Differences in the wage distribution of predicted matches are driven instead by firms being willing to pay higher wages for male workers. This manifests itself in the wage distribution by a long right tail male wages, beyond the support of the female wage distribution. 

%By contrast in Figure \ref{health}, in Health Service occupations, we see that the reservation wage distribution for men is much higher than for women. This means that men do not value this occupation, or have higher outside options. The high reservation wages for men drives the wage gap in this occupation. The estimated willingness-to-pay gap is very small because the support of the right tails of the male and female wage distributions broadly overlaps.




%Recall that worker utility depends on non-wage utility from occupation $o$, denoted $\alpha^g_o$, the fraction female through coefficient $\gamma^g$, and the reservation wages through the scale parameter of the worker heterogeneity in tastes for occupation $\sigma^g_{\eta}$, where $\frac{1}{\sigma^g_{\eta}}$ is the coefficient on the reservation wage.

%In the first scenario we will see a low center of the female wage distribution and with women being drawn primarily from the lower end of that distribution. In the second scenario we will see a large spread in the female wage distribution with women being matched relatively uniformly from all parts of it. In summary, the center of the observed distribution identifies the $W^g_o$ and the relative spread and shape identifies how many workers from that part of the distribution are being chosen relative to the other gender. 



%Depending on the relative values of $WTP^F_o$ and $WTP^M_o$ we will observe a higher share of jobs hiring women or men in a given occupation. Theoretically the scale of the willingness-to-pay parameters is identified also from the shape of the wage distribution, specifically from the degree of truncation on the right tail. In practice this source of variation does not perform well, in part because the right tail of finite sample wage data is censored and thin. The direct implication of the scale of both $WTP^F_o$ and $WTP^M_o$ is that we will see more or fewer jobs left unfilled, which is not directly observable without additional data. 



%To further illustrate identification consider the following plots of the reservation wage distributions for men and women, and the observed wage distributions for men and women. The vertical black lines denote $W^F_{o}$ and $W^M_{o}$, the centers of the offer distributions.

%\begin{center}
%%\includegraphics[width=.9\textwidth]{offerwages1reg1trunc}
%\includegraphics[width=.9\textwidth]{offerwages_occfinal8reg1trunc}
%\end{center}
%
%In this example\footnote{For more examples see Appendix.} we see that the willingness-to-pay from hiring a man must be higher than a woman since we observe male wages well to the right of female wages. Simultaneously the cost of hiring a man is relatively high because we observe that the male wage distribution is flattened


%\begin{align*}
%Pr(\text{$j$ hire M} | Wage_{Mj} ) &=exp(-e^{\frac{-(  \pi_{MY} - Wage_{Mj} -( \pi_{FY} - W_{FY}) )}{\sigma_{XY}}} ) \\
%Pr(Wage_{Mj} ) &=e^{-\frac{ W_{MY} -Wage_{Mj}   }{\sigma_{XY}}} exp(-e^{-\frac{W_{MY} -Wage_{Mj}  }{\sigma_{XY}}}) \\
%Pr(\text{$j$ hire F} | Wage_{Fj}) &=exp(-e^{\frac{-(\pi_{FY} - Wage_{Fj} - (\pi_{MY} - W_{MY}))}{\sigma_{XY}}} ) \\
%Pr(Wage_{Fj}) &=e^{-\frac{W_{FY} -Wage_{Fj} }{\sigma_{XY}}} exp(-e^{-\frac{W_{FY} - Wage_{Fj} }{\sigma_{XY}}}) \\
%Pr(\text{$j$ unmatched}) &=  exp (- \sum_{X \in M,F} e^{ \frac{\pi_{XY} - W_{XY}}{\sigma_{X}}  } ) \\
%\end{align*}

%\begin{align*}
%F_{1}(0|y_{3j}) &=exp(-e^{\frac{-(  \widebar{WTP}^M_o - \bar{W}^M_o - y_{3j}  -( \widebar{WTP}^F_o - \bar{W}^F_o - W_{FY}) )}{\sigma^g_{\xi}}} ) \\
%f_3(y_{3j}) &= \frac{1}{\sigma^g_{\xi}} e^{-\frac{ W_{MY} -y_{3j}  }{\sigma^g_{\xi}}} exp(-e^{-\frac{W_{MY} -y_{3j} }{\sigma^g_{\xi}}}) \\
%F_{-1}( 0| y_{2j}) &=exp(-e^{\frac{-(\widebar{WTP}^F_o - \bar{W}^F_o - y_{2j} - (\widebar{WTP}^M_o - \bar{W}^M_o - W_{MY}))}{\sigma^g_{\xi}}} ) \\
%f_2(y_{2j}) &= \frac{1}{\sigma^g_{\xi}} e^{-\frac{W_{FY} - y_{2j} }{\sigma^g_{\xi}}} exp(-e^{-\frac{W_{FY} - y_{2j} }{\sigma^g_{\xi}}}) \\
%\end{align*}


%Note also that once we know $ \widebar{WTP}^g_o$ and  $W^g_o$, we can solve for the number of jobs that remain unfilled. This will be important for the counterfactual because unfilled jobs becoming filled will affect the wage dynamics because less desirable jobs must compensate workers more for their dis-amenities $\xi^g_j$.

%Lastly, $W^g_o$ are taken as data to estimate the components of the worker utility function. 



%$$ \sigma^g_{\xi} ln(  \mu^g_o) - \sigma^g_{\xi}  ln(\mu_{0Y}) = \Pi^g_o = \pi^g_o - W^g_o $$
%$$ \implies  ln(\mu_{0Y}) = ln(  \mu^g_o) +\frac{W^g_o- \pi^g_o  }{\sigma^g_{\xi}} $$



%\begin{align*}
% \widebar{WTP}^M_o - \bar{W}^M_o - \bar{\xi}^M_j < 0 \hspace{5mm} \text{and} \hspace{5mm} \widebar{WTP}^F_o - \bar{W}^F_o -  \bar{\xi}^F_j < 0 \\
%\end{align*}

%The $Pr( \text{$j$ filled}) = 1 - Pr( \text{$j$ unfilled}) $ is then
%\begin{align*}
%1- Pr(\xi^F_j< -(\widebar{WTP}^F_o - \bar{W}^F_o), \hspace{3mm} \xi^M_j< -(\widebar{WTP}^M_o - \bar{W}^M_o )) & \\
%= 1 - \Phi_{0,\sigma^F_{\xi}}(-(\widebar{WTP}^F_o - \bar{W}^F_o)* \Phi_{0,\sigma^M_{\xi}}(-(\widebar{WTP}^M_o - \bar{W}^M_o))
%\end{align*}
%
%
%
%%Recall that
%%
%%$$ \widebar{Wage}^g_j = \bar{W}^g_o + \bar{\xi}^g_j $$
%
%Then other terms in the likelihood are as follows:






% in the notation of \citeA{Amemiya1985} and my notation.

%Since we observe log wages $\widebar{Wage}^g_j$, the expectation of observed wages suggests itself as an estimator for $\bar{W}^g_o$. Unfortunately there is a wrinkle to this strategy. There is a selection problem in that . 



\subsection{Step 2: Worker Estimation and Identification}




%This equation \ref{eq:1} cannot be estimated in the cross section because there are only as many moments as occupations, and I must estimate $\gamma^g$ and $\sigma^g_{\eta}$ in addition to an $\alpha^g_o$ for each occupation. Therefore In order to better match time variation in shares, a time effect $\beta^g_t$ is added. 

In the second stage of estimation, I take the centers of the wage offer distributions estimated in the previous step, ($W^g_{o,t}$), and treat them as data in occupation level regressions. I pool all six cross sections of Census/ACS data (1960-2012) into a single regression for each gender. This allows me to estimate the occupation-specific intercepts, $\alpha^g_o$ and use time variation in fraction female and reservation wages to identify their coefficients in the worker utility function, $\frac{\gamma^g}{\sigma^g_{\eta}}$ and $\frac{1}{\sigma^g_{\eta}}$. Following \citeA{Berry1994}, I add an error term $\epsilon^g_{o,t}$ to represent changes over time in the utility of workers due to changes in unobserved occupation attributes. Any changes not due to movement in the fraction female or the wage offers will be captured by $\epsilon^g_{o,t}$. Let $s^g_{o,t}$ be the share of workers of gender $g$ choosing occupation $o$ in time $t$, and $s^g_{N,t}$ the share of non-employed workers. Then the final estimating equation is as follows.

\begin{align*}
ln(s^g_{o,t}) - ln(s^g_{N,t}) =   \frac{  \alpha^g_o + \beta^g_t  + \gamma^g F_{o} +  W^g_{o,t} + \epsilon^g_{o,t}}{\sigma^g_{\eta}}  \\
\end{align*}

%In the sections below I introduce several strategies to identify the parameters $\frac{\gamma^g}{\sigma^g_{\eta}}$ and $\frac{1}{\sigma^g_{\eta}}$. 

Unfortunately using time variation means that estimates of $\frac{\gamma^g}{\sigma^g_{\eta}}$ and $\frac{1}{\sigma^g_{\eta}}$ likely suffer from omitted variable bias. Changes over time in both the gender ratio and the reservation wage are likely correlated with changes in unobserved occupation attributes $\epsilon^g_{o,t}$. For example if an occupation is becoming more family friendly over time, and this causes more women to enter the occupation, the coefficient on fraction female will be biased upward for women. Similarly if occupation amenities deteriorate over time, this may be correlated with increases in wages, causing a downward bias on the coefficient on the reservation wage $W^g_{o,t}$. I therefore need instruments to get clean variation in $W^g_{o,t}$ and fraction female $F_{o,t-1}$ to identify the worker utility parameters.



% \subsection{Fixed Effects Specification}
 
% Recall from equation \ref{utility} that worker utility is given by 
%
%$$\bar{u}^i_o =  \alpha^g_o + \gamma^g_o F^g_o + \bar{W}^g_o  + \bar{\eta}^i_o $$

 
% Let worker utility in occupation $o$ in time $t$ for either men or women be given by
%\begin{align*}
%u^{i}_{o,t} &= \alpha_o +   \gamma_t +  \gamma F_{o,t} + \delta log(W^g_{o,t}) + \eta^i_{o,t} + \epsilon_{o,t}   \\
%\end{align*}



%One concern with estimating this equation is that, leaving aside the structural interpretation of the discrete choice model, it is unclear if the coefficients reflect labor demand or labor supply. Skill requirements of occupations for example is one factor, unmodeled on the firm side, that could lead to a spurious negative correlation between the wage and the share of workers in an occupation. Simultaneously, there is the problem of omitted variable bias due to correlation between time varying occupation amenities $\epsilon_{o,t}$ and $F_{o,t}$ and $W^g_{o,t}$ discussed above. Ideally instruments driven by labor demand will also be uncorrelated with $\epsilon_{o,t}$ and therefore solve the endogeneity problem.

%In order to trace out the labor supply curve, variation in the fraction female and the wage due to labor demand shocks is needed.

% Instruments that use shocks to labor demand will ensure that $\gamma$ and $\delta$ reflect labor supply. 






% \footnote{There are 138 firm side parameters estimated in the first stage in each of 6 waves of data, for a total of 828 parameters. There are 86 worker parameters estimated in the second stage over all waves of data.}

%Combining the two estimation steps would be feasible but would be less natural because the worker side is well suited to gMM and the job side to MLE. It would be possible to estimate the worker mean utility levels simultaneously in the job side MLE, however the incorporation of instruments would still require a separate second stage of estimation, unless joint normality is assumed for the reduced form errors to allow for LIML. Estimation of both sides simultaneously using gMM would also be possible, but in this case information from the shape of wage distribution would need to be collapsed into an arbitrary set of higher order moments, which is not as desirable as using the full information in MLE.

%%%%%In the following sections I lay out the assumptions that allow the ML estimation of the willingness-to-pay and reservation wages. The reservation wages are then taken as data in the estimation of worker utility parameters, which is described in the following section.

%The job MLE produces the locations of the unconditional reservation wage distributions faced by workers in equilibrium as they make their occupation choices. These parameters are then taken as data in the worker estimation. 



%\footnote{Making this alternative assumption would not substantially change the identification strategy, but would reduce reliance on the tails of the wage distribution and increase reliance on the shape more generally.} 

%In a fully specified firm model this could make sense with a complementary production function and other inputs, but without firms such complementarities should already be approximated in the job level willingness-to-pay term. 



%The joint probability of hiring male or female and hiring any worker at all can be broken out into conditional components as follows, in the case of hiring female:

%e use the distribution of $\bar{\xi}^F_j$ conditional on $\widebar{WTP}^F_o - \bar{W}^F_o + \bar{\xi}^F_j>0$:

%\begin{align*}
%Pr(j \in o \text{ choose } F , j \text{ filled}) &= \\
%  Pr(\widebar{WTP}^F_o - \bar{W}^F_o + \bar{\xi}^F_j \geq \widebar{WTP}^M_o - \bar{W}^M_o + \bar{\xi}^M_j | \widebar{WTP}^F_o - \bar{W}^F_o + \bar{\xi}^F_j \geq 0) ( 1-Pr ( \bar{\xi}^F_j < {-\widebar{WTP}^F_o + \bar{W}^F_o} ))   \\
%%&= \int_{0}^{\infty} \frac{ exp(-e^{-(\widebar{WTP}^F_o - \bar{W}^F_o - \widebar{WTP}^M_o - \bar{W}^M_o + \xi^F_j)} ) e^{-\xi^F_j} exp(-e^{-\xi^F_j})}{ 1-exp(-e^{\widebar{WTP}^F_o - \bar{W}^F_o} ) } d\xi^F_j\\
%%&= \frac{exp(\widebar{WTP}^F_o - \bar{W}^F_o)}{exp(\widebar{WTP}^F_o - \bar{W}^F_o) + exp(\widebar{WTP}^M_o - \bar{W}^M_o)} (1 - exp(- e^{\widebar{WTP}^F_o - \bar{W}^F_o} - e^{\widebar{WTP}^M_o - \bar{W}^M_o})) \frac{1}{1-exp(-e^{\widebar{WTP}^F_o - \bar{W}^F_o} ) } \\
%\end{align*}

%\begin{align*}
%Pr(j \in Y \text{ choose } F ,  \widebar{WTP}^F_o - \bar{W}^F_o + \xi^F_j \geq 0) &=Pr(j \in Y \text{ choose } F |  \widebar{WTP}^F_o - \bar{W}^F_o + \xi^F_j \geq 0) * Pr(  \widebar{WTP}^F_o - \bar{W}^F_o + \xi^F_j \geq 0)\\
%&= \frac{exp(\widebar{WTP}^F_o - \bar{W}^F_o)}{ exp(\widebar{WTP}^F_o - \bar{W}^F_o) + exp(\widebar{WTP}^M_o - \bar{W}^M_o)}  (1 - exp(- e^{\widebar{WTP}^F_o - \bar{W}^F_o}) - exp(- e^{\widebar{WTP}^M_o - \bar{W}^M_o})) \\
%\end{align*}




%\subsubsection{Likelihood Function}
%%The parameters that we need to identify for the job side of the market are the type level market clearing common component of wages $W^g_o$, the willingness-to-pays $WTP^g_o$,  and the scale parameter of the job amenities shock $\xi^g_j$, which is $\sigma^g_{\xi}$.
%
%Parameters of the model are
%
%$$ \theta = \{ WTP^g_o,  \alpha^g_o, \gamma^g, \sigma^g_\eta, \sigma^g_\xi  \} $$
%
%Recall that $WTP^g_o$ are the firms' willingness-to-pay parameters, $\alpha^g_o$ the workers' non-wage utility unrelated to the fraction female, $\gamma^g$ the value of the fraction female to the worker, $\sigma^g_\eta$ the scale of the worker taste heterogeneity, and $\sigma^g_\xi$ the scale of the job dis-amenities heterogeneity. The unconditional centers of the reservation wage distributions, $W^g_o$, are reduced form outcomes determined in equilibrium as a function of the primitives $\theta$. Nevertheless recovering the $W^g_o$ from the observed conditional wage distributions is critical to recovering the fundamental parameters in the worker's utility function.
%



%%%Recall the log wage specification, $$ \widebar{Wage}^g_j = \bar{W}^g_o -  \bar{\xi}^g_j $$

%\footnote{The distributions of the $\epsilon$ given they are observed at the maximized utility only is the same as the unconditional distributions \cite{DePalma2007}, therefore we can estimate the scale parameter  $\sigma^g_{\xi}$ using the individual wage data as suggested in \citeA{Salanie2014}. The variance of observed wages $w^g_j = W^g_o -  \xi^g_j$ will be $var(W^g_o) + var(\xi^g_j)$. However since $var(W^g_o)$ is a constant within $X,Y$ and the $\xi^g_j$ are independent of $X,Y$, we can estimate $\sigma_{X}$ with the sample variance of $w^g_j$ within each type $X,Y$. }

%Since we observe $w^g_j$, the worker-job match wage, for the matches that are made, it appears we could estimate the $\sigma^g_{\xi}$ from the scale of the observed wage distribution (especially given that the scale of a maximum of extreme value distributions is the same as scale of the original distributions\footnote{See Appendix}). However it is not possible to take this approach because of the positive willingness-to-pay product assumption, which implies a truncation of the job payoffs, and therefore a truncation of the observed wages at a truncation point for each type ($ \pi^g_o + W^g_o + \xi^g_j \geq 0 \implies \xi^g_j \geq - \pi^g_o - W^g_o$). This truncation point is unknown because $W^g_o$ is unknown.


%\subsection{Identification}
%I first discuss the identification in the first stage maximum likelihood estimation of the firm parameters ($\widebar{WTP}^g_{o}$ and $\sigma^g_{\xi}$) and centers of the equilibrium wage offer distributions ($W^g_{o}$). Then I discuss identification in the second stage instrumental variables regression of the worker parameters ($\alpha^g_o$, $\gamma^g$, and $\sigma^g_{\eta}$).
%ADD MORE INTRO TO IVs HERE???


\subsubsection{Worker Identification}

%In this section I introduce Bartik-style instruments that exploit variation in the industry composition of occupations. These instruments allow me to isolate changes over time in the fraction female and wage by occupation that are caused by industry level changes. I assume industry level changes do not affect labor supply to occupations except through changes to occupation wage and gender composition directly. This assumption allows me to identify the labor supply parameters of the worker.

%Recall that the worker's occupation choice depends on a fixed preference for occupations, as well as wages and the fraction female, which may change across cohorts of workers. Since the occupation fixed effects ($\alpha^g_o$) capture the fixed value of the occupations to men and women, the identification concern is that we would expect changes in the fraction female and the wage offer to be correlated with changes in the value of the occupation not captured in the $\alpha^g_o$. 

In this section I introduce Bartik-style instruments to identify the labor supply parameters of the worker. Occupations exist in a variety of different industries, and these industries have different wage offers and fraction females, and also experience different changes in wage offers and fraction female. The idea behind my first two instruments is to use the exposure that occupations have to changes in industries to predict changes in wage offers and fraction female by occupation. The key assumption is that wage and gender composition are the only attributes of occupations that are affected by changes to industries. 

For example, suppose that wages in manufacturing occupations, other than the occupation being instrumented for, are going up over time. This could be due to changes to the output market or production technology. Then workers might expect to see an increase in the wages of administrative assistants working in the manufacturing industry. The more administrative assistants work in manufacturing, the more we would expect wages to go up in the occupation. Identification will be threatened if wages are going up because the job amenities are getting worse for all occupations in manufacturing. Likewise if manufacturing is becoming more female over time, a worker might view administrative assistance as a more female occupation since they are more likely to work with women. Again the more administrative assistants work in manufacturing the greater the impact. Identification is threatened if all jobs in manufacturing are becoming more welcoming to women at the same time.



%industry level changes are driven by labor demand and not confounded by labor supply changes. 

%This type of instrument is commonly called a Bartik instrument.
 
 %\footnote{The instrument is similar in style to \citeA{Autor2013e} where import demand is assumed uncorrelated across countries but import supply correlated across countries. It is also similar to Hausman instruments where product cost shocks are assumed correlated across markets but demand shocks are independent \cite{Hausman1996a, Nevo2001}.} 



%Intuition:
%Occupations exist in different industries
%Industries experience growth and changes to fraction female and wages We can predict changes in occupation fraction female and wages using changes in industries
%(Using labor demand shifts to trace out labor supply)
%Threats to identification:
%Worker preferences for industry vary over time
%Changes in worker preferences for occupations are correlated across occupations

%The identifying assumptions for the first set of instruments are that changes in industry wage levels and gender ratios over time are uncorrelated with changes in how workers value industries, and that changes in occupation attributes are independent across occupations. The identifying assumptions for the second pair of instruments is that changes in industry size over time that are correlated with initial wage and gender ratios, are uncorrelated with changes in the worker valuation of the industry. Lastly I introduce an instrument that interacts the initial gender ratio by occupation and the relative growth rates of men's and women's labor force participation over time. The identifying assumption is that changes in unobserved occupation attributes are independent across occupations.

 %two main variations on a Bartik instrument. The first, which can be constructed in the cross section, are the occupation wage levels and gender ratios that could be predicted solely from the industry composition of the occupation. The key assumption for these is the independence across occupations of worker preferences over industries. The second, which relies on panel data, are the changes in wage and gender ratio due solely to changes in industry size or the relative labor force size of men and women.
 

%\subsubsection{Instrumental Variables}
%\subsubsection{Changes Over Time in Industry Wage and Gender Ratio}
%The first set of instruments uses industry variation in wages and gender ratios, and variation in the presence of industries within occupations, to predict occupation level wages and gender ratios. For example, the reservation wage for administrative assistants will be the weighted sum of the wages by industry for all industries that exist in that occupation.

%The basic idea is that various labor demand factors may cause variation in the reservation wage or gender ratio by industry. For example production technology or consumer demand may lead reservation wages for administrative assistants in manufacturing to be higher than in retail trade, and the gender ratio in that occupation to be highest in professional service and lowest in manufacturing. 

For the instrument to work it is also necessary that each industry contain multiple occupations. At the level of aggregation I use, 14 major industries,\footnote{Industries used are aggregates of the harmonized IPUMS codes of ind1990. Industries are as follows: Agriculture, Forestry and Fisheries; Mining; Construction; Manufacturing; Transportation, Communications, and other public utilities; Wholesale Trade; Retail Trade; Finance, Insurance, and Real Estate; Business and Repair Service; Personal Services; Entertainment and Recreation Services; Professional and Related Services; Public Administration.} every industry has employment in almost all occupations. The predicted fraction female and wage in occupation $o$ in time $t$ are the weighted sum across industries of industry level fraction female and wage as follows:

\begin{align*}
\hat{F}_{o,t} &= \sum_I  p_{Io,1950}*\hat{F}_{Io,t} \\
\hat{W}_{o,t} &= \sum_I  p_{Io,1950}*\hat{W}_{Io,t} 
\end{align*}

Where $p_{Io,1950}$ is the fraction of occupation $o$ in industry $I$ fixed in 1950, prior to the sample data, and $\hat{F}_{Io,t}$ and $\hat{W}_{Io,t}$ are the fraction female and wage, respectively, in industry $I$ in year $t$, excluding workers in occupation $o$. I exclude the occupation that is being instrumented for so that changes to that occupation will not contaminate estimates of changes in industry wages and fraction female. 


%\footnote{Note that in order for industry level fraction female or wage level to be estimated excluding occupation $o$, it is necessary that each industry contain multiple occupations. In fact at the level of aggregation I use, 14 major industries, every industry has employment in almost all occupations.} 

 %Fraction female is measured as the fraction female in the older generations (ages 36-65) in the Census cross section. Industry wages are measured by gender as lifetime incomes of those beginning their careers in a given industry as simulated using the Census and SIPP data described above.

%Pooling all six waves of data (1960-2010) is critical to the quality of the instruments in two ways. First, the industry composition can be fixed in 1950, prior to the sample data. This alleviates any concern that changes in industry composition are correlated with changes in unobserved occupation amenities, which is critical for the validity of the instrument \cite{SorkinBartik}. 

%Second, occupation fixed effects absorb the mean utility coming from the industry composition. Therefore we must only assume that \textit{changes} in how workers value industries are independent across occupation, but changes in industry demand are correlated across occupations. 



%One might worry that the reason reservation wages are increasing in manufacturing is due to a compensating differential from exactly the unobserved amenities that confound the OLS regression on reservation wages. To avoid this concern, the reservation wage level in the industry is calculated excluding the reservation wages in the occupation being instrumented. So for example in the case of administrative assistants, manufacturing wages would be calculated taking the mean over only other occupations in manufacturing. 

%The instrument will be invalidated if changes in industry amenities are correlated across occupation. For example, if the working conditions for all workers in manufacturing are getting worse over time, and this is causing the working conditions for administrative assistants in the manufacturing sector to get worse over time, then the instrument will predict an increase in wages among administrative assistants that is correlated with the change in working conditions.



%\subsubsection{Growth in Industry Employment}
My second instrument uses changes in the relative size of industries to predict changes in occupation wages offers and fraction female. For example, if manufacturing is declining, then administrative assistants will be less impacted by the prevailing fraction female and wage level in manufacturing. Workers considering the occupation will see that more jobs are now in the service sector, not manufacturing, and update their expectations of wages and fraction female accordingly. The key identifying assumption is that occupations are the same regardless of what industry they are in, except for expected wage and fraction female. The predicted fraction female or wage in occupation $o$ and time $t$ respectively can be written as a weighted sum over industry, of initial industry fraction female or wage, times the growth rate by industry as follows:

\begin{align*}
\hat{F}_{o,t} = \sum_I p_{Io,1950}*F_{Io,1950}*\frac{size_{Io,t}}{size_{Io,1950}}\\
\hat{W}_{o,t} = \sum_I p_{Io,1950}*W_{Io,1950}*\frac{size_{Io,t}}{size_{Io,1950}}
\end{align*}


%The growth rate of industries since 1950 is the treatment and the initial wages and industry fraction female\footnote{Industry composition by gender is set in 1950. Wages are set in 1960 because the sample size in the 1950 Census is too small to calculate lifetime wages} are the exposure to the treatment. Changes in employment by industry is a fundamentally different source of variation. 

%The instrument predicts changes in occupation wage and gender ratio over time due to t, not including the impact of these factors on changes in wage or gender ratio since these are fixed in the initial period, and therefore controlled by the occupation fixed effects.

I fix both the occupation*industry composition ($p_{Io,1950}$) and the industry fraction female and wage ($ F_{Io,1950}$ and $ W_{Io,1950}$ at 1950. $\frac{size_{Io,t}}{size_{Io,1950}}$ is the growth in industry $I$ excluding occupation $o$ relative to 1950.



%The changes over time in the predicted $\hat{F}_{o,t}$ are driven by changes in industry size.

%One requirement for validity of the instrument is that the growth in the size of industries, and therefore the exposure of occupations to the wages and gender ratios predicted by those industries, is due to product demand or productive efficiency changes over time. So for example say that demand for manufacturing is growing over time and therefore more jobs are available in the occupations in manufacturing. We would expect the wages and gender ratios in occupations found in manufacturing to be more and more reflective of the growing prevalence of manufacturing.

%The main threat to validity of the instrument is if the wages and gender ratio in manufacturing are reflective of underlying amenity values that are common across all occupations in manufacturing. In that case, even if the growth of an industry is exogenous, its increased prevalence in an occupation will be correlated with a change in the amenities in that occupation. This is less important to the extent that my paper is concerned with occupation amenities not industry amenities.

%This instrument would be invalid if industry growth rates are driven by changes in occupation amenities, and these changes in amenities are correlated with wage levels or fraction female in the initial period. We might worry, for example, that industries with initially low wages grow at a faster rate and this growth is due to the expansion of unobserved amenities correlated with the low wages in the initial period. If this is the case then we would be predicting low wages in occupations that 

%Workers could be valuing being a manager in manufacturing more and more and this could be growing manufacturing and therefore impacting the wage and gender ratio in management and also how much workers value management as an occupation. As long as the things workers like about being a manager in manufacturing are unique to being a manager and not correlated across all occupations in manufacturing, then we should be capturing only labor demand factors using this instrument since the initial wages and gender composition are captured by the fixed effects. 

% I could also argue the independence of industry amenities across occupation here but I do not think it is necessary given the paragraph below.

The last Bartik-style instrument predicts changes in the fraction female of occupations over time based on the initial fraction female in occupations in 1950, and the relative growth rates of male and female labor force participation. The idea is that as women enter the labor force they are more likely to enter occupations that historically have had more women.\footnote{This is similar to instrumenting for immigration patterns based on overall flows of immigrants and initial shares \cite{Altonji1991a}.} The instrument is constructed as follows. Let the number of men and women employed in all occupations except $o$ in time period $t$ be $\#M_t$ and $\#W_t$. I define the relative growth in female vs. male employment $r_t$ as

$$ r_t = \frac{ \frac{\#F_t}{\#M_t} }{ \frac{\#F_{1950}}{\#M_{1950}} }$$
%  due to higher fixed values of those occupations ($\alpha^o_g$

Let $F_{o,1950}$ be the fraction female in the occupation in 1950. Then the fraction female in occupation $o$ predicted by the instrument in time $t$, $\hat{F}_{o,t}$ is as follows:

$$\hat{F}_{o,t} = F_{o,initial} * r_t $$

The instrument will be invalid if, for example, occupations that had high fraction female in 1950 are becoming relatively more attractive to women over time, and this change is driving the increase in female labor force participation. It is more likely that the relative increase in female labor force participation was the result of a broader change in norms and the value of home work than driven by changes to female-dominated occupations.
%%  ADD A CITE HERE AS TO WHAT ACTUALLY CAUSED RISE IN WOMEN'S LFP
 
 %I assume that the growth in the ratio of employed to non-employed in all other occupations will not be correlated with growth in amenities in the instrumented occupation. 
% So for example we see a high fraction female for teaching non-postsecondary in 1960 and also growth in the number of female teachers over time. Under the assumption of the instrument the growth in the number of female teachers is due to an overall increase in female labor force participation and the fixed amenities component of teaching being attractive for women. However, if the increase in female labor force participation is driven by growth in the amenity value of teaching, and the amenity value growth is correlated with the initial amenity value, then the instrument is endogenous.


%\subsubsection{Changes Over Time in Employment by Gender}
%Another source of variation made possible by the panel is variation in the relative value of employment and non-employment for men vs. women over time. Fixing occupation gender ratios in the initial period, the instrument predicts changes in occupation gender ratios due only to the gender ratio of those employed in all other occupations. Assuming changes in occupation attributes are independent across occupation, growth in the ratio of employed to non-employed in all other occupations will not be correlated with growth in amenities in the given occupation. 

%The identifying variation is changes in the relative value of home vs. work by gender, projected onto current occupation gender ratios based on previous gender ratios. This is similar to instrumenting for immigration patterns based on overall flows of immigrants and initial shares \cite{Altonji1991a}. One might expect initial fraction female to impact flows into an occupation through occupation attributes that are fixed over time. Since fixed attributes are controlled by fixed effects, variation over time is assumed to be due only to changes in labor force attachment of men relative to women relative to the initial period. For recent evidence of changes in labor force attachment by gender see \citeA{Albanesi2017}. 





%\subsubsection{Firm Willingness-to-Pay Parameters}
Lastly, I include the willingness-to-pay estimates $WTP^g_{o,t}$ from the firm side of the model as instruments. The $WTP^g_{o,t}$ are a measure of how much firms value male vs. female workers, and therefore should be uncorrelated with changes in unobserved occupation amenities.

%should be uncorrelated with unobserved amenities, assuming amenities are fixed, and therefore are a good proxy for labor demand side factors that will shift the reservation wage and the relative number of men or women through firm preference for hiring men vs. women, whether this be consumer demand driven, productivity differences, or discrimination.

%The ratio of $\frac{WTP^M_o}{WTP^F_o}$ is also used as a proxy for labor demand.

\section{Model Estimates} \label{results}
\subsection{Model Parameters and Fit}

%Model fit is good for the fraction female observed in the Census/ACS data in the cross section, which is targeted in estimation. There is some discrepancy when I allow the model to endogenously update the fraction female over time. This is because workers do not necessarily stay in their starting occupation for their lifetime, as assumed in the simulations. Histograms comparing the model predicted and actual lifetime income distributions can be found in the online appendix, and generally show a good fit.

%The moments used in estimation are shares of workers by gender and occupation, and wage distributions. Therefore these are good moments to compare to model predicted moments to examine model fit. 

The fraction female by occupation observed in the Census/ACS data matches the model predicted fraction female for the young cohort, since shares of workers by gender and occupation are moments targeted in estimation. There is some discrepancy when I allow the model to endogenously update the overall fraction female over time. This is because workers do not necessarily stay in their starting occupation for their lifetime, as assumed in the simulations. Histograms comparing the model predicted and actual lifetime income distributions can be found in the online appendix, and generally show a good fit, since wage distributions by occupation and gender are also targeted in estimation.

%and the endogenous updating of the fraction female is not a moment targeted in the model.

%The fit of the model can be assessed by comparing the top left panel, which is observed fraction female by occupation in the Census data,\footnote{Both observed and simulated fraction females are for age ranges 35-64} to the right and bottom panels, which are model simulations. Note that the model simulations do not exactly match the Census data patterns in the top left because the dynamic updating of the fraction female across cohorts was not a moment targeted by the model. In the data workers also do not necessarily stay in their starting occupation for their lifetime.

Table \ref{modelestimates} shows the results of the first stage of estimation. Parameters $WTP^g_{o,t}$ and $W^g_{o,t}$ were estimated separately for each of the six cohorts of workers, but the results in the table are the averages within occupation across these six cohorts ($\widebar{WTP^g_{o}}$ and $\widebar{W^g_{o}}$) for ease of interpretation. The first column is the observed fraction female, varying from .02 to .89, and all 34 occupations are ordered from highest fraction female (administrative support) to the lowest fraction female (construction and extraction). In the second column we see the ratio of the average female wage offer to male wage offer ($  \frac{\widebar{W^F_{o}}}{\widebar{W^M_{o}}}$, which varies from .36 to 2.66, and in the third column we see the ratio of the firms' wilingness-to-pay for women vs. men ($ \frac{\widebar{WTP^F_{o}}}{\widebar{WTP^M_{o}}}$), which varies from .37 to 1.14. 

In general, women have lower wage offers than men ($  \frac{\widebar{W^F_o}}{\widebar{W^M_o}} < 1$) and are less valued by firms ($ \frac{\widebar{WTP^F_o}}{\widebar{WTP^M_o}} < 1$). In general, the higher the fraction female, the lower the wage offers for women are in the occupation, relative to men. On the other hand, firms tend to be willing to pay more for women relative to men in high fraction female occupations. Intuitively these results are consistent with the idea that women like, and are most valued at, highly female-dominated occupations such as Administrative Support, Financial Records Processing Occupations, and Health Service Occupations. Similarly, women dislike, and are least valued at, highly male occupations such as Engineers, Architects, and Surveyors, Mechanics and Repairers, and Construction and Extraction.

%This is consistent with a Roy model with women sorting into occupations in which they are more valued by jobs.

% (thought not through a wage mechanism necessarily since both observed wages and reservation wages are lower for women in female-dominated occupations).

It has been noted that both men and women have lower wages the higher the female share in an occupation (see eg. \citeA{Macpherson1995a, Levanon2009, Addison2017,Harris2018}). Unlike the mean wage by occupation, the centers of the wage offer distributions $W^g_o $ control for the fact that in observed wages, we see only the most attractive firms filling their vacancies in each occupation, and by the workers that satisfy the firm's payoff maximization. Thus, using the model allows me to look at the correlation between wage offers and fraction female unconditional on firm selection effects, which would downward bias any estimated correlation. I find that female wage offers are negatively correlated with the fraction female (correlation coefficient $-0.77$), while male wage offers are positively correlated with the fraction female (correlation coefficient $0.6$), despite no statistically significant correlation between my raw estimates of lifetime income and fraction female.

% Nathan comment: spell out implications

%and reflects the common component of the reservation wage to every worker of gender $g$ at any job in occupation $o$. 

%Estimation results indicate that in contrast to observed wages, the model estimated reservation wages for men, $\bar{W}^M_o $, are positively correlated with the fraction female (estimated correlation coefficient of $0.6$), where the unit of observation is an occupation-year. The correlation coefficient between fraction female and observed labor income is $-0.16$. Reservation wages for women ($\bar{W}^F_o $) are negatively correlated with the fraction female (correlation coefficient $-0.77$), just like observed wages (correlation coefficient $-0.12$). 



% \cite{Baker2001} for canada evidence and non-linearities

%Reservation wages are more closely related to the value that the worker has for an occupation than observed wages, since observed wages reflect additional selection by the firm. This explains why reservation wages could be so strongly correlated with the fraction female, but lifetime income not so much. 

%Although reservation wages for men in female-dominated occupations are quite high, in the observed wage data we see only those jobs within the occupation that are so attractive to men that the men can be hired an comparable wages to women. In contrast, the higher the fraction female in an occupation, the lower the reservation wage to women $\bar{W}^F_o$, but since women are cheap to hire relative to men, we see even the least attractive jobs being filled by women. 

Lower wage offers for women in female-dominated occupations could be consistent with a female preference for working with women producing a compensating differential, but could also be the result of a strong preference for certain occupations. Similarly, the high wage offers for men in female occupations could be consistent with a male preference against working with women or strong tastes for occupations. In the next section, I present the results of the instrumental variables regression of worker utility, and discuss whether the fraction female has a causal impact on worker utility.

% this follows mechanically from the model maybe??

%*** include table comparing $\bar{W}^g_o$ to the mean observed lifetime income by occupation and gender?




%\begin{tablenotes}
%      \footnotesize
%  things
%    \end{tablenotes}
%blah
%\captionof{figure}{my table}
%\end{minipage}





%*** include the parameter table here with confidence intervals comparing to the counterfactual wage vector

%My first step is to see if my simulation of the model over cohorts can match the actual evolution of the fraction female over cohorts. The potential problem is that in real life workers do change careers after the first period, meaning that my model predicted older generation gender ratio will not match the actual observed older generation gender ratio exactly. I need to see how good of an approximation my model is to reality before I can interpret any counterfactual results.



\subsection{Worker Preference Results}
As discussed above, I estimate worker utility parameters by regressing worker utility on occupation fixed effects, model predicted wage offers, and the fraction female, using Bartik-style instruments for the fraction female and wage offers. Tables \ref{IV1} and \ref{IV2} show the results both instrumented and un-instrumented fixed effects regressions by gender described above. The first stage has a Kleibergen-Papp F statistic of around 10 for women and 8.6 for men.\footnote{I report only the linear specification for men and the cubic specification for women because these have the highest first stage F statistics in the IV specification, but results for other functional forms including a beta distribution are qualitatively similar.} Estimation is done using limited information maximum likelihood for robustness to weak instruments, but two stage GMM results are similar, and standard errors clustered at the occupation level.\footnote{I expect errors correlated within occupation due to occupation fixed effects and possible differences in model fit across occupations.} 

%I report only the linear specification for men and the cubic specification for women because these seem to fully capture the functional form and obtain the highest first stage F statistics in the IV specification.


% 10 \% increase in wages implies a 10/100 increase in log utility
% then 4* change in F implies a 10/100 increase in log utility
% then change in F must be 10/400= 2.5% or .025... 10/200 would be .05

%\footnote{Denoted ``KP rk F" in the tables.}

%The first column is an un-instrumented fixed effects regression. The second column (IV) includes the instruments using variation in industry wage and fraction female over time, variation in industry size over time, variation in male and female labor force participation, and firm willingness-to-pay parameters $WTP^g_o$. This is the preferred specification because it has the strongest first stage, achieving a Kleibergen-Papp F statistic of around 10 for women and 8.6 for men.\footnote{Denoted ``KP rk F" in the tables.}

%Pooling all six waves of data (1960-2010) allows for the addition of occupation and time fixed effects. These fixed effects should improve the quality of the instruments in two ways. First, the industry composition can be fixed in 1950, prior to the sample data. This alleviates any concern that changes in industry composition are correlated with changes in unobserved occupation amenities. Second, occupation fixed effects makes the assumption that industry amenities are independent across occupation easier to swallow since the assumption becomes industry amenities are independent net of mean occupation utility. 

%Finally, a classic Bartik IV exploiting change over time in the the size of industries can be used to instrument for $W^g_o$ and the gender ratio respectively. In this case the growth rate of industries since 1950 is the treatment and the initial wages and industry composition by gender\footnote{Industry composition by gender is set in 1950. Wages are set in 1960 because the sample size in the 1950 Census is too small to calculate lifetime wages} are the exposure to the treatment. Results for these instruments are below.


%\input{Stata_pvfirstW_sex1_}
%\input{Stata_pvfirstg_sex1_}
%\input{Stata_pvIV_sex1_}
%\input{Stata10_pvIV_sex1_}
%\input{Stata10_pvIV_PDVsex1_}
%\input{Stata10_pvIV_PDVstartsex1_}
%\input{Stata_pvfirstW_sex2_}
%\input{Stata_pvfirstg_sex2_}
%\input{Stata_pvIV_sex2_}
%\input{Stata10_pvIV_sex2_}
%\input{Stata10_pvIV_PDVsex2_}
%\input{Stata10_pvIV_PDVstartsex2_}



In the un-instrumented regression in the first column, both men and women have a negative wage coefficient. This implies that there may be time-varying omitted variables, such as changes in occupation amenities or skill requirements, not controlled for by the fixed effects. The instrumented specifications should avoid this endogeneity by identifying off of industry level wage changes (excluding the instrumented occupation). Indeed, the wage coefficient becomes positive in the instrumented regression in the second column. By contrast, the signs of the fraction female coefficients do not change with the addition of instruments. In both specifications men have no preference over the fraction female and women have a strong preference over the fraction female. However, the standard errors for men are high, so I cannot rule out moderate effects for men in either direction. 



%The lack of precision could be due to a lack of variation over time in men's labor market outcomes, which is corroborated by the much higher total sum of squares in the female regression.

%In fact, a regression of the log difference in shares on only time and occupation dummies produces and F statistic of 823 for women and 3167 for men.



%For women the results imply a preference for entering female occupations on the same order of magnitude as the preference for wage. Under the fixed effects specification, moving the fraction female from 0\% to 100\% would have an equivalent impact on log utility as increasing lifetime income by \$700,000 for all workers in this occupation. Under the $IV4$ specification this figure is \$1.56 million.



% PRoBLEM: this is log wages and log utility.... so if gender ratio moved from 0 to 1 then Log utility goes up by 1, similarly if wages go up by 1 million, Log utility goes up by 1

% so ubar =  beta * F  + delta * W, what is the meaning of beta?? but also we have log(u) = ubar + Wbar + eta 

% and workers maximize log utility which is equivalent to maximizing utilty for an individual worker but in terms of whether ToTAL surplus gets maximized could be different since u+p relatively equal > u +p unequal (when logged). I could make it so that multiplicative surplus is maximized so pairwise stability defined in terms of u*p then the wage cancels on the side where the match actually occurs (which is the counterfactual side) so u*p <= u'*p' where u' and p' are with whoever they are with which  might not be each other. if workers are maximizing logged utiltiy and firms logged profit than they should be maximizing also total logged surplus... so this is a different allocation than the standard but it is the allocation that is consistent with my context.

% but if I do this for a single occupation this will have an impact on other occupations through the wage equilibrium... so this is a partial equilibrium outcome... should I calculate expected utility if all occupations are 0% female vs. 100% female including the equilibrium wage effects?? or expected utility in one occupation if it is 0% or 100% including equilibrium wage effect

%So if lifetime income went up by one million dollars, utility would increase similarly as moving the number of women in the chosen occupation from 0\% to 100\%. 

%%%%%%%%%%%%%%%%%%%%%%%%%%%%%%%%%%%%%%%%%
% why am I reporting partial equilibrium results? report it with the wage moving??
% Problem: need to solve for new equilibrium 34 times for p25 for each occupation and the 34 more times for p75 for each occupation... need to run this overnight. at least I can do men and women at the same time?

% also do 10-20, 80-90 instead?

% also forget moving the wage since wage is in equilibrium? so make two columns, row 1 moving 10-20\% female, row 2, 80-90\% female, first column labor supply effect, second column equilibrium effect.



Figure \ref{prefs} shows the relationship between log utility and the fraction female for men and women. Women have increasing utility in the fraction female, and the increase is steeper the fewer women there are in the occupation. By contrast for men, utility is relatively flat in the fraction female and not statistically different from zero. The magnitude of the female preference is around twice the preference for log wages, meaning that if the log wage offers in an occupation went up by 10\%, this would have an equivalent effect on log utility ($u^F_o$) as a 5\% increase to the fraction female. For women the estimated preference for fraction female is also economically significant in terms of the impact on occupation choice. The average marginal effect of moving the fraction female in a single occupation from 20\% ($F=.2$) to 80\% ($F=.8$) would be to entice 124\% more women to enter that occupation in equilibrium, that is allowing wages to adjust to clear the market. So if an occupation moved from relatively male-dominated to relatively female-dominated that would just over double the number of women who would enter that occupation in equilibrium.

%In partial equilibrium, holding wages fixed, moving the fraction female in a single occupation from 20\% ($F=.2$) to 80\% ($F=.8$) would have an average marginal effect of moving 1352\% more women into that occupation under the $IV$ specification (see Table \ref{sumstat}). 


The non-wage utility that workers get from each of the occupations ($\alpha^g_o$) are reported in Table \ref{FEs}, measured in log millions of dollars. These are the fixed effects from the regressions in Tables \ref{IV1} and \ref{IV2} with the excluded category being ``Teachers, Postsecondary". Generally we see that women have higher utility than men for female occupations and vice versa. The only male-dominated occupations ($F<.2$) in which women have higher non-wage utility than men are Protective Service and Health Diagnosing Occupations, and there are no female-dominated occupations ($F>.8$) in which men have higher non-wage utility.

%In combination with the estimated willingness-to-pay from Table \ref{modelestimates}, they track relatively closely with the observed fraction female by occupation. Occupations with higher female utility and higher willingness-to-pay for women have more women in them. 

%However this effect is exaggerated because it is partial equilibrium. Holding equilibrium wages fixed means that this result only reflects the decisions of workers without firm response. 

%The impact of moving the log reservation wage from the 25th to 75th percentile is offered for comparison purposes and is much smaller, only an average marginal effect of 90\% more women or 53\% more men. A change in fraction female from 0 to 100 would be equivalent to about a 300\% increase in log wages in terms of the effect on women's utility.

%Fraction female from 0 to 100 increases female utility about 6 in cubic spec. so 6 = beta/100%*change in wage. Beta is 2. So 6 = 2/100%* change in wage. So the equivalent change in log wage is 100%*6/2, or 3*100% so 300%



% this would be a nice spot to run the counterfactuals setting um=uf and pim=pif and contrasting which contributes more to segregation



%\newpage
%\subsection{Adding Quadratic in Fraction Female}
%\input{Stata10_pvIV_PDVstartsex1__v2_s}
%\input{Stata10_pvIV_PDVstartsex2__v2_s}
%\clearpage
%
%%\input{Stata10_pvIV_sex1_s}
%%\input{Stata10_pvIV_PDVstartsex1_s}
%%\input{Stata10_pvIV_sex2_s}
%%\input{Stata10_pvIV_PDVstartsex2_s}
%%\clearpage
%
%\newpage
%\subsection{Adding Cubic in Fraction Female}
%\input{Stata10_pvIV_PDVstartsex1__v2_sc}
%\input{Stata10_pvIV_PDVstartsex2__v2_sc}
%\clearpage


%\subsection{Adding distance from parity in the fraction female}
%Let genderM be defined as distance from parity in a male dominated occupation and genderF be the same in a female dominated occupation.
%
%\input{Stata10_pvIV_sex1_MF}
%\input{Stata10_pvIV_PDVstartsex1_MF}
%\input{Stata10_pvIV_sex2_MF}
%\input{Stata10_pvIV_PDVstartsex2_MF}



%\citeA{Stock2008} recommend against the use of robust standard errors in the case of panel data with fixed time periods. 

%\subsection{Scaling in terms of wage coefficient}
%\input{testing}

%\newpage
%\subsection{Adding in interaction of Wage and gender.}
%
%\input{Stata_pvfirstW_sex1_Wgender}
%\input{Stata_pvfirstg_sex1_Wgender}
%\input{Stata_pvIV_sex1_Wgender}
%\clearpage
%\input{Stata_pvfirstW_sex2_Wgender}
%\input{Stata_pvfirstg_sex2_Wgender}
%\input{Stata_pvIV_sex2_Wgender}


%  also IV with size of the labor force???

%BLP instruments have no variation across time so they should be almost collinear with the occupation fixed effects as is the problem with Nevo's cereal application





\section{Counterfactuals}\label{counterfactual}
% summarize impact of preference on segregation and wage gap here?

% two policy implications
% answer both
% then how to explain?
% wage adjustment section
	% status quo wage
	% why fixing wages messes it up
	% why equal pay is even worse

There are two policy implications from the result that women prefer to enter more female occupations. First, the gender composition of occupations might be path-dependent, allowing past preferences to affect current sorting. For example, Food Preparation and Service Occupations is majority female at 74\%, but employers are willing to pay more for men (See Table \ref{modelestimates} $ \frac{\widebar{WTP}^F_o}{\widebar{WTP}^M_o} = .74$), and men value the occupation more than women (See Table \ref{FEs}, fixed effect of 1.66 vs. 1.35), so we might not expect ex ante for this occupation to be majority female. However the occupation was historically female, and the preference for women to work with women means that firms can hire women for cheaper. It is an empirical question as to whether this occupation could become majority male if enough men chose it and drove up wages for women. I use my model to test empirically whether temporarily nudging more men or women into any given occupation might lead to convergence to a new (and possibly more efficient) gender composition.

Second, the estimated gender preference has implications for policies that target gender equity through wage equity. Because employers compensate women through a combination of wage and non-wage amenities, including the fraction female, allowing employers to pay women less in female-dominated fields and more in male-dominated fields moderates segregation. Therefore policies to enforce equal pay for men and women will have two countervailing impacts on the gender wage gap: equalizing wages within occupation and increasing segregation across occupation.

\subsection{Homophily and Path-Dependence}

%\subsubsection{Finding Stable Equlibria}



%Although the male preference for working with men is close to zero and not statistically significant, these predicted future patterns hold broadly even if men are given a coefficient on fraction female at the lower bound of the confidence interval. So even if men had a relatively strong preference against working with women, most occupations would see little change in segregation based on endogenous evolution of the fraction female alone.




%\subsection{Steady States by Occupation}
The homophily I have estimated means that segregation could depend on initial conditions. If segregation is path dependent, then temporarily increasing the number of workers of one gender could have long run consequences through moving an occupation from one steady state fraction female to another.\footnote{For an illustrative model of multiple steady states in the fraction female, see appendix of \citeA{Pan2010}.} 



%My model can answer the question of whether occupations might have more than one steady state segregation pattern, where steady state means that the relative number of women entering the occupation is the same as the relative number of women leaving the occupation.

% On the other hand, I estimate that firms value men and women in Administrative Support roughly equally and women appear to like the occupation much more than men. So it seems unlikely that this occupation is 89\% female for purely for historical reasons. 

%(See Table \ref{modelestimates} $ \frac{\widebar{WTP}^F_o}{\widebar{WTP}^M_o} = .97$), (See Table \ref{FEs}, fixed effect of 2.04 vs. .36)

Using my model it is possible to search systematically for steady states in the fraction female. To do so, I allow wage offers ($ W^{g}_{o,t}$) to clear the market for each cohort of workers, and the fraction female to update endogenously across cohorts. I then graph the mapping between fraction female in the current period and the next period, by occupation, at ten equidistant starting points between 0\% female and 100\% female using 2012 parameter values, and show a fitted line through these points. By examining these graphs it is easy to find steady states where the fraction female this period is the same as the fraction female in next period, by observing intersections with the 45 degree line. If the fraction female crosses the 45 degree line multiple times, this implies multiple steady states in the fraction female.

%I fix all attributes of other occupations, including the fraction female, to focus only on the occupation at hand. I graph ten equidistant starting points between 0\% female and 100\% female using 2012 parameter values, and show a fitted line through these points on the graphs.



I find that every occupation has one unique steady state in the fraction female. This means that given my estimated parameter values, the model predicts current segregation patterns to emerge regardless of historical segregation patterns. The top left panels of Figures \ref{transitions17} and \ref{transitions83} show the unique steady states for Health Technologists and Technicians and Engineers, Architects, and Surveyors. Like all occupations, the unique steady states in these occupations are stable, meaning that any arbitrary perturbation to the fraction female will eventually converge back to the unique steady state. 

If the gender preference were approximately doubled, many occupations would have multiple steady states in the fraction female. The bottom left panel of Figures \ref{transitions17} and \ref{transitions83} shows these same occupations, but with homophily twice as large. In this case Health Technologists and Technicians still has a unique steady state, but Engineers, Architects, and Surveyors now has three steady states, two of which are stable. Note that a doubling of the gender preference seems unlikely since the estimated preference is already economically large.


%Furthermore, the predicted steady states are close to the observed 2012 values of fraction female. 

%%Almost all occupations are currently in their unique steady state. To test this I run simulations that allow the fraction female to evolve endogenously over time, holding every other aspect of labor supply and demand fixed. The only changes to worker utility are due to changes in the fraction female $F_{o,t-1}$, and changes to the equilibrium wage offers ($ W^{*g}_{o,t}$). Figure \ref{fig:sq1} shows the four occupations that are predicted to become substantially more female in the future, and the two occupations that are predicted to become substantially less female in the future, holding all else constant. All other occupations are close to their unique steady state and therefore remain similarly segregated over time, holding all else fixed.

%\footnote{I solve for the new $ W^{*g}_{o,t}$ the equates the supply and demand for workers and jobs, given the changes to worker utility due to the evolution of fraction female.}

%Health diagnosing occupations  is predicted to move from 38\% to 60\% female, engineers, architects and surveyors from 17\% to 30\%, math computer and natural science from  32\% to 43\%, and precision production occupations 25\% to 38\%. Meanwhile machine operators, fabricators, assemblers, testers moves from 20\% to 5\% female, and metal, wood, plastic, print, textile from 30\% to 10\%. 

%All other inputs are fixed at the 2012 values, the last year of data.  



%The first question I can answer with this model is whether occupation segregation is stable over time. Since I find that women have a preference over the fraction female, we might expect the fraction female to evolve endogenously causing some occupations to become more male or female over time, even if every other aspect of labor supply and demand is fixed. 



%Given my parameter estimates, the model predicts that only a few occupations will see substantial evolution in the fraction female in the future. 




%The results of this simulation indicate that most occupations may have already converged to a steady state in the fraction female, meaning that the same number of women are entering and leaving each occupation with each successive cohort. The next question is whether these steady states are stable, or whether, for example, a shock to the number of women in an occupation in a given cohort could cause the occupation to converge to a new steady state.  







%Wage adjustment in equilibrium is the main reason that we observe unique steady states in every occupation. To take a closer look at how wages adjust to compensate for the preference over fraction female, I plot the response of wage to a change in the fraction female in two case studies. 

%Specifically I set a female dominated occupation to be 0\% female, and separately, a male dominated occupation to be 100\% female, and observe how wage adjustment facilitates convergence back to the unique stable steady state, which occurs after about eight cohorts of workers.

% How do I explain that the transition graphs show no state dependence even without wage adjustment, but the graphs with all of them moving do so state dependence??? mechanics and repairers stays female without wage adjustment...... and nurses stays male.... what the hell... moving all occupatoins at the same time causes more polarization?


%Firm preferences generally mitigate the impact of the worker side gender preferences in determining sorting patterns. 

Since every occupation has a unique steady state fraction female, policies that only temporarily alter number of men or women in an occupation will not have long run consequences. I run simulations to illustrate the mechanisms that render temporary policies ineffective in the long run and find that compensating differentials play a key role in making segregation stable in equilibrium. 

My first simulation mimics a policy that temporarily encourages men to enter nursing to overcome gender barriers. In Figure \ref{nurses}, I set nursing (``Health Technologists and Technicians") to be 0\% female in 1960, when in reality nursing was close to 100\% female in 1960. The equilibrium wage offer for women in nursing that clears the market increases dramatically when nursing is a male occupation, to compensate women for their strong estimated preference against being in a highly male occupation. These high wage offers mean that some women in the next cohort (with high taste for nursing) are enticed to enter nursing. This entry of women then makes nursing more attractive for the following cohort of women, which in turn makes women cheaper for firms to hire, increasing demand for female nurses. This feedback loop continues until nursing is female-dominated and simulated wage offers have dropped to the levels estimated in the data.

%However, the gender preference is not strong enough to completely price women out of nursing once wages adjust to reflect labor demand. It is still cost-effective for firms to hire some women who really love nursing jobs, and so the female wage offer rate must be set high to equate supply and demand by compensating women for the disutility of the low fraction female.

%In the simulation, 0\% male is a stable equilibrium if we fix wages and thereby ignore firm preferences. Without equilibrium wages, no women want to enter nursing when it is 0\% female because of the preference for working with women. 

 % and women are cheaper to hire.

In the second simulation, shown in Figure \ref{mechanics}, I set ``Mechanics and Repairers" to be a 100\% female in 1960 when in reality it was close to 0\% female. Men with the highest taste for the occupation still choose it. Female wages then skyrocket as more and more men start to become Mechanics and Repairers. After about eight cohorts, women are no longer affordable to hire and the occupation has converged to its unique steady state at around 0\% female.

If it were not for compensating wage differentials both of these occupations would be path dependent in the fraction female. The top right panel of Figure \ref{transitions17} illustrates the steady states in nursing if wages are not allowed to clear the market. Nursing would remain 0\% female if it ever dropped below around 35\% female. Similarly, mechanics and repairers would converge to a majority female steady state at around 80\% female if it ever reached 80\% or more female. With wage offers free to adjust, however, policies would need to change underlying job amenities, or labor demand through quotas or wage subsidies, in order to have long-run impact.


%shown in Figures \ref{fig:sq1} through \ref{fig:sq4}, where the fraction female is allowed to evolve in all occupations at once. 

%\footnote{As discussed above, in Figures \ref{fig:sq1} through \ref{fig:sq4} all occupations are allowed to evolve to a fixed point at the same time. In these graphs, changes in fraction female in any one occupation have repercussions for the fraction female in all other occupations.}

%The transition graphs also show that the further away the starting point is from the stable equilibrium, the faster the convergence towards that equilibrium. 

%The location of the stable equilibria depend on the location and shape of the male and female labor supply curves. More occupations have an equilibrium at close to 0\% female than 100\% female. This is partly due to the lack of male preferences over fraction female, meaning that it is possible for women to be priced out of a very male occupation, but difficult for men to be priced out of female occupations. This is exacerbated by the fact that in most occupations firms have a higher willingness to pay for male workers.


\subsection{Homophily and Equal Pay for Equal Work}

%Although the estimated parameter values produce only one steady state in the fraction female for each occupation, the model does allow for multiple steady states to emerge under certain conditions. Below, I explore two mechanisms that could lead to multiple equilibria in the fraction female: first, doubling the magnitude of the preference for women to work with women, and second, fixing equilibrium wages so they are not allowed to adjust and form compensating differentials.

%predict multiple equilibria to emerge with a stronger gender preference, or by fixing equilibrium wages. Below I run simulations to find the multiple equilibria by occupation with a higher gender preference, and with a fixed wage. In the following section I further discuss the mechanisms by which wages prevent multiple equilibria in the fraction female.

%I find that doubling the preference over the fraction female produces multiple equilibria in some occupations, such as Postsecondary Teachers, and Engineers Architects and Surveyors. These occupations have one steady state at close to 0\% female and another that is majority female. Intuitively, a very strong gender preference makes it cheapest for an occupation to hire either all men or majority women. The initial fraction female would determine to which steady state the occupation converges. However, a preference that doubles the magnitude of my estimate is likely implausible, given that my estimate is already economically large.

%It turns out that allowing wages to vary freely to equate supply and demand plays a key role in eliminating multiple equilibria.

%Next, I simulate the endogenous evolution of the fraction female when wages are not allowed to adjust to clear the market. Fixing the wage offer distributions at their 2012 locations, I find that most occupations (27 out of 34) have two stable equilibria, one that is near zero percent female and one that is majority female. Modeling equilibrium wages is therefore critical to accurately assess the impact of the gender preference.

One policy to intended to increase equity in the labor market is equal pay for equal work. In this last set of simulations, I fix the ratio of female to male wage offers $\left( \frac{W^F_{o,t}}{W^M_{o,t}} \right) $ to be equal to the ratio of female to male labor force participation, as measured in hours over the lifetime by gender and occupation. This fixed ratio means that on average men and women must be offered the same per hour worked within an occupation.\footnote{Note that individual firms can still offer men and women differently based on firm-specific amenities and this can still result in wage gaps if, for instance, only the most female-friendly firms hire women.} Factors that would normally affect $\left( \frac{W^F_{o,t}}{W^M_{o,t}} \right) $, such as willingness-to-pay by occupation or differential taste for occupation (including based on the fraction female), can no longer directly influence wage offers.

In this simulation 13 out of 34 occupations have multiple steady states in the fraction female.  Of the remaining 21 occupations that have only one steady state, 16 are only stable at 0\% female, and 5 at close to 100\% female. The bottom right panels of Figures \ref{transitions17} and \ref{transitions83} illustrate the steady states for nursing and engineering under equal pay for equal work. In both cases the stable steady states are more segregated then when wages are free to adjust. 

Overall, equal pay for equal work results in even higher segregation, and a larger gender pay gap, than the status quo. Table \ref{counterfactuals} shows a comparison of segregation and gender wage gaps under various counterfactual regimes.  In the second column of Table \ref{counterfactuals} we see that if there were no preference on the part of women against working in male-dominated occupations, women would earn 85\% of what men earn, and only 24\% of workers would have to change occupation to reach 50\% female.\footnote{This is the Duncan Segregation Index \cite{OtisDudleyDuncan1955}.} Requiring that men and women be paid equally per hour of work in a given occupation wildly increases segregation. Under equal pay for equal work 80\% of workers would have to change occupations for each occupation to have 50\% women, compared to 41\% in the status quo. As a result of this increase in segregation, equal pay actually widens the pay gap slightly from the status quo, putting the wages of women at 65\% those of men instead of 71\% in the status quo.

%Table \ref{counterfactuals} summarizes the gender wage gap and a measure of segregation under various counterfactual regimes, as compared with the status quo. In the status quo model as estimated, the ratio of female to male lifetime wages, or gender wage gap, is 0.71, and 41\% of worker would have to change occupations to achieve parity, a measure called the Duncan Segregation Index \cite{OtisDudleyDuncan1955}. If there were no preference on the part of women against working in male-dominated occupations, the gender wage gap would be reduced to 0.85, and the Duncan Segregation Index would be only 24\%. Not allowing wage adjustment to compensate for changes in the fraction female would result in a similar wage gap as in the status quo, 0.70, and a higher segregation index of 54\%. 

% men cannot be offered higher wages than women in female dominated occupations, and

The mechanism for the increased segregation under equal pay for equal work is that women cannot be offered higher wages than men in male-dominated occupations. Similarly, men cannot be offered higher wages than women in occupations they dislike, such as nursing. Therefore under equal pay, women exit male occupations such as engineering, and men exit female occupations such as nursing, resulting in more segregated occupations. Female-dominated occupations have the lowest wages, in part to negatively compensate for the increase in the fraction female, so this increased segregation results in an increase in the gender wage gap. The overall wage levels for both men and women also fall resulting in an increase in non-employment. Note that this result depends heavily on the strong gender preference I estimate on the part of women. My model simulations show that equal pay would reduce segregation and the gender wage gap in the absence of the estimated homophily preference.

%\footnote{Note that women still earn less than men within each occupation, both in the status quo and equal pay simulation. This is because they are valued less by firms, so only the firms that have the best amenities for women and can pay women the least hire women.}
% Nathan note: put this in appendix?

%In most cases this means that women must be offered a higher percent of what is offered to men, as compared to the status quo.

%\footnote{The actual wages paid will still depend on which jobs are accepted, with the most attractive jobs being filled first at the lowest wages.}

%As a result, equal pay for equal work produces even more segregation in the counterfactual than fixing the wages at their 2012 values (which may reflect compensating differentials by gender or productivity differences in addition to hours differences). The mechanism is that 

%Why does fixed ratio mean fewer multiple equilibria than the completely fixed simulation? but the single equilibria are very segregated? Hence you get more segregation?





%\subsection{Counterfactual Summary}



%Figures \ref{transitions17} and \ref{transitions83} summarize the steady state results under the various counterfactual scenarios for two example occupations. Health Technologists and Technicians has a single female dominated steady state, but with fixed wages it also has a steady state at 0\% female. Under equal pay for equal work, both of these steady states become more female. Engineers, Architects, and Surveyors has a single male dominated steady state. With fixed wages it has two steady states, one close to 0\% female and one close to 100\% female, and under equal pay for equal work only the 0\% female steady state remains.

%Adjusters and Investigators follows a similar pattern.



%The equal pay requirement tends to increase women's pay by pinning it to that of men, but at the same time, the inability to adjust pay according to how men and women value occupations differently, and how firms value men and women differently, creates steady states that are close to 0\% or 100\% female, which drastically increases segregation. The increased segregation in turn lowers wages for women since concentration in certain occupations means low wage offers through the compensating differential of a high fraction female under the gender preference. Thus overall in my simulation, equal pay laws actually result in lower relative wages for women given the strong gender preference I estimate.







%%%%% TABLE RESULTS:
%		homophily 	no homophily		equal pay	 equal pay, no homophily   equal pay, 1/10 homophily equal pay half homophily
%wage gap:	.71		.85			.65			.82			.82				.81
%segregation:	41%		24%			80%			32%			35%				50%

%under each regime:
%				number of equilibria		location 1 		location 2		wage gap
%top 3 female occs
%top 3 male occs


%Wage adjustment produces less segregated outcomes through compensating differentials: as more women enter an occupation women's wages go down, and as fewer women enter women's wages go up. 

%Figures \ref{ftransitions17} through \ref{ftransitions83} show the same set of occupations as Figures \ref{transitions17} through \ref{transitions83} but with fixed wages. 

%In all of these examples we have an equilibrium that is majority female, and an equilibrium that is close to zero percent female, implying that allowing wages to vary freely to equate supply and demand plays a key role i limiting the number of steady states in the fraction female.

%The wage moderates the impact of the preference of women to enter more female occupations, as it would over any endogenous amenity. 

%As more women enter an occupation it becomes more attractive, but at the same time wages go down as employers are able to attract more women at lower cost, thus ultimately dampening the supply of women to the occupation. Likewise as women leave an occupation, the wage offered to women in that occupation goes up, which increase female labor supply to that occupation and dampens the movement towards 0\% female. It is therefore crucial that the estimated model allows wages to adjust to equate supply and demand.

%Figure \ref{fig:tipping} illustrates the stylized model for the case in which the gender preference produces both an all male and a mixed equilibrium (mixed meaning between 0\% and 100\% female). I estimate that all occupations have only the mixed equilibrium. This is because the gender preference is not strong enough to cause female labor supply to actually cross male labor supply, meaning that employers are always able to hire a few women who really love the job for cheap, making a 0\% female occupation unsustainable.



%If the preference for working with women is strong enough, as women leave a male occupation female wages will rise enough to price women out of the occupation and produce a 0\% female equilibrium. Because I do not estimate a preference on the part of men to work with men, they will not be similarly priced out of female occupations, so a 100\% female equilibrium is unlikely. As a result with a female preference, I would expect occupations to have equilibria at either 0\% female, somewhere in between 0\% and 100\% female, or both. 

% IS THIS INTERESTING?? CUT THIS?
%In simulating future sorting patterns with endogenous wages, I see that some occupations still converge to almost 0\% female, whereas no occupation lands above 90\% female. This is because firms are willing to pay more for men, and men do not have a preference against working with women. So if enough women leave a male occupation, female wages can become so high that it is no longer profitable to hire any women. On the other hand, as men leave a female occupation, female wages cannot go negative, so firms are still willing to pay for some men.




%So rather than having stable equilibria at 0\% or 100\% female as might be predicted by the \citeA{Pan2010} model, starting at such extreme points actually leads to large movements to the more moderate stable equilibrium in most cases. 

%Furthermore, only a few occupations are characterized by very male or female dominated equilibria. At greater than 80\% female we have only Health Services Occupations with a fixed point at around 87\% female, and Health Assessment and Treating and Therapists at around 85\% female. At under 20\% female we have Agriculture Forestry and Fishing at around 12\% female, Machine Operators Fabricators Assemblers and Testers at around 15\% female, Road Rail and Water Transportation at around 10\%, and both Mechanics and Repairers, and Construction and Extraction, at around 5\% female.


%% INCLUDE THESE WAGE RATIO RESULTS???
%To explore the role of wages I run two sets of simulations. First I fix wages at their current value, allowing for no market clearing adjustment, and use transitions graphs as described above to observe the equilibria in fraction female. Next I impose only that the ratio of male to female wages is fixed, but that otherwise wages can adjust freely up or down. In both sets of simulations multiple equilibria emerge in some occupations.




% no wage adjustment results:
% Engineers only has female equilibrium
% Only male equilibrium: Machine operators, fabricators, assemblers, testers; metal wood plastic print textile; food preparation and service occupations; sales workers, retail, and personal services; writers artists entertainers athletes; social scientists, lawyers, judges, urban planners, librarians; teachers except postsecondary, 

% how does teachers only have a male equilibrium??? women like it more and are good at it......

% two stable equilibria, Sf is upward sloping for first few women, then downward sloping as more women enter, then upward sloping again at high fraction female. WHY? the equilibrium at 0% makes more sense than the once I'm seeing at like 20% female. but this is in the pan model where wages adjust. here we are talking no wage adjustment.





	%include fixedratio graphs ?
	
%To explore the role of wages, I fix wages at their current value, allowing for no market clearing adjustment, and use transitions graphs as described above to observe the equilibria in fraction female. In the case of fully fixed wages, 27 of the 34 occupations have two stable equilibria as opposed to one. Figures \ref{ftransitions17} through \ref{ftransitions83} show the same set of occupations as Figures \ref{transitions17} through \ref{transitions83} but with fixed wages. 

%Since wages are fixed, these simulations effectively hold the employers' responses fixed. In the absence of firms adjusting wages to meet demand, the impact of fraction female on worker labor supply is less dramatic. The result is that whether an occupation converges to more male or female can depend on starting point, see for example \ref{ftransitions17}. Freely moving wages reinforce convergence, and convergence to a single equilibrium.

% start here
	








%Movement to 100\% female dominated is mitigated by the fact that wages are bounded below by zero, so without male gender preferences moving the male reservation wage distribution up, it is unlikely that every woman will be cheaper to hire than every man. This is exacerbated by the fact that in most occupations firms have a higher willingness to pay for male workers.

%On the other hand, in occupations with low fraction female, wage adjustment does allow movement to close to 100\% male. In these cases, the female reservation wage distribution moves up to the point that it is entirely above the firm's willingness-to-pay. This is again a phenomenon that hinges on the willingness-to-pay gap.

%The extent to which occupations exhibit more or less tipping depends on the extent to which they are not yet in a stable equilibrium, which depends on the relative non-wage utility of male and female workers, the willingness-to-pay gap, and also these values in all other occupations since the workers' and firms' outside options matter for labor supply and demand. 

%\subsection{Simulation from Initial Parity}
%If preferences over an endogenous amenity like the fraction female are strong enough, there could be multiple equilibria. In this case long run sorting patterns could depend on the initial conditions, or historical segregation, which would determine which equilibrium is selected. As a first pass at testing this hypothesis, I begin my simulation with all occupations at 50\% female in the initial year of 1960. The figures \ref{fig:ip1} through \ref{fig:ip4} show that this does not affect the long run outcome. The patterns are shockingly similar today as if occupations had begun from the observed segregated position. 
%
%This result suggests that we are currently in a stable equilibrium, and that parity is close enough to this stable equilibrium that it leads to convergence. It could also suggest that each occupation has in fact only one stable equilibrium in the fraction female, and would converge to this equilibrium regardless of any starting point. In the next section I further explore this question by searching for all equilibria in the fraction for each occupation separately.


\section{Conclusion}

%Most occupations appear to be close to their stable equilibrium based on the limited amount of movement that occurs both in the status quo future simulation, and in counterfactuals.

%I examine the consequences of gender segregation as an endogenous occupation attribute and find that current preferences could result in long run tipping patterns in the future akin to those identified in \citeA{Pan2010}. However 



%\textit{``Supporting women STEM students and researchers is not only an essential part of America's strategy to out-innovate, out-educate, and out-build the rest of the world; it is also important to women themselves"} (Office of Science and Technology Policy under the Obama administration) %\cite{ObamaSTEM}.

%\textit{``One of the things that I really strongly believe in is that we need to have more girls interested in math, science, and engineering. We've got half the population that is way underrepresented in those fields and that means that we've got a whole bunch of talent...not being encouraged the way they need to."} (President Obama 2013)\footnote{\url{https://obamawhitehouse.archives.gov/administration/eop/ostp/women}}


%\textit{``We've got half the population that is way underrepresented in those fields [math, science, and engineering] and that means that we've got a whole bunch of talent...not being encouraged the way they need to."} (President Obama 2013)\footnote{\url{https://obamawhitehouse.archives.gov/administration/eop/ostp/women}}

Nursing was historically female due to explicit gender barriers, and has continued to be female-dominated to this day. In this paper, I study whether occupations such as nursing continue to be segregated today because workers prefer to work in occupations with the same gender (homophily), as opposed to persistent gender barriers or preferences for occupations. I distinguish homophily using a structural model of the labor market. I impose a specific payoff structure within transferable utility matching that allows me to leverage the wage distribution to separately identify firm and worker preferences, and I use industry-level demand shocks to trace out worker taste for fraction female and wages.

%Occupations have been highly gender segregated without much change for the past 30 years. It is an open question as to whether this segregation results from fixed worker and firm preferences, or rather from historical inertia from the explicit occupational barriers of the 1950s, 60s or later. For example, nursing could be a female occupation because women like nursing, or because nursing was historically female and women prefer female occupations, and likewise for male dominated occupations such as engineering.

%I find that women prefer more female occupations, but firms' and workers' preferences are far more important in explaining segregation. This answer is critical for thinking about policies to reduce segregation and the gender wage gap.

I find that women prefer not to work in male-dominated occupations, and this increases segregation (from around 24\% to 41\%) and the gender wage gap (from around 85\% to 71\%). I find that, for the estimated parameters, each occupation has a single unique steady state in the fraction female. This means, for example, that pushing more women into STEM or men in nursing will not lead more to follow, rather the occupation will tend back to its original sorting pattern. Equity policies need to target the underlying causes of segregation.

%Policies that seek to change segregation alone are likely ineffectual and treating a symptom not a cause. 

%It appears that reducing segregation requires changing the underlying utility workers get from occupations, or how firms view workers by gender.


%%%%%%%%%%%%%%%%%%%% Add this stuff???
%There are two important takeaways from this paper for policymakers. First, short run shocks to the fraction female in occupations, such as temporarily pushing more men into nursing or women into STEM, will not have long run consequences since there is no historical inertia to overcome. Policymakers seeking to reduce segregation should find specific supply and demand factors that concern them and address them directly, whether it be occupation amenities or discrimination. Second, policymakers should keep in mind that changes to labor supply and demand might have an outsized impact on segregation due to the feedback loop of women wanting to work with women. Making an occupation more attractive to women might have the unintended consequence of making the occupation highly female dominated, and thereby lowering the wages for those women over time through a compensating differential. 

%Lastly, future research should address the source of the gender preference to determine it is due to gender identity, or workplace environment and amenities. If workplace environment and amenities are the cause, then policymakers who care about reducing segregation and the gender wage gap might consider how to address these to make male-dominated occupations more welcoming to women.

% James: "I think this is very interesting and I think you could expand it a bit.  One idea:
%"While it is beyond the scope of this analysis, future work may seek to understand the source of preferences for gender composition.  First, preferences for gender composition could reflect simple homophily whereby women prefer to work with other women.  If this is the case, firms face a bit of a Catch-22 where the only way to attract more women is to first employ more women.  Alternatively, preferences for gender composition could actually reflect preferences for inclusive and supportive environments.  If this is the case, male dominated firms can still attract women by improving their culture."



%Although occupations would be less segregrated if women did not care about the fraction female, it is not the case that occupations are segregated because of historical patterns. 
%women do prefer female-dominated occupations, but that this preference is not strong enough to overcome workers' preferences for occupations and firms' preferences for workers. 

%. As the more women enter an occupation, the occupation becomes more attractive to women and therefore wages for women go down, which slows the entry of women into the occupation. I find that the gender preference would have to be twice as large as I what I estimate, which is already economically quite large, in order for wages to be unable to adjust enough to facilitate movement back to the original fraction female. 

My model is well positioned to study gendered policies with equilibrium consequences, such as equal pay for equal work laws.  Given my estimated parameters, equal pay actually leads to drastically more segregation, which ultimately increases the gender wage gap despite narrower gaps within occupation. The mechanism for this result, and the unique steady state result, is compensating wage differentials. To equate supply and demand, women must be paid more to work in male-dominated occupations.

%Compensating wage differentials play a key role in producing a unique steady state, which begs the question of what would happen if policymakers were to introduce stickiness. 

%The obvious example is equal pay for equal work laws. It is difficult to measure the impact of these laws empirically because they have existed federally since 1963, and overlap with other state level gender equity laws. 



%The impact of these laws is hard to measure empirically since they are generally implemented with a suite of other gender equity policies that are variously enforced.

The mechanism of segregation explored in this paper, homophily preference over the fraction female, merits future research. In particular, understanding the cause of women preferring not to work in male-dominated occupations could be policy relevant. Reducing this preference could be an important policy target if it is the result of male sexism, and this could be explored using regional variation in sexism as in \citeA{CGP19}. Future work could also embed a Roy-style model within the matching model to account for unobserved productivity heterogeneity. Lastly, although this paper focused only on the fraction female in occupations, future work could extend this model to look at race, age, or any other group preference or endogenous amenity, and the mitigating effects of compensating differentials on segregation.

%This assumption might bias my estimates, depending on the distributions of unobserved productivity heterogeneity by gender and occupation. Future work could embed a Roy model.

%Understanding the mechanism is important because the preference appears to cause greater segregation and lower wages for women.
% On the other hand the preference may result from other workplace amenities correlated with the fraction female. 
	
%	so people should be aware of this mechanism and research more
%adding women could change "fixed" attributes of occupations- but I find it doesn't??
%I find it's partly pref for fraction female- why? need to research more
%this hurts women

%The estimates of this model are likely imperfectly predictive of the future of occupation gender segregation, but the model does prove useful for learning about the role of wages in two-sided matching with an endogenous amenity. This paper focused on the fraction female in occupations, but future work could use this model to look at race, age, or any other group preference or endogenous amenity, and the moderating effect of price adjustment.

%In addition, although I do not find evidence of tipping between multiple equilibria given my estimated parameter values, it is clear that this could occur under different circumstances, such as a stronger gender preference or stickier wages (for example equal pay for equal work laws). Future work is needed to fully understand occupation gender segregation, and could benefit from considering how tipping is mitigated by compensating differentials.

%Talk about tipping patterns???

%It is an open policy question as to what sort of encouragement would lead more women to enter male occupations and vice versa. Ideally we might see that a few men or women entering a field might lead to a flood of followers. I find women do prefer to go into occupations that already have more women. However, womens' preference is not strong enough that simply putting more women in a field will lead more women to enter in the long run. I also find no evidence that men prefer to enter occupations that already have more men. 

%As barriers have fallen and women's labor force participation has skyrocketed we have seen some occupations ``tip" from male to female while others remain virtually unchanged. To understand these patterns better, I 

%The preference on the part of women to work with women increases gender segregation by creating a feedback loop that amplifies the impact of gender differences in labor supply and demand. According to my simulations, the``tipping" patterns documented by \citeA{Pan2010} might be the result of changes in the perceived productivity of men and women, compounded by a feedback loop from the preference of women to work with women. This feedback loop mechanism also exacerbates the gender wage gap by lowering wages in female dominated occupations through compensating differentials.





%Around 40\% of men or women would have to change occupations in order for all occupations to be 50\% female.

%Are occupations segregated because this is what workers and firms prefer? or is it because of historical patterns?


%What causes segregation?
%need to know if we care about changing it
%and this causes dynamics
%can you encourage and have long run impact?



%Policymakers view reducing occupation gender segregation as a way to narrow the gender wage gap and even increase economic productivity. 

%From the Office of Science and Technology Policy under the Obama administration: 

% GAO or BLS reports on gender... inspector general
% former policy people, paul krugman, 
% we need everyone in STEM and compsci
% obama quote on pushing women into stem, then my results indicate maybe this is a bad idea

%One way to justify the economic benefit of reducing segregation is if workers have preferences over the gender of their colleagues, and this results in historical dependence. 



%On the other hand my simulations indicate that changes to occupation amenities or the perceived productivity of men or women would lead to long run changes in the gender of occupations.

%Could encouragement of say women to enter STEM, or men to enter nursing, lead to long run changes in the gender composition of these fields? The answer is that it depends on the type of encouragement. I find a strong preference on the part of women to work with other women, but that this preference is not strong enough that simply putting more women in a field will lead more women to enter in the long run. Long run gender composition is only responsive to changes in labor supply and demand, such as occupation amenities or the valuation of men and women by firms.



% If wages were not allowed to adjust in my model, the level of gender preference that I estimate would imply that occupations could ``tip" between male or female based on small changes in composition. 

%A model without endogenous wages would in fact imply multiple equilibria in most occupations. If the gender preference were twice as strong this would also be too strong for wages to be able to compensate, causing multiple equilibria. This suggests that thinking about how changes to occupations interact dynamically with changes to wages, and gender preference, could be important for future work on gender segregation.



% I estimate that occupations do not have the potential to be either male or female depending on historical patterns. Therefore short-run shocks to the fraction female in occupations do not cause tipping, rather convergence back to the original fraction female.
%Policymakers hoping to reduce gender segregation should focus on changing fundamental economic parameters such as non-wage amenities of jobs and the perceived productivity of men and women, as these things can move the stable equlibrium and cause long run changes in segregation. 


%\citeA{Schelling1971} showed that even a small degree of homophily can result dramatic changes in group composition. \citeA{Pan2010} documents such dramatic changes in occupation gender composition in the United States. This motivates the question of whether slight encouragement of say, women to enter STEM, could result in tipping patterns as observed in the past. 

%I find a strong preference on the part of women to work with other women. However, I estimate that this preference is not strong enough to produce multiple stable equilibria in the fraction female. Therefore, according to my estimates, putting more women in an occupation does not mean that more women will follow. In fact in my simulations the occupation will always converge back to its unique stable equilibrium in the long-run.

%According to my estimates, convergence to a more productive stable equilibrium is not possible based on shocks to gender composition alone.



% consistent with tipping from stable male to stable female equilibria. This motivates the question of whether current segregation patterns are optimal, or could some occupations produce more surplus at a more male or female equilibrium.

%However, this preference is not strong enough to produce tipping points in the fraction female by occupation as proposed by \citeA{Pan2010}. I estimate that all occupations have only one stable fraction female. 



%Sorting patterns depend on the extent to which men and women value occupations differently, and occupations value men and women differently. Changes to occupational attributes or the perceived productivity of men and women could therefore move the stable equlibrium and cause long run changes in segregation.

%  talk about the firm vs. worker preferences simulation here???



%how changes to labor supply and demand interact dynamically with wages, and gender preference to cause changes in segregation.

%or in a more general context, how allowing prices to vary freely to equate supply and demand can moderate the outcome of a group preference.

%We are stable! Don't push people around it's not optimal! but keep an eye out for male prefs in the future!? also for equal pay laws! Cuz tipping is real and could happen, just not now given my estimates!



%The evidence that I uncover points to the importance of womens' preferences for entering female-dominated occupations and against entering male-dominated occupations, with little to no role for mens' preferences against entering female-dominated occupations. 

%Furthermore the matching model allows market clearing wages to adjust, which can exacerbate or attenuate the tipping dynamics depending on the preference structure, and where an occupation is relative to its stable equilibrium. 

%Most occupations appear to be close to their stable equilibrium based on the limited amount of movement that occurs both in the status quo future simulation, and in the counterfactual simulations of initial parity and shocks to nurses and mechanics.

%In future work I will simulate the impact of integrative policies such as wage subsidies or amenity requirements on segregation dynamics. 

%\section{Preliminary First Stage: 386 occupations}
%Below are preliminary results of the decomposition of approximately $u^g_o$. \textbf{Results need to be adjusted for scale estimation!}
%
%\begin{align*}
% u^g_o = Z_y\beta_X + Z^g_o\beta_X + g_X(F)+ \xi_{y} \\ 
%\end{align*}
%
%Where the parameters of interest are the $\delta$ in $g(F)$:
%$$g_X(F)= \delta_{x}^1 F + \delta_{x}^2 \mathcal{I}(F<.5)(F-.5)^2 + \delta_{x}^3\mathcal{I}(F>.5)(F-.5)^2 $$
%
%The following plots are of $g_X(F)$ for each type of worker.

%Each column represents a type of worker, with columns 1-4 education categories (less than high school, high school degree, some college, college degree). The tables are for childless males, males with children, childless females, and females with children, in that order.


%\clearpage
%\subsection{Linear and preference against minority in fraction female}
%\begin{center}
%\includegraphics[width=.8\textwidth]{utility_sexratio_3}
%\includegraphics[width=.8\textwidth]{utility_sexratio_4}
%\includegraphics[width=.8\textwidth]{utility_sexratio_1}
%\includegraphics[width=.8\textwidth]{utility_sexratio_2}
%\end{center}


%
%\section{Instrument for Fraction Female}
%With a strong instrument it would no longer be necessary to make $g_X(F)$ a flexible non-linear function, although these results highlighting a quadratic preference against being a minority for women are interesting. With an instrument we may believe that any impact of the fraction female is due to preferences over the fraction female rather than unobserved attributes in $\xi_Y$ that are correlated with fraction female. 
%
%I propose to use a Bartik instrument to 

%\section{Questions}
%\begin{enumerate}
%\item Do limitations on the sign of transfers (wages) preclude socially optimal outcome?
%\item What are the implications of assuming the coefficient on wages is one for both sides of the market?
%\item Can I make substitution patterns more reasonable, eg. nested logit?
%\item Can I allow the data to pick types of workers and tipping point for $g_X(F)$?
%\item Extension to continuous types?
%\item Convert wages to PDV lifetime income using method outlined by Hoxby or other.
%\end{enumerate}

%
%\section{Further work}
%\begin{enumerate}
%\item To what extent is sex ratio relationship mechanical if not instrumented?
%\item What is the intuition for the separation of worker and firm utility? 
%\item What is the intuition behind the additive separability assumption?
%\item What is the best choice for firm outside option?
%\item Find an instrument for fraction female in the oLS stage, use state variation? maternity leave policies?
%\item Do limitations on the sign of transfers (wages) preclude socially optimal outcome?
%\item Continuous types (avoid two stage estimation process), guarantees unique equilibrium?
%\item Identify types and occupation specific tipping points and use them instead of .5 in the parametric form of $g(F)$, or use semi-parametric estimator in logit.
%\end{enumerate}


\newpage

\clearpage


\subsection{Figures and Tables}



\begin{figure}[H]
\centering
\caption{Simulated Impact of Worker Preferences on Observed Wage Distribution}
\label{workerpref}
%\footnotesize{\textbf{Sales Representatives, Finance, and Business Services}}
%\includegraphics[width=.7\textwidth]{"/Users/Miriam/OneDrive/Box Sync/TYPlocal/output/graphs/matlabwages/offerwages_occfinal8reg1trunc_1960_v3"}
\includegraphics[width=.75\textwidth]{"/Users/Miriam/OneDrive/Box Sync/TYPlocal/output/graphs/matlabwages/IDwages_workerpref_v3_2019"}
\end{figure}

\begin{figure}[H]
\centering
\caption{Simulated Impact of Firm Preferences on Observed Wage Distribution}
\label{firmpref}
%\footnotesize{\textbf{Health Service occupations}}
%\includegraphics[width=.7\textwidth]{"/Users/Miriam/OneDrive/Box Sync/TYPlocal/output/graphs/matlabwages/offerwages_occfinal18reg1trunc_1960_v3"}
\includegraphics[width=.75\textwidth]{"/Users/Miriam/OneDrive/Box Sync/TYPlocal/output/graphs/matlabwages/IDwages_firmpref_v3_2019"}
\end{figure}



\begin{figure}[H]
\centering
\caption{Lifetime Wages in Health Service Occupations}
\label{health}
%\footnotesize{\textbf{Health Service occupations}}
%\includegraphics[width=.7\textwidth]{"/Users/Miriam/OneDrive/Box Sync/TYPlocal/output/graphs/matlabwages/offerwages_occfinal18reg1trunc_1960_v3"}
\includegraphics[width=.6\textwidth]{"/Users/Miriam/OneDrive/Box Sync/TYPlocal/output/graphs/matlabwages/pofferwages_occfinal18reg1trunc_1960"}
\end{figure}

\begin{figure}[H]
\centering
\caption{Lifetime Wages in Sales Representatives, Finance, and Business Services}
\label{sales}
%\footnotesize{\textbf{Sales Representatives, Finance, and Business Services}}
%\includegraphics[width=.7\textwidth]{"/Users/Miriam/OneDrive/Box Sync/TYPlocal/output/graphs/matlabwages/offerwages_occfinal8reg1trunc_1960_v3"}
\includegraphics[width=.6\textwidth]{"/Users/Miriam/OneDrive/Box Sync/TYPlocal/output/graphs/matlabwages/pofferwages_occfinal8reg1trunc_1960"}
\end{figure}

%
%\begin{figure}[H]
%\centering
%\caption{Model Fit}
%\label{fig:modelwages1}
%\includegraphics[width=.8\textwidth]{logwages13reg1trunc_occfinal}
%\end{figure}
%
%\begin{figure}[H]
%\centering
%\caption{Model Fit}
%\label{fig:modelwages2}
%\includegraphics[width=.8\textwidth]{logwages14reg1trunc_occfinal}
%\end{figure}

\begin{figure}[H]
\caption{Male and Female Log Utility by Fraction Female in Occupation}
\label{prefs}
\begin{center}
\includegraphics[width=.8\textwidth]{"/Users/Miriam/OneDrive/Box Sync/TYPlocal/output/graphs/Prefs_cublin"}
%\includegraphics[width=.8\textwidth]{"/Users/Miriam/OneDrive/Box Sync/TYPlocal/output/graphs/Prefs_cubic"}
%\begin{minipage}{.8\textwidth}
%\begin{tablenotes}
%\footnotesize
%\item Results of instrumental variables regressions with occupation fixed effects on 6 waves of Census and ACS data (1960-2012), 34 occupations, using wage offers estimated earlier using MLE.
%\end{tablenotes}
%\end{minipage}
\end{center}
\end{figure}


%\begin{center}
%\begin{figure}[H]
%\centering
%\caption{Status Quo: Simulated Occupation Segregation Patterns}
%\label{fig:sq1}
%%\includegraphics[width=.8\textwidth]{"/Users/Miriam/OneDrive/Box Sync/TYPlocal/output/graphs/Final_v2_clinv2_nowW_occ61"}
%\includegraphics[width=.8\textwidth]{"/Users/Miriam/OneDrive/Box Sync/TYPlocal/output/graphs/HPDVstart__clinv2_sc"}
%\end{figure}

%% in these the wage updates after 1960
%\begin{figure}[H]
%\centering
%\caption{Status Quo: Simulated Occupation Segregation Patterns}
%\label{fig:sq1}
%%\includegraphics[width=.8\textwidth]{"/Users/Miriam/OneDrive/Box Sync/TYPlocal/output/graphs/Final_v2_clinv2_nowW_occ61"}
%\includegraphics[width=.8\textwidth]{"/Users/Miriam/OneDrive/Box Sync/TYPlocal/output/graphs/FPDVstart_1_clinv4_sc"}
%\end{figure}
%\begin{figure}[H]
%\centering
%\caption{Status Quo: Simulated Occupation Segregation Patterns}
%\label{fig:sq2}
%%\includegraphics[width=.8\textwidth]{"/Users/Miriam/OneDrive/Box Sync/TYPlocal/output/graphs/Final_v2_clinv2_nowW_occ62"}
%\includegraphics[width=.8\textwidth]{"/Users/Miriam/OneDrive/Box Sync/TYPlocal/output/graphs/FPDVstart_2_clinv4_sc"}
%\end{figure}
%\begin{figure}[H]
%\centering
%\caption{Status Quo: Simulated Occupation Segregation Patterns}
%\label{fig:sq3}
%%\includegraphics[width=.8\textwidth]{"/Users/Miriam/OneDrive/Box Sync/TYPlocal/output/graphs/Final_v2_clinv2_nowW_occ63"}
%\includegraphics[width=.8\textwidth]{"/Users/Miriam/OneDrive/Box Sync/TYPlocal/output/graphs/FPDVstart_3_clinv4_sc"}
%\end{figure}
%\begin{figure}[H]
%\centering
%\caption{Status Quo: Simulated Occupation Segregation Patterns}
%\label{fig:sq4}
%%\includegraphics[width=.8\textwidth]{"/Users/Miriam/OneDrive/Box Sync/TYPlocal/output/graphs/Final_v2_clinv2_nowW_occ64"}
%\includegraphics[width=.8\textwidth]{"/Users/Miriam/OneDrive/Box Sync/TYPlocal/output/graphs/FPDVstart_2_clinv4_sc"}
%\end{figure}
%\end{center}

%%%%%%%%%%%%%%% wage simulations
\begin{figure}[H]
\centering
\caption{Setting Health Technologists and Technicians to 0\% Female in 1960, Changes in Wage Offers}
\label{nurses}
\includegraphics[width=.9\textwidth]{"/Users/Miriam/OneDrive/Box Sync/TYPlocal/output/graphs/Wg2PDVstart_occ17_clinv2_0_17sc"}
\end{figure}


\begin{figure}[H]
\centering
\caption{Setting Mechanics and Repairers to 100\% Female in 1960, Changes in Wage Offers}
\label{mechanics}
\includegraphics[width=.9\textwidth]{"/Users/Miriam/OneDrive/Box Sync/TYPlocal/output/graphs/Wg2PDVstart_occ51_clinv2_1_51sc"}
\end{figure}

%\begin{center}
%\begin{figure}[H]
%\centering
%\caption{Initial Parity: Simulated Occupationx Segregation Patterns}
%\label{fig:ip1}
%\includegraphics[width=.8\textwidth]{"/Users/Miriam/OneDrive/Box Sync/TYPlocal/output/graphs/Final_v2_clinv2_f_occ61"}
%\end{figure}
%\begin{figure}[H]
%\centering
%\caption{Initial Parity: Simulated Occupation Segregation Patterns}
%\label{fig:ip2}
%\includegraphics[width=.8\textwidth]{"/Users/Miriam/OneDrive/Box Sync/TYPlocal/output/graphs/Final_v2_clinv2_f_occ62"}
%\end{figure}
%\begin{figure}[H]
%\centering
%\caption{Initial Parity: Simulated Occupation Segregation Patterns}
%\label{fig:ip3}
%\includegraphics[width=.8\textwidth]{"/Users/Miriam/OneDrive/Box Sync/TYPlocal/output/graphs/Final_v2_clinv2_f_occ63"}
%\end{figure}
%\begin{figure}[H]
%\centering
%\caption{Initial Parity: Simulated Occupation Segregation Patterns}
%\label{fig:ip4}
%\includegraphics[width=.8\textwidth]{"/Users/Miriam/OneDrive/Box Sync/TYPlocal/output/graphs/Final_v2_clinv2_f_occ64"}
%\end{figure}
%\end{center}


% transitions with moving wage
%\begin{figure}[H]
%\centering
%\caption{Transitions in Fraction Female Across Periods}
%\label{transitions10}
%\includegraphics[width=.9\textwidth]{"/Users/Miriam/OneDrive/Box Sync/TYPlocal/output/graphs/Splot24_occ_10all"}
%\end{figure}

\begin{figure}[H]
\centering
\caption{Transitions in Fraction Female Across Periods}
\label{transitions17}
\includegraphics[width=.9\textwidth]{"/Users/Miriam/OneDrive/Box Sync/TYPlocal/output/graphs/Splot24_occ_17all"}
\end{figure}

%\begin{figure}[H]
%\centering
%\caption{Transitions in Fraction Female Across Periods}
%\label{transitions35}
%\includegraphics[width=.9\textwidth]{"/Users/Miriam/OneDrive/Box Sync/TYPlocal/output/graphs/Splot24_occ_35all"}
%\end{figure}

%\begin{figure}[H]
%\centering
%\caption{Transitions in Fraction Female Across Periods}
%\label{transitions51}
%\includegraphics[width=.9\textwidth]{"/Users/Miriam/OneDrive/Box Sync/TYPlocal/output/graphs/Splot24_occ_51all"}
%\end{figure}

\begin{figure}[H]
\centering
\caption{Transitions in Fraction Female Across Periods}
\label{transitions83}
\includegraphics[width=.9\textwidth]{"/Users/Miriam/OneDrive/Box Sync/TYPlocal/output/graphs/Splot24_occ_83all"}
\end{figure}

%% transitions with fixed wage
%\begin{figure}[H]
%\centering
%\caption{Transitions in Fraction Female Across Periods: Fixed Wages}
%\label{ftransitions10}
%\includegraphics[width=.6\textwidth]{"/Users/Miriam/OneDrive/Box Sync/TYPlocal/output/graphs/Splot_nowW_occ_10"}
%\end{figure}
%
%\begin{figure}[H]
%\centering
%\caption{Transitions in Fraction Female Across Periods: Fixed Wages}
%\label{ftransitions17}
%\includegraphics[width=.6\textwidth]{"/Users/Miriam/OneDrive/Box Sync/TYPlocal/output/graphs/Splot_nowW_occ_17"}
%\end{figure}
%
%\begin{figure}[H]
%\centering
%\caption{Transitions in Fraction Female Across Periods: Fixed Wages}
%\label{ftransitions35}
%\includegraphics[width=.6\textwidth]{"/Users/Miriam/OneDrive/Box Sync/TYPlocal/output/graphs/Splot_nowW_occ_35"}
%\end{figure}
%
%\begin{figure}[H]
%\centering
%\caption{Transitions in Fraction Female Across Periods: Fixed Wages}
%\label{ftransitions51}
%\includegraphics[width=.6\textwidth]{"/Users/Miriam/OneDrive/Box Sync/TYPlocal/output/graphs/Splot_nowW_occ_51"}
%\end{figure}
%
%\begin{figure}[H]
%\centering
%\caption{Transitions in Fraction Female Across Periods: Fixed Wages}
%\label{ftransitions83}
%\includegraphics[width=.6\textwidth]{"/Users/Miriam/OneDrive/Box Sync/TYPlocal/output/graphs/Splot_nowW_occ_83"}
%\end{figure}
%
%
%%  transitions with double the preference parameter for women
%\begin{figure}[H]
%\centering
%\caption{Transitions in Fraction Female Across Periods: Gender Preference Doubled}
%\label{ftransitions10}
%\includegraphics[width=.6\textwidth]{"/Users/Miriam/OneDrive/Box Sync/TYPlocal/output/graphs/Splot_occ_10double"}
%\end{figure}
%
%\begin{figure}[H]
%\centering
%\caption{Transitions in Fraction Female Across Periods: Gender Preference Doubled}
%\label{ftransitions17}
%\includegraphics[width=.6\textwidth]{"/Users/Miriam/OneDrive/Box Sync/TYPlocal/output/graphs/Splot_occ_17double"}
%\end{figure}
%
%\begin{figure}[H]
%\centering
%\caption{Transitions in Fraction Female Across Periods: Gender Preference Doubled}
%\label{ftransitions35}
%\includegraphics[width=.6\textwidth]{"/Users/Miriam/OneDrive/Box Sync/TYPlocal/output/graphs/Splot_occ_35double"}
%\end{figure}
%
%\begin{figure}[H]
%\centering
%\caption{Transitions in Fraction Female Across Periods: Gender Preference Doubled}
%\label{ftransitions51}
%\includegraphics[width=.6\textwidth]{"/Users/Miriam/OneDrive/Box Sync/TYPlocal/output/graphs/Splot_occ_51double"}
%\end{figure}
%
%\begin{figure}[H]
%\centering
%\caption{Transitions in Fraction Female Across Periods: Gender Preference Doubled}
%\label{ftransitions83}
%\includegraphics[width=.6\textwidth]{"/Users/Miriam/OneDrive/Box Sync/TYPlocal/output/graphs/Splot_occ_83double"}
%\end{figure}

% 




%\subsection{Tables}
\input{Model_estimates_v4}

%\subsection{Comparison of Average Shares: Modal occupation vs. starting occupation 1960-2012}
%
%\input{share_census_lifeinc}
\newpage

\input{Stata10_pvIV_PDVstartsex1__v4_}

%\input{Stata10_pvIV_PDVstartsex2__v2_}

\newpage

\input{Stata10_pvIV_PDVstartsex2__v4_sc}

%\input{peffect}

%\input{peffect2018}  corrected for coding error not updating deltas


%\input{peffect20182080} % corrected for other coding error no updating wage THIS IS MOST UP TO DATE

%\input{FE_nowW_PDVstart_clinv2_sc}\label{FEs}
\newpage
\input{FE3_PDVstart_clinv2_sc}

\input{Counterfactuals}

%\input{"/Users/Miriam/OneDrive/Box Sync/TYPlocal/output/tables/Wage_gap_occs_clinv4_mw0.tex"}

\clearpage
\newpage
\section{Appendix}

\subsection{Transferable Utility Matching}
When a worker and a job match, total surplus is created from the match. In the worker's case the value of a match reflects the amenities of the job. A job might have a particularly collegial environment, or free child care for example. Amenities may be valued differently by gender. On the job side the payoff is the willingness-to-pay for a worker, which could reflect productivity, and differ by gender due to gender differences in turnover, differences in search cost by gender, differences in productivity, or devaluation, for example. The wage determines the split of the total surplus between the worker and the firm.

The most general payoff structure in a matching model would allow each possible match between a worker $i$ and a job $j$ to have its own unobserved match quality. To make the problem empirically tractable, I assume that no portion of the payoff depends on unobservable characteristics of both firm and worker, which is a standard assumption in empirical matching. So although the surplus may depend on $i$ or $j$, it may not depend on $i$ and $j$.

\begin{assumption}
Additive Separability: No component of surplus depends on unobserved characteristics of both workers and firms.
\end{assumption}

Formally, let $g$ denote gender, which is observed as either male ($M$) or female ($F$) in this model. Let $o$ denote occupation. We therefore have workers $i \in g \in G= \{M,F\}$ and jobs $j \in o \in O = \{1,2,...,34\}$.\footnote{Thirty-four occupations are chosen according to data constraints discussed below.} Under additive separability we have that the total surplus from a match between worker $i$ and job $j$, $S^i_j$, can be decomposed:

\begin{align}
S^i_j = S^g_o + \eta^i_o + \xi^g_j
\end{align}


Note that there are components that vary at the occupation*gender level ($S^g_o$), the occupation*worker level ($\eta^i_o$), and the gender*job level ($\xi^g_j$), but never the worker*job level. In other words, additive separability implies that there is no $\xi^i_j$ or $\eta^i_j$. This assumption is important because it allows me to separate the matching problem into two separate discrete choice problems, one for each side of the market \cite{Galichon2015}.

The components of total surplus that depend on unobservables of either the worker ($\eta^i_o$) or the job ($\xi^g_j$) can theoretically come from the worker's utility function, the job payoff function, or both. In order to gain identifying power from the observed wage distribution, and because my research question is focused the the role of worker utility in occupation choice, I assume all unobserved components of surplus originate from the worker's utility. This means that only workers have preferences over unobservables, and jobs care only about whether they chose to hire a male or female worker.

\begin{assumption}
$\eta^i_o$ and $\xi^g_j$ are primitives in the worker's utility function.
\end{assumption}

In other words, each worker has an individual taste for each occupation ($\eta^i_o$) and each job differs in how attractive it is to men and women ($\xi^g_j$). The job amenity heterogeneity can be thought of as any component of the attractiveness of a job that is orthogonal to the overall attractiveness of the occupation, which is included in $S^g_o$. For example child care offerings at a particular employer might differ relative to the average child care offerings in that occupation.


\subsubsection{Equilibrium Wages} \label{equilibrium}
In the following section I outline conditions for a matching to be feasible and stable. I then introduce the equilibrium wage vector and show that it supports feasibility and stability.

\subsubsection{Feasibility}
A matching is feasible if every worker is matched to at most one job and every job matched to at most one worker. Formally, following \citeA{Galichon2015}, let $\mu^i_j$ be equal to either $0$ or $1$ where $1$ indicates a match between worker $i$ and job $j$. Then for every $i$ and $j$ a feasible matching has

$$ \sum_{k \in \mathcal{J}} \mu^i_k \leq 1 \text{ and }  \sum_{k \in \mathcal{I}} \mu^k_j \leq 1$$

Similarly following \citeA{Galichon2015}, the matching must be feasible given the number of men and women and jobs in each occupation available in the market, or

$$ \sum_{j \in \mathcal{J}} \mu^g_j \leq n_g, \text{ } \forall g \text{ and }  \sum_{i \in \mathcal{I}} \mu^i_o \leq n_o, \text{ } \forall o$$

\subsubsection{Stability}
%Pairwise stability implies that for any worker and any job in the candidate equilibrium, the sum of their individual surplus must be greater than the total surplus if they were to form a match with each other. 

Intuitively, pairwise stability implies that no worker and job that are not currently matched with each other, would prefer to match with each other. Let $i$ and $j$ be a so called ``blocking pair", and let $i$ be currently matched to $j(i)$ and $j$ to $i(j)$. Then pairwise stability states that the sum of the individual surpluses from the existing matches ($i$ with $j(i)$ and $j$ with $i(j)$) must be greater than the surplus of the blocking pair ($i$ and $j$). Therefore even with any possible transfer, $i$ and $j$ will not both prefer to match with each other, because the total possible surplus is lower.

\begin{definition}
Pairwise Stability: In a matching where $i$ is paired with $j(i)$ and $j$ is paired with $i(j)$, it must be the case that $u^{i}_{j(i)}+ \pi^{i(j)}_j \geq u^{i}_{j} + \pi^{i}_j, \hspace{3mm}  \forall i,j$. In addition, each worker and job must attain higher surplus than their outside option, or $u^{i}_{j(i)} \geq u^{i}_{N}$ and $\pi^{i(j)}_j \geq \pi^{N}_j $, where $N$ represents not working for the worker, and not hiring for the firm.
 \end{definition}
 
Note that on the left hand side $u^{i}_{j(i)}+ \pi^{i(j)}_j$ includes the wage paid out to the worker and by the job in their respective matches. On the right hand side $u^{i}_{j} + \pi^{i}_j$ the wage will cancel within the match leaving the underlying total surplus. 

%Therefore, even though it could be the case that in the absence of equilibrium wages both the worker and the job would prefer to match with each other, it must be the case that the total surplus they get from their respective matchings including wages must be higher.

Following \citeA{Shapley1972a}, the pairwise stable matching will be unique and the competitive equilibrium will coincide with the pairwise stable matching, but the competitive equilibrium wage vector may not be unique. I assume the observed wages are the equilibrium wages described in \citeA{Galichon2015} and \citeA{Salanie2014}. These are the wages that make workers indifferent over jobs within each occupation, and jobs indifferent over workers within each gender. As the sample size of men and women goes to infinity, the equilibrium wages will be unique \cite{Galichon2015}.

\subsubsection{Proof of Pairwise Stability}\label{sec.Stability}

Workers choose an occupation to maximize utility, and firms choose a worker to maximize rate of return, so the chosen job $j^*$ and worker $i^*$ respectively must satisfy

\begin{align*}
j^* \in o^* &= \argmax_o ( u^{g}_o + W^g_o   + \eta^i_o ) \\
i^* \in g^* &= \argmax_g (WTP^g_o -  W^g_o -\xi^g_j)
\end{align*}

From this it is clear than within an occupation, workers are indifferent to which job they are matched to, and likewise within gender, jobs are indifferent to which worker they are matched to.

This implies that if worker $i$ were to match with a different job within the same occupation, we would have $u^{i}_j =u^{i}_{-j}$, and likewise for job $j$, $\pi^{i}_j = \pi^{-i}_j $, therefore the pairwise stability inequality holds trivially for observationally equivalent (same $g$ and $o$) candidate matches:

$$u^{i}_{j(i)}+ \pi^{i(j)}_j = u^{i}_{j} + \pi^{i}_j \hspace{5mm} \forall i,i(j) \in g \hspace{3mm} \forall j,j(i) \in o$$

Now consider matching worker $i$ to a job in a different occupation. Both workers and jobs choose the occupation or gender that produces the highest payoff for them, given the wage vector. Let the optimal occupation be $o^*$ and optimal gender $g^*$. Therefore we know that for worker $i$

$$u^{i}_{j(i)} > u^{i}_{j} \hspace{5mm} \forall j(i) \in o^* \text{  and  } \forall j \in o \neq o^*$$

and for job $j$

$$\pi^{i(j)}_{j} > \pi^{i}_j \hspace{5mm} \forall i(j) \in g^* \text{  and  } \forall i \in g \neq g^*$$

Therefore pairwise stability holds with strict inequality for all candidate matches that are not observationally equivalent (different $g$ or $o$) to the competitive equilibrium.

The second part of pairwise stability is the requirement that the choice payoffs be greater than the outside option payoffs. Recall that the outside option for the worker is remaining unemployed is equal to the idiosyncratic taste for non-employment, $ u^i_N = \eta^i_N$. The value to the firm of not hiring a worker is simply zero, $\pi^j_N = 0$.

%An interesting result from \citeA{Salanie2013a} (p.15) is that although the preference structure may be such that the jobs have an ordering $\eta^i_o$ over workers and workers have an ordering $\xi^g_j$ over jobs, the post-transfer utility will always accumulate such that job $j$ receives $\xi^g_j$ and worker $i$ receives $\eta^i_o$. Intuitively this results from the transfers serving to make workers indifferent over jobs within each type, and jobs indifferent over workers within each type.


Another key aspect of the equilibrium wage vector is that it must be feasible, which in the case of this labor market is equivalent to equating supply and demand at the level of male and female workers and occupations. \citeA{Crawford1981} and \citeA{RothSotomayor} prove the existence of such an equilibrium in a model with transfers. Intuitively, as long as the common component of wage, or $W^g_o$, is free to adjust, supply and demand can adjust until the market clears. 

%The empirical implications of market clearing will be discussed in the empirical section below.

% Roth and Sotomayor chapter 9: we need continuous utilty functions and must assume that some amount of money could induce you to switch jobs. They also assume that the utility functions map from R to R...  this is ok for me since log(W) is normal and therefore infinites support

% Crawford and Knoer also lay out the mechanism for feasible matching.


%\begin{align*}
%u^i_g &= \frac{ u^g_o*W^g_o * \xi^g_j * e^{ \eta^i_o} } {\xi^g_j}    \\
%&= u^g_o*W^g_o * e^{ \eta^i_o} 
%\end{align*}


   


%%% THESE gRAPHS USE STARTINg oCCUPATIoN AND 2000-2012 DATA FoR THE SIPP NoT 1960
%\begin{center}
%\clearpage
%\includegraphics[width=.8\textwidth]{"/Users/Miriam/OneDrive/Box Sync/TYPlocal/output/graphs/SIPPlifeinc_age_occ_0"}
%\newline
%\includegraphics[width=.8\textwidth]{"/Users/Miriam/OneDrive/Box Sync/TYPlocal/output/graphs/PSIDlifeinc_age_occ_0"}
%\clearpage
%\includegraphics[width=.8\textwidth]{"/Users/Miriam/OneDrive/Box Sync/TYPlocal/output/graphs/SIPPlifeinc_age_occ_1"}
%\newline
%\includegraphics[width=.8\textwidth]{"/Users/Miriam/OneDrive/Box Sync/TYPlocal/output/graphs/PSIDlifeinc_age_occ_1"}
%\clearpage
%\includegraphics[width=.8\textwidth]{"/Users/Miriam/OneDrive/Box Sync/TYPlocal/output/graphs/SIPPlifeinc_age_occ_2"}
%\newline
%\includegraphics[width=.8\textwidth]{"/Users/Miriam/OneDrive/Box Sync/TYPlocal/output/graphs/PSIDlifeinc_age_occ_2"}
%\clearpage
%\includegraphics[width=.8\textwidth]{"/Users/Miriam/OneDrive/Box Sync/TYPlocal/output/graphs/SIPPlifeinc_age_occ_3"}
%\newline
%\includegraphics[width=.8\textwidth]{"/Users/Miriam/OneDrive/Box Sync/TYPlocal/output/graphs/PSIDlifeinc_age_occ_3"}
%\clearpage
%\includegraphics[width=.8\textwidth]{"/Users/Miriam/OneDrive/Box Sync/TYPlocal/output/graphs/SIPPlifeinc_age_occ_4"}
%\newline
%\includegraphics[width=.8\textwidth]{"/Users/Miriam/OneDrive/Box Sync/TYPlocal/output/graphs/PSIDlifeinc_age_occ_4"}
%\clearpage
%\includegraphics[width=.8\textwidth]{"/Users/Miriam/OneDrive/Box Sync/TYPlocal/output/graphs/SIPPlifeinc_age_occ_5"}
%\newline
%\includegraphics[width=.8\textwidth]{"/Users/Miriam/OneDrive/Box Sync/TYPlocal/output/graphs/PSIDlifeinc_age_occ_5"}
%\clearpage
%\includegraphics[width=.8\textwidth]{"/Users/Miriam/OneDrive/Box Sync/TYPlocal/output/graphs/SIPPlifeinc_age_occ_6"}
%\newline
%\includegraphics[width=.8\textwidth]{"/Users/Miriam/OneDrive/Box Sync/TYPlocal/output/graphs/PSIDlifeinc_age_occ_6"}
%\clearpage
%\end{center}







%\subsection{Plots of Model Estimates over Time}
%\begin{center}
%\includegraphics[width=.4\textwidth]{"/Users/Miriam/OneDrive/Box Sync/TYPlocal/output/graphs/Wpi_results_2"}
%\includegraphics[width=.4\textwidth]{"/Users/Miriam/OneDrive/Box Sync/TYPlocal/output/graphs/Wpi_results_3"}
%\newline
%\includegraphics[width=.4\textwidth]{"/Users/Miriam/OneDrive/Box Sync/TYPlocal/output/graphs/Wpi_results_4"}
%\includegraphics[width=.4\textwidth]{"/Users/Miriam/OneDrive/Box Sync/TYPlocal/output/graphs/Wpi_results_5"}
%\newline
%\includegraphics[width=.4\textwidth]{"/Users/Miriam/OneDrive/Box Sync/TYPlocal/output/graphs/Wpi_results_6"}
%\includegraphics[width=.4\textwidth]{"/Users/Miriam/OneDrive/Box Sync/TYPlocal/output/graphs/Wpi_results_7"}
%\newline
%\includegraphics[width=.4\textwidth]{"/Users/Miriam/OneDrive/Box Sync/TYPlocal/output/graphs/Wpi_results_8"}
%\includegraphics[width=.4\textwidth]{"/Users/Miriam/OneDrive/Box Sync/TYPlocal/output/graphs/Wpi_results_9"}
%\end{center}
%\clearpage
%
%\begin{center}
%\includegraphics[width=.4\textwidth]{"/Users/Miriam/OneDrive/Box Sync/TYPlocal/output/graphs/Wpi_results_10"}
%\includegraphics[width=.4\textwidth]{"/Users/Miriam/OneDrive/Box Sync/TYPlocal/output/graphs/Wpi_results_11"}
%\newline
%\includegraphics[width=.4\textwidth]{"/Users/Miriam/OneDrive/Box Sync/TYPlocal/output/graphs/Wpi_results_12"}
%\includegraphics[width=.4\textwidth]{"/Users/Miriam/OneDrive/Box Sync/TYPlocal/output/graphs/Wpi_results_13"}
%\newline
%\includegraphics[width=.4\textwidth]{"/Users/Miriam/OneDrive/Box Sync/TYPlocal/output/graphs/Wpi_results_14"}
%\includegraphics[width=.4\textwidth]{"/Users/Miriam/OneDrive/Box Sync/TYPlocal/output/graphs/Wpi_results_15"}
%\newline
%\includegraphics[width=.4\textwidth]{"/Users/Miriam/OneDrive/Box Sync/TYPlocal/output/graphs/Wpi_results_16"}
%\includegraphics[width=.4\textwidth]{"/Users/Miriam/OneDrive/Box Sync/TYPlocal/output/graphs/Wpi_results_17"}
%\end{center}
%\clearpage
%
%\begin{center}
%\includegraphics[width=.4\textwidth]{"/Users/Miriam/OneDrive/Box Sync/TYPlocal/output/graphs/Wpi_results_18"}
%\includegraphics[width=.4\textwidth]{"/Users/Miriam/OneDrive/Box Sync/TYPlocal/output/graphs/Wpi_results_19"}
%\newline
%\includegraphics[width=.4\textwidth]{"/Users/Miriam/OneDrive/Box Sync/TYPlocal/output/graphs/Wpi_results_20"}
%\includegraphics[width=.4\textwidth]{"/Users/Miriam/OneDrive/Box Sync/TYPlocal/output/graphs/Wpi_results_21"}
%\newline
%\includegraphics[width=.4\textwidth]{"/Users/Miriam/OneDrive/Box Sync/TYPlocal/output/graphs/Wpi_results_22"}
%\includegraphics[width=.4\textwidth]{"/Users/Miriam/OneDrive/Box Sync/TYPlocal/output/graphs/Wpi_results_23"}
%\newline
%\includegraphics[width=.4\textwidth]{"/Users/Miriam/OneDrive/Box Sync/TYPlocal/output/graphs/Wpi_results_24"}
%\includegraphics[width=.4\textwidth]{"/Users/Miriam/OneDrive/Box Sync/TYPlocal/output/graphs/Wpi_results_25"}
%\end{center}
%\clearpage
%
%\begin{center}
%\includegraphics[width=.4\textwidth]{"/Users/Miriam/OneDrive/Box Sync/TYPlocal/output/graphs/Wpi_results_26"}
%\includegraphics[width=.4\textwidth]{"/Users/Miriam/OneDrive/Box Sync/TYPlocal/output/graphs/Wpi_results_27"}
%\newline
%\includegraphics[width=.4\textwidth]{"/Users/Miriam/OneDrive/Box Sync/TYPlocal/output/graphs/Wpi_results_28"}
%\includegraphics[width=.4\textwidth]{"/Users/Miriam/OneDrive/Box Sync/TYPlocal/output/graphs/Wpi_results_29"}
%\newline
%\includegraphics[width=.4\textwidth]{"/Users/Miriam/OneDrive/Box Sync/TYPlocal/output/graphs/Wpi_results_30"}
%\includegraphics[width=.4\textwidth]{"/Users/Miriam/OneDrive/Box Sync/TYPlocal/output/graphs/Wpi_results_31"}
%\newline
%\includegraphics[width=.4\textwidth]{"/Users/Miriam/OneDrive/Box Sync/TYPlocal/output/graphs/Wpi_results_31"}
%\includegraphics[width=.4\textwidth]{"/Users/Miriam/OneDrive/Box Sync/TYPlocal/output/graphs/Wpi_results_32"}
%\end{center}
%\clearpage
%
%\begin{center}
%\includegraphics[width=.4\textwidth]{"/Users/Miriam/OneDrive/Box Sync/TYPlocal/output/graphs/Wpi_results_33"}
%\includegraphics[width=.4\textwidth]{"/Users/Miriam/OneDrive/Box Sync/TYPlocal/output/graphs/Wpi_results_34"}
%\newline
%\includegraphics[width=.4\textwidth]{"/Users/Miriam/OneDrive/Box Sync/TYPlocal/output/graphs/Wpi_results_35"}
%\end{center}
%\clearpage


%\subsection{Static Worker Utility Decomposition}
%
%\input{Estimation_utility_year_1960}
%\input{Estimation_utility_year_1960IV}
%\input{Estimation_utility_year_1960IV_nocont}
%\clearpage
%\input{Estimation_utility_year_1970}
%\input{Estimation_utility_year_1970IV}
%\input{Estimation_utility_year_1970IV_nocont}
%\clearpage
%\input{Estimation_utility_year_1980}
%\input{Estimation_utility_year_1980IV}
%\input{Estimation_utility_year_1980IV_nocont}
%\clearpage
%\input{Estimation_utility_year_1990}
%\input{Estimation_utility_year_1990IV}
%\input{Estimation_utility_year_1990IV_nocont}
%\clearpage
%\input{Estimation_utility_year_2000}
%\input{Estimation_utility_year_2000IV}
%\input{Estimation_utility_year_2000IV_nocont}
%\clearpage
%\input{Estimation_utility_year_2012}
%\input{Estimation_utility_year_2012IV}
%\input{Estimation_utility_year_2012IV_nocont}



%\subsection{Reservation Wage Distributions}
%
%\begin{center}
%%\includegraphics[width=.9\textwidth]{offerwages1reg1trunc}
%\includegraphics[width=.9\textwidth]{offerwages_occfinal30reg1trunc}
%\end{center}
%\begin{center}
%%\includegraphics[width=.9\textwidth]{offerwages1reg1trunc}
%\includegraphics[width=.9\textwidth]{offerwages_occfinal1reg1trunc}
%\end{center}
%\begin{center}
%%\includegraphics[width=.9\textwidth]{offerwages1reg1trunc}
%\includegraphics[width=.9\textwidth]{offerwages_occfinal2reg1trunc}
%\end{center}
%\begin{center}
%%\includegraphics[width=.9\textwidth]{offerwages1reg1trunc}
%\includegraphics[width=.9\textwidth]{offerwages_occfinal3reg1trunc}
%\end{center}
%\begin{center}
%%\includegraphics[width=.9\textwidth]{offerwages1reg1trunc}
%\includegraphics[width=.9\textwidth]{offerwages_occfinal4reg1trunc}
%\end{center}
%\begin{center}
%%\includegraphics[width=.9\textwidth]{offerwages1reg1trunc}
%\includegraphics[width=.9\textwidth]{offerwages_occfinal5reg1trunc}
%\end{center}
%\begin{center}
%%\includegraphics[width=.9\textwidth]{offerwages1reg1trunc}
%\includegraphics[width=.9\textwidth]{offerwages_occfinal6reg1trunc}
%\end{center}
%\begin{center}
%%\includegraphics[width=.9\textwidth]{offerwages1reg1trunc}
%\includegraphics[width=.9\textwidth]{offerwages_occfinal7reg1trunc}
%\end{center}
%\begin{center}
%%\includegraphics[width=.9\textwidth]{offerwages1reg1trunc}
%\includegraphics[width=.9\textwidth]{offerwages_occfinal8reg1trunc}
%\end{center}
%\begin{center}
%%\includegraphics[width=.9\textwidth]{offerwages1reg1trunc}
%\includegraphics[width=.9\textwidth]{offerwages_occfinal9reg1trunc}
%\end{center}
%\begin{center}
%%\includegraphics[width=.9\textwidth]{offerwages1reg1trunc}
%\includegraphics[width=.9\textwidth]{offerwages_occfinal10reg1trunc}
%\end{center}




%
%
%\subsection{Distribution of Maximum of Extreme Value Distributions}
%
%The distributions of the $\epsilon$ given they are observed at the maximized utility only is the same as the unconditional distributions \cite{DePalma2007}, therefore we can estimate the scale parameter  $\sigma^g_{\xi}$ using the individual wage data as suggested in \citeA{Salanie2014}. To see this, follow the original proof of from \citeA{Domenchich1975} page 64, outlined here.
%
%First, note that 
%
%$$Pr(\Pi^g_o + \xi^g_j \leq Z) = Pr( \xi^g_j \leq Z - \Pi^g_o )=exp(-e^{-\frac{1}{\sigma^g_{\xi}} (Z - \Pi_{X_jY} + \sigma^g_{\xi} )})$$
%
%So we can define the CDF associated with alternative $X_k$ as the $F_{X_k}$, the CDF of $\xi^g_j$ with mean parameter shifted by $\Pi_{X_kY}$.
%
%$$F_{X_k} = exp(-e^{-\frac{1}{\sigma^g_{\xi}} (Z - \Pi_{X_kY} + \sigma^g_{\xi} )})$$
%
%Then we can express the probability of choosing a worker of type $X_1$ over a worker of type $X_2$ as 
%\begin{align*}
% Pr(\Pi_{X_2Y} + \epsilon_{X_2j} \leq \Pi_{X_1Y} + \epsilon_{X_1j}) &= F_{X_2}(  \Pi_{X_1Y} + \epsilon_{X_1j}) \\
%\end{align*}
%
%
%Then the probability that a number $Z$ is the maximum over all choices in $X$, indexed $k=1,...,n_k$:
%\begin{align*}
%Pr( \max_k \Pi_{X_kY} + \epsilon_{X_kj} \leq Z) &=  \prod_{k} F_{X_k}(  Z) \\
%&= \prod_{k} exp(-e^{-\frac{1}{\sigma^g_{\xi}} (Z - \Pi_{X_kY} + \sigma^g_{\xi} )}) \\
%&= exp(- \sum_{k} e^{-\frac{1}{\sigma^g_{\xi}} (Z - \Pi_{X_kY} + \sigma^g_{\xi} )}) \\
%&= exp(- e^{-\frac{1}{\sigma^g_{\xi}} (Z + \sigma^g_{\xi} )}  \sum_{k}  e^{\frac{\Pi_{X_jY}}{\sigma^g_{\xi}} }) \\
%\end{align*}
%
%Now we have the CDF of the maximum value of the random variable, it remains to recognize the scale and location parameters. Let the location parameter of the distribution of the maximum value be $\mu$:
%
%\begin{align*}
%e^{\frac{\mu}{\sigma^g_{\xi}}} = \sum_{k}  e^{\frac{\Pi_{X_jY}}{\sigma^g_{\xi}} }
%\end{align*}
%
%Then we have $\mu =  \sigma^g_{\xi} ln \sum_{k} e^{\frac{\Pi_{X_jY}}{\sigma^g_{\xi}}}$, or $\sigma^g_{\xi}$ times the log inclusive value. Thus the CDF of the maximized random variable is
%
%$$exp(- e^{ - \frac{1}{\sigma^g_{\xi}} (Z - \mu + \sigma^g_{\xi} )}  ) $$
%
%Therefore the scale parameter of the maximum value remains $\sigma^g_{\xi}$, the scale parameter of $\xi^g_j$.
%

%\subsection{Industry Wages by Sex and Year}
%\includegraphics[width=.8\textwidth]{"/Users/Miriam/OneDrive/Box Sync/TYPlocal/output/graphs/Industry_sex1"}
%\newline
%\includegraphics[width=.8\textwidth]{"/Users/Miriam/OneDrive/Box Sync/TYPlocal/output/graphs/Industry_sex2"}
%
%\newpage
%\includegraphics[width=.8\textwidth]{"/Users/Miriam/OneDrive/Box Sync/TYPlocal/output/graphs/Industry_lifeinc_sex1"}
%\newline
%\includegraphics[width=.8\textwidth]{"/Users/Miriam/OneDrive/Box Sync/TYPlocal/output/graphs/Industry_lifeinc_sex2"}


%
%
%\begin{center}
%\begin{scriptsize}
%\input{logit_female_nokid_v2_occ1990}
%\end{scriptsize}
%\end{center}
%
%\begin{center}
%\begin{scriptsize}
%\input{logit_female_kid_v2_occ1990}
%\end{scriptsize}
%\end{center}
%
%\begin{center}
%\begin{scriptsize}
%\input{logit_male_nokid_v2_occ1990}
%\end{scriptsize}
%\end{center}
%
%\begin{center}
%\begin{scriptsize}
%\input{logit_male_kid_v2_occ1990}
%\end{scriptsize}
%\end{center}
%
%
%\clearpage
%\subsection{Linear in fraction female}
%\begin{center}
%\includegraphics[width=.8\textwidth]{utility_sexratio_3_line}
%\includegraphics[width=.8\textwidth]{utility_sexratio_4_line}
%\includegraphics[width=.8\textwidth]{utility_sexratio_1_line}
%\includegraphics[width=.8\textwidth]{utility_sexratio_2_line}
%\end{center}
%
%
%
%\clearpage
%\subsection{$F+F^2+F^3$}
%\begin{center}
%\includegraphics[width=.8\textwidth]{utility_sexratio_3_cube}
%\includegraphics[width=.8\textwidth]{utility_sexratio_4_cube}
%\includegraphics[width=.8\textwidth]{utility_sexratio_1_cube}
%\includegraphics[width=.8\textwidth]{utility_sexratio_2_cube}
%\end{center}
%
%\clearpage
%\subsection{$F+F^2+F^3+F^4$}
%\begin{center}
%\includegraphics[width=.8\textwidth]{utility_sexratio_34}
%\includegraphics[width=.8\textwidth]{utility_sexratio_44}
%\includegraphics[width=.8\textwidth]{utility_sexratio_14}
%\includegraphics[width=.8\textwidth]{utility_sexratio_24}
%\end{center}
%
%\clearpage
%\section{Preliminary First Stage: 80 occupations}
%
%\begin{center}
%\begin{scriptsize}
%\input{logit_male_nokid_occ80}
%\end{scriptsize}
%\end{center}
%
%\begin{center}
%\begin{scriptsize}
%\input{logit_male_kid_occ80}
%\end{scriptsize}
%\end{center}
%
%\begin{center}
%\begin{scriptsize}
%\input{logit_female_nokid_occ80}
%\end{scriptsize}
%\end{center}
%
%\begin{center}
%\begin{scriptsize}
%\input{logit_female_kid_occ80}
%\end{scriptsize}
%\end{center}


\newpage

\bibliographystyle{apacite}
\bibliography{library} 





\end{document}